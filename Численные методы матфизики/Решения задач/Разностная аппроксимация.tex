\documentclass[a4paper, 12pt]{article}
\usepackage{cmap}
\usepackage{amssymb}
\usepackage{amsmath}
\usepackage{graphicx}
\usepackage{amsthm}
\usepackage{upgreek}
\usepackage{setspace}
\usepackage{color}
\usepackage{pgfplots}
\pgfplotsset{compat=1.9}
\usepackage[T2A]{fontenc}
\usepackage[utf8]{inputenc}
\usepackage[normalem]{ulem}
\usepackage{mathtext} % русские буквы в формулах
\usepackage[left=2cm,right=2cm, top=2cm,bottom=2cm,bindingoffset=0cm]{geometry}
\usepackage[english,russian]{babel}
\usepackage[unicode]{hyperref}
\newenvironment{Proof} % имя окружения
{\par\noindent{$\blacklozenge$}} % команды для \begin
{\hfill$\scriptstyle\boxtimes$}
\newcommand{\Rm}{\mathbb{R}}
\newcommand{\Cm}{\mathbb{C}}
\newcommand{\Z}{\mathbb{Z}}
\newcommand{\I}{\mathbb{I}}
\newcommand{\N}{\mathbb{N}}
\newcommand{\rank}{\operatorname{rank}}
\newcommand{\Ra}{\Rightarrow}
\newcommand{\ra}{\rightarrow}
\newcommand{\FI}{\Phi}
\newcommand{\Sp}{\text{Sp}}
\renewcommand{\leq}{\leqslant}
\renewcommand{\geq}{\geqslant}
\renewcommand{\alpha}{\upalpha}
\renewcommand{\beta}{\upbeta}
\renewcommand{\gamma}{\upgamma}
\renewcommand{\delta}{\updelta}
\renewcommand{\varphi}{\upvarphi}
\renewcommand{\phi}{\upvarphi}
\renewcommand{\tau}{\uptau}
\renewcommand{\lambda}{\uplambda}
\renewcommand{\psi}{\uppsi}
\renewcommand{\mu}{\upmu}
\renewcommand{\omega}{\upomega}
\renewcommand{\d}{\partial}
\renewcommand{\xi}{\upxi}
\renewcommand{\epsilon}{\upvarepsilon}
\newcommand{\intx}{\int\limits_{x_0}^x}
\newcommand\Norm[1]{\left\| #1 \right\|}
\newcommand{\sumk}{\sum\limits_{k=0}^\infty}
\newcommand{\sumi}{\sum\limits_{i=0}^\infty}
\newtheorem*{theorem}{Теорема}
\newtheorem*{cor}{Следствие}
\newtheorem*{lem}{Лемма}
\begin{document}
	\section*{Разностная аппроксимация дифференциального оператора}
	\subsubsection*{Условия}
	\begin{enumerate}
		\item Используя метод неопределенных коэффициентов, построить разностный оператор, аппроксимирующий $u''(x)$ на шаблоне $\text{Ш}(x) = \{x, x+h, x+2h\}$. Определить порядок аппроксимации и главный член погрешности. (\hyperlink{t1}{Решение})
		\item Используя метод неопределенных коэффициентов, построить разностный оператор, аппроксимирующий $u''(x)$ на шаблоне $\text{Ш}(x) = \{x-h, x, x+h\}$. Определить порядок аппроксимации и главный член погрешности. (\hyperlink{t2}{Решение})
		\item Используя метод неопределенных коэффициентов, построить разностный оператор, аппроксимирующий $u'(x)$ на нерегулярном шаблоне $\text{Ш}(x) = \{x-h_1, x, x+h_2\}$, $h_1 \ne h_2$. Определить порядок аппроксимации и главный член погрешности. (\hyperlink{t3}{Решение})
	\end{enumerate}
	
	\newpage
	\subsubsection*{Решения}
	\begin{enumerate}
		\item \hypertarget{t1}{}
		Итак, мы имеем дифференциальный оператор $$Lu(x) = u''(x),$$ и шаблон $$\text{Ш}(x) = \{x, x+h, x+2h\}.$$ Разностную аппроксимацию будем искать в виде линейной комбинации значений функции $u$ в точках шаблона, то есть $$L_h u(x) = a_0 u(x) + a_1 u(x+h) + a_2 u(x+2h).$$ Выпишем погрешность аппроксимации $\psi(x) = L_h u(x) - L u(x)$ в соответствии с нашими данными: $$\psi(x) = a_0 u(x) + a_1 u(x+h) + a_2 u(x+2h) - u''(x).$$
		Разложим правую часть этого уравнения в ряд Тейлора в окрестности точки $x$:
		\begin{multline*}
			\psi(x) = a_0 u(x) + a_1 \left(u(x) + h u'(x) + \dfrac{h^2}{2}u''(x) + \dfrac{h^3}{6}u'''(x)\right) +\\+ a_2 \left(u(x) + 2h u'(x) + \dfrac{4h^2}{2}u''(x) + \dfrac{8h^3}{6}u'''(x)\right) + O(h^4) - u''(x) .
		\end{multline*}
		При необходимости можно записать и больше членов разложения. Теперь преобразуем выражение справа, вынеся общие множители, следующим образом
		\begin{multline*}
			\psi(x) = (a_0 + a_1 + a_2)\cdot u(x) + (ha_1 + 2ha_2)\cdot  u'(x) +\\+ \left(\dfrac{h^2}{2}a_1 + \dfrac{4h^2}{2}a_2 - 1\right)\cdot u''(x) + \left(\dfrac{h^3}{6}a_1 +\dfrac{8h^3}{6}a_2\right)\cdot u'''(x) +O(h^4).
		\end{multline*}
		Нас интересует возможность найти неизвестные коэффициенты $a_k$ такими, чтобы погрешность аппроксимации была минимальной. Для этого коэффициенты при $u(x)$, $u'(x)$, $u''(x)$ мы приравниваем к нулю и получаем следующую систему линейных уравнений для отыскания неизвестных коэффициентов 
		$$\begin{cases}
			a_0 + a_1 + a_2 = 0,\\
			ha_1  + 2ha_2  = 0,\\
			\dfrac{h^2}{2}a_1 + \dfrac{4h^2}{2}a_2 - 1 = 0.
		\end{cases}$$
		Домножим второе уравнение на $h$, а третье на 2 и получим
		$$\begin{cases}
			a_0 + a_1 + a_2 = 0,\\
			h^2 a_1+ 2 h^2 a_2= 0,\\
			h^2 a_1 + 4h^2a_2 = 2.
		\end{cases}$$
		Вычтем из третьего уравнения второе и получим 
		$$\begin{cases}
			a_0 + a_1 + a_2 = 0,\\
			h^2 a_1+ 2 h^2 a_2= 0,\\
			2h^2a_2 = 2.
		\end{cases}$$
		Отсюда $$a_2 = \dfrac{1}{h^2}.$$
		Тогда из второго уравнения $$a_1 = -\dfrac{2}{h^2},$$ а из первого уравнения $$a_0 = \dfrac{1}{h^2}.$$
		Таким образом, мы можем построить разностную аппроксимацию дифференциального оператора, подставив найденные коэффициенты в записанный ранее общий вид,
		$$L_h u(x) = \dfrac{u(x)  -2 u(x+h) +  u(x+2h)}{h^2}.$$ 
		Остается лишь определить порядок аппроксимации и главный член погрешности. Для этого найденные коэффициенты подставляем в нашу крайнюю запись для погрешности аппроксимации (учитываем, что первые три слагаемых обращаются в ноль при подстановке)
		$$
		\psi(x) = \left(-\dfrac{h^3}{6}\cdot \dfrac{2}{h^2} +\dfrac{8h^3}{6} \cdot \dfrac{1}{h^2}\right)\cdot u'''(x) +O(h^4) = h u'''(x) + O(h^4) = O(h).
		$$
		То есть мы получаем аппроксимацию первого порядка, а главный член погрешности это $hu'''(x)$.
		
		\newpage
		\item 
		\hypertarget{t2}{}
		Итак, мы имеем дифференциальный оператор $$Lu(x) = u''(x),$$ и шаблон $$\text{Ш}(x) = \{x-h, x, x+h\}.$$ Разностную аппроксимацию будем искать в виде линейной комбинации значений функции $u$ в точках шаблона, то есть $$L_h u(x) = a_{-1} u(x-h) + a_0 u(x) + a_1 u(x+h).$$ Выпишем погрешность аппроксимации $\psi(x) = L_h u(x) - L u(x)$ в соответствии с нашими данными: $$\psi(x) = a_{-1} u(x-h) + a_0 u(x) + a_1 u(x+h) - u''(x).$$
		Разложим правую часть этого уравнения в ряд Тейлора в окрестности точки $x$:
		\begin{multline*}
			\psi(x) =  a_{-1} \left(u(x) - h u'(x) + \dfrac{h^2}{2}u''(x) - \dfrac{h^3}{6}u'''(x) + \dfrac{h^4}{24}u''''(x)\right) +a_0 u(x)\\+ a_1 \left(u(x) + h u'(x) + \dfrac{h^2}{2}u''(x) + \dfrac{h^3}{6}u'''(x) + \dfrac{h^4}{24}u''''(x)\right) + O(h^5) - u''(x) .
		\end{multline*}
		При необходимости можно записать и больше членов разложения. Теперь преобразуем выражение справа, вынеся общие множители, следующим образом
		\begin{multline*}
			\psi(x) = (a_{-1} + a_0 + a_1)\cdot u(x) + (-ha_{-1} + ha_1)\cdot  u'(x) + \left(\dfrac{h^2}{2}a_{-1} + \dfrac{h^2}{2}a_1 - 1\right)\cdot u''(x) +\\+ \left(-\dfrac{h^3}{6}a_{-1} +\dfrac{h^3}{6}a_1\right)\cdot u'''(x) + \left(\dfrac{h^4}{24}a_{-1} +\dfrac{h^4}{24}a_1\right)\cdot u''''(x) +O(h^5).
		\end{multline*}
		Нас интересует возможность найти неизвестные коэффициенты $a_k$ такими, чтобы погрешность аппроксимации была минимальной. Для этого коэффициенты при $u(x)$, $u'(x)$, $u''(x)$ мы приравниваем к нулю и получаем следующую систему линейных уравнений для отыскания неизвестных коэффициентов 
		$$\begin{cases}
			a_{-1} + a_0 + a_1= 0,\\
			-ha_{-1} + ha_1= 0,\\
			\dfrac{h^2}{2}a_{-1} + \dfrac{h^2}{2}a_1 - 1= 0.
		\end{cases}$$
		Домножим второе уравнение на $h$, а третье на 2 и получим
		$$\begin{cases}
			a_{-1} + a_0 + a_1= 0,\\
			-h^2 a_{-1}+ h^2 a_1= 0,\\
			h^2 a_{-1} + h^2a_1 = 2.
		\end{cases}$$
		Прибавим к третьему уравнению второе и получим
		$$\begin{cases}
			a_{-1} + a_0 + a_1= 0,\\
			-h^2 a_{-1}+ h^2 a_1= 0,\\
			2h^2a_1 = 2.
		\end{cases}$$
		Отсюда $$a_1 = \dfrac{1}{h^2}.$$
		Тогда из второго уравнения $$a_{-1} = \dfrac{1}{h^2},$$ а из первого уравнения $$a_0 = -\dfrac{2}{h^2}.$$
		Таким образом, мы можем построить разностную аппроксимацию дифференциального оператора, подставив найденные коэффициенты в записанный ранее общий вид,
		$$L_h u(x) = \dfrac{u(x-h)  -2 u(x) +  u(x+h)}{h^2}.$$ 
		Остается лишь определить порядок аппроксимации и главный член погрешности. Для этого найденные коэффициенты подставляем в нашу крайнюю запись для погрешности аппроксимации (учитываем, что первые три слагаемых обращаются в ноль при подстановке)
		\begin{multline*}
			\psi(x) = \left(-\dfrac{h^3}{6}\cdot \dfrac{1}{h^2} +\dfrac{h^3}{6} \cdot \dfrac{1}{h^2}\right)\cdot u'''(x) + \left(\dfrac{h^4}{24}\cdot \dfrac{1}{h^2} +\dfrac{h^4}{24} \cdot \dfrac{1}{h^2}\right)\cdot u''''(x)+\\ +O(h^5) = \dfrac{h^2}{12} u''''(x) + O(h^5) = O(h^2).
		\end{multline*}
		То есть мы получаем аппроксимацию второго порядка, а главный член погрешности это $\dfrac{h^2}{12} u''''(x)$.
	
	\newpage
	\item 
	\hypertarget{t3}{}
	Итак, мы имеем дифференциальный оператор $$Lu(x) = u'(x),$$ и нерегулярный шаблон (то есть с разным шагом) $$\text{Ш}(x) = \{x-h_1, x, x+h_2\},\ h_1 \ne h_2.$$ Разностную аппроксимацию будем искать в виде линейной комбинации значений функции $u$ в точках шаблона, то есть $$L_h u(x) = a_{-1} u(x-h_1) + a_0 u(x) + a_1 u(x+h_2).$$ Выпишем погрешность аппроксимации $\psi(x) = L_h u(x) - L u(x)$ в соответствии с нашими данными: $$\psi(x) = a_{-1} u(x-h_1) + a_0 u(x) + a_1 u(x+h_2) - u'(x).$$
	Разложим правую часть этого уравнения в ряд Тейлора в окрестности точки $x$:
	\begin{multline*}
		\psi(x) = a_0\left(u(x) - h_1u'(x) + \dfrac{h_1^2}{2}u''(x)-\dfrac{h_1^3}{6}u'''(x)\right) + a_1 u(x)\\+ a_2 \left(u(x) + h_2 u'(x) + \dfrac{h_2^2}{2}u''(x) +\dfrac{h_2^3}{6}u'''(x)\right) + O(h_1^4+h_2^4) - u'(x) .
	\end{multline*}
	При необходимости можно записать и больше членов разложения. Теперь преобразуем выражение справа, вынеся общие множители, следующим образом
	\begin{multline*}
		\psi(x) = (a_0 + a_1 + a_2)\cdot u(x) + (-h_1a_{0} + h_2a_2 - 1)\cdot  u'(x) +\\+ \left(\dfrac{h_1^2}{2}a_{0} + \dfrac{h_2^2}{2}a_2\right)\cdot u''(x) + \left(-\dfrac{h_1^3}{6}a_{0} + \dfrac{h_2^3}{6}a_2\right)u'''(x)+O(h_1^4+h_2^4).
	\end{multline*}
	Нас интересует возможность найти неизвестные коэффициенты $a_k$ такими, чтобы погрешность аппроксимации была минимальной. Для этого коэффициенты при $u(x)$, $u'(x)$, $u''(x)$ мы приравниваем к нулю и получаем следующую систему линейных уравнений для отыскания неизвестных коэффициентов 
	$$\begin{cases}
		a_0 + a_1 + a_2= 0,\\
		-h_1a_{0} + h_2a_2 - 1= 0,\\
		\dfrac{h_1^2}{2}a_{0} + \dfrac{h_2^2}{2}a_2 = 0.
	\end{cases}$$
	Домножим второе уравнение на $h_1$, а третье на 2 и получим
	$$\begin{cases}
		a_0 + a_1 + a_2= 0,\\
		-h_1^2 a_{0}+ h_1h_2 a_2= h_1,\\
		h_1^2 a_{0} + h_2^2a_2 = 0.
	\end{cases}$$
	Прибавим к третьему уравнению второе и получим
	$$\begin{cases}
		a_0 + a_1 + a_2= 0,\\
		-h_1^2 a_{0}+ h_1h_2 a_2= h_1,\\
		(h_1h_2 + h_2^2)a_2 = h_1.
	\end{cases}$$
	Отсюда $$a_2 = \dfrac{h_1}{h_2 (h_1+h_2)}.$$
	Тогда из второго уравнения $$a_{0} = \dfrac{h_2}{h_1 (h_1+h_2)},$$ а из первого уравнения $$a_1 = -\left(\dfrac{h_1^2+h_2^2}{h_1h_2}\right)\cdot \dfrac{1}{h_1+h_2}.$$
	Таким образом, мы можем построить разностную аппроксимацию дифференциального оператора, подставив найденные коэффициенты в записанный ранее общий вид,
	$$L_h u(x) = \dfrac{1}{h_1+h_2}\cdot\left(\dfrac{h_2^2 u(x-h_1) -(h_1^2+h_2^2)u(x) + h_1^2u(x+h_2)}{h_1h_2}\right).$$ 
	Остается лишь определить порядок аппроксимации и главный член погрешности. Для этого найденные коэффициенты подставляем в нашу крайнюю запись для погрешности аппроксимации (учитываем, что первые три слагаемых обращаются в ноль при подстановке)
	\begin{multline*}
		\psi(x) = \left(-\dfrac{h_1^3}{6}\cdot \dfrac{h_2}{h_1 (h_1+h_2)} +\dfrac{h_2^3}{6} \cdot \dfrac{h_1}{h_2 (h_1+h_2)}\right)\cdot u'''(x) +\\+O(h_1^4+h_2^4) = \dfrac{h_1h_2}{6(h_1+h_2)}(h_1-h_2)\cdot u''''(x) + O(h_1^4+h_2^4) = O(h_1^2+h_2^2).
	\end{multline*}
	То есть мы получаем аппроксимацию второго порядка, а главный член погрешности это $\dfrac{h_1h_2}{6(h_1+h_2)}(h_1-h_2)\cdot u''''(x)$.
	\end{enumerate}
	
\end{document} 