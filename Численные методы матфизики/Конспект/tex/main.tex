\documentclass[a4paper, 12pt]{report}
\usepackage{cmap}
\usepackage{amssymb}
\usepackage{amsmath}
\usepackage{graphicx}
\usepackage{amsthm}
\usepackage{upgreek}
\usepackage{setspace}
\usepackage{mathtools}
\setcounter{secnumdepth}{5}
\setcounter{tocdepth}{5}
\numberwithin{equation}{section}
\renewcommand{\theequation}{\arabic{equation}}
\usepackage[T2A]{fontenc}
\usepackage[utf8]{inputenc}
\usepackage[normalem]{ulem}
\usepackage{mathtext} % русские буквы в формулах
\usepackage[left=2cm,right=2cm, top=2cm,bottom=2cm,bindingoffset=0cm]{geometry}
\usepackage[english,russian]{babel}
\usepackage[unicode]{hyperref}
\newenvironment{Proof} % имя окружения
{\par\noindent{$\blacklozenge$}} % команды для \begin
{\hfill$\scriptstyle\square$}
\newcommand{\Rm}{\mathbb{R}}
\newcommand{\Cm}{\mathbb{C}}
\newcommand{\Z}{\mathbb{Z}}
\newcommand{\I}{\mathbb{I}}
\newcommand{\N}{\mathbb{N}}
\newcommand{\rank}{\operatorname{rank}}
\newcommand{\Ra}{\Rightarrow}
\newcommand{\ra}{\rightarrow}
\newcommand{\FI}{\Phi}
\newcommand{\Sp}{\text{Sp}}

\renewcommand{\leq}{\leqslant}
\renewcommand{\geq}{\geqslant}

\renewcommand{\alpha}{\upalpha}
\renewcommand{\beta}{\upbeta}
\renewcommand{\gamma}{\upgamma}
\renewcommand{\delta}{\updelta}
\renewcommand{\varphi}{\upvarphi}
\renewcommand{\phi}{\upvarphi}
\renewcommand{\tau}{\uptau}
\renewcommand{\theta}{\uptheta}
\renewcommand{\eta}{\upeta}
\renewcommand{\lambda}{\uplambda}
\renewcommand{\sigma}{\upsigma}
\renewcommand{\psi}{\uppsi}
\renewcommand{\mu}{\upmu}
\renewcommand{\omega}{\upomega}
\renewcommand{\xi}{\upxi}
\renewcommand{\epsilon}{\upvarepsilon}
\renewcommand{\rho}{\uprho}
\renewcommand{\varepsilon}{\upvarepsilon}

\renewcommand{\d}{\partial}
\renewcommand{\Re}{\operatorname{Re}}
\newcommand{\const}{\operatorname{const}}
\newcommand{\intx}{\int\limits_{x_0}^x}
\newcommand\Norm[1]{\left\| #1 \right\|}
\newcommand{\sumk}{\sum\limits_{k=0}^\infty}
\newcommand{\sumi}{\sum\limits_{i=0}^\infty}
\newtheorem*{theorem}{Теорема}
\newtheorem*{cor}{Следствие}
\newtheorem*{lem}{Лемма}
\title{\textbf{\Huge{Численные методы математической физики}}\\Конспект по 4 курсу 
	специальности «прикладная математика»\\(лектор А. М. Будник)}
\date{}
\begin{document}
	\maketitle
	\tableofcontents{}
	\newpage
	\section*{Введение.}
	В данном курсе мы будем рассматривать задачи математической физики в частных производных. Основной принцип решения состоит в том, что дифференциальное уравнение мы заменяем разностным и ищем приближенное решение на сетке узлов. Такой способ называется \textit{методом конечных разностей} (\textit{методом сеток}). А раздел численных методов, посвященный теории метода конечных разностей, носит название \textit{теория разностных схем}. 
	\\\\
	Выделим два основных момента при решении:
	\begin{enumerate}
		\item построение дискретных разностных аппроксимаций для задачи математической физики и исследование основных характеристик этих аппроксимаций: погрешности, устойчивости и точности разностных схем;
		\item решение разностных уравнений прямыми или итерационными методами.
	\end{enumerate}
	\chapter{Способы построения и исследования разностных схем.}
	\section{Сетки и сеточные функции.}
	При решении той или иной математической задачи мы не можем воспроизвести приближенное решение для всех значений аргумента. Поэтому в области задания функций выбирается конечное множество точек, и приближенное решение задачи ищется в этих точках.
	\\\\
	$\bullet$ \textit{Это множество называется \textbf{сеткой}, а отдельные точки -- \textbf{узлами сетки}.}
	\\\\
	$\bullet$ \textit{Функция, определенная в узлах сетки, называется \textbf{сеточной функцией}.}
	\\\\
	Заменяя области непрерывного изменения аргумента сеткой, осуществляем аппроксимацию пространства решения дифференциального уравнения пространством сеточной функции.
	\\\\
	\textbf{Пример сетки на отрезке.}\\\\
	В качестве области определения мы рассматриваем отрезок на оси $x$.
	\begin{enumerate}
		\item \textbf{Равномерная сетка.}
		Возьмем отрезок $[0,1]$ и разобьем его на $N$ равных частей $$x_0=0,\ x_1,\ \ldots,\ x_{N-1},\ x_N.$$
		Обозначим через $h$ шаг сетки, а через $x_i$ -- узлы, $i=\overline{0,N}$. Тогда множество $x_i$ составляют сетку 
		$$\overline \omega _ h = \left\{x_i = ih,\ i=\overline{0, N},\ h = \dfrac1N\right\}.$$
		Множество граничных узлов обозначим как $$\gamma_h = \{x_0, x_N\}.$$
		А остальные точки образуют множество внутренних узлов
		$$ \omega _ h = \left\{x_i = ih,\ i=\overline{1, N-1},\ h = \dfrac1N\right\}.$$
		Таким образом, можно записать 
		$$\overline \omega _h = \omega _h \cup \gamma _h.$$
		\item \textbf{Неравномерная сетка.}
		Возьмем отрезок $[0,1]$ и разобьем его на $N$ равных частей $$0=x_0 < x_1 <\ldots< x_{N-1} < x_N=1.$$
		Тогда мы можем записать неравномерную сетку с граничными узлами 
		$$\hat {\overline \omega} _ h = \left\{x_i,\ i=\overline{0, N},\ x_0 = 0, \ x_N=1\right\}.$$
		Шаг неравномерной сетки зависит от номера узла и подлежит нормировке $$\sum_{i=1}^{N} h_i=1,\ h_i=x_i - x_{i-1}.$$
		Аналогично первому примеру можно записать $$\hat {\overline \omega} _ h = \hat \omega_h \cup \hat\gamma _h.$$
	\end{enumerate}
	\end{document}
	


	