\documentclass[a4paper, 12pt]{article}
\usepackage{cmap}
\usepackage{amssymb}
\usepackage{amsmath}
\usepackage{graphicx}
\usepackage{amsthm}
\usepackage{upgreek}
\usepackage{setspace}
\usepackage[T2A]{fontenc}
\usepackage[utf8]{inputenc}
\usepackage[normalem]{ulem}
\usepackage{mathtext} % русские буквы в формулах
\usepackage[left=2cm,right=2cm, top=2cm,bottom=2cm,bindingoffset=0cm]{geometry}
\usepackage[english,russian]{babel}
\usepackage[unicode]{hyperref}
\newenvironment{Proof} % имя окружения
{\par\noindent{$\blacklozenge$}} % команды для \begin
{\hfill$\scriptstyle\boxtimes$}
\newcommand{\Rm}{\mathbb{R}}
\newcommand{\Cm}{\mathbb{C}}
\newcommand{\I}{\mathbb{I}}
\renewcommand{\phi}{\upvarphi}
\renewcommand{\varphi}{\upvarphi}
\renewcommand{\alpha}{\upalpha}
\renewcommand{\psi}{\uppsi}
\renewcommand{\tau}{\uptau}
\renewcommand{\mu}{\upmu}
\renewcommand{\omega}{\upomega}
\renewcommand{\d}{\partial}
\newcommand{\N}{\mathbb{N}}
\newcommand{\Ln}{L_n = D^n + a_{n-1}D^{n-1} + \ldots + a_1D + a_0D^0}
\begin{document}
	\subsection*{Уравнение в полных дифференциалах (УПД).}
	\textbf{Условие.} $P(x,y)dx + Q(x,y)dy = 0$ и $P'_y = Q'_x$.\\\\
	\textbf{Решение.} $$\int\limits_{(x_0,y_0)}^{(x,y)}P(x,y)dx + Q(x,y)dy = \int\limits_{x_0}^xP(x,y)dx + \int\limits_{y_0}^yQ(x_0,y)dy = \int\limits_{x_0}^xP(x,y_0)dx + \int\limits_{y_0}^yQ(x,y)dy = C.$$
	\subsection*{Уравнение с разделенными переменными.}
	\textbf{Условие.} $P(x)dx + Q(y)dy = 0$.\\\\
	\textbf{Решение.} $$\int\limits_{x_0}^xP(x)dx + \int\limits_{y_0}^yQ(y)dy = C.$$
	\subsection*{Интегрирующий множитель.}
	\textbf{Условие.} $P(x,y)dx + Q(x,y)dy = 0$ и $P'_y \ne Q'_x$.\\\\
	\textbf{Решение.} $$\dfrac{P'_y - Q'_x}{Q\omega'_x - P\omega'_y} = \psi(\omega)\quad\Rightarrow\quad \mu(\omega) = e^{\int\limits_{\omega_0}^\omega \psi(\tau)d\tau}\quad\Rightarrow\quad \mu(\omega)\cdot P(x,y)dx +\mu(\omega)\cdot Q(x,y)dy = 0 \text{ --- УПД}.$$
	\subsection*{Уравнения с разделяющимися переменными (УРП).}
	\textbf{Условие.} $P_1(x)Q_1(y)dx + P_2(x)Q_2(y)dy = 0.$\\\\
	\textbf{Решение.} $$\dfrac{P_1(x)}{P_2(x)}dx + \dfrac{Q_2(y)}{Q_1(y)}dy = 0 \quad\Rightarrow\quad \int\limits_{x_0}^x\dfrac{P_1(x)}{P_2(x)}dx + \int\limits_{y_0}^y\dfrac{Q_2(y)}{Q_1(y)}dy = C. $$
	\subsection*{Линейные уравнения первого порядка (ЛУ-1).}
	\textbf{Условие.} $y' + P(x)\cdot y = Q(x)$.\\\\
	\textbf{Решение.} $$y =  e^{-\int\limits_{x_0}^xP(\tau)d\tau}\cdot \Big(C +\int\limits_{x_0}^x Q(t)\cdot e^{\int\limits_{t_0}^tP(\tau)d\tau} dt \Big).$$
	\subsection*{Уравнение Бернулли.}
	\textbf{Условие.} $y' + P(x)\cdot y = Q(x)\cdot y^m$.\\\\
	\textbf{Решение.} $$u = y^{1-m}\quad\Rightarrow\quad u' + (1-m)\cdot P(x)\cdot u = (1-m)\cdot Q(x) \text{ --- ЛУ-1}.$$
	$$y^{1-m} = e^{(m-1)\int\limits_{x_0}^xP(\tau)d\tau}\cdot \Big(C +(1-m)\int\limits_{x_0}^x Q(t)\cdot e^{(1-m)\int\limits_{t_0}^tP(\tau)d\tau} dt \Big).$$
\end{document}