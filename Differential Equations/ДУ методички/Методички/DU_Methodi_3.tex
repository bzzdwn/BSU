\documentclass[a4paper, 12pt]{article}
\usepackage{cmap}
\usepackage{amssymb}
\usepackage{amsmath}
\usepackage{graphicx}
\usepackage{amsthm}
\usepackage{upgreek}
\usepackage{setspace}
\usepackage[T2A]{fontenc}
\usepackage[utf8]{inputenc}
\usepackage[normalem]{ulem}
\usepackage{mathtext} % русские буквы в формулах
\usepackage[left=2cm,right=2cm, top=2cm,bottom=2cm,bindingoffset=0cm]{geometry}
\usepackage[english,russian]{babel}
\usepackage[unicode]{hyperref}
\newenvironment{Proof} % имя окружения
{\par\noindent{}} % команды для \begin
{\hfill$\scriptstyle$}
\newcommand{\Rm}{\mathbb{R}}
\newcommand{\Cm}{\mathbb{C}}
\newcommand{\I}{\mathbb{I}}
\newcommand{\N}{\mathbb{N}}
\newtheorem*{thrm}{Теорема}
\newcommand{\Ln}{L_n = D^n + a_{n-1}D^{n-1} + \ldots + a_1D + a_0D^0}
\begin{document}
	\section*{Устойчивость решений ДУ. Устойчивость по Ляпунову. Асимптотическая устойчивость.}
	В данном уроке мы не будем полностью разбирать теорию устойчивости ДУ, а лишь рассмотрим примеры того, как исследовать функцию на устойчивость. Для полного погружения читайте теоретический материал. Однако перед примерами введем некоторые определения, которые нам понадобятся.\\
	Рассмотрим уравнение $$L_nx = f(t),\quad t \in \I\eqno(1)$$ с непрерывной на промежутке $\I$ функцией $f(t)$.\\\\
	$\bullet$ \textit{Решение $x_0(t)$ уравнения $(1)$ называется \textbf{устойчивым по Ляпунову}, если оно непрерывно зависит от начальных данных на промежутке $\I = [t_0; +\infty)$.}\\\\
	$\bullet$ \textit{Если кроме того $\lim\limits_{t \rightarrow + \infty} \Delta_x(t) = 0$, то решение $x_0(t)$ называется \textbf{асимптотически устойчивым}. Решение не являющееся устойчивым называется \textbf{неустойчивым}.}\\\\
	Все решения уравнения (1) либо одновременно устойчивы, либо одновременно неустойчивы.\\\\
	$\bullet$ \textit{Уравнение называется \textbf{устойчивым}, если все его решения устойчивы (аналогично неустойчивым и асимптотически устойчивым).}\\\\
	\textbf{Замечание.} \textit{Неоднородность \textbf{не влияет} на устойчивость уравнения. Следовательно, исследование устойчивости любого решения уравнения $(1)$ можно заменить исследованием устойчивости нулевого решения соответствующего однородного уравнения $L_nx = 0$.}
	\subsection*{Устойчивость по Ляпунову.}
	Теперь введем теорему, с помощью которой и будем исследовать устойчивость уравнений.
	\begin{thrm}[Критерий устойчивости]
	Уравнение $L_nx = f(t)$ устойчиво $\Longleftrightarrow$ действительные части корней характеристического уравнения оператора $L_n$ неположительны, причем корни с нулевой действительной частью имеют кратность $k = 1$.
	\end{thrm}\begin{Proof}\end{Proof}\\
	Для "отсеивания"\ неустойчивых уравнений также можно использовать следующую теорему.
	\begin{thrm}[Необходимое условие устойчивости]
		Для устойчивости линейного уравнения необходимо, чтобы все коэффициенты характеристического многочлена были неотрицательны.
	\end{thrm}\begin{Proof}\end{Proof}\\
	\textbf{Пример 1.} Исследовать устойчивость уравнения $$D^2x - Dx - 30 = t^2e^{2t} + 2t.$$
	\textbf{Решение.} Как говорилось в замечании, интересует нас лишь левая часть равенства. То есть это и последующие неоднородные уравнения можно рассматривать как соответствующие им однородные. Таким образом, заменим правую часть на 0 и получим СтЛОУ $D^2x - Dx - 30 = 0$.\\
	Найдем корни характеристического уравнения: $\lambda_1 = -5$, $k_1 = 1$; $\lambda_2 = 6$, $k_2 = 1$. Так как корень $\lambda_2$ положительный, то по критерию устойчивости уравнение не является устойчивым.\\\\
	Иначе по необходимому условию устойчивости, так как среди коэффициентов характеристического уравнения $$\lambda^2 - \lambda - 30 = 0$$ есть отрицательные, то уравнение не является устойчивым.\\\\
	\textbf{Ответ:} Неустойчиво.\\\\
	На основе проделанного решения можно выделить определенный алгоритм исследования уравнений:\begin{itemize}
		\item строим характеристическое уравнение оператора $L_n$; если среди коэффициентов есть отрицательные, то уравнение неустойчиво;
		\item находим корни характеристического уравнения;
		\item если действительная часть положительная, то уравнение неустойчиво; если кратность корней с нулевой действительной частью $k \ne 1$, то уравнение неустойчиво;
	\end{itemize}
\textbf{Пример 2.} Исследовать устойчивость уравнения
$$D^4x + 4D^2x + 3x = f(t).$$
\textbf{Решение.} Построим характеристическое уравнение $\lambda^4 + 4\lambda^2 + 3 =0$ и найдем его корни: $\lambda_1 = i\sqrt{3}$, $\lambda_2 = -i\sqrt{3}$, $\lambda_3 = i$, $\lambda_4 = -i$, $k_1 = k_2 = k_3 = k_4 = 1$. Все корни, которые мы получили, имеют нулевую действительную часть. Кратность всех корней 1, следовательно уравнение устойчиво.\\\\
\textbf{Ответ:} Устойчиво.\\\\
\textbf{Пример 3.} Исследовать устойчивость уравнения
$$D^4x + x^2 = \cos t.$$
\textbf{Решение.} Запишем характеристическое уравнение и вынесем общие множители:
$$\lambda^4 + \lambda^2 = \lambda^2(\lambda^2 + 1) = 0.$$
Тогда корни этого уравнения следующие:
$\lambda_1 = 0$, $k_1 = 2$; $\lambda_2 = i$, $k_2 = 1$; $\lambda_3 = -i$, $k_3 = 1$. Действительная часть всех корней нулевая, однако нулевой корень $\lambda_1$ имеет кратность $k_1 = 2\ne 1$. Тогда по критерию устойчивости уравнение не является устойчивым.\\\\
\textbf{Ответ:} Неустойчиво.\\\\
\textbf{Пример 3.1}.  Исследовать устойчивость уравнения
$$D^5x + 2D^3x + Dx= 0.$$
\textbf{Решение.} Построим характеристический многочлен:
$$\lambda^5 + 2\lambda^3 + \lambda = \lambda(\lambda^2 + 1)^2 = 0.$$
Значения корней данного уравнения аналогичны предыдущему уравнению. Однако сейчас $\lambda_1 = 0$, $k_1 = 1$; $\lambda_2 = i$, $k_2 = 2$; $\lambda_3 = -i$, $k_3 = 2$. Так как чисто мнимые числа не однократные, то уравнение также неустойчиво.\\\\
\textbf{Ответ:} Неустойчиво.\\\\
\textbf{Пример 4.} Исследовать устойчивость уравнения
$$D^3x + 4D^2x + 9Dx + 36x = \sin t.$$
\textbf{Решение.} Составим характеристическое уравнение. Разобьем многочлен на множители:
$$\lambda^3 + 4\lambda^2 + 9\lambda + 36 = (\lambda^2 + 9)(\lambda + 4) = 0.$$
Тогда уравнение имеет корни $\lambda_1 = -4$, $\lambda_2 = 3i$, $\lambda_3 = -3i$,  $k_1 = k_2 = k_3 = 1$; Так как действительная часть всех корней неположительная, то по критерию устойчивости уравнение устойчиво.\\\\
\textbf{Ответ:} Устойчиво.
\subsection*{Асимптотическая устойчивость.}
Основываясь на определении устойчивости, мы можем составить аналогичный критерий для исследования на асимптотическую устойчивость.
\begin{thrm}[Критерий асимптотической устойчивости]
	Уравнение $L_nx = f(t)$ асимптотически устойчиво $\Longleftrightarrow$ действительные части корней характеристического уравнения отрицательны.
\end{thrm}\begin{thrm}[Необходимое условие асимптотической устойчивости]
Для асимптотической устойчивости линейного уравнения необходимо, чтобы все коэффициенты характеристического многочлена были положительны.
\end{thrm}\begin{Proof}
\end{Proof}\\
\textbf{Замечание.} \textit{Из критерия следует, что условие асимптотической устойчивости более строгое, чем устойчивости. Следовательно, если уравнение асимптотически устойчиво, то оно устойчиво. Обратное, вообще говоря, утверждать нельзя.}\\\\
$\bullet$ \textit{Определитель порядка $n$ вида $$\begin{vmatrix}
	a_{n-1} & 1 & 0 & 0 & \dots & 0\\
	a_{n-3} & a_{n-2} & a_{n-1} & 1 & \dots & 0\\
	a_{n-5} & a_{n-4} & a_{n-3} & a_{n-2} & \dots & 0\\
	\vdots & \vdots & \vdots & \vdots & \ddots & \vdots\\
	a_{n-(2n+1)} & a_{n-2n} & a_{n-(2n -1)} & a_{n-(2n - 2)} & \dots & a_0\\
\end{vmatrix}$$ называется \textbf{определителем Гурвица}, или \textbf{гурвицианом} ($a_j = 0$, если $j < 0$)}.
\begin{thrm}
[Критерий Гурвица] Действительные части всех корней характеристического многочлена $\lambda^n + a_{n-1}\lambda^{n-1} + \ldots + a_{1}\lambda + a_0$ матрицы $A$ отрицательны $\Longleftrightarrow$ все главные миноры гурвициана положительны.  
\end{thrm}\begin{Proof}\end{Proof}\\
\textbf{Замечание.} \textit{Невыполнение критерия Гурвица (хотя бы один из главных миноров неположительный) не гарантирует неустойчивость.}\\\\
То есть, с помощью гурвициана мы можем исследовать уравнение \textbf{только} на асимптотическую устойчивость.\\\\
Построение гурвициана в общем случае лучше запомнить. Однако для понимания рассмотрим несколько частных случаев. Для уравнения порядка $n = 2$ гурвициан имеет вид: $$\begin{vmatrix}
	a_1 & 1\\
	0 & a_0
\end{vmatrix};$$
для уравнения порядка $n = 3$:
 $$\begin{vmatrix}
	a_2 & 1 & 0\\
	a_0 & a_1 & a_2\\
	0 & 0 & a_0
\end{vmatrix};$$
для уравнения порядка $n = 4$:
$$\begin{vmatrix}
	a_3 & 1 & 0 & 0\\
	a_1 & a_2 & a_3 & 1\\
	0 & a_0 & a_1 & a_2\\
	0 & 0 & 0 & a_0
\end{vmatrix};$$
Теперь можно переходить к исследованию уравнений. Для начала рассмотрим устойчивые уравнения из предыдущих примеров.\\\\
\textbf{Пример 2.1.} Исследовать устойчивость и асимптотическую устойчивость уравнения
$$D^4x + 4D^2x + 3x = f(t).$$
\textbf{Решение.} Поскольку корни $\lambda_1 = i\sqrt{3}$, $\lambda_2 = -i\sqrt{3}$, $\lambda_3 = i$, $\lambda_4 = -i$, $k_1 = k_2 = k_3 = k_4 = 1$ чисто мнимые, то уравнение по критерию асимптотической устойчивости асимптотически неустойчиво, так как все корни имеют неотрицательную (нулевую) действительную часть. Для того, чтобы окончательно убедиться в этом, построим гурвициан характеристического уравнения $$\begin{vmatrix}
	0 & 1 & 0 & 0\\
	0 & 4 & 0 & 1\\
	0 & 3 & 0 & 4\\
	0 & 0 & 0 & 3
\end{vmatrix};\quad \Delta_1 = \Delta_2 = \Delta_3 = \Delta_4 = 0.$$
Все главные миноры неположительны, следовательно, уравнение не является асимптотически устойчивым.\\\\
\textbf{Ответ:} Устойчиво.\\\\
\textbf{Пример 4.1.} Исследовать устойчивость и асимптотическую устойчивость уравнения
$$D^3x + 4D^2x + 9Dx + 36x = \sin t.$$
\textbf{Решение.} Характеристическое уравнение имеет корни $\lambda_1 = -4$, $\lambda_2 = 3i$, $\lambda_3 = -3i$,  $k_1 = k_2 = k_3 = 1$. Корни $\pm 3i$ имеют неотрицательную действительную часть, следовательно, по критерию асимптотической устойчивости уравнение не является асимптотически устойчивым. Опять же для проверки построим гурвициан $$\begin{vmatrix}
	4 & 1 & 0\\
	36 & 9 & 4\\
	0 & 0 & 36
\end{vmatrix};\quad \Delta_1 = 4,\ \Delta_2 = \Delta_3 = 0.$$
\textbf{Ответ:} Устойчиво.\\\\
\textbf{Пример 5.} Исследовать устойчивость и асимптотическую устойчивость уравнения $$D^2x + 6Dx + 13x = \sin t+\cos t.$$
\textbf{Решение.} Для уравнения второго порядка, на мой взгляд, проще сразу построить гурвициан характеристического уравнения, так как нам нужно найти всего 2 определителя вместо нахождения корней. Гурвициан имеет вид $$\begin{vmatrix}
	6 & 1\\
	0 & 13
\end{vmatrix};\quad \Delta_1 = 6,\ \Delta_2 = 78.$$
Главные миноры положительны, следовательно, уравнение асимптотически устойчиво.\\\\
\textbf{Ответ:} Асимптотически устойчиво.\\\\
\textbf{Пример 6.} Исследовать устойчивость и асимптотическую устойчивость уравнения
$$D^4x + 6D^3x + 13D^2x + 12Dx + 4x = 0.$$
\textbf{Решение.} Построим характеристическое уравнение и вынесем общие множители $$\lambda^4 + 6\lambda^3 + 13\lambda^2 + 12\lambda + 4 = (\lambda + 1)^2(\lambda + 2)^2 = 0.$$
Тогда корни уравнения $\lambda_1 = -1$, $\lambda_2 = -2$, $k_1 = k_2 = 2$. Следовательно, так как действительная часть корней уравнения отрицательная, то уравнение асимптотически устойчиво. Для проверки построим гурвициан $$\begin{vmatrix}
	6 & 1 & 0 & 0\\
	12 & 13 & 6 & 1\\
	0 & 4 & 12 & 13\\
	0 & 0 & 0 & 4
\end{vmatrix};\quad \Delta_1 = 6,\ \Delta_2 = 66,\ \Delta_3 = 1188,\ \Delta_4 = 4752.$$
\textbf{Ответ:} Асимптотически устойчиво.\\\\
В некоторых примерах мы сначала находили корни, а только затем для проверки строили гурвициан. Однако для того, чтобы ускорить исследование, сначала стоит начать с построения гурвициана. Как мы ранее замечали, условие асимптотической устойчивости более строгое, чем условие устойчивости. Соответственно, построив гурвициан мы можем сразу получить асимптотическую устойчивость.\\ Таким образом мы можем составить алгоритм исследования уравнения на асимптотическую устойчивость: \begin{itemize}
	\item строим характеристическое уравнение оператора $L_n$; если среди коэффициентов есть неположительные, то уравнение не является асимптотически устойчивым;
	\item строим гурвициан, рассматриваем его главные миноры;
	\item если все миноры положительны, то уравнение асимптотически устойчиво; иначе переходим к алгоритму исследования функции на устойчивость.
\end{itemize}
\textbf{Пример 7.} Исследовать устойчивость и асимптотическую устойчивость уравнения $$D^3x - 6D^2x + 11Dx - 6 = t^2e^t.$$
\textbf{Решение.} Для начала построим гурвициан характеристического уравнения 
$$\begin{vmatrix}
	-6 & 1 & 0\\
	-6 & 11 &-6\\
	0 & 0 & -6
\end{vmatrix};\quad \Delta_1 = -6 < 0.$$
Мы выяснили, что уравнение не является асимптотически устойчивым. Следовательно, остается проверить, является ли оно устойчивым. Построим характеристическое уравнение и разобьем его на множители $$\lambda^3 - 6\lambda^2 + 11\lambda - 6 = (\lambda - 1)(\lambda -2)(\lambda - 3) = 0.$$
Получили корни $\lambda_1 = 1$, $\lambda_2 = 2$, $\lambda_3 = 3$, кратностей 1. Значит уравнение неустойчиво.\\\\
Иначе по необходимому условию асимптотической устойчивости уравнение не является асимптотически устойчивым. По необходимому условию устойчивости уравнение не является устойчивым.\\\\
\textbf{Ответ:} Неустойчиво.\\\\
\textbf{Пример 8.} С помощью критерия Гурвица исследовать, при каких значениях параметров уравнение асимптотически устойчиво
$$D^3x + aD^2x + bDx + 2x = 0.$$
\textbf{Решение.} Построим характеристическое уравнение:
$$\lambda^3 + a\lambda^2 + b\lambda + 2 = 0.$$
Построим его Гурвициан:
$$\begin{vmatrix}
	a & 1 & 0\\
	2 & b & a\\
	0 & 0 & 2
\end{vmatrix}\quad \Delta_1 = a > 0,\ \Delta_2 = ab - 2 > 0,\ \Delta_3 = 2(ab - 2) > 0.$$
Таким образом, для асимптотической устойчивости исходного СтЛУ необходимо, чтобы $$\begin{cases}
	a > 0,\\
	ab > 2.
\end{cases}$$
\textbf{Ответ:} $a > 0$, $ab > 2$.\\\\
\end{document}