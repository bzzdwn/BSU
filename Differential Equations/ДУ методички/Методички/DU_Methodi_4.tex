\documentclass[a4paper, 12pt]{article}
\usepackage{cmap}
\usepackage{amssymb}
\usepackage{amsmath}
\usepackage{graphicx}
\usepackage{amsthm}
\usepackage{upgreek}
\usepackage{setspace}
\usepackage[T2A]{fontenc}
\usepackage[utf8]{inputenc}
\usepackage[normalem]{ulem}
\usepackage{mathtext} % русские буквы в формулах
\usepackage[left=2cm,right=2cm, top=2cm,bottom=2cm,bindingoffset=0cm]{geometry}
\usepackage[english,russian]{babel}
\usepackage[unicode]{hyperref}
\newenvironment{Proof} % имя окружения
{\par\noindent{}} % команды для \begin
{\hfill$\scriptstyle$}
\newcommand{\Rm}{\mathbb{R}}
\newcommand{\Cm}{\mathbb{C}}
\newcommand{\I}{\mathbb{I}}
\newcommand{\N}{\mathbb{N}}
\newtheorem*{thrm}{Теорема}
\newcommand{\Ln}{L_n = D^n + a_{n-1}D^{n-1} + \ldots + a_1D + a_0D^0}
\begin{document}
\section*{Фазовая плоскость. Классификация точек покоя.}
	Пусть линейное уравнение имеет вид $$D^2x + a_1Dx + a_0x = 0,\quad t \in \Rm.\eqno(1)$$ 
	$\bullet$ \textit{\textbf{Фазовым графиком} решения $x(t)$ называется график параметрически заданной функции вида $$\begin{cases}
			x = x(t),\\
			y = Dx(t);
		\end{cases}\quad t \in \Rm.$$}
	$\bullet$ \textit{Решение, сохраняющее постоянное значение при всех $t$, называется \textbf{стационарным}.}\\\\
	Фазовый график стационарного решения $x(t)\equiv C$ состоит из единственной точки $(C, 0)$.\\\\
	$\bullet$ \textit{Точка, являющаяся фазовым графиком стационарного решения, называется \textbf{точкой покоя} уравнения.}\\\\
	Любое уравнение вида (1) имеет стационарное решение $x(t)\equiv C$. Следовательно, точка $O(0, 0)$ является точкой покоя для этого уравнения.\\\\
	Пусть $\lambda_1$, $\lambda_2$ --- корни характеристического уравнения для уравнения (1). Тогда тип точки покоя $O$ при $a_0 \ne 0$ определяется следующим образом:\begin{enumerate}
		\item Если $\lambda_1$, $\lambda_2$ $\in \Rm$ и
		\begin{enumerate}
			\item $\lambda_1\cdot \lambda_2 < 0$, то точка покоя называется \textbf{седлом};
			\item $\lambda_1\cdot \lambda_2 > 0$, $\lambda_1 \ne \lambda_2$, то точка покоя называется \textbf{бикритическим узлом}, причем, при $\lambda_1 < \lambda_2 < 0$ \textbf{устойчивым}; при $\lambda_2 > \lambda_1 > 0$ \textbf{неустойчивым};
			\item $\lambda_1\cdot \lambda_2 > 0$, $\lambda_1 = \lambda_2$, то точка покоя называется \textbf{монокритическим узлом}, причем, при $\lambda_1 = \lambda_2 < 0$ \textbf{устойчивым}; при $\lambda_2 = \lambda_1 > 0$ \textbf{неустойчивым};
		\end{enumerate}
	\item Если $\lambda_{1,2} = \alpha \pm \beta i$ и\begin{enumerate}
		\item $\alpha \ne 0$, $\beta \ne 0$, то точка покоя называется \textbf{фокусом}, причем, при $\alpha < 0$ \textbf{устойчивым}; при $\alpha > 0$ \textbf{неустойчивым};
		\item $\alpha = 0$, $\beta \ne 0$, то точка покоя называется \textbf{центром}.
	\end{enumerate}
	\end{enumerate}
Если линейное уравнение имеет вид $D^2x + a_1Dx = 0$, где $a_1 \geqslant 0$, то прямая $y=0$ состоит из точек покоя и называется \textbf{прямой покоя}.\\\\
\textbf{Пример 1}. Установить тип точки покоя для уравнения $$D^2x - 4Dx + 3 =0.$$
\textbf{Решение.} Найдем корни характеристического уравнения: $\lambda_1 = 1$, $\lambda_2 = 3$. Таким образом, $\lambda_1\cdot \lambda_2 > 0$, $\lambda_1 \ne \lambda_2$, $\lambda_2 > \lambda_1 > 0$. Следовательно, точка покоя $O$ --- неустойчивый бикритический узел.\\\\
\textbf{Ответ:} $O$ --- неустойчивый бикритический узел.\\\\
\textbf{Пример 2}. Установить тип точки покоя для уравнения $$D^2x + 9Dx =0.$$
\textbf{Решение.} Так как коэффициент $a_0 = 0$, то прямая $y = 0$ является прямой покоя.\\\\\textbf{Ответ:} $y = 0$ --- прямая покоя.\\\\
\textbf{Пример 3.} Определить тип точки покоя уравнения $$D^2x + 3\alpha Dx + 3x = 0$$ в зависимости от значений параметра $\alpha$.\\\\
\textbf{Решение.} Построим характеристическое уравнение $$\lambda^2 + 3\alpha\lambda + 3 = 0.$$
Тогда корни уравнения имеют вид $$\lambda_1 = \dfrac{-3\alpha + \sqrt{9\alpha^2 - 12}}{2},\quad \lambda_2 = \dfrac{-3\alpha - \sqrt{9\alpha^2 - 12}}{2}.$$\begin{enumerate}
	\item Пусть $9\alpha^2 - 12 > 0$. Тогда $|\alpha| > \dfrac{2}{\sqrt{3}}$. Подставим $\alpha$ в $\lambda_1$ и $\lambda_2$ и получим $\lambda_1, \lambda_2 \in \Rm$, $\lambda_1 \ne \lambda_2$, $$\lambda_1 \cdot \lambda_2 = \dfrac{9\alpha^2 - 9\alpha^2 + 12}{4} = 3 > 0,$$ следовательно, точка $O$ --- бикритический узел. Причем при $\alpha > \dfrac{2}{\sqrt{3}}$ получаем устойчивый бикритический узел, а при $\alpha < -\dfrac{2}{\sqrt{3}}$ --- неустойчивый.
	\item Пусть $9\alpha^2 - 12 = 0$. Тогда $|\alpha| = \dfrac{2}{\sqrt{3}}$. Подставим $\alpha$ в $\lambda_1$ и $\lambda_2$ и получим $\lambda_1, \lambda_2 \in \Rm$, $\lambda_1 = \lambda_2 = -\dfrac{3}{2}$, $\lambda_1\cdot\lambda_2 > 0$. Таким образом, точка $O$ --- монокритический узел. Причем при $\alpha = \dfrac{2}{\sqrt{3}}$ получаем устойчивый бикритический узел, а при $\alpha = -\dfrac{2}{\sqrt{3}}$ --- неустойчивый.
	\item Пусть $9\alpha^2 - 12 < 0$. Тогда получаем два случая:\begin{enumerate}
		\item $0 < |\alpha| < \dfrac{2}{\sqrt{3}}$. Таким образом, $\lambda_{1,2} = \alpha \pm \beta i$ и $\alpha \ne 0$. Тогда точка $O$ --- фокус, причем при $0 <\alpha < \dfrac{2}{\sqrt{3}}$ устойчивый, а при  $0 > \alpha >- \dfrac{2}{\sqrt{3}}$ неустойчивый.
		\item $\alpha = 0$. Таким образом, $\lambda_{1,2} = \pm \beta i$, и точка $O$ --- центр.
	\end{enumerate}
\end{enumerate}
\end{document}