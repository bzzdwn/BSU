\documentclass[a4paper, 12pt]{article}
\usepackage{cmap}
\usepackage{amssymb}
\usepackage{amsmath}
\usepackage{graphicx}
\usepackage{amsthm}
\usepackage{upgreek}
\usepackage{setspace}
\usepackage[T2A]{fontenc}
\usepackage[utf8]{inputenc}
\usepackage[normalem]{ulem}
\usepackage{mathtext} % русские буквы в формулах
\usepackage[left=2cm,right=2cm, top=2cm,bottom=2cm,bindingoffset=0cm]{geometry}
\usepackage[english,russian]{babel}
\newenvironment{Proof} % имя окружения
{\par\noindent{$\blacklozenge$}} % команды для \begin
{\hfill$\scriptstyle\boxtimes$}
\newcommand{\Rm}{\mathbb{R}}
\newcommand{\Cm}{\mathbb{C}}
\newcommand{\I}{\mathbb{I}}
\newcommand{\N}{\mathbb{N}}
\newtheorem*{theorem}{Теорема}
\newtheorem*{cor}{Следствие}
\newtheorem*{lem}{Лемма}
\newcommand{\FI}{\text{Ф}}
\newcommand{\Ln}{L_n = D^n + a_{n-1}D^{n-1} + \ldots + a_1D + a_0D^0}
\begin{document}
	\section*{Матричный метод решения СтЛВУ.}
	Рассмотрим уравнение $$DX = AX,\eqno(1)$$ где $A$ --- матрица $n\times n$, $X$ --- вектор-функция.\\\\
	Рассмотрим ряд $$e^A = E + \dfrac{A}{1!}+\dfrac{A^2}{2!} + \ldots + \dfrac{A^k}{k!} + \ldots. \eqno (2)$$
	$\bullet$ \textit{Ряд $(2)$ является сходящимся матричным рядом и называется \textbf{матричной экспонентой}}.\\\\
	\textbf{\textit{Свойства матричной экспоненты:}}
	\begin{enumerate}
		\item $e^0 = E$\textit{ $(0$ --- нулевая матрица$)$.}
		\item \textit{Если матрицы $A$, $B$ перестановочны, то есть $AB = BA$, то $e^A \cdot e^B = e^{A+B}$.}
		\item $(e^A)^{-1} = e^{-A}$.
	\end{enumerate}
	Рассмотрим матричную экспоненту $$e^{At} = E + \dfrac{A}{1!}t+\dfrac{A^2}{2!}t^2 + \ldots + \dfrac{A^k}{k!}t^k + \ldots,\eqno(3)$$
	где $t$ --- некоторая действительная переменная. При любом фиксированном $t$ ряд (3) является сходящимся и непрерывно дифференцируемым. Тогда $$D(e^{At}) =  Ae^{At} = e^{At}A.$$
	Перейдем к рассмотрению примеров. Поскольку данный метод предусматривает построение ряда, для которого нам нужно вычислять степени матрицы $A$, матричным методом находятся решения только для достаточно простых матриц $A$. Иначе вычисления могут получиться очень громоздкими, следовательно, рациональнее будет использовать иной метод.\\\\
	\textbf{Пример 1.} Используя представление матричной экспоненты в виде ряда, найти решение уравнения $DX = AX$, где
	$$A = \begin{pmatrix}
		4 & 0 \\
		0 & 5
	\end{pmatrix}.$$
\textbf{Решение.} Для представления в виде матричной экспоненты необходимо построить разложение в ряд (3). Для этого нужно вычислить степени матрицы $A$. Найдем их:
$$A^0 = E = \begin{pmatrix}
	1 & 0\\
	0 & 1
\end{pmatrix},\quad A = \begin{pmatrix}
4 & 0 \\
0 & 5
\end{pmatrix},\quad A^2 = \begin{pmatrix}
4^2 & 0 \\
0 & 5^2
\end{pmatrix},\ldots, A^k = \begin{pmatrix}
4^k & 0 \\
0 & 5^k
\end{pmatrix}.$$
Для построения матрицы $k$-ой степени необходимо построить матрицы некоторых малых степеней (1, 2, 3, 4 и так далее) и проследить закономерность.\\\\
Построим ряд (2), подставив полученные матрицы:
$$e^{At} = \begin{pmatrix}
	1 & 0\\
	0 & 1
\end{pmatrix} + \dfrac{t}{1!}\begin{pmatrix}
4 & 0 \\
0 & 5
\end{pmatrix} +\dfrac{t^2}{2!}\begin{pmatrix}
4^2 & 0 \\
0 & 5^2
\end{pmatrix} + \ldots + \dfrac{t^k}{k!}\begin{pmatrix}
4^k & 0 \\
0 & 5^k
\end{pmatrix} + \ldots.$$
Перепишем ряд в виде одной матрицы $2\times2$:
$$e^{At} = \begin{pmatrix}
	1 + \dfrac{4t}{1!} + + \dfrac{4^2t^2}{2!} + \ldots + \dfrac{4^kt^k}{k!} + \ldots & 0\\
	0 &  1 + \dfrac{5t}{1!} + + \dfrac{5^2t^2}{2!} + \ldots + \dfrac{5^kt^k}{k!} + \ldots
\end{pmatrix}$$
Теперь освежим в памяти разложение в ряд Тейлора:
$$e^t = 1 + \dfrac{t}{1!} + + \dfrac{t^2}{2!} + \ldots + \dfrac{t^k}{k!} + \ldots = \sum_{k = 1}^{\infty}\dfrac{t^k}{k!}.$$
Соответственно на главной диагонали полученной матрицы имеем разложения $e^{\lambda_it}$. Свернём их и получим
$$e^{At} = \begin{pmatrix}
	e^{4t} & 0\\
	0 & e^{5t}
\end{pmatrix}.$$
Тогда общее решение векторного уравнения имеет вид:
$$X(t) = e^{At}\cdot C =  \begin{pmatrix}
	e^{4t} & 0\\
	0 & e^{5t}
\end{pmatrix}\begin{pmatrix}
C_1\\C_2
\end{pmatrix} = (C_1e^{4t}, C_2e^{5t})^T.$$
Для того, чтобы убедиться, что мы правильно нашли правильное решение уравнения, Вы можете попробовать решить данное СтЛВУ уже известными Вам методами.\\\\
\textbf{Ответ:} $X(t) = (C_1e^{4t}, C_2e^{5t})^T.$\\\\
\textbf{Замечание.} \textit{Полученная матрица $e^{At}$ является \textbf{фундаментальной матрицей} ФСР нормированной в точке $t_0 = 0$ исходной СтЛВУ. Поэтому как методом Эйлера, так и матричным методом мы можем построить ФСР и ФМ СтЛВУ. Однако ФСР полученные методом Эйлера и полученные матричным методом будут отличаться. Так как матричным методом мы находим ФСР нормированную в точке $t_0 = 0$. В свою очередь, в методе Эйлера мы не ищем нормированную в точке ФСР.}\\\\
Основываясь на решении примера, можно подметить отличительные черты данного метода. Для решения СтЛВУ матричным методом необходимо уметь правильно умножать матрицы и раскладывать элементарные функции в ряд Тейлора.\\\\
\textbf{Пример 2.} Используя представление матричной экспоненты в виде ряда, найти решение уравнения $DX = AX$, где
$$A = \begin{pmatrix}
	0 & -1 \\
	1 & 0
\end{pmatrix}.$$
\textbf{Решение.} Найдем степени матрицы $A$:
$$A^0 = E = \begin{pmatrix}
	1 & 0\\
	0 & 1
\end{pmatrix},\quad A = \begin{pmatrix}
	0 & -1 \\
	1 & 0
\end{pmatrix},\quad A^2 = \begin{pmatrix}
	-1 & 0 \\
	0 & -1
\end{pmatrix},\quad A^3 = \begin{pmatrix}
	0 & 1 \\
	-1 & 0
\end{pmatrix}, \quad A^4 = E = \begin{pmatrix}
1 & 0\\
0 & 1
\end{pmatrix}.$$
Соответственно получили закономерность $A^{4k} = A^0, A^{4k + 1} = A^1, A^{4k + 2} = A^2, A^{4k+3} = A^3$. Построим ряд (2):
$$e^{At} = \begin{pmatrix}
	1 & 0\\
	0 & 1
\end{pmatrix} + \dfrac{t}{1!}\begin{pmatrix}
	0 & -1 \\
	1 & 0
\end{pmatrix} +\dfrac{t^2}{2!}\begin{pmatrix}
	-1 & 0 \\
	0 & -1
\end{pmatrix} + \dfrac{t^3}{3!}\begin{pmatrix}
	0 & 1 \\
	-1 & 0
\end{pmatrix} + \ldots.$$
Запишем в виде матрицы:
$$e^{At} = \begin{pmatrix}
	1 - \dfrac{t^2}{2!} + \ldots + \dfrac{(-1)^kt^{2k}}{(2k)!}+\ldots & -t + \dfrac{t^3}{3!} + \ldots + \dfrac{(-1)^{k}t^{2k-1}}{(2k-1)!}+\ldots\\
	t - \dfrac{t^3}{3!} + \ldots + \dfrac{(-1)^{k+1}t^{2k-1}}{(2k-1)!}+\ldots &  1 - \dfrac{t^2}{2!} + \ldots + \dfrac{(-1)^kt^{2k}}{(2k)!}+\ldots
\end{pmatrix}.\eqno(4)$$
Вспомним разложения
$$cos(t) = 	1 - \dfrac{t^2}{2!} + \ldots + \dfrac{(-1)^kt^{2k}}{(2k)!}+\ldots.$$
$$sin(t) = t - \dfrac{t^3}{3!} + \ldots + \dfrac{(-1)^{k+1}t^{2k-1}}{(2k-1)!}+\ldots.$$
Тогда матричную экспоненту можно записать в виде
$$e^{At} = \begin{pmatrix}
	cos(t) & -sin(t)\\
	sin(t) & cos(t)
\end{pmatrix}.$$
Следовательно, общее решение СтЛВУ имеет вид
$$X(t) =  \begin{pmatrix}
	cos(t) & -sin(t)\\
	sin(t) & cos(t)
\end{pmatrix}\begin{pmatrix}
C_1\\C_2
\end{pmatrix}.$$
\textbf{Ответ:} $(C_1cos(t) - C_2sin(t),\ C_1sin(t) + C_2cos(t))^T.$
\begin{theorem}
	Задача Коши $DX = AX$, $X|_{t=t_0} = \xi$, $t_0 \in \Rm$, $\xi \in \Rm_{n,1}$ имеет единственное решение $$X(t) = e^{A(t-t_0)}\xi.$$
\end{theorem}
\begin{cor}
	Матрица $e^{A(t-t_0)}$ является фундаментальной матрицей уравнения $(1)$, нормированной в точке $t =t_0$.
\end{cor}
\textbf{Пример 3.} Используя представление матричной экспоненты в виде ряда, решить задачу Коши $DX = AX$, $X|_{t=t_0} = \xi$, где
$$A = \begin{pmatrix}
	4 & 0 \\
	0 & 5
\end{pmatrix}, \quad t_0 = 1,\quad \xi = \begin{pmatrix}
3\\4
\end{pmatrix}.$$
\textbf{Решение.} Решение задачи Коши мы можем найти по формуле $X(t) = e^{A(t-t_0)}\xi.$ Тогда из примера 1
$$e^{A(t-t_0)} = \begin{pmatrix}
	e^{4(t-t_0)} & 0\\
	0 & e^{5(t-t_0)}
\end{pmatrix}.$$ И при $t_0 = 1$ получаем
$$e^{A(t-1)} = \begin{pmatrix}
	e^{4(t-1)} & 0\\
	0 & e^{5(t-1)}
\end{pmatrix}.$$ Теперь сможем найти решение задачи Коши:
$$X(t) = \begin{pmatrix}
	e^{4(t-1)} & 0\\
	0 & e^{5(t-1)}
\end{pmatrix} \begin{pmatrix}
3\\4
\end{pmatrix} = (3	e^{4(t-1)}, 4e^{5(t-1)})^T.$$
\textbf{Ответ:} $X(t) = (3	e^{4(t-1)}, 4e^{5(t-1)})^T.$\\\\
\textbf{Замечание.} \textit{Иногда проследить закономерность в степенях матрицы $A$ не удается. В таком случае в качестве фундаментальной матрицы можно взять матрицу, состоящую из рядов (например $(4)$).}
	\subsection*{Вычисление матричной экспоненты.}
	$\forall A \in \Cm_{n,n}$ $\exists J$ --- жорданова нормальная форма, то есть $\forall A \in \Cm_{n,n}$ $\exists S: A = S JS^{-1}$. Следовательно, $$e^{At} = Se^{Jt}S^{-1}.$$
	Вычислим матричную экспоненту для клетки Жордана:
	$$e^{Jt} = e^{\lambda t}\begin{pmatrix}
		1 & \dfrac{t}{1!} & \dfrac{t^2}{2!} & \dfrac{t^3}{3!} & \dots\\
		0 & 1 & \dfrac{t}{1!} & \dfrac{t^2}{2!} & \dots\\
		0 & 0 & 1 & \dfrac{t}{1!} & \dots\\
		0 & 0 & 0 & 1 & \dots\\
		\vdots & \vdots & \vdots & \vdots & \ddots
	\end{pmatrix}\eqno(5)$$
Таким образом, для вычисления матричной экспоненты более сложных матриц $A$ можно воспользоваться жордановой нормальной формой матрицы. Для освежения в памяти повторите главу "Полиномиальные матрицы"\ в Линейной Алгебре.\\\\
\textbf{Пример 4.} Используя экспонентное представление решения, найти общее решение уравнения $DX = AX$, где $$A = \begin{pmatrix}
	-2 & 8 & 6\\
	-4 & 10 & 6\\
	4 & -8 & -4
\end{pmatrix}.$$
\textbf{Решение.} Построим характеристический многочлен матрицы и найдем собственные значения: $$\det(A -\lambda E) = \begin{vmatrix}
	-2 - \lambda & 8 & 6\\
	-4 & 10 - \lambda & 6\\
	4 & -8 & -4 - \lambda
\end{vmatrix} = \lambda(\lambda -2)^2.$$
Следовательно, матрица имеет собственные значения $\lambda_1 = 0$; $\lambda_2 = 2$, $k_2 = 2$. Рассмотрим каждое собственное значение по отдельности:
\begin{enumerate}
	\item Так как собственное значение $\lambda_1 = 0$ имеет алгебраическую кратность 1, а значит и геометрическую кратность 1, то ему соответствует одна жорданова клетка. Найдем собственный вектор, соответствующий данному собственному значению. Для этого решим СЛАУ $(A - \lambda_1 E)\upgamma = 0$, $\upgamma(\upgamma_1, \upgamma_2, \upgamma_3)$:
	$$A - 0E = \begin{pmatrix}
		-2 & 8 & 6\\
		-4 & 10 & 6\\
		4 & -8 & -4
	\end{pmatrix} \sim \begin{pmatrix}
	-2 & 8 & 6\\
	0 & -6 & -6\\
	0 & 8 & -8
\end{pmatrix}\sim \begin{pmatrix}
1 & -4 & -3\\
0 & 1 & 1\\
0 & 1 & -1
\end{pmatrix}\sim \begin{pmatrix}
1 & -4 & -3\\
0 & 1 & 1
\end{pmatrix}\sim $$ $$\sim\begin{pmatrix}
1 & 0 & \vline & 1\\
0 & 1 &\vline & 1
\end{pmatrix}.$$
Возьмем третий столбец за произвольную переменную. Тогда ФСР СЛАУ $A - 0 E = 0$ имеет вид $\upgamma(-\alpha, -\alpha, \alpha)$. Следовательно, собственный вектор, соответствующий $\lambda_1 = 0$ равен
$$a_1(-1, -1, 1).$$
\item Собственное значение $\lambda_2 = 2$ имеет алгебраическую кратность $k_2 = 2$. Найдем его геометрическую кратность $r_2$:
$$r_2 = n - rank(A - 2E) = 3 - rank\begin{pmatrix}
	-4 & 8 & 6\\
	-4 & 8 & 6\\
	4 & -8 & -6
\end{pmatrix} = 2.$$ Следовательно, собственному значению соответствуют 2 жордановы клетки и оно имеет 2 линейно независимых собственных вектора. Для нахождения собственных векторов решим СЛАУ $(A - \lambda_2 E)\upgamma = 0$, $\upgamma(\upgamma_1, \upgamma_2, \upgamma_3)$: $$A - 2E = \begin{pmatrix}
-4 & 8 & 6\\
-4 & 8 & 6\\
4 & -8 & -6
\end{pmatrix}\sim \begin{pmatrix}
2 & -4 & -3\\
\end{pmatrix}\sim \begin{pmatrix}
1 &\vline & -2 &\vline& -\dfrac{3}{2}\\
\end{pmatrix}.$$
В качестве произвольных переменных $\alpha$ и $\beta$ возьмем второй и третий столбцы соответственно. Тогда ФСР имеет вид $\upgamma(2\alpha + \frac{3}{2}\beta, \alpha, \beta)$. Следовательно, собственными векторами, соответствующими этому собственному значению, являются $$a_2(2,1,0),\quad a_3(3,0,2).$$
\end{enumerate}
Теперь можно построить жорданову нормальную форму матрицы и трансформирующую матрицу. Так как мы выяснили, что жорданова нормальная имеет 3 клетки, то $$J = \begin{pmatrix}
	0 & 0 & 0\\
	0 & 2 & 0\\
	0 & 0 & 2
\end{pmatrix}.$$
\textbf{Замечание.} \textit{Жордановы блоки в жордановой нормальной форме будем записывать по возрастанию собственного значения и по убыванию размерностей жордановых клеток.}\\\\
Тогда матричная экспонента для жордановой нормальной формы имеет вид $$e^{Jt} = \begin{pmatrix}
	e^{0t} & 0 & 0\\
	0 & e^{2t} & 0\\
	0 & 0 & e^{2t}
\end{pmatrix} = \begin{pmatrix}
1 & 0 & 0\\
0 & e^{2t} & 0\\
0 & 0 & e^{2t}
\end{pmatrix}.$$
\textit{Заметим, что если бы собственному значению $\lambda_2 = 2$ соответствовала одна жорданова клетка, то матричная экспонента имела бы вид} $e^{Jt} =\begin{pmatrix}
	1 & 0 & 0\\
	0 & e^{2t} & te^{2t}\\
	0 & 0 & e^{2t}
\end{pmatrix}.$\\\\
Трансформирующая матрица строится из собственных векторов. Причем собственные векторы ставятся в те столбцы, в которых находятся соответствующие собственные значения в жордановой нормальной форме. Тогда трансформирующая матрица имеет вид $$S = \begin{pmatrix}
	-1 & 2 & 3\\
	-1 & 1 & 0\\
	1 & 0 & 2
\end{pmatrix}.$$
Найдем обратную матрицу. Вы можете воспользоваться известными Вам методами. Я же расширю матрицу, добавив к ней единичную. Затем элементарными преобразованиями приведу матрицу слева к единичному виду: $$(S\ |\ E) = \begin{pmatrix}
	-1 & 2 & 3 & \vline & 1 & 0 & 0\\
	-1 & 1 & 0 & \vline & 0 & 1 & 0\\
	1 & 0 & 2 & \vline & 0 & 0 & 1
\end{pmatrix}\sim  \begin{pmatrix}
	1 & 0 & 0 & \vline & -2 & 4 & 3 \\
	0 & 1 & 0 & \vline &-2 & 5 & 3 \\
	0 & 0 & 1 & \vline & 1 & -2 & -1
\end{pmatrix} = (E\ |\ S^{-1}).$$
Тогда матричная экспонента имеет вид
$$e^{At} =\begin{pmatrix}
	-1 & 2 & 3\\
	-1 & 1 & 0\\
	1 & 0 & 2
\end{pmatrix}\begin{pmatrix}
1 & 0 & 0\\
0 & e^{2t} & 0\\
0 & 0 & e^{2t}
\end{pmatrix}\begin{pmatrix}
-2 & 4 & 3\\
-2 & 5 & 3\\
1 & -2 & -1
\end{pmatrix} = \begin{pmatrix}
2 - e^{2t} & 4e^{2t} - 4 & 3e^{2t}-3\\
2-2e^{2t} & 5e^{2t} - 4 & 3e^{2t} - 3\\
2e^{2t}-2 & 4-4e^{2t} & 3-2e^{2t}
\end{pmatrix}.$$
Тогда общее решение СтЛВУ имеет вид 
$$X(t) = \begin{pmatrix}
	2 - e^{2t} & 4e^{2t} - 4 & 3e^{2t}-3\\
	2-2e^{2t} & 5e^{2t} - 4 & 3e^{2t} - 3\\
	2e^{2t}-2 & 4-4e^{2t} & 3-2e^{2t}
\end{pmatrix}\begin{pmatrix}
C_1\\C_2\\C_3
\end{pmatrix}.$$
Умножать получившиеся матрицы мы не будем, так как это будет объемно.\\\\
\textbf{Ответ:} $X(t) = \begin{pmatrix}
	2 - e^{2t} & 4e^{2t} - 4 & 3e^{2t}-3\\
	2-2e^{2t} & 5e^{2t} - 4 & 3e^{2t} - 3\\
	2e^{2t}-2 & 4-4e^{2t} & 3-2e^{2t}
\end{pmatrix}\begin{pmatrix}
	C_1\\C_2\\C_3
\end{pmatrix}.$
\\\\
\textbf{Замечание.} \textit{Трансформирующая матрица неоднозначна, поэтому решения одного уравнения могут отличаться в зависимости от выбора матрицы $S$.}\\\\
\textbf{Замечание.} \textit{Фундаментальная матрица $\FI(t) = e^{At} = Se^{Jt}S^{-1}$ может отличаться от матрицы, полученной с помощью решения методом Эйлера. Всё дело опять же в том, что матричным методом мы получаем представление в виде матричной экспоненты, которая, в свою очередь, является фундаментальной матрицей ФСР нормированной в точке $t = t_0$. ФСР уравнения является матрица $Se^{Jt}$, но, умножая ее на $S^{-1}$, мы нормируем ее в точке $t_0 = 0$. Аналогичные действия мы проводили, когда нормировали ФСР в СтЛУ например при поиске частного решения методом Коши.}\\\\
\textbf{Пример 5.} Используя экспонентное представление решения, найти общее решение уравнения $DX = AX$, где $$A = \begin{pmatrix}
	4 & 6 & -15\\
	1 & 3 & -5\\
	1 & 2 & -4
\end{pmatrix}.$$
\textbf{Решение.} Характеристический многочлен имеет единственный корень $\lambda = 1$ с кретностью $k=3$. Выясним количество жордановых клеток, то есть геометрическую кратность, этого собственного значения:
$$r = n - rank(A - E) = 3 - rank\begin{pmatrix}
	3 & 6 & -15\\
	1 & 2 & -5\\
	1 & 2 & -5
\end{pmatrix} = 2.$$
Следовательно, собственному значению соответствуют две жордановых клетки, причем одна размера $2\times2$, а другая $1\times1$ и два собственных вектора. Таким образом, построим ЖНФ
$$J = \begin{pmatrix}
	1 & 1 & 0\\
	0 & 1 & 0\\
	0 & 0 & 1
\end{pmatrix}.$$ Матричная экспонента ЖНФ имеет вид
$$e^{Jt} = e^t \begin{pmatrix}
	1 & t & 0\\
	0 & 1 & 0\\
	0 & 0 & 1
\end{pmatrix} = \begin{pmatrix}
e^t & te^t & 0\\
0 & e^t & 0\\
0 & 0 & e^t
\end{pmatrix}.$$
Теперь остается составить трансформирующую матрицу. Для этого найдем два линейно независимых собственных вектора и присоединенный к ним. Для поиска собственных векторов решим СЛАУ $(A-E)\upgamma = 0$:
$$A - E = \begin{pmatrix}
	3 & 6 & -15\\
	1 & 2 & -5\\
	1 & 2 & -5
\end{pmatrix} \sim \begin{pmatrix}
1 & 2 & -5\\
\end{pmatrix}.$$
В качестве свободных переменный возьмем второй столбец за $\alpha$ и третий столбец за $\beta$. Тогда ФСР такой СЛАУ имеет равна $\upgamma(-2\alpha + 5 \beta, \alpha, \beta)$. Запишем два собственных вектора образованных этой ФСР: $$a_1(3,1,1),\quad a_2(5,0,1).$$
Теперь найдем присоединенный вектор. В качестве вектора, для которого мы будем находить присоединенный, можно взять любой. Поэтому возьмем вектор $a_1$, чтобы наша следующая система $(A-E\ |\ a_2) \upgamma = 0$ имела решения и не было проблем с поиском вектора $a_3$
$$(A-E\ |\ a_2) = \begin{pmatrix}
	3 & 6 & -15 & \vline & 3\\
	1 & 2 & -5& \vline & 1\\
	1 & 2 & -5& \vline & 1
\end{pmatrix}\sim \begin{pmatrix}
1 & 2 & -5& \vline & 1
\end{pmatrix}.$$
В качестве свободных переменный возьмем второй столбец за $\alpha$ и третий столбец за $\beta$. Тогда ФСР такой СЛАУ имеет равна $\upgamma(-2\alpha + 5 \beta + 1, \alpha, \beta)$. Тогда в качестве присоединенного вектора возьмем вектор $$a_3(1,0,0).$$ В итоге мы получаем следующую трансформирующую матрицу: $$S = \begin{pmatrix}
	3 & 1 & 5\\
	1 & 0 & 0\\
	1 & 0 & 1
\end{pmatrix}.$$
\textbf{Замечание.} \textit{В трансформирующей матрице присоединенный вектор записывается рядом с тем, для которого он является присоединенным. Поэтому вектор $(3,1,1)$ составляет первый столбец, а присоединенный к нему --- второй. Иначе, если бы он находился во втором столбце, то присоединенный находился бы справа от него, а клетка размерности $2\times2$ в ЖНФ располагалась бы после клетки $1\times1$.}\\\\
\textbf{Замечание.} \textit{Если в задаче стоит вопрос лишь о нахождении общего решения, то необязательно вычислять матрицу $S^{-1}$. Так как общему решению будет удовлетворять любая ФСР, включая нормированную в точке $t$. Вообще говоря, заниматься поиском нормированной ФСР есть смысл только в случае, когда нам нужно решить задачу Коши. И то не всегда, а только лишь при нахождении решения СтЛНВУ методом Коши (который мы рассмотрим в следующем уроке).}\\\\
На основании крайнего замечания мы можем утвержать, что общим решением СтЛВУ является столбец $$X(t) = \FI(t)C = Se^{Jt}C = e^t\begin{pmatrix}
	3 & 1 & 5\\
	1 & 0 & 0\\
	1 & 0 & 1
\end{pmatrix}\begin{pmatrix}
	1 & t & 0\\
	0 & 1 & 0\\
	0 & 0 & 1
\end{pmatrix} \begin{pmatrix}
C_1\\C_2\\C_3
\end{pmatrix}= e^t\begin{pmatrix}
3 & 3t+ 1 & 5\\
1 & t & 0\\
1 & t & 1\\
\end{pmatrix}\begin{pmatrix}
C_1\\C_2\\C_3
\end{pmatrix} =$$ $$= e^t\begin{pmatrix}
3C_1 + C_2(3t+1) + 5C_3\\
C_1 + C_2t\\
C_1 + C_2t + C_3
\end{pmatrix}.$$
Стоит также обратить внимание, что такой подсчет матриц занимает больше времени. Поскольку умножение матриц ассоциативно, то умножение мы можем начать не слева направо, а справа налево. Тогда 
$$X(t) = \FI(t)C = Se^{Jt}C = e^t\begin{pmatrix}
	3 & 1 & 5\\
	1 & 0 & 0\\
	1 & 0 & 1
\end{pmatrix}\begin{pmatrix}
	1 & t & 0\\
	0 & 1 & 0\\
	0 & 0 & 1
\end{pmatrix} \begin{pmatrix}
	C_1\\C_2\\C_3
\end{pmatrix}= e^t\begin{pmatrix}
3 & 1 & 5\\
1 & 0 & 0\\
1 & 0 & 1
\end{pmatrix}\begin{pmatrix}
	C_1 + C_2t\\
	C_2\\
	C_3
\end{pmatrix} =$$ $$= e^t\begin{pmatrix}
	3C_1 + C_2(3t+1) + 5C_3\\
	C_1 + C_2t\\
	C_1 + C_2t + C_3
\end{pmatrix}.$$
Таким образом мы нашли общее решение, сократив количество операций умножения.\\\\
Если же нам необходимо найти ФСР нормированную в точке $t = 0$ или посчитать матричную экспоненту, то нужно посчитать обратную к матрице $S$ матрицу $S^{-1}$. Она имеет вид
$$S^{-1} = \begin{pmatrix}
	0 & 1 & 0\\
	1 & 2 & -5\\
	0 & -1 & 1
\end{pmatrix}.$$
Таким образом, матричная экспонента исходной матрицы $A$ равна
$$e^{At} = \FI(t) = Se^{Jt}S^{-1} = e^t\begin{pmatrix}
	3 & 1 & 5\\
	1 & 0 & 0\\
	1 & 0 & 1
\end{pmatrix}\begin{pmatrix}
1 & t & 0\\
0 & 1 & 0\\
0 & 0 & 1
\end{pmatrix}\begin{pmatrix}
0 & 1 & 0\\
1 & 2 & -5\\
0 & -1 & 1
\end{pmatrix}=$$ $$= e^t\begin{pmatrix}
3 & 3t+ 1 & 5\\
1 & t & 0\\
1 & t & 1\\
\end{pmatrix}\begin{pmatrix}
0 & 1 & 0\\
1 & 2 & -5\\
0 & -1 & 1
\end{pmatrix}=e^t\begin{pmatrix}
3t+1 & 6t& -15t\\
t & 2t + 1& -5t\\
t & 2t & 1 - 5t
\end{pmatrix}.$$
Тогда общее решение СтЛВУ будет иметь вид
$$X(t) = e^t\begin{pmatrix}
	3t+1 & 6t& -15t\\
	t & 2t + 1& -5t\\
	t & 2t & 1 - 5t
\end{pmatrix}\begin{pmatrix}
C_1\\C_2\\C_3
\end{pmatrix}.$$
Также для сокращения количества операций умножения про поиске общего решения уравнения мы могли дописать справа столбец $C$ и умножение начать справа налево.\\\\
\textbf{Ответ:} $X(t) = e^t\begin{pmatrix}
	3C_1 + C_2(3t+1) + 5C_3\\
	C_1 + C_2t\\
	C_1 + C_2t + C_3
\end{pmatrix}$ или (и) $X(t) = e^t\begin{pmatrix}
	3t+1 & 6t& -15t\\
	t & 2t + 1& -5t\\
	t & 2t & 1 - 5t
\end{pmatrix}\begin{pmatrix}
	C_1\\C_2\\C_3
\end{pmatrix}.$\\\\
\textbf{Пример 6.} Используя экспонентное представление решения, найти общее решение уравнения $DX = AX$, где $$A = \begin{pmatrix}
	4 & 5 & -2\\
	-2 & -2 & 1\\
	-1 & -1 & 1
\end{pmatrix}.$$
\textbf{Решение.} Матрица имеет одно собственное значение $\lambda = 1$ кратности $k = 3$. Возможно вы уже догадываетесь, чем данный пример будет отличаться от предыдущего. Сперва найдем геометрическую кратность собственного значения:
$$r = n - rank(A - E)=3 - rank\begin{pmatrix}
	3 & 5 & -2\\
	-2 & -3 & 1\\
	-1 & -1 & 0
\end{pmatrix} = 3 - rank\begin{pmatrix}
0 & 1 & -1\\
0 & 1 & -1\\
-1 & -1 & 0
\end{pmatrix} = 1.$$
Следовательно, собственному значению соответствует один собственный вектор и одна жорданова клетка. Построим матричную экспоненту ЖНФ:
$$e^{Jt} = e^t\begin{pmatrix}
	1 & t & \frac{t^2}{2}\\
	0 & 1 & t\\
	0 & 0 & 1
\end{pmatrix}.$$
Теперь найдем собственный вектор, соответствующий собственному значению $$A - E = \begin{pmatrix}
		3 & 5 & -2\\
	-2 & -3 & 1\\
	-1 & -1 & 0
\end{pmatrix}\sim \begin{pmatrix}
1 & \vline & 1 & \vline & 0\\
0 & \vline & -1 &\vline & 1
\end{pmatrix}.$$
В качестве свободной переменной возьмем второй столбец. Тогда ФСР имеет вид $(-\alpha, \alpha, \alpha)$, а собственный вектор $$a_1(1, -1, -1).$$
Теперь займемся поиском присоединенных векторов. Для поиска присоединенного к $a_1$ решим СЛАУ $(A-E\ |\ a_1)\upgamma = 0$:
$$(A-E\ |\ a_1) = \begin{pmatrix}
	3 & 5 & -2 & \vline & 1\\
	-2 & -3 & 1 & \vline & -1\\
	-1 & -1 & 0 & \vline & -1
\end{pmatrix}\sim \begin{pmatrix}
1 & 1 & 0 & \vline & 1\\
0 & -1 & 1 & \vline & 1\\
\end{pmatrix}.$$
Возьмем в качестве свободной переменной второй столбец. Тогда ФСР имеет вид $(1 - \alpha, \alpha, 1 + \alpha)$ и собственный вектор $$a_2(1,0,1).$$
Осталось найти последний присоединенный вектор. Искать мы его будем для вектора $a_2$:
$$(A-E\ |\ a_2) = \begin{pmatrix}
	3 & 5 & -2 & \vline & 1\\
	-2 & -3 & 1 & \vline & 0\\
	-1 & -1 & 0 & \vline & 1
\end{pmatrix}\sim\begin{pmatrix}
1 & 1 & 0 & \vline & -1\\
0 & -1 & 1 & \vline & -2\\
\end{pmatrix}.$$
Снова возьмем второй столбец в качестве свободной переменной и получим ФСР $(-1-\alpha, \alpha, -2 + \alpha)$ и собственный вектор 
$$a_3(-1, 0, -2).$$
Тогда трансформирующая матрица равна
$$S =\begin{pmatrix}
	1 & 1 & -1\\
	-1 & 0 & 0\\
	-1 & 1 & -2
\end{pmatrix}.$$
А обратная к ней 
$$S^{-1} = \begin{pmatrix}
	0 & -1 & 0\\
	2 & 3 & -1\\
	1 & 2 & -1
\end{pmatrix}.$$
Таким образом, общее решение СтЛВУ имеет вид $$X(t) = e^{At}C = e^t\begin{pmatrix}
	1 & 1 & -1\\
	-1 & 0 & 0\\
	-1 & 1 & -2
\end{pmatrix}\begin{pmatrix}
1 & t & \frac{t^2}{2}\\
0 & 1 & t\\
0 & 0 & 1
\end{pmatrix}\begin{pmatrix}
0 & -1 & 0\\
2 & 3 & -1\\
1 & 2 & -1
\end{pmatrix}\begin{pmatrix}
C_1\\C_2\\C_3
\end{pmatrix}=$$ $$=e^t\begin{pmatrix}
1 & t+1 & \frac{t^2}{2} + t - 1\\
-1 & -t & -\frac{t^2}{2}\\
-1 & 1-t & -\frac{t^2}{2} + t - 2
\end{pmatrix}\begin{pmatrix}
0 & -1 & 0\\
2 & 3 & -1\\
1 & 2 & -1
\end{pmatrix}\begin{pmatrix}
C_1\\C_2\\C_3
\end{pmatrix}.$$
Также общее решение уравнения можно получить из ненормированной в точке $t= 0$ ФСР, то есть без матрицы $S^{-1}$, на этапе $$X(t) =e^t\begin{pmatrix}
	1 & t+1 & \frac{t^2}{2} + t - 1\\
	-1 & -t & -\frac{t^2}{2}\\
	-1 & 1-t & -\frac{t^2}{2} + t - 2
\end{pmatrix}\begin{pmatrix}
	C_1\\C_2\\C_3
\end{pmatrix}.$$
\textbf{Ответ:} $X(t)=e^t\begin{pmatrix}
	\frac{t^2}{2} + 2t + 1 & t^2+5t & -\frac{t^2}{2} -2t\\
	-\frac{t^2}{2} -t & -\frac{t^2}{2} -3t + 1 & \frac{t^2}{2} + t\\
	-\frac{t^2}{2} -3 t & -t^2-t & \frac{t^2}{2} + 1
\end{pmatrix}\begin{pmatrix}
	C_1\\C_2\\C_3
\end{pmatrix}.$\\\\
Напоследок рассмотрим решение задачи Коши для такого способа вычисления матричной экспоненты.\\\\
\textbf{Пример 7.} Решить задачу Коши $DX = AX$, $X|_{t=2} = \xi$, где 
$$A = \begin{pmatrix}
	1 & 0 & 0\\
	0 & 2 & 0\\
	0 & 0 & -3
\end{pmatrix},\quad \xi = \begin{pmatrix}
1 \\ -1 \\ 0
\end{pmatrix}.$$
\textbf{Решение.} Собственные значения матрицы равны $\lambda_1 = -3$, $\lambda_2 = 1$, $\lambda_3 = 2$, кратности которых равны 1, и каждому соответствует 1 жорданова клетка. Тогда
$$e^{Jt} = \begin{pmatrix}
	e^{-3t} & 0 & 0\\
	0 & e^{t} & 0\\
	0 & 0 & e^{2t} 
\end{pmatrix}.$$
Данным собственным значениям соответствуют собственные векторы $a_1(0, 0, 1)$, $a_2(1, 0, 0)$, $a_3(0, 1, 0)$. Тогда
$$S = \begin{pmatrix}
	0 & 1 & 0\\
	0 & 0 & 1\\
	1 & 0 & 0
\end{pmatrix};\quad S^{-1} = \begin{pmatrix}
0 & 0 & 1\\
1 & 0 & 0\\
0 & 1 & 0
\end{pmatrix}.$$
Таким образом $$e^{At} = \begin{pmatrix}
	e^t & 0 & 0\\
	0 & e^{2t} & 0\\
	0 & 0 & e^{-3t}
\end{pmatrix}.$$
Мы получили ФСР нормированную в точке $t_0 = 0$. В общем имеем $$e^{A(t-t_0)} = \begin{pmatrix}
	e^{(t-t_0)} & 0 & 0\\
	0 & e^{2(t-t_0)} & 0\\
	0 & 0 & e^{-3(t-t_0)}
\end{pmatrix}.$$
Подставим наши начальные условия: $t_0 = 2$ и умножим на столбец $\xi$. Тогда решение задачи Коши имеет вид
$$X(t) = \begin{pmatrix}
	e^{(t-2)} & 0 & 0\\
	0 & e^{2(t-2)} & 0\\
	0 & 0 & e^{-3(t-2)}
\end{pmatrix}\begin{pmatrix}
1 \\ -1 \\ 0
\end{pmatrix} = \begin{pmatrix}
e^{t-2}\\
-e^{2t-4}\\
0
\end{pmatrix}.$$
Получить данное решение можно было бы и другим образом. Например, если мы вычислим ФСР $\FI(t) = Se^{Jt}$, то для решения задачи Коши необходимо умножить ее на матрицу $\FI^{-1}(\uptau)\xi$ аналогично методу Эйлера для СтЛВУ.\\\\
\textbf{Ответ:} $X(t) = \begin{pmatrix}
	e^{t-2}\\
	-e^{2t-4}\\
	0
\end{pmatrix}.$
\subsection*{Задачи для самостоятельного решения.}
 Используя экспонентное представление решения, найти общие решения уравнений $DX = AX$, где\begin{enumerate}
 	\item $$A = \begin{pmatrix}
 		2 & 0\\
 		0 & 3
 	\end{pmatrix};$$
 \item $$A = \begin{pmatrix}
 	1 & 1\\
 	0 & -1
 \end{pmatrix};$$
\item $$A = \begin{pmatrix}
	0 & 1\\
	-1 & 0
\end{pmatrix};$$
\item $$A = \begin{pmatrix}
	2 & 0 & 0\\
	1 & 2 & 0\\
	0 & 0 & -1
\end{pmatrix};$$
\item $$A = \begin{pmatrix}
	1 & 0 & 1\\
	0 & 1 & 1\\
	1 & -1 & 1
\end{pmatrix};$$
\item $$A = \begin{pmatrix}
	2 & -1 & 1\\
	1 & 2 & 0\\
	1 & 2 & 0
\end{pmatrix};$$
\end{enumerate}
и решить задачи Коши
\begin{enumerate}
\item $$A = \begin{pmatrix}
	1 & 1\\
	-2 & 4
\end{pmatrix};\quad X|_{t=1} = \begin{pmatrix}
1\\2
\end{pmatrix};$$
\item $$A = \begin{pmatrix}
	4 & -5 & -6\\
	-2 & 7 & 6\\
	2 & -5 & -4
\end{pmatrix};\quad X|_{t=0} = \begin{pmatrix}
	1\\1\\1
\end{pmatrix};$$
\item $$A = \begin{pmatrix}
	5 & 4 & 6\\
	4 & 5 & 6\\
	-4 & -4 & -5
\end{pmatrix};\quad X|_{t=0} = \begin{pmatrix}
	1\\0\\1
\end{pmatrix};$$
 \end{enumerate}
\end{document}