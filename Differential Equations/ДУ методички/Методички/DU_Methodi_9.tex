\documentclass[a4paper, 12pt]{article}
\usepackage{cmap}
\usepackage{amssymb}
\usepackage{amsmath}
\usepackage{graphicx}
\usepackage{amsthm}
\usepackage{upgreek}
\usepackage{setspace}
\usepackage[T2A]{fontenc}
\usepackage[utf8]{inputenc}
\usepackage[normalem]{ulem}
\usepackage{mathtext} % русские буквы в формулах
\usepackage[left=2cm,right=2cm, top=2cm,bottom=2cm,bindingoffset=0cm]{geometry}
\usepackage[english,russian]{babel}
\newenvironment{Proof} % имя окружения
{\par\noindent{}} % команды для \begin
{\hfill$\scriptstyle$}
\newcommand{\Rm}{\mathbb{R}}
\newcommand{\Cm}{\mathbb{C}}
\newcommand{\I}{\mathbb{I}}
\newcommand{\N}{\mathbb{N}}
\newtheorem*{theorem}{Теорема}
\newtheorem*{cor}{Следствие}
\renewcommand{\epsilon}{\upvarepsilon}
\newcommand\Norm[1]{\left\| #1 \right\|}
\newtheorem*{lem}{Лемма}
\newcommand{\FI}{\text{Ф}}
\renewcommand{\tau}{\uptau}
\newcommand{\Ln}{L_n = D^n + a_{n-1}D^{n-1} + \ldots + a_1D + a_0D^0}
\begin{document}
	\section*{Устойчивость СтЛВУ.}
	Введем аналогичные СтЛУ свойства и определения.\\\\
	Рассмотрим уравнение $$DX = AX + f(t),\ t \in \I = [t_0; + \infty)\eqno (1)$$ с непрерывной на $\I$ векторной функцией $f(t)$.\\\\
		$\bullet$ \textit{Решение $X_0(t)$ уравнения $(1)$ называется \textbf{устойчивым по Ляпунову}, если}$$\forall \epsilon\ \exists\delta : \forall X(t) \Norm{X(t_0) - X_0(t_0)} < \delta \Rightarrow \Norm{X(t) - X_0(t)} < \epsilon\ \forall t > t_0.$$
		Таким образом, решение $X_0(t)$ называется устойчивым по Ляпунову, если оно непрерывно зависит от начальных данных на промежутке $\I$ вида $[t_0;+\infty)$.\\\\
		$\bullet$ \textit{Если кроме того $\lim\limits_{t\to+\infty}\Norm{X(t) - X_0(t)} = 0$, то решение $X_0(t)$ называется \textbf{асимптотически устойчивым}.}\\\\
		Все решения уравнения либо одновременно устойчивы, либо нет.\\\\
		$\bullet$ \textit{Уравнение, все решения которого устойчивы, называется \textbf{устойчивым} (аналогично \textbf{неустойчивым}, \textbf{асимптотически устойчивым}).}\\\\
		Устойчивость уравнения не зависит от неоднородности $f(t)$. Следовательно, в дальнейшем вместо (1) можно рассматривать соответствующее ему СтЛВУ $DX = AX$.
		\subsection*{Устойчивость по Ляпунову.}
		Теперь введем теорему, с помощью которой и будем исследовать устойчивость уравнений.
		\begin{theorem}[Критерий устойчивости]
			Уравнение $DX = AX + f(t)$ устойчиво $\Longleftrightarrow$ действительные части собственных значений матрицы $A$ неположительны, причем собственным значениям с нулевой действительной частью в Жордановой матрице соответствует клетка размерности 1, то есть геометрическая и алгебраическая кратности совпадают.
		\end{theorem}\begin{Proof}
	\end{Proof}\\
	\textbf{Пример 1.} Исследовать на учтойчивость СтЛВУ вида $DX = AX$, где 
	$$A = \begin{pmatrix}
		-2 & 0 & -2\\
		1 & 0 & 1\\
		-2 & 0 & -2
	\end{pmatrix}.$$
\textbf{Решение.} Матрица $A$ имеет собственные значения $\lambda_1 = 0$, $k_1 = 2$; $\lambda_2 = -4$. Геометрическая кратность значения $\lambda_1 = 0$ равна $r_1 = 2$. Следовательно, этому значению соответствуют две клетки Жордановой матрицы. Значит по критерию устойчивости СтЛВУ является устойчивым.\\\\
\textbf{Ответ:} Устойчиво.\\\\
\textbf{Пример 2.} Исследовать на учтойчивость СтЛВУ вида $DX = AX$, где 
$$A = \begin{pmatrix}
	1 & 1 & -1 & 0\\
	-3 & -3 & 3 & 5\\
	-2 & -2 & 2 & 1\\
	0 & 0 & 0 & -8
\end{pmatrix}.$$
\textbf{Решение.} Матрица $A$ имеет собственные значения $\lambda_1 = 0$, $k_1 = 3$; $\lambda_2 = -8$. Так как геометрическая кратность собственного значения $\lambda_1 = 0$ равна $r_1 = 2$, то данному собственному значению соответствуют 2 клетки Жордановой матрицы. Причем одна из этих клеток имеет размерность 2. Значит по критерию устойчивости СтЛВУ не является устойчивым.\\\\
\textbf{Ответ:} Неустойчиво.\\\\
Остальные уравнения рассматриваются аналогично. То есть\begin{itemize}
	\item находим собственные значения матрицы $A$;
	\item если среди собственных значений есть хотя бы одно с положительной действительной частью, то уравнение неустойчиво;
	\item если есть собственное значение равно нулю алгебраической кратности большей, чем 1, то находим его геометрическую кратность; если получается так, что клетка жордановой матрицы, соответствующая этому собственному значению имеет размерность больше 1, то СтЛВУ неустойчиво;
	\item иначе СтЛВУ устойчиво.
\end{itemize} 
	\subsection*{Асимптотическая устойчивость.}
	Основываясь на определении устойчивости, мы можем составить аналогичный критерий для исследования на асимптотическую устойчивость.
	\begin{theorem}[Критерий асимптотической устойчивости]
		Уравнение $DX = AX + f(t)$ асимптотически устойчиво $\Longleftrightarrow$ действительные части собственных значений матрицы $A$ отрицательны.
	\end{theorem}\begin{Proof}\end{Proof}\\
	\textbf{Замечание.} \textit{Из критерия следует, что условие асимптотической устойчивости более строгое, чем устойчивости. Следовательно, если уравнение асимптотически устойчиво, то оно устойчиво. Обратное, вообще говоря, утверждать нельзя.}\\\\
	$\bullet$ \textit{Определитель порядка $n$ вида $$\begin{vmatrix}
			a_{n-1} & 1 & 0 & 0 & \dots & 0\\
			a_{n-3} & a_{n-2} & a_{n-1} & 1 & \dots & 0\\
			a_{n-5} & a_{n-4} & a_{n-3} & a_{n-2} & \dots & 0\\
			\vdots & \vdots & \vdots & \vdots & \ddots & \vdots\\
			a_{n-(2n+1)} & a_{n-2n} & a_{n-(2n -1)} & a_{n-(2n - 2)} & \dots & a_0\\
		\end{vmatrix}$$ называется \textbf{определителем Гурвица}, или \textbf{гурвицианом} ($a_j = 0$, если $j < 0$)}.
	\begin{theorem}
		[Критерий Гурвица] Действительные части всех корней характеристического многочлена $\lambda^n + a_{n-1}\lambda^{n-1} + \ldots + a_{1}\lambda + a_0$ матрицы $A$ отрицательны $\Longleftrightarrow$ все главные миноры гурвициана положительны.  
	\end{theorem}\begin{Proof}\end{Proof}\\
	\textbf{Замечание.} \textit{Невыполнение критерия Гурвица (хотя бы один из главных миноров неположительный) не гарантирует неустойчивость.}\\\\
	\textbf{Пример 3.} Исследовать на асимптотическую устойчивость уравнение  вида $DX = AX$, где $$A = \begin{pmatrix}
		-2 & -1 & -1 & 9\\
		3 & 2 & 1 & 6\\
		1 & 0 & -1 & 5\\
		0 & 0 & 0 & -3
	\end{pmatrix}.$$
\textbf{Решение.} Для исследования данного уравнения составим характеристический многочлен матрицы $A$:
$$\det(A-\lambda E) = \begin{vmatrix}
	-2-\lambda & -1 & -1 & 9\\
	3 & 2-\lambda & 1 & 6\\
	1 & 0 & -1-\lambda & 5\\
	0 & 0 & 0 & -3-\lambda
\end{vmatrix} = \lambda^4 +4\lambda^3 + 3\lambda^2 - 2\lambda - 6 = 0.$$
Не будем вычислять собственные значения, а составим Гурвициан этого многочлена:
$$\begin{vmatrix}
	4 & 1 & 0 & 0\\
	-2 & 3 & 4 & 1\\
	0 & -6 & -2 & 3\\
	0 & 0 & 0 & -6
\end{vmatrix},\quad \Delta_1 = 4 > 0,\ \Delta_2 = 10 > 0,\ \Delta_3 = 68 > 0,\ \Delta_4 = -6\cdot 68 < 0.$$
Следовательно, уравнение не является асимптотически устойчивым. Можно попытатьтся исследовать на устойчивость данное уравнение, но для этого уже нужно найти корни характеристического уравнения, а для этого необходимо разложить характеристический многочлен на простейшие множители. Возьмем $\lambda = 1$ и проверим, является ли он оно корнем. Подставим в характеристическое уравнение и получим $1 + 4 + 3 -2 - 6 = 0$. Следовательно, $\lambda = 1$ является корнем, и уравнение не является устойчивым.\\\\
\textbf{Ответ:} Не является асимптотически устойчивым; неустойчиво.\\\\
\textbf{Пример 4.} Исследовать на асимптотическую устойчивость уравнение  вида $DX = AX$, где $$A = \begin{pmatrix}
	1 & -3 & 5\\
	4 & -7 & 8\\
	6 & -7 & -7
\end{pmatrix}.$$
\textbf{Решение.} Построим характеристический многочлен матрицы $A$:
$$\det(A-\lambda E) = \begin{vmatrix}
	1-\lambda & -3 & 5\\
	4 & -7-\lambda & 8\\
	6 & -7 & -7-\lambda
\end{vmatrix} = -\lambda^3-13\lambda^2-73\lambda-53 = 0.$$
Домножим характеристическое уравнение на $-1$, чтобы старший коэффициент был равен 1. Построим Гурвициан характеристического многочлена:
$$\begin{vmatrix}
	13 & 1 & 0\\
	53 & 73 & 13\\
	0 & 0 & 53
\end{vmatrix},\quad \Delta_1 = 13 > 0,\ \Delta_2 = 13\cdot 73 -53 > 0,\ \Delta_3 = 53\cdot\Delta_2 > 0.$$
Следовательно, уравнение является асимптотически устойчивым по критерию асимптотической устойчивости.\\\\
\textbf{Ответ:} Асимптотически устойчиво.
В общем и целом исследование СтЛВУ на асимптотическую устойчивость аналогично исследованию СтЛУ:\begin{itemize}
	\item строим характеристический многочлен матрицы $A$;
	\item если нет возможности сразу определить корни характеристического уравнения, то строим Гурвициан и исследуем его главные миноры; если хотя бы один минор неположителен, то уравнение не является асимптотически устойчивым и мы переходим к исследованию на устойчивость;
	\item если мы смогли определить собственные значения матрицы $A$, то их действительные части должны быть отрицательны; иначе уравнение не является асимптотически устойчивым и мы переходим к исследованию на устойчивость.
\end{itemize}
Рассмотрим также один пример с параметром. Сразу хочется подметить то, что для таких заданий нет одного четкого алгоритма, кроме как построить характеристический многочлен. Просматривать же зависимость от параметров можно совершенно различными способами: вычисление дискриминанта, теорема Виета, построение Гурвициана и так далее.\\\\
\textbf{Пример 5.} Определить область плоскости параметров асимптотической устойчивости линейных уравнений вида $DX = AX$, где
$$A = \begin{pmatrix}
	0 & 1 & 0\\
	0 & 0 & 1\\
	-2 & -b & -a
\end{pmatrix}.$$
\textbf{Решение.} Построим характеристический многочлен матрицы $A$:
$$\det(A-\lambda E) = \begin{vmatrix}
	 - \lambda & 1 & 0\\
	0 & -\lambda & 1\\
	-2 & -b & -a - \lambda
\end{vmatrix} = - (\lambda^3 + a\lambda^2 + b\lambda + 2) = 0.$$
Домножим характеристическое уравнение на $-1$ и построим его Гурвициан:
$$\begin{vmatrix}
	a & 1 & 0\\
	2 & b & a\\
	0 & 0 & 2
\end{vmatrix}\quad \Delta_1 = a > 0,\ \Delta_2 = ab - 2 > 0,\ \Delta_3 = 2(ab - 2) > 0.$$
Таким образом, для асимптотической устойчивости исходного СтЛВУ необходимо, чтобы $$\begin{cases}
	a > 0,\\
	ab > 2.
\end{cases}$$
\textbf{Ответ:} $a > 0$, $ab > 2$.\\\\
\textbf{Пример 6.}  Исследовать на устойчивость и асимптотическую устойчивость уравнение вида $DX = AX$, где $$A = \begin{pmatrix}
	-1 & 2 & 0\\
	0 & -2 & 0\\
	-4 & -3 & 0
\end{pmatrix}.$$ 
\textbf{Решение.} Построим характеристический многочлен матрицы $A$:
$$\det(A - \lambda E) = \begin{pmatrix}
	-1 - \lambda & 2 & 0\\
	0 & -2 -\lambda& 0\\
	-4 & -3 & 0 - \lambda
\end{pmatrix} = -(\lambda^4 + 3\lambda^3 + 2\lambda) =0.$$
Для него построим Гурвициан:
$$\begin{vmatrix}
	3 & 1 & 0\\
	0 & 2 & 3\\
	0 & 0 & 0
\end{vmatrix},\quad \Delta_1 = 3>0,\ \Delta_2 = 6 > 0,\ \Delta_3 = 0.$$
Значит уравнение не является асимптотически устойчивым. Тогда исследуем его на устойчивость. Для этого найдем корни характеристического уравнения:
$\lambda_1 = -2$, $\lambda_2 = -1$, $\lambda_3 = 0$. Таким образом, так как кратность нулевого собственного значения равна 1, то уравнение является устойчивым по критерию устойчивости.\\\\
\textbf{Ответ:} Устойчиво.
\end{document}