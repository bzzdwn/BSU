\documentclass[a4paper, 12pt]{article}
\usepackage{cmap}
\usepackage{amssymb}
\usepackage{amsmath}
\usepackage{graphicx}
\usepackage{amsthm}
\usepackage{upgreek}
\usepackage{setspace}
\usepackage[T2A]{fontenc}
\usepackage[utf8]{inputenc}
\usepackage[normalem]{ulem}
\usepackage{mathtext} % русские буквы в формулах
\usepackage[left=2cm,right=2cm, top=2cm,bottom=2cm,bindingoffset=0cm]{geometry}
\usepackage[english,russian]{babel}
\usepackage[unicode]{hyperref}
\newenvironment{Proof} % имя окружения
{\par\noindent{}} % команды для \begin
{\hfill$\scriptstyle$}
\newcommand{\Rm}{\mathbb{R}}
\newcommand{\Cm}{\mathbb{C}}
\newcommand{\I}{\mathbb{I}}
\newcommand{\N}{\mathbb{N}}
\newtheorem*{thrm}{Теорема}
\renewcommand{\varphi}{\upvarphi}
\renewcommand{\alpha}{\upalpha}
\renewcommand{\psi}{\uppsi}
\renewcommand{\tau}{\uptau}
\renewcommand{\beta}{\upbeta}
\newcommand{\Ln}{L_n = D^n + a_{n-1}D^{n-1} + \ldots + a_1D + a_0D^0}
\begin{document}
	\section*{СтЛНУ. Метод Коши. Метод Лагранжа. Метод Эйлера.}
	Сейчас мы переходим к рассмотрению другого типа стационарных линейных уравнений, а именно, к неоднородным СтЛУ. Рассмотрим уравнение $$D^nx + a_{n-1}D^{n-1}x + \ldots + a_1Dx + a_0D^0x = f(t),\quad t\in \I\subseteq \Rm.\eqno(1)$$
	Обозначим через $L_n = D^n + a_{n-1}D^{n-1} + \ldots + a_1D + a_0$ линейный оператор дифференцирования. \\\\
	$\bullet$ \textit{Уравнение $L_nx = f(t)$ будем называть \textbf{линейным неоднородным стационарным уравнением n-ого порядка (СтЛНУ-n)}, а функцию $f(t)$ --- \textbf{неоднородностью}.}\\\\
	\textit{Общее решение неоднородного} уравнения представляет собой сумму \textit{общего решения соответствующего ему однородного} уравнения и \textit{частного решения неоднородного} уравнения. То есть $$x_\text{он}(t) = x_\text{oo}(t) + x_\text{чн}(t).$$ Находить общее решение однородного уравнения мы научились в прошлом уроке. Сейчас необходимо научиться находить частные решения неоднородного уравнения. И для нахождения частных решений существуют три метода: метод Коши, метод Лагранжа и метод Эйлера.
	\subsection*{Метод Коши.}
	Первый метод, который мы рассмотрим, называется \textbf{методом Коши}.\\\\
		$\bullet$ \textit{Пусть $\varphi_0(t), \ldots, \varphi_{n-1}(t)$ --- ФСР уравнения $L_nx = 0$, нормированная при $t_0 = 0$. Тогда функция $\varphi_{n-1}(t)$ называется \textbf{функцией Коши} линейного оператора $L_n$.}\\\\
		То есть, исходя из определения, функция Коши --- последняя функция ФСР нормированной в точке $t_0 = 0$. ФСР нормированную в точке мы уже умеем искать из прошлого урока.\\\\
		Иначе можно сформулировать это определение как\\\\
		$\bullet$ \textit{Функция, являющаяся решением задачи Коши $$\begin{cases}
				L_nx = 0,\\
				x|_{t=0} = 0,\\
				Dx|_{t=0} = 0,\\
				\dotfill\\
				D^{n-1}x|_{t=0} = 1.
			\end{cases}$$ называется \textbf{функцией Коши}.}
	\newtheorem*{2_4_1}{Теорема}\begin{2_4_1}[Метод Коши]
		Пусть функция $f(t)$ непрерывна на $\I$ и пусть $\varphi_{n-1}(t)$ --- функция Коши оператора $L_n$. Тогда функция $$x(t) = \int\limits_{t_0}^{t}\varphi_{n-1}(t-\uptau) f(\uptau)d\uptau\eqno(2)$$ является решением уравнения $L_nx = f(t)$ $\forall t_0 \in \I$.
	\end{2_4_1}
Теперь сам алгоритм нахождения частного решения уравнений по методу Коши:\begin{itemize}
	\item находим корни и строим общее решение $x_\text{oo}(t)$ соответствующего для СтЛНУ однородного уравнения;
	\item находим все производные до ($n-1$)-го порядка от $x_\text{oo}(t)$;
	\item составляем систему уравнений из найденных производных $D^ix(t)$ в точке $t = 0$, где $D^ix(t)|_{t=0} = 0\quad \forall i =\overline{0,n-2}$ и $D^{n-1}x(t)|_{t=0} = 1$;
	\item находим из составленной системы уравнений константы $C_i$ и подставляем их в $x_\text{oo}(t)$, полученная функция и будет функцией Коши ($\varphi_{n-1}(t)$);
	\item вычисляем интеграл (2) и получаем $x_\text{чн}(t)$.
\end{itemize}
Для построения общего решения всего уравнения, как говорилось ранее, применяем формулу $x_\text{он}(t) = x_\text{oo}(t) + x_\text{чн}(t).$\\\\
\textbf{Пример 1.} Найти общее решение неоднородного уравнения по методу Коши
$$D^2x + 2Dx + x = e^{2t},\quad \I = \Rm.$$
\textbf{Решение.} Запишем для нашего исходного уравнения соответствующее однородное $$D^2x + 2Dx + x = 0,$$ у которого характеристическое уравнение имеет корень $\lambda_1 = -1$, $k_1 = 2$. Cоставим общее решение однородного уравнения (теперь будем индексировать его как $x_\text{oo}(t)$ дабы различать общее решение однородного и неоднородного уравнений): $$x_\text{oo}(t) = C_1te^{-t} + C_2e^{-t}.$$
Для нахождения частного решения нам необходимо знать функцию Коши, следовательно, приступим к её нахождению (алгоритм похож на нахождение ФСР из прошлого урока, однако теперь нужно найти только $\varphi_{n-1}(t)$). Порядок уравнения $n = 2$, следовательно, нам необходимо найти только первую производную $Dx$ (т.к. в нашем случае $n-1 = 1$). Тогда $$Dx(t) = C_1e^{-t} - C_1te^{-t} - C_2e^{-t}.$$
Теперь найдем значения полученных функций $D^ix(t)$ в точке $t = 0$:
$$x(t)|_{t=0} = C_1\cdot 0\cdot e^0 + C_2\cdot e^0 = C_2;$$
$$Dx(t)|_{t=0} = C_1\cdot e^{0} - C_1\cdot 0 \cdot e^{0} - C_2\cdot e^{0} = C_1 - C_2.$$
Так как мы ищем функцию Коши, то нам нужно составить и решить следующую систему уравнений: $$\begin{cases}
	x(t)|_{t=0} = 0,\\
	Dx(t)|_{t=0} = 1;
\end{cases}\Rightarrow\begin{cases}
C_2 = 0,\\
C_1 - C_2 = 1.
\end{cases}$$ Для решения воспользуемся методом Гаусса. Составим матрицу, в которой слева у нас будут коэффициенты при постоянных $C_i$, а справа столбец, полученный из правой части построенной выше системы уравнений:
$$\begin{pmatrix}
	0 & 1 & \vline & 0\\
	1 & -1 & \vline & 1
\end{pmatrix}\sim \begin{pmatrix}
1 & 0 & \vline & 1\\
0 & 1 & \vline & 0
\end{pmatrix}.$$
Подставим полученные значения из правой части матрицы вместо коэффициентов $C_i$ в $x_\text{oo}(t)$ для нахождения функции Коши:
$$\varphi_1(t) = te^{-t}.$$
По формуле (2) мы можем построить частное решение нашего СтЛНУ.\\\\ Нижний предел $t_0$ будем считать равным 0, но можно взять произвольное значение \textbf{из промежутка} $\I$ (то есть такое $t$, в котором функция $f(t)$ будет непрерывна).\\\\ Функцию $f(\uptau) = e^{2\tau}$ возьмем из правой части исходного уравнения. Тогда по формуле (2) 
$$x_\text{чн}(t) = \int\limits_{0}^t(t - \uptau)e^{-(t-\uptau)}e^{2\uptau}d\uptau.$$
Значение этого интеграла мы можем вычислить [интегрирование по частям], и оно равно $$x_\text{чн}(t) = \dfrac{e^{2t}}{9} - \dfrac{te^{-t}}{3} - \dfrac{e^{-t}}{9}.$$
Теперь остается лишь найти общее решение всего уравнения по формуле $x_\text{он} = x_\text{oo} + x_\text{чн}$:
$$x_\text{он}(t) = (C_1te^{-t} + C_2e^{-t}) + \Big(\dfrac{e^{2t}}{9} - \dfrac{te^{-t}}{3} - \dfrac{e^{-t}}{9}\Big).$$
\textbf{Ответ:} $x_\text{он}(t) = e^{-t}(C_1t + C_2) + \dfrac{e^{2t}}{9} - \dfrac{te^{-t}}{3} - \dfrac{e^{-t}}{9}.$\\\\
\textbf{Замечание:} \textit{Проверить правильность найденного частного решения можно путем подстановки.}\\\\
\textbf{Замечание:} \textit{Если интеграл $x_\text{чн}(t)$ неберущийся, то в ответ записывается сумма общего решения и самого интеграла.}\\\\
То есть, если бы у нас получился, к примеру, $$x_\text{чн}(t) = \int\limits_{1}^{t}\dfrac{e^\tau}{\tau}d\tau,$$
то общее решение неоднородного уравнения имело бы вид
$$x_\text{он}(t) = (C_1te^{-t} + C_2e^{-t}) + \int\limits_{1}^{t}\dfrac{e^\tau}{\tau}d\tau.$$
\textbf{Пример 2.} Найти общее решение неоднородного уравнения по методу Коши
$$D^3x - 3D^2x + 3Dx - x = e^t,\quad \I = \Rm.$$
\textbf{Решение.} Составим соответствующее однородное уравнение для исходного:
$$D^3x - 3D^2x + 3Dx - x = 0.$$
Его общее решение имеет вид
$$x_\text{oo}(t) = C_1t^2e^t +C_2te^t + C_3e^t.$$
Займемся поиском частного решения, а для этого нужна функция Коши. Найдем производные 1-го и 2-го порядков:
$$Dx = C_1t^2e^t + 2C_1te^t + C_2te^t + C_2e^t + C_3 e^t,$$
$$D^2x = C_1t^2e^t + 4C_1te^t + 2C_1e^t + C_2te^t + 2C_2e^t + C_3 e^t.$$
Составим систему уравнений, чтобы найти функцию Коши
$$\begin{cases}
	x|_{t=0} = C_3 = 0,\\
	Dx|_{t=0} = C_2 + C_3 = 0,\\
	D^2x|_{t=0} = 2C_1 + 2C_2 + C_3 = 1
\end{cases}\Rightarrow \begin{cases}
C_1 = \dfrac{1}{2},\\
C_2 = 0,\\
C_3 = 0.
\end{cases}$$
Отсюда
$$\varphi_{n-1}(t) = \dfrac{t^2e^t}{2}.$$
Подставим полученную функцию в уравнение (2)
\begin{multline*}
	x_\text{чн}(t) = \int\limits_0^t\dfrac{(t-\tau)^2\cdot e^{t-\tau}}{2}\cdot e^\tau d\tau = e^t\int\limits_0^t\Big(\dfrac{t^2}{2} - t\tau + \dfrac{\tau^2}{2}\Big)d\tau =\\= \dfrac{t^2e^t}{2}\int\limits_0^td\tau - te^t\int\limits_0^t\tau d\tau + \dfrac{e^t}{2}\int\limits_0^t\tau^2d\tau = \dfrac{t^3e^t}{2} - \dfrac{t^3e^t}{2} + \dfrac{t^3e^t}{6} = \dfrac{t^3e^t}{6}.
\end{multline*}
Таким образом, сложим полученные решения общего однородного и частное неоднородного и получим
$$x_\text{он}(t) = C_1t^2e^t +C_2te^t + C_3e^t + \dfrac{t^3e^t}{6}.$$
\textbf{Ответ:} $x_\text{он}(t) = C_1t^2e^t +C_2te^t + C_3e^t + \dfrac{t^3e^t}{6}.$\\\\
Иногда нужно не только найти общее решение неоднородного уравнения, но и решить задачу Коши. Для этого применяется следующее следствие.\\\\
\textbf{Следствие} (из метода Коши)\textbf{.}
\textit{Если $\varphi_0(t),\ldots,\varphi_{n-1}(t)$ --- ФСР $L_nx = 0$, нормированная при $t=0$, то задача Коши $L_nx = f(t)$, $D^ix|_{t=t_0} = \xi_i$, $\forall i = \overline{0,n-1}$ имеет решение $$x(t) = \xi_0\varphi_0(t-t_0)+\ldots+\xi_{n-1}\varphi_{n-1}(t-t_0) + \int\limits_{t_0}^{t}\varphi_{n-1}(t-\uptau)f(\uptau)d\uptau.$$}\\
\textbf{Следствие.} \textit{Уравнение $(2)$ является решением нулевой задачи Коши при $t=t_0$.}\\\\
Стоит также учесть, что если стоит вопрос о решении задачи Коши для СтЛНУ, то для поиска решения используется только метод Коши.\\\\
Рассмотрим пример 1, но уже с задачей Коши.\\\\
\textbf{Пример 3.} Найти решение задачи Коши $x|_{t=7} = 2,\ Dx|_{t=7} = 3$ для уравнения
$$D^2x + 2Dx + x = e^{2t}, \quad \I = \Rm.$$
\textbf{Решение.} Из примера 1 возьмём общее решение однородного уравнения $$x_\text{oo}(t) = C_1te^{-t} + C_2e^{-t}$$
и найдем ФСР нормированную в точке $t_0 = 7$ (как находить ФСР нормированную в точке мы рассматривали в прошлом уроке). Для начала построим ФСР нормированную в точке $t = 0$: если
$$Dx(t) = C_1e^{-t} - C_1te^{-t} - C_2e^{-t},$$
то
$$x|_{t=0} = C_2;$$
$$Dx|_{t=0} = C_1-C_2.$$
Составим системы уравнений $$\begin{cases}
	x|_{t=0} = C_1 = 1,\\
	Dx|_{t=0} = C_1 - C_2 = 0;
\end{cases}\begin{cases}
x|_{t=0}= C_1 = 0,\\
Dx|_{t=0} = C_1 - C_2 = 1;
\end{cases}$$ и решим их методом Гаусса:$$\begin{pmatrix}
0 & 1 & \vline & 1 & 0\\
1 & -1 & \vline & 0 & 1
\end{pmatrix}\sim \begin{pmatrix}
1 & 0 & \vline &1 & 1\\
0 & 1 & \vline &1 & 0
\end{pmatrix}.$$ Теперь определим сами функции, образующие ФСР нормированную в точке $t = 0$ (последняя функция является функцией Коши):
$$\varphi_0(t) = te^{-t} + e^{-t};$$
$$\varphi_1(t) =  te^{-t}.$$
Сделаем сдвиг функций в точку $t_0 = 7$:
$$\varphi_0(t - 7) = (t-7)e^{7-t} + e^{7-t};$$
$$\varphi_1(t - 7) =  (t-7)e^{7-t}.$$
Найдем частное решение неоднородного уравнения (приняв в (2) $t_0 = 7$ по условию задачи Коши):
$$x_\text{чн}(t) = \int\limits_{7}^t(t - \uptau)e^{-(t-\uptau)}e^{2\uptau}d\uptau = \dfrac{e^{2t}}{9} + \dfrac{20e^{21-t}}{9} - \dfrac{te^{21-t}}{3}.$$
Теперь по следствию из метода Коши построим решение задачи Коши. Для этого возьмем формулу $$x(t) = \xi_0\varphi_0(t-t_0)+\ldots+\xi_{n-1}\varphi_{n-1}(t-t_0) + \int\limits_{t_0}^{t}\varphi_{n-1}(t-\uptau)f(\uptau)d\uptau,$$ где вместо $\xi_i$ подставим значения соответствующих $D^ix$ из задачи Коши, вместо $\varphi_i(t-t_0)$ подставим полученную нами ФСР нормированную в точке $t_0 = 7$. Тогда получим $$x(t) = (2\cdot (t-7)e^{7-t} + 2\cdot e^{7-t} + 3\cdot(t-7)e^{7-t}) + \Big(\dfrac{e^{2t}}{9} + \dfrac{20e^{21-t}}{9} - \dfrac{te^{21-t}}{3}\Big)=$$
$$ = (5t - 33)e^{7-t} + \Big(\dfrac{e^{2t}}{9} + \dfrac{20e^{21-t}}{9} - \dfrac{te^{21-t}}{3}\Big).$$
\textbf{Ответ:} $x(t) = (5t - 33)e^{7-t} + \dfrac{e^{2t}}{9} + \dfrac{20e^{21-t}}{9} - \dfrac{te^{21-t}}{3}$.
	\subsection*{Метод Лагранжа.}
	Следующий метод нахождения частного решения неоднорогодного СтЛУ называется \textbf{методом Лагранжа}, или \textbf{методом вариации произвольных постоянных}.
\begin{thrm}[Метод Лагранжа] Пусть $\varphi_1(t),\ldots,\varphi_n(t)$ --- ФСР $L_nx = 0$. Тогда функция $$x_\text{чн}(t) = u_1(t)\varphi_1(t) + \ldots + u_n(t)\varphi_n(t)$$ является решением уравнения $L_nx = f(t)$, $t\in \I$, если функции $u_i(t)$ удовлетворяют системе $$\begin{cases}
			Du_1\varphi_1 + \ldots + Du_n\varphi_n = 0,\\
			Du_1D\varphi_1 + \ldots + Du_nD\varphi_n = 0,\\
			\dotfill\\
			Du_1D^{n-2}\varphi_1 + \ldots + Du_nD^{n-2}\varphi_n = 0,\\
			Du_1D^{n-1}\varphi_1 + \ldots + Du_nD^{n-1}\varphi_n = f(t);
		\end{cases}\eqno(3)$$
\end{thrm}
Соответственно, из формулировки теоремы мы можем сделать определенные выводы: для нахождения частного решения уравнения мы должны \begin{itemize}
	\item найти корни и построить общее решение $x_\text{oo}(t)$ соответствующего для СтЛНУ однородного уравнения;
	\item найти ФСР соответствующего СтЛОУ;
	\item составить функцию $x_\text{чн}$;
	\item составить линейную систему уравнений (3) из соответствующих производных;
	\item найти функции $u_i(t)$ из системы уравнений;
	\item подставить $u_i(t)$ в функцию $x_\text{чн}$;
\end{itemize}
Теперь пришло время рассмотрения задач.\\\\
\textbf{Пример 4.} Применить метод Лагранжа нахождения общего решения уравнения
$$D^2x - 4Dx + 4 = 2te^{2t}, \quad \I = \Rm.$$
\textbf{Решение.} По аналогии с методом Коши найдем общее решение соответствующего однорогодно уравнения $D^2x - 4Dx + 4 = 0$: $\lambda_1 = 2$, $k_1 = 2$. Тогда $$x_\text{oo}(t) = C_1te^{2t} + C_2e^{2t}.$$
Далее находим ФСР (значения при постоянных $C_i$, то есть $\varphi_1(t) = te^{2t}$, $\varphi_2(t) = e^{2t}$) и составляем функцию, являющуюся частным решением уравнения: $$x_\text{чн}(t) = u_1(t)\cdot te^{2t} + u_2(t)\cdot e^{2t}.$$ Теперь на основе этой функции составляем систему $n$ уравнений (3), учитывая, что в последнем равенстве справа стоит функция $f(t) = 2te^{2t}$: $$\begin{cases}
	u_1'(t)\cdot te^{2t} + u_2'(t)\cdot e^{2t} = 0,\\
	u_1'(t)\cdot e^{2t} + 2u_1'(t)\cdot te^{2t} + 2u_2'(t)\cdot e^{2t} = 2te^{2t};
\end{cases}$$
Домножим первое равенство на $-2$ и прибавим ко второму. Тогда получим $$\begin{cases}
	u_1'(t)\cdot te^{2t} + u_2'(t)\cdot e^{2t} = 0,\\
	u_1'(t)\cdot e^{2t} = 2te^{2t};
\end{cases}$$
Теперь мы можем найти функцию $u_1(t)$, проинтегрировав уравнение $u_1'(t) = 2t$:
$$u_1(t)=\int2tdt = 2\int tdt = t^2.$$
Возвращаясь к нашей системе, находим $u_2(t)$: $$\begin{cases}
	u_2'(t) = -u_1'(t)t,\\
	u_1'(t) = 2t;
\end{cases}$$
$$u_2(t) = \int(-2)t^2dt = -\dfrac{2t^3}{3}.$$
Подставим полученные функции $u_i(t)$ в $x_\text{чн}(t)$. Тогда частное решение уравнения имеет вид $$x_\text{чн}(t) =  2t^3e^{2t} -\dfrac{2t^3}{3}e^{2t} = \dfrac{4t^3}{3}e^{2t}.$$
Последним действием находим общее решение неоднородного уравнения уравнения $$x_\text{он}(t) = (C_1te^{2t} + C_2e^{2t}) + \Big( \dfrac{4t^3}{3}e^{2t}\Big).$$
\textbf{Ответ:} $x_\text{он}(t) = C_1te^{2t} + C_2e^{2t} +  \dfrac{4t^3}{3}e^{2t}.$\\\\
\textbf{Замечание:} \textit{Вообще говоря, основываясь на определении неопределенного интеграла, значения функций $u_1(t)$ и $u_2(t)$ были найдены неправильно. К нашим найденным функциям необходимо добавить константы $C_1$ и $C_2$ в функции соответственно. Тогда при подстановке $u_i(t)$ в $x_\textbf{чн}(t)$ мы сразу получаем общее решение. То есть, подводя итог, в случае, когда интеграл берущийся, для правильного решения необходимо добавлять к $u_i(t)$ константы $C_i$, и решение уравнения имеет вид $$x(t) = \sum_{i = 0}^{n-1}(u_i(t) + C_i)\varphi_i(t).$$ В противном случае, если интеграл неберущийся, записываем в привычном виде $x_\text{он} = x_\text{oo} + x_\text{чн}.$ Однако для берущегося интеграла также можно использовать привычную формулу для $x_\text{он}$.}\\\\
\textbf{Пример 5.} Применить метод Лагранжа нахождения общего решения уравнения $$D^2x + x = \dfrac{1}{\cos t},\quad \I = \Big(-\dfrac{\pi}{2}; \dfrac{\pi}{2}\Big).$$
\textbf{Решение.} Корни характеристического уравнения соответствующего однородного уравнения следующие: $\lambda_1 = i$, $k_1 = 1$; $\lambda_2 = -i$, $k_2 = 1$. Составим общее решение однородного: $$x_\text{oo}(t) = {C}_1\cos t + {C}_2\sin t.$$
Тогда ФСР $\varphi_1(t) = \cos t$, $\varphi_2(t) = \sin t$ и $$x_\text{чн}(t) = u_1(t)\cdot \cos t + u_2(t)\cdot \sin t.$$
Составляем систему уравнений: $$\begin{cases}
	u_1'(t)\cdot \cos t + u_2'(t)\cdot \sin t = 0,\\
	-u_1'(t)\cdot \sin t + u_2'(t)\cdot \cos t = \dfrac{1}{\cos t};
\end{cases}$$
Выводим $u_2(t)$ из верхнего уравнения и подставляем в нижнее: $$\begin{cases}
	u_2'(t) = -u_1'(t)\cdot \dfrac{\cos t}{\sin t},\\
	u_1'(t)\cdot \sin t + u_1'(t)\cdot \dfrac{\cos^2t}{\sin t} = -\dfrac{1}{\cos t};
\end{cases}$$
$$\begin{cases}
	u_2'(t) = -u_1'(t)\cdot \dfrac{\cos t}{\sin t},\\
	u_1'(t) = -\dfrac{\sin t}{\cos t};
\end{cases}$$
$$\begin{cases}
	u_2'(t) = 1,\\
	u_1'(t) = -\dfrac{\sin t}{\cos t};
\end{cases}$$
Проинтегрируем нижнее уравнение и получим $u_2(t) = t$.
Теперь проинтегрируем нижнее уравнение: 
$$u_1(t) = \int\dfrac{d(\cos t)}{\cos t} = \ln|\cos t| = \ln(\cos t).$$
Подставим все полученные функции и получим общее решение СтЛУ $$x_\text{он}(t) = {C}_1\cos t + {C}_2\sin t + \ln(\cos t)\cdot \cos t + t\sin t.$$
\textbf{Ответ:} $x_\text{он}(t) ={C}_1\cos t + {C}_2\sin t + \ln(\cos t)\cdot \cos t + t\sin t.$
\subsection*{Метод Эйлера.}
Последним методом в этом уроке будет \textbf{метод Эйлера}. Методом Эйлера решаются СтЛНУ со специальной правой частью.\\\\
Пусть $\Delta(\lambda) = \lambda^n + a_{n-1}\lambda^{n-1} + \ldots + a_1\lambda + a_0$ --- характеристический многочлен уравнения $L_nx = 0$.\\\\
\textbf{Теорема}.
	\textit{Уравнение $$L_nx = P(t)e^{\upgamma t},$$ где $x(t)$ --- неизвестная действительная функция, $P(t)\in\Rm[t]$, $\deg P(t) = m$, $\upgamma \in \Rm$, имеет частное решение вида $$x_1(t) = t^kQ(t)e^{\upgamma t},$$ где $Q(t) \in \Rm[t], \deg Q(t) \leqslant m$, $k$ --- кратность корня $\upgamma$ характеристического многочлена $\Delta(\lambda)$.}\\\\
	Из теоремы сделаем следующий вывод: в СтЛНУ, которое мы будем решать данным методом, справа должно быть обязательно произведение многочлена, зависящего от $t$ на $e^{\upgamma t}$.\\\\
	$\bullet$ \textit{Число $\upgamma$ будем называть \textbf{контрольным числом} правой части.}\\\\
	\textbf{Пример 6.} Применить метод Эйлера для нахождения общего решения уравнения: $$D^2x - x = 2e^t - t^2.$$
	\textbf{Решение.} Как всегда построим общее решение для левой части: $\lambda_1 = 1$, $\lambda_2 = -1$.
	$$x_\text{oo}(t) = C_1e^t + C_2e^{-t}.$$
	Теперь перейдем к рассмотрению правой части равенства. Немного перепишем её, чтобы получить специальную правую часть: $$P_1(t) e^{\upgamma_1 t} + P_2(t) e^{\upgamma_2 t} = 2e^t - t^2e^{0t}.$$
	В итоге мы получили сумму двух многочленов. Обозначим их $P_1(t) = 2$ и $P_2(t) = -t^2$ с контрольными числами $\upgamma_1 = 1$ и $\upgamma_2 = 0$ соответственно.\\\\
	Вспомним, что частное решение должно иметь вид $x_1(t) = t^{k_1}Q_1(t)e^{\upgamma_1 t} + t^{k_2}Q_2(t)e^{\upgamma_2 t}$, где $k$ --- кратность корня $\upgamma$ (если он является корнем). Значит проверим, являются ли контрольные числа $\upgamma_i$ корнями СтЛОУ $\lambda_i$. Поскольку $\upgamma_1 = \lambda_1$, $k_1 = 1$, то контрольное число $\upgamma_1$ будет также иметь кратность 1. Значит первое слагаемое частного решения будет умножаться на $t^1$. Второе же слагаемое на $t^0 = 1$, потому что $\upgamma_2$ не является корнем (значит кратность контрольного числа будет 0). Таким образом частное решение имеет вид
	$$x_\text{чн}(t) = tQ_1(t)e^{t} + Q_2(t)e^{0t}$$
	Теперь нужно определить вид многочленов $Q_i(t)$. Для этого рассмотрим многочлены $P_i(t)$: $\deg P_1(t) = 0$, значит $\deg Q_1(t) = 0$, тогда можем заменить $Q_1(t) = C_1$, где $C_1$ --- какой-то постоянный коэффициент; $\deg P_2(t) = 2$, значит $\deg Q_1(t) \leqslant 2$, тогда применима замена $Q_2(t) = A_2t^2 + B_2t + C_2$, где $A_2,\ B_2,\ C_2$ --- какие-то постоянные коэффициенты.\\\\
	Теперь наше частное решение будет иметь вид $$x_\text{чн}(t) = tC_1e^t + (A_1t^2 + B_2t + C_2).$$
	Далее нужно найти $D^2x - x$ (левую часть исходного уравнения) и приравнять ее к правой части исходного уравнения. Для этого посчитаем первую и вторую производную от $x$:
	$$Dx = tC_1e^t + C_1e^t + 2A_2t + B_2;$$
	$$D^2x = tC_1e^t + 2C_1e^t + 2A_2.$$
	Теперь подставляем всё необходимое в исходное уравнение и получаем $$tC_1e^t + 2C_1e^t + 2A_2 - tC_1e^t - A_2t^2 - B_2t - C_2  = 2e^t - t^2.$$
	Сократим и получим
	$$2C_1e^t + 2A_2 - A_2t^2 - B_2t - C_2  = 2e^t - t^2.$$
	Данное равенство лучше всего решать методом неопределенных коэффициентов: $$\begin{cases}
		e^t : 2C_1 = 2 \Rightarrow C_1 = 1;\\
		t^2 : -A_2 = -1 \Rightarrow A_2 = 1;\\
		t\; : -B_2 = 0 \Rightarrow B_2 = 0\\
		t^0 : 2A_2 - C_2 = 0 \Rightarrow C_2 = 2.
	\end{cases}$$
	Полученные коэффициенты подставим в $x_\text{чн}(t)$ и получим $$x_\text{чн}(t) = te^t + (t^2 + 2).$$
	Отсюда $$x_\text{он} = C_1e^t + C_2e^{-t} + te^t + t^2 + 2.$$
	\textbf{Ответ:} $x_\text{он} = C_1e^t + C_2e^{-t} + te^t + t^2 + 2.$\\\\
	Теперь на основе решения задачи можем составить определенный алгоритм нахождения частного решения СтЛНУ методом Эйлера:\begin{itemize}
		\item находим корни и строим общее решение $x_\text{oo}(t)$ соответствующего для СтЛНУ однородного уравнения;
		\item сравниваем контрольное число (или числа, если их несколько) с корнями характеристического уравнения СтЛОУ;
		\item находим кратность контрольного значения (если значение совпало с корнем, то кратность значения равна кратности корня, иначе 0);
		\item считаем степени многочленов $P_i(t)$ и на основе этой степени составляем $Q_i(t)$ (к примеру, если $\deg P_i = 0$, то $Q_i = A_i$; если $\deg P_i = 1$, то $Q_i = A_it + B_i$; если $\deg P_i = 2$, то $Q_i = A_it^2 + B_it + C_i$);
		\item составляем $x_\text{чн}(t)$ в виде $t^kQ(t)e^{\upgamma t}$;
		\item находим находим все производные до $n$ порядка уравнения и подставляем левую часть исходного уравнения (если производные были найдены правильно, то $e^{\upgamma t}$ сократится);
		\item методом неопределенных коэффциентов находим коэффициенты $A_i$, $B_i$, $C_i$ и т.д, затем подставляем их в $Q_i(t)$ для $x_\text{чн}(t)$;
	\end{itemize}
	\textbf{Пример 7.} Применить метод Эйлера для нахождения общего решения уравнения: $$D^2x + Dx = 4t^2e^t.$$
	\textbf{Решение.} Находим корни соответствующего однородного уравнения: $\lambda_1 = 0$, $k_1 = 1$; $\lambda_2 = -1$, $k_2 = 1$. Составляем общее решение $$x_\text{oo}(t) = C_1 + C_2e^{-t}.$$
	Контрольное число правой части $\upgamma = 1$. Такого корня нет, следовательно его кратность $k = 0$. Рассмотрим многочлен $P(t) = 4t^2$: $\deg P(t) = 2$, следовательно, $Q(t) = At^2 + Bt + C$.
	Тогда $$x_\text{чн} (t)= (At^2 + Bt + C)e^t.$$
	Найдем первую и вторую производные от этой функции:
	$$Dx = (2At + B + At^2 + Bt + C)e^t;$$
	$$D^2x = (2A + 2At + B + 2At + B + At^2 + Bt + C)e^t.$$
	Подставим их в исходное уравнение и получим $$(2At^2 + (6A + 2B)t + (2A + 3B + 2C))e^t = 4t^2 e^t.$$
	 $$\begin{cases}
		t^2 : A = 2;\\
		t\; : 6A+2B = 0 \Rightarrow B = -6;\\
		t^0 : 2A + 3B + 2C = 0 \Rightarrow C = 7.
	\end{cases}$$
Подставим полученные коэффициенты в частное решение и получим $$x_\text{чн} (t)= (2t^2 - 6t + 7)e^t.$$
Тогда общее решение неоднородного уравнения имеет вид $$x_\text{он}(t) =C_1 + C_2e^{-t} +(2t^2 - 6t + 7)e^t.$$
\textbf{Ответ:} $x_\text{он}(t) =C_1 + C_2e^{-t} +(2t^2 - 6t + 7)e^t.$\\\\
Также разрешимыми по методу Эйлера являются уравнения с правой частью выраженной по формуле Эйлера. Для нахождения решений таких уравнений будем применять следующее следствие.\\\\
\textbf{Следствие.}
	\textit{Уравнение $$L_nx = e^{\alpha t}(P_1(t)\cos(\beta t) + P_2(t)\sin (\beta t)),$$ где $P_1(t),P_2(t)\in \Rm[t]$, причем $\max\{\deg P_1(t), \deg P_2(t)\} = m$, и $(\alpha + \beta i)$ --- корень многочлена $\Delta (\lambda)$ крастности $k$ имеет решение вида $$x_1(t) = t^ke^{\alpha t}(Q_1(t)\cos(\beta t) + Q_2(t)\sin(\beta t)),$$ где $Q_1(t), Q_2(t) \in \Rm[t], \deg Q_1(t) \leqslant m$, $\deg Q_2(t) \leqslant m$, $k$ --- кратность корня $(\alpha + \beta i)$ характеристического многочлена $\Delta(\lambda)$.}\\\\
	\textbf{Замечание}. \textit{Число $\alpha + \beta i$ из следствия будем называть аналогично \textbf{контрольным числом}.}\\\\
\textbf{Пример 8.} Применить метод Эйлера для нахождения общего решения уравнения: $$D^2x + x = t\sin t.$$
\textbf{Решение.} Характеристическое уравение имеет корни $\lambda_{1,2} = \pm i$, $k_{1,2} = 1$ и общее решение вида $$x_\text{oo}(t)= C_1\cos t + C_2\sin t.$$ В правой части равенства $$e^{\alpha t}(P_1(t)\cos(\beta t) + P_2(t)\sin (\beta t)) = e^{0t}(0\cdot \cos t + t\sin t).$$ Число $\alpha + \beta i = 0 + 1\cdot i$ является корнем характеристического уравнения, значит кратность контрольного числа $k = k_1 = 1$. Рассмотрим многочлены $P_i(t)$: $\max\{\deg P_1(t), \deg P_2(t)\} = \deg\{0, t\} = 1$. Следовательно, $\deg Q_1(t) = \deg Q_2(t) \leqslant 1$. Примем степени обоих многочленов за 1 и построим частное решение:
$$x_\text{чн}(t) =te^{0 t}((A_1t+B_1)\cos t + (A_2t + B_2)\sin t).$$ Переходим к самой страшной части решения: подсчету производных 1-го и 2-го порядков:
$$Dx = (B_2 - t(A_1t + B_1 - 2A_2))\sin t + (t(2A_1 + A_2t + B_2)+B_1)\cos t;$$
$$D^2x = (-4A_1t - 2B_1 - A_2t^2 + 2A_2-2B_2t)\sin t + (-A_1t^2 + 2A_1-B_1t+4A_2t+2B_2);$$
Подставим полученные функции в исходное уравнение и получим:
$$(-4A_1t - 2B_1 + 2B_2)\sin t + (2A_1 + 4A_2t + 2B_2)\cos t = t\sin t.$$
Отсюда получаем $$\begin{cases}
	A_1 = -\dfrac{1}{4},\\
	B_1 = 0,\\
	A_2 = 0,\\
	B_2 = \dfrac{1}{4;}
\end{cases}$$
Подставляем полученные коэффициенты в частное решение и получаем $$x_\text{чн}(t) = \dfrac{1}{4}(-t^2\cos t + \sin t).$$
И находим общее решение уравнения как обычно
$$x_\text{он}(t) =C_1\cos t + C_2\sin t +\dfrac{1}{4}(-t^2\cos t + \sin t).$$
\textbf{Ответ:} $x_\text{он}(t) =C_1\cos t + C_2\sin t +\dfrac{1}{4}(-t^2\cos t + \sin t).$\\\\
\textbf{Пример 9.} Применить метод Эйлера для нахождения общего решения уравнения: $$x'' - x - e^t - 4\sin^3t = 0.$$
\textbf{Решение.} Приведем уравнение к виду (1), то есть оставим слева всё, что связано с $x(t)$, а остальное перенесем в правую часть:
$$x'' - x = e^t + 4\sin^3t.$$
Найдем общее решение для соответствующего однородного уравнения. Так как корни характеристического многочлена $\lambda_1 = -1$, $\lambda_2 = 1$, то общее решение будет $$x_{\text{оо}}(t) = C_1e^{-t} + C_2 e^t.$$
Для поиска частного решения преобразуем правую уравнения. Для применения метода Эйлера нам необходимо, чтобы в правой части синус был в первой степени. Воспользуемся формулой $$\sin(3x) = 3\sin x - 4\sin^3x.$$
Тогда получим уравнение $$x'' - x = e^t + 3\sin t - \sin(3t).$$
Запишем частное решение в общем виде. Для этого разобьем частное решение на 3 части, каждая из которых соответствует одному из слагаемых справа, то есть $$x'' - x = \underbrace{e^t}_{x_{\text{чн}_1}} + \underbrace{3\sin t}_{x_{\text{чн}_2}} - \underbrace{\sin(3t)}_{x_{\text{чн}_3}}.$$ Сначала рассмотрим слагаемое $e^t$. Число $\upgamma_1 = 1$ является корнем характеристического многочлена, а $\deg$ многочлена при $e^t$ равна 1. Следовательно, $$x_{\text{чн}_1} = tA_1e^t.$$
Рассмотрим $3\sin t$. Контрольное число $\upgamma_2 = 0 + i$, и оно не является корнем характеристического уравнения. $\deg$ многочлена при этом слагаемом также равна 0, то есть в общем виде $$x_{\text{чн}_2} = A_2\sin t + A_3 \cos t.$$
Рассотрим $-\sin (3t)$. Контрольное число $\upgamma_3 = 0 + 3i$, также не является корнем характеристического уравнения, а $\deg$ многочлена при этом слагаемом также равна 1. Тогда в общем виде $$x_{\text{чн}_3} = A_4\sin (3t) + A_5 \cos (3t).$$
Сложим получившиеся части, тогда $$x_{\text{чн}} = tA_1e^t + A_2\sin t + A_3 \cos t + A_4\sin (3t) + A_5 \cos (3t).$$
Найдем вторую производную от $x_\text{чн}$:
$$x' = tA_1e^t + A_1e^t + A_2\cos t - A_3\sin t + 3A_4\cos(3t) - 3A_5\sin(3t).$$
$$x'' = tA_1e^t + 2A_1e^t - A_2\sin t - A_3\cos t - 9A_4\sin(3t) - 9A_5\cos(3t).$$
Подставим это в уравнение $x'' - x = e^t + 3\sin t - \sin(3t)$ и получим \begin{multline*}
	tA_1e^t + 2A_1e^t - A_2\sin t - A_3\cos t - 9A_4\sin(3t) - 9A_5\cos(3t) -\\- tA_1e^t - A_2\sin t - A_3 \cos t -A_4\sin (3t) - A_5 \cos (3t) = e^t + 3\sin t - \sin(3t).
\end{multline*}
Отсюда получаем $$\begin{cases}
	A_1 = \dfrac{1}{2},\\
	A_2 = -\dfrac{3}{2},\\
	A_3 = A_5 = 0,\\
	A_4 = \dfrac{1}{10}.
\end{cases}$$
Подставим эти коэффициенты в $x_\text{чн}$ и получим 
$$x_\text{чн} = \dfrac{te^t}{2} - \dfrac{3}{2}\sin t + \dfrac{\sin(3t)}{10}.$$
Таким образом, общее решение исходного уравнения имеет вид
$$x_\text{он} = C_1e^{-t} + C_2e^t + \dfrac{te^t}{2} - \dfrac{3}{2}\sin t + \dfrac{\sin(3t)}{10}.$$
\textbf{Ответ:} $x_\text{он} = C_1e^{-t} + C_2e^t + \dfrac{te^t}{2} - \dfrac{3}{2}\sin t + \dfrac{\sin(3t)}{10}.$
\\\\
Подводя итог, хотелось бы отметить, что чаще всего на практике студенты применяют методы Лагранжа и Эйлера, так как в методе Коши мы вынуждены заниматься подсчётом не всегда лёгкого интеграла. Однако вы можете использовать тот метод, который вам нравится больше.
\end{document}