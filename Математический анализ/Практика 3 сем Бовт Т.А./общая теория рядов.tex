\documentclass[a4paper, 12pt]{report}
\usepackage{cmap}
\usepackage{amssymb}
\usepackage{amsmath}
\usepackage{graphicx}
\usepackage{amsthm}
\usepackage{upgreek}
\usepackage{setspace}
\usepackage{empheq}
\usepackage{mathtools}
\setcounter{secnumdepth}{5}
\setcounter{tocdepth}{5}
\numberwithin{equation}{section}
\renewcommand{\theequation}{\arabic{equation}}
\usepackage[T2A]{fontenc}
\usepackage[utf8]{inputenc}
\usepackage[normalem]{ulem}
\usepackage{mathtext} % русские буквы в формулах
\usepackage[left=2cm,right=2cm, top=2cm,bottom=2cm,bindingoffset=0cm]{geometry}
\usepackage[english,russian]{babel}
\usepackage[unicode]{hyperref}
\newenvironment{Proof} % имя окружения
{\par\noindent{$\blacklozenge$}} % команды для \begin
{\hfill$\scriptstyle\square$}
\newcommand{\Rm}{\mathbb{R}}
\newcommand{\Cm}{\mathbb{C}}
\newcommand{\Z}{\mathbb{Z}}
\newcommand{\I}{\mathbb{I}}
\newcommand{\N}{\mathbb{N}}
\newcommand{\rank}{\operatorname{rank}}
\newcommand{\Ra}{\Rightarrow}
\newcommand{\ra}{\rightarrow}
\newcommand{\FI}{\Phi}
\newcommand{\Sp}{\text{Sp}}
\newcommand{\ol}{\overline}

\renewcommand{\leq}{\leqslant}
\renewcommand{\geq}{\geqslant}

\renewcommand{\alpha}{\upalpha}
\renewcommand{\beta}{\upbeta}
\renewcommand{\gamma}{\upgamma}
\renewcommand{\delta}{\updelta}
\renewcommand{\varphi}{\upvarphi}
\renewcommand{\phi}{\upvarphi}
\renewcommand{\tau}{\uptau}
\renewcommand{\theta}{\uptheta}
\renewcommand{\eta}{\upeta}
\renewcommand{\lambda}{\uplambda}
\renewcommand{\sigma}{\upsigma}
\renewcommand{\psi}{\uppsi}
\renewcommand{\mu}{\upmu}
\renewcommand{\omega}{\upomega}
\renewcommand{\xi}{\upxi}
\renewcommand{\epsilon}{\upvarepsilon}
\renewcommand{\rho}{\uprho}
\renewcommand{\varepsilon}{\upvarepsilon}

\renewcommand{\d}{\partial}
\renewcommand{\Re}{\operatorname{Re}}
\newcommand{\const}{\operatorname{const}}
\newcommand{\intx}{\int\limits_{x_0}^x}
\newcommand\Norm[1]{\left\| #1 \right\|}
\newcommand{\sumn}{\sum\limits_{n=1}^\infty}
\newcommand{\sumnk}{\sum\limits_{n=1}^k}
\newtheorem*{theorem}{Теорема}
\newtheorem*{cor}{Следствие}
\newtheorem*{lem}{Лемма}

\date{}
\begin{document}

\subsection{Общая идея и суть рядов}

Фундаментальная идея, лежащая в основе теории рядов, заключается в том, чтобы \textbf{представить сложный математический объект в виде бесконечной суммы очень простых объектов}. Если эта сумма <<сходится>> к исходному объекту, мы получаем огромные преимущества:
\begin{itemize}
	\item \textbf{Аппроксимация:} Мы можем вычислить значение сложного объекта (например, функции $\sin(x)$ или числа $\pi$) с любой желаемой точностью, взяв конечное число слагаемых из суммы.
	\item \textbf{Анализ:} Мы можем изучать свойства сложного объекта (непрерывность, дифференцируемость, интегрируемость), анализируя свойства простых слагаемых в его разложении.
\end{itemize}

\subsection{Числовые ряды: Основа основ}

\subsubsection{В чем суть?}
Числовой ряд — это бесконечная сумма чисел вида:
$$ \sum_{n=1}^{\infty} a_n = a_1 + a_2 + a_3 + \dots $$
Главный вопрос теории — \textbf{сходится ли} эта сумма к некоторому конечному числу, или она \textbf{расходится} (например, уходит в бесконечность).


\subsubsection{Применение}
Числовые ряды являются теоретическим фундаментом для всех остальных типов рядов. Они подобны алфавиту, на котором строится язык математического анализа.

\subsection{Функциональные ряды: Обобщение на функции}

\subsubsection{В чем суть?}
Функциональный ряд — это бесконечная сумма функций:
$$ \sum_{n=1}^{\infty} f_n(x) = f_1(x) + f_2(x) + f_3(x) + \dots $$
Здесь возникают два новых вопроса:
\begin{enumerate}
	\item Для каких значений $x$ (область сходимости) этот ряд сходится?
	\item \textbf{Как} он сходится?
\end{enumerate}

\subsubsection{Поточечная и равномерная сходимость}
\begin{description}
	\item[Поточечная сходимость] — слабая форма сходимости. Для каждого фиксированного $x$ мы получаем числовой ряд, который может сходиться. Однако <<хорошие>> свойства функций (например, непрерывность) могут не наследоваться суммой.
	\item[Равномерная сходимость] — сильная и очень важная форма. Она означает, что скорость сходимости не зависит от $x$ на всем множестве. Формально:
	$$ \lim_{n \to \infty} \sup_{x \in E} |f_n(x) - f(x)| = 0 $$
	Главное свойство: если ряд из непрерывных/дифференцируемых функций сходится равномерно, то его сумма также будет непрерывной/дифференцируемой, и ряд можно \textbf{почленно дифференцировать и интегрировать}.
\end{description}

\subsubsection{Применение}
\begin{itemize}
	\item \textbf{Ряды Фурье:} Представление периодических функций (сигналов, колебаний) в виде суммы синусов и косинусов. Это основа всей цифровой обработки сигналов.
	\item \textbf{Решение дифференциальных уравнений:} Поиск решений в виде функциональных рядов.
\end{itemize}

\subsection{Степенные ряды: Лучшие из функциональных}

\subsubsection{В чем суть?}
Степенной ряд — это функциональный ряд специального, наиболее удобного вида, <<бесконечный многочлен>>:
$$ \sum_{n=0}^{\infty} c_n (x-a)^n = c_0 + c_1(x-a) + c_2(x-a)^2 + \dots $$
Они наследуют все удобные свойства многочленов: их легко дифференцировать, интегрировать и вычислять.

\subsubsection{Ряды Тейлора и Маклорена}
Это вершина теории. Любую достаточно гладкую функцию можно разложить в степенной ряд (ряд Тейлора) в окрестности точки $a$, где коэффициенты вычисляются через производные:
$$ c_n = \frac{f^{(n)}(a)}{n!} $$
Примеры знаменитых разложений в ряд Маклорена (ряд Тейлора при $a=0$):
\begin{align*}
	e^x &= \sum_{n=0}^{\infty} \frac{x^n}{n!} = 1 + x + \frac{x^2}{2!} + \frac{x^3}{3!} + \dots \\
	\sin(x) &= \sum_{n=0}^{\infty} (-1)^n \frac{x^{2n+1}}{(2n+1)!} = x - \frac{x^3}{3!} + \frac{x^5}{5!} - \dots \\
	\cos(x) &= \sum_{n=0}^{\infty} (-1)^n \frac{x^{2n}}{(2n)!} = 1 - \frac{x^2}{2!} + \frac{x^4}{4!} - \dots
\end{align*}

\subsubsection{Применение в прикладных задачах}
\begin{enumerate}
	\item \textbf{Вычисления и аппроксимация:} Калькуляторы и компьютеры вычисляют значения трансцендентных функций ($\sin, \cos, \ln$) с помощью нескольких первых членов их рядов Тейлора.
	
	\item \textbf{Интегрирование:} <<Неберущиеся>> интегралы, например, интеграл Пуассона из теории вероятностей, вычисляются путем разложения подынтегральной функции в ряд и его почленного интегрирования.
	$$ \int_{0}^{x} e^{-t^2} dt = \int_{0}^{x} \left( \sum_{n=0}^{\infty} \frac{(-t^2)^n}{n!} \right) dt = \sum_{n=0}^{\infty} \frac{(-1)^n x^{2n+1}}{n! (2n+1)} $$
	
	\item \textbf{Решение дифференциальных уравнений:} Многие уравнения в физике и технике решаются путем поиска решения в виде степенного ряда.
	
	\item \textbf{Физика и инженерия:}
	\begin{itemize}
		\item \textbf{Линеаризация:} Сложные нелинейные зависимости при малых отклонениях заменяются первыми членами ряда Тейлора, что сильно упрощает анализ систем.
		\item \textbf{Теория относительности:} Разложение релятивистской энергии в ряд по степеням $(v/c)^2$ дает классическую кинетическую энергию как первое приближение:
		$$ E = \frac{mc^2}{\sqrt{1-v^2/c^2}} \approx mc^2 + \frac{1}{2}mv^2 $$
	\end{itemize}
	
	\item \textbf{Компьютерные науки:} Ряды Фурье используются в алгоритмах сжатия изображений (JPEG) и звука (MP3), отбрасывая <<менее важные>> члены ряда.
\end{enumerate}

\subsection{Итог}
Суть всех рядов — \textbf{разбиение сложного на бесконечное число простого}. Это одна из самых мощных и плодотворных идей во всей современной науке, позволяющая решать задачи, которые были бы недоступны другими методами.
\subsection{В чем суть рядов Фурье? Аналогия с призмой}

Представьте себе сложный луч белого света. Сам по себе он просто <<белый>>. Но когда вы пропускаете его через призму, он раскладывается на свои фундаментальные компоненты — чистые цвета радуги. Вы видите, \textit{из чего состоит} белый свет.

\begin{center}
	\textbf{Ряды Фурье — это математическая <<призма>> для функций и сигналов.}
\end{center}

\textbf{Суть:} Любой сложный периодический сигнал (звуковая волна, электрическое колебание, температурный цикл) можно представить как \textbf{сумму простых синусоидальных волн} (синусов и косинусов) разной частоты и амплитуды.

\begin{itemize}
	\item \textbf{Основная частота (фундаментальная гармоника):} Это <<главная>> волна, определяющая основной период сигнала.
	\item \textbf{Обертоны (высшие гармоники):} Это волны с частотами, кратными основной (в 2, 3, 4 раза выше и т.д.). Их амплитуды и фазы определяют уникальную <<форму>> или <<тембр>> сложного сигнала.
\end{itemize}

Математически разложение в тригонометрический ряд Фурье для функции $f(x)$ с периодом $2\pi$ выглядит так:
\begin{equation*}
	f(x) = \frac{a_0}{2} + \sum_{n=1}^{\infty} \left( a_n \cos(nx) + b_n \sin(nx) \right)
\end{equation*}
Здесь коэффициенты $a_n$ и $b_n$ показывают <<вес>> или <<амплитуду>> каждой гармоники в общем сигнале. Процесс их нахождения называется \textbf{анализом Фурье}.

\subsection{Главное преимущество: Переход в частотную область}

Ряды Фурье позволяют полностью сменить точку зрения. Вместо того чтобы анализировать сигнал как функцию, зависящую от времени или пространства (во \textbf{временной/пространственной области}), мы начинаем смотреть на него как на \textbf{спектр} — набор частот и их мощностей (в \textbf{частотной области}).

Этот сдвиг парадигмы фундаментален, поскольку многие задачи, которые чрезвычайно сложны во временной области, становятся тривиальными в частотной.

\subsection{Применение в задачах и дисциплинах}

\subsubsection{Цифровая обработка сигналов}
\begin{description}
	\item[Проблема:] Как сжать аудиофайл (MP3) или изображение (JPEG) без видимой потери качества?
	\item[Роль Фурье:] Сигнал (звук или изображение) раскладывается на частотный спектр. Алгоритм сжатия \textbf{отбрасывает} коэффициенты, отвечающие за частоты, которые человек плохо воспринимает (очень высокие или тихие звуки, мелкие и неконтрастные детали на фото). Оставшихся данных становится значительно меньше, что и приводит к сжатию файла.
\end{description}

\subsubsection{Физика и Инженерия}
\begin{description}
	\item[Проблема (Акустика):] Почему скрипка и рояль, играющие одну и ту же ноту (440 Гц), звучат по-разному?
	\item[Роль Фурье:] Основная частота у них одинакова, но набор и амплитуды обертонов (880 Гц, 1320 Гц и т.д.) совершенно разные. Анализ Фурье показывает этот уникальный <<частотный отпечаток>> или \textbf{тембр} инструмента.
	
	\item[Проблема (Анализ вибраций):] Как спроектировать мост, чтобы он не разрушился от резонанса?
	\item[Роль Фурье:] У любой конструкции есть собственные (резонансные) частоты. Анализ Фурье позволяет инженерам вычислить эти опасные частоты и спроектировать конструкцию так, чтобы внешние воздействия (ветер, шаги людей) не совпадали с ними.
\end{description}

\subsubsection{Электротехника и Электроника}
\begin{description}
	\item[Проблема:] Как работает эквалайзер в музыкальной системе?
	\item[Роль Фурье:] Эквалайзер — это набор частотных фильтров. Он получает аудиосигнал, неявно раскладывает его на частоты, и вы ползунками можете усилить или ослабить определенные частотные диапазоны (басы, средние, высокие).
	
	\item[Проблема:] Как очистить полезный сигнал от шума?
	\item[Роль Фурье (Фильтрация):] Если полезный сигнал и шум находятся в разных частотных диапазонах (например, низкочастотный голос и высокочастотное шипение), можно применить <<фильтр низких частот>>. В частотной области это означает просто обнуление всех коэффициентов Фурье выше определенной частоты.
\end{description}

\subsubsection{Математика}
\begin{description}
	\item[Проблема:] Как решать сложные дифференциальные уравнения в частных производных (уравнения теплопроводности, волновое уравнение)?
	\item[Роль Фурье:] Метод Фурье (метод разделения переменных) позволяет свести одно сложное уравнение к бесконечному набору более простых обыкновенных дифференциальных уравнений для каждой гармоники в разложении.
\end{description}

\section{Общая идея и суть}

Обычный определенный интеграл $\int_{a}^{b} f(x) dx$ вычисляется на \textbf{конечном} отрезке $[a,b]$, где функция $f(x)$ \textbf{непрерывна}. Несобственные интегралы — это обобщение этого понятия на два <<проблемных>> случая:
\begin{enumerate}
	\item Когда отрезок интегрирования \textbf{бесконечен} (например, $[a, +\infty)$).
	\item Когда функция $f(x)$ имеет \textbf{вертикальную асимптоту} (разрыв второго рода) внутри отрезка.
\end{enumerate}
Главный вопрос, на который отвечает теория: можно ли такой <<бесконечной>> площади приписать \textbf{конечное} числовое значение? Если да, то интеграл \textbf{сходится}, если нет — \textbf{расходится}.

\section{Несобственные интегралы I рода (по бесконечному промежутку)}
\subsection{В чем суть?}
Здесь мы пытаемся найти площадь под кривой на бесконечно длинном <<куске>> оси. Это похоже на попытку сложить бесконечный числовой ряд.

\textbf{Определение:} Интеграл от функции $f(x)$ на промежутке $[a, +\infty)$ определяется через предел:
\begin{equation*}
	\int_{a}^{+\infty} f(x) dx = \lim_{b \to +\infty} \int_{a}^{b} f(x) dx
\end{equation*}

\subsection{Где используются?}
\begin{description}
	\item[Теория вероятностей:] Плотность распределения вероятностей $p(x)$ для непрерывной случайной величины должна быть нормирована на единицу. Это означает, что полная вероятность найти частицу где-либо на всей числовой оси равна 1. Математически это записывается как несобственный интеграл I рода:
	$$ \int_{-\infty}^{+\infty} p(x) dx = 1 $$
	
	\item[Физика (Электростатика и Гравитация):] Чтобы вычислить работу, необходимую для перемещения тела из точки $a$ в <<бесконечность>> из гравитационного поля, необходимо вычислить интеграл от силы по бесконечному промежутку:
	$$ A = \int_{a}^{+\infty} F(r) dr = \int_{a}^{+\infty} \frac{GmM}{r^2} dr $$
	Если этот интеграл сходится, то работа конечна (именно поэтому существует вторая космическая скорость).
\end{description}

\section{Несобственные интегралы II рода (от разрывных функций)}
\subsection{В чем суть?}
Здесь мы пытаемся найти площадь фигуры, которая <<уходит в бесконечность>> не в длину, а в высоту. То есть функция имеет вертикальную асимптоту.

\textbf{Определение:} Если функция $f(x)$ имеет разрыв в точке $b$, то интеграл определяется через предел:
\begin{equation*}
	\int_{a}^{b} f(x) dx = \lim_{\varepsilon \to 0^+} \int_{a}^{b-\varepsilon} f(x) dx
\end{equation*}

\subsection{Где используются?}
\begin{description}
	\item[Физика (Потенциальная энергия):] Потенциал $U(r)$ или напряженность поля $E(r)$, создаваемые точечным зарядом или точечной массой, обращаются в бесконечность при $r \to 0$. Однако многие физические величины, связанные с ними (например, энергия поля в некотором объеме), могут быть конечными и вычисляются с помощью несобственных интегралов II рода. Например, расчет потенциала:
	$$ \varphi = \int_{r}^{\infty} E(x) dx = \int_{r}^{\infty} \frac{kQ}{x^2} dx $$
	Когда $r \to 0$, мы имеем дело с несобственным интегралом.
	
	\item[Геометрия и Механика:] Вычисление массы или моментов инерции для объектов с плотностью, которая неограниченно возрастает в некоторой точке.
\end{description}

\section{Несобственные интегралы, зависящие от параметра}
\subsection{В чем суть?}
Это вершина теории. Здесь под знаком несобственного интеграла стоит функция, которая зависит не только от переменной интегрирования $x$, но и от некоторого параметра $\alpha$. В результате сам интеграл становится функцией этого параметра:
$$ I(\alpha) = \int_{a}^{b} f(x, \alpha) dx $$
Основная идея — изучать свойства функции $I(\alpha)$ (непрерывность, дифференцируемость), а также использовать мощный прием — \textbf{дифференцирование по параметру под знаком интеграла} (правило Лейбница).

\subsection{Где используются?}
Это один из самых мощных инструментов математического анализа, породивший целые разделы науки.
\begin{description}
	\item[Специальные функции (Гамма- и Бета-функции):] Многие важные функции, обобщающие понятия факториала и др., определяются через такие интегралы.
	\begin{itemize}
		\item \textbf{Гамма-функция Эйлера:} Обобщение факториала на комплексные числа.
		$$ \Gamma(z) = \int_{0}^{+\infty} t^{z-1} e^{-t} dt $$
	\end{itemize}
	
	\item[Интегральные преобразования (Фурье, Лаплас):] Это основа основ для решения дифференциальных уравнений в электротехнике, теории управления, обработке сигналов.
	\begin{itemize}
		\item \textbf{Преобразование Лапласа:} Превращает сложные дифференциальные уравнения в простые алгебраические.
		$$ F(s) = \mathcal{L}\{f(t)\} = \int_{0}^{+\infty} e^{-st} f(t) dt $$
	\end{itemize}
	
	\item[Вычисление сложных интегралов:] Метод дифференцирования по параметру позволяет элегантно вычислять интегралы, которые не берутся стандартными методами.
\end{description}

\section*{Итог: Краткая таблица-шпаргалка}

\begin{center}
	\renewcommand{\arraystretch}{1.5}
	\begin{tabular}{| p{3cm} | p{5.5cm} | p{5.5cm} |}
		\hline
		\textbf{Тип интеграла} & \textbf{Суть (простыми словами)} & \textbf{Ключевые приложения} \\
		\hline \hline
		\textbf{I рода} (бесконечный предел) & Площадь фигуры бесконечной длины. Сходится, если кривая достаточно быстро <<прижимается>> к оси. & Теория вероятностей (нормировка), физика (работа по перемещению на бесконечность), астрономия (космические скорости). \\
		\hline
		\textbf{II рода} (разрывная функция) & Площадь фигуры бесконечной высоты. Сходится, если <<ширина>> пика у асимптоты сужается достаточно быстро. & Физика (потенциалы и поля вблизи точечных источников), механика (центры масс для тел с неограниченной плотностью). \\
		\hline
		\textbf{Зависящие от параметра} & Интеграл как функция. Позволяет <<менять>> подынтегральную функцию, изучая, как меняется результат. & Определение спецфункций ($\Gamma$-функция), решение дифф. уравнений (Преобразование Лапласа), обработка сигналов (Преобразование Фурье). \\
		\hline
	\end{tabular}
\end{center}

\end{document}