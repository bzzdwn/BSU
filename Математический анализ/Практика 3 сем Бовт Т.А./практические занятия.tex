\documentclass[a4paper, 12pt]{report}
\usepackage{cmap}
\usepackage{amssymb}
\usepackage{amsmath}
\usepackage{graphicx}
\usepackage{amsthm}
\usepackage{upgreek}
\usepackage{setspace}
\usepackage{empheq}
\usepackage{mathtools}
\setcounter{secnumdepth}{5}
\setcounter{tocdepth}{5}
\numberwithin{equation}{section}
\renewcommand{\theequation}{\arabic{equation}}
\usepackage[T2A]{fontenc}
\usepackage[utf8]{inputenc}
\usepackage[normalem]{ulem}
\usepackage{mathtext} % русские буквы в формулах
\usepackage[left=2cm,right=2cm, top=2cm,bottom=2cm,bindingoffset=0cm]{geometry}
\usepackage[english,russian]{babel}
\usepackage[unicode]{hyperref}
\newenvironment{Proof} % имя окружения
{\par\noindent{$\blacklozenge$}} % команды для \begin
{\hfill$\scriptstyle\square$}
\newcommand{\Rm}{\mathbb{R}}
\newcommand{\Cm}{\mathbb{C}}
\newcommand{\Z}{\mathbb{Z}}
\newcommand{\I}{\mathbb{I}}
\newcommand{\N}{\mathbb{N}}
\newcommand{\rank}{\operatorname{rank}}
\newcommand{\Ra}{\Rightarrow}
\newcommand{\ra}{\rightarrow}
\newcommand{\FI}{\Phi}
\newcommand{\Sp}{\text{Sp}}
\newcommand{\ol}{\overline}

\renewcommand{\leq}{\leqslant}
\renewcommand{\geq}{\geqslant}

\renewcommand{\alpha}{\upalpha}
\renewcommand{\beta}{\upbeta}
\renewcommand{\gamma}{\upgamma}
\renewcommand{\delta}{\updelta}
\renewcommand{\varphi}{\upvarphi}
\renewcommand{\phi}{\upvarphi}
\renewcommand{\tau}{\uptau}
\renewcommand{\theta}{\uptheta}
\renewcommand{\eta}{\upeta}
\renewcommand{\lambda}{\uplambda}
\renewcommand{\sigma}{\upsigma}
\renewcommand{\psi}{\uppsi}
\renewcommand{\mu}{\upmu}
\renewcommand{\omega}{\upomega}
\renewcommand{\xi}{\upxi}
\renewcommand{\epsilon}{\upvarepsilon}
\renewcommand{\rho}{\uprho}
\renewcommand{\varepsilon}{\upvarepsilon}

\renewcommand{\d}{\partial}
\renewcommand{\Re}{\operatorname{Re}}
\newcommand{\const}{\operatorname{const}}
\newcommand{\intx}{\int\limits_{x_0}^x}
\newcommand\Norm[1]{\left\| #1 \right\|}
\newcommand{\sumn}{\sum\limits_{n=1}^\infty}
\newcommand{\sumnk}{\sum\limits_{n=1}^k}
\newtheorem*{theorem}{Теорема}
\newtheorem*{cor}{Следствие}
\newtheorem*{lem}{Лемма}

\date{}
\begin{document}
	\newpage
	\section{Сходимость и расходимость числовых рядов}
	Пусть задача числовая последовательность $a_1, a_2,\ldots, a_n,\ldots \in \mathbb R$. 
	Сумма членов этой последовательности записывается в виде
	\begin{equation}
		a_1 + a_2 + \ldots + a_n + \ldots = \sumn a_n
	\end{equation}
	и называется \textbf{числовым рядом}.
	Сумма членов последовательности до $k$ называется \textbf{частной суммой ряда}
	\begin{equation}
		S_k = a_1 + a_2 + \ldots + a_k = \sumnk a_n.
	\end{equation}
	Рассмотрим последовательность частных сумм ряда
	\begin{equation*}
		S_1 = a_1,\ S_2 = a_1 + a_2,\ \ldots,\ S_k = a_1 + a_2 + \ldots + a_k,\ \ldots.
	\end{equation*}
	Если существует конечный предел последовательности частных сумм ряда
	\begin{equation*}
		\lim\limits_{k\to \infty}S_k = S,
	\end{equation*}
	то ряд называется \textbf{сходящимся}, иначе \textbf{расходящимся}.
	\begin{theorem}
		[Необходимое условие сходимости]
		Если числовой ряд $\sumn a_n$ сходится, то $$a_n \xrightarrow[n\to \infty]{}0.$$
	\end{theorem}
	\noindent\textbf{Следствие.} Если $a_n \not\xrightarrow[n\to \infty]{}0$, то числовой ряд $\sumn a_n$ расходится.
	\begin{theorem}
		[Критерий сходимости положительных числовых рядов ($a_n > 0$)]
		Числовой ряд $\sumn a_n$, $a_n > 0$ сходится тогда и только тогда, когда последовательность частных сумм $(S_n)_{n=1}^\infty$ ограничена, то есть $|S_n| \leq M$, $\forall n$, $M = \const$.
	\end{theorem}
	\begin{theorem}
		[Признаки сравнения]
		Пусть $a_n \geq 0$, $a_n \leq b_n$, $\forall n \in \mathbb N$. Тогда
		\begin{enumerate}
			\item Из сходимости $\sumn b_n$ следует сходимость $\sumn a_n$;
			\item Из расходимости $\sumn a_n$ следует расходимость $\sumn b_n$.
		\end{enumerate}
	\end{theorem}
	\begin{theorem}
		[Предельный признак сравнения]
		Пусть $b_n > 0$. Если \begin{equation}
			\lim\limits_{n\to\infty} \dfrac{a_n}{b_n} = l,\ 0 < l < +\infty,
		\end{equation}
		то $\sumn a_n$ и $\sumn b_n$ сходятся или расходятся одновременно.
	\end{theorem}
	\section*{Практика}
	\textbf{2546.}
	$$1 - \frac{1}{2} + \frac{1}{4} - \frac{1}{8} + \dots + \frac{(-1)^{n-1}}{2^{n-1}} + \dots$$ Перед нами бесконечно убывающая геометрическая прогрессия. Следовательно, используя формулу для такой прогрессии, сразу найдем сумму ряда
	$$\sum_{n=1}^{\infty} \frac{(-1)^{n-1}}{2^{n-1}} = \left[ \frac{b_1}{1-q} \right] = \frac{1}{1 - \left(-\frac{1}{2}\right)} = \frac{1}{\frac{3}{2}} = \frac{2}{3}.$$
	\textbf{2547.}
	$$\left(\frac{1}{2} + \frac{1}{3}\right) + \left(\frac{1}{2^2} + \frac{1}{3^2}\right) + \dots + \left(\frac{1}{2^n} + \frac{1}{3^n}\right) + \dots$$ 
	В данном случае имеем сумму двух бесконечно убывающих геометрических прогрессии:
	$$\sum_{n=1}^{\infty} \frac{1}{2^n} + \sum_{n=1}^{\infty} \frac{1}{3^n} = \left[ \frac{b_1}{1-q} \right] = \frac{\frac{1}{2}}{1-\frac{1}{2}} + \frac{\frac{1}{3}}{1-\frac{1}{3}} = 1 + \frac{1}{2} = \frac{3}{2}.$$
	\textbf{2549.}$$
	\frac{1}{1 \cdot 2} + \frac{1}{2 \cdot 3} + \frac{1}{3 \cdot 4} + \dots + \frac{1}{n(n+1)} + \dots = \sum_{n=1}^{\infty} \frac{1}{n(n+1)}
	$$
	Разложим $n$-ый член последовательности $a_n$ на простейшие дроби:
	$$\dfrac{A}{n} + \dfrac{B}{n+1} = \left[ A(n+1) + B \cdot n = 1 \Rightarrow \begin{cases} A+B=0 \\ A=1 \end{cases} \Rightarrow \begin{cases} A=1 \\ B=-1 \end{cases} \right] = \frac{1}{n} - \frac{1}{n+1}.
	$$ 
	$$\sum_{n=1}^{\infty} \left( \frac{1}{n} - \frac{1}{n+1} \right) = \left(1 - \frac{1}{2}\right)+\left(\frac{1}{2}-\frac{1}{3}\right)+\dots+\left(\frac{1}{n}-\frac{1}{n+1}\right) + \dots.
	$$ 
	Тогда рассмотрим частную сумму ряда
	$$
	S_k = \sum_{n=1}^{k} \left( \frac{1}{n} - \frac{1}{n+1} \right) = 1 - \frac{1}{k+1} \xrightarrow[k \to \infty]{} 1.
	$$ 
	То есть мы доказали, что существует конечный предел последовательности частных сумм. А значит по определению ряд сходится и его сумма равна 
	$$\sum_{n=1}^{\infty} \frac{1}{n(n+1)} = 1.$$
	\textbf{2550.}
	Обозначим исследуемый ряд
	$$\dfrac{1}{1\cdot 4} + \dfrac{1}{4\cdot 7} + \ldots + \frac{1}{(3n-2)(3n+1)} + \ldots = \sum_{n=1}^{\infty} \frac{1}{(3n-2)(3n+1)}.$$
	Разложим $a_n$ на сумму простейших дробей
	$$ \frac{A}{3n-2}+\frac{B}{3n+1} = \left[ \begin{gathered} A(3n+1) + B(3n-2) = 1 \\ \begin{cases} 3A+3B=0 \\ A-2B=1 \end{cases} \Rightarrow \begin{cases} A+B=0 \\ 3B=-1 \\ \end{cases} \Rightarrow \begin{cases} A=\frac{1}{3} \\ B=-\frac{1}{3} \end{cases} \end{gathered} \right] = \frac{1}{3(3n-2)} - \frac{1}{3(3n+1)} $$
	Тогда рассмотрим частную сумму ряда
	$$ S_k = \sum_{n=1}^{k} \left( \frac{1}{3(3n-2)} - \frac{1}{3(3n+1)} \right) = \left( \frac{1}{3} - \frac{1}{12} \right) + \left( \frac{1}{12} - \frac{1}{21} \right) + \dots $$
	$$ \dots + \left( \frac{1}{3(3k-2)} - \frac{1}{3(3k+1)} \right) = \frac{1}{3} - \frac{1}{3(3k+1)} \xrightarrow[k \to \infty]{} \frac{1}{3}.$$
	Последовательность часть ных сумм имеет конечный предел, следовательно ряд сходится его сумма равна
	$$ \sum_{n=1}^{\infty} \frac{1}{(3n-2)(3n+1)} = \frac{1}{3}.$$
	\textbf{2552.}
	$$ \sum_{n=1}^{\infty} (\sqrt{n+2} - 2\sqrt{n+1} + \sqrt{n})$$ 
	Распишем сумму по членам
	$$(\sqrt{3} - 2\sqrt{2} + 1) + (\sqrt{4} - 2\sqrt{3} + \sqrt{2}) + (\sqrt{5} - 2\sqrt{4} + \sqrt{3}) + (\sqrt{6} - 2\sqrt{5} + \sqrt{4}) + \dots $$ $$\ldots + (\sqrt{n+2} - 2\sqrt{n+1} + \sqrt{n}) + \dots $$
	Таким образом, частную сумму ряда можно записать в виде
	$$ S_k = 1 - \sqrt{2} + \sqrt{k+2} - \sqrt{k+1} = 1 - \sqrt{2} + \frac{1}{\sqrt{k+2} + \sqrt{k+1}} \xrightarrow[k \to \infty]{} 1-\sqrt{2}.$$
	Таким образом,
	$$ \sum_{n=1}^{\infty} (\sqrt{n+2} - 2\sqrt{n+1} + \sqrt{n}) = 1 - \sqrt{2}.$$
	\textbf{2556.}
	$$ 1-1+1-1+\dots = \sum_{n=1}^{\infty} (-1)^{n-1} $$
	$$ \lim_{n \to \infty} a_n = \lim_{n \to \infty} (-1)^{n-1} = \begin{cases} 1 \\ -1 \end{cases} \not\to 0, \text{предела не существ.} $$
	Тогда по необходимому условию сходимости числовой ряд расходится.
	\\\\
	\textbf{2557.}
	$$ 0.001 + \sqrt{0.001} + \sqrt[3]{0.001} + \dots = \sum_{n=1}^{\infty} (0.001)^{\frac{1}{n}}.$$
	Рассмотрим $n$-ый член последовательности
	$$ a_n = (0.001)^{\frac{1}{n}} \xrightarrow{n \to \infty} 1 \ne 0 \Rightarrow \text{расходится по необходимому условию сходимости.} $$
	\textbf{2558.}
	$$ \frac{1}{1!} + \frac{1}{2!} + \dots = \sum_{n=1}^{\infty} \frac{1}{n!} $$
	Рассмотрим частную сумму ряда
	$$ S_k = \frac{1}{1!} + \frac{1}{2!} + \dots + \frac{1}{k!} = \sum_{n=1}^{k} \frac{1}{n!}.$$
	Так как
	$$ S_{k+1} = S_k + \frac{1}{(k+1)!} \ge S_k,$$
	то последовательность монотонно возрастает. Нужно найти верхнюю грань.
	Из неравенства $$ n! \ge 2^{n-1} $$ следует, что
	$$ \frac{1}{k!} \le \frac{1}{2^{k-1}} \quad \left(n-1, \text{чтобы } \frac{1}{1!} \le \frac{1}{2^0}\right).$$
	Следовательно,
	$$ \sum_{k=1}^{n} \frac{1}{k!} \le \sum_{k=1}^{n} \frac{1}{2^{k-1}} = 1 + \frac{1}{2} + \dots + \frac{1}{2^{n-1}} = [\text{геом. прогр. }] = $$
	$$ = \left[ \frac{b_1(1-q^n)}{1-q} \right] = \frac{1 \cdot (1-\frac{1}{2^n})}{1-\frac{1}{2}} = 2 - \frac{1}{2^{n-1}} \le 2 \quad \forall n \Rightarrow |S_n| \le 2 \quad \forall n. $$
	Тогда по критерию сходимости исходный ряд сходится, так как последовательность частных сумм является ограниченной.
	\\\\
	\textbf{2559.}
	$$ 1 + \frac{1}{3} + \frac{1}{5} + \frac{1}{7} + \dots + \frac{1}{2n-1} + \dots = \sum_{n=1}^{\infty} \frac{1}{2n-1} $$
	Используем признак сравнения:
	$$ \sum_{n=1}^{\infty} \frac{1}{2n-1} \ge \sum_{n=1}^{\infty} \frac{1}{2n} = \frac{1}{2} \sum_{n=1}^{\infty} \frac{1}{n} $$
	Необходимо доказать расходимость гармонического ряда.
	Доказательство Орема:
	$$ \sum \frac{1}{n} = 1 + \left[\frac{1}{2}\right] + \left[\frac{1}{3} + \frac{1}{4}\right] + \left[\frac{1}{5} + \frac{1}{6} + \frac{1}{7} + \frac{1}{8}\right] + \dots $$
	$$ > 1 + \left[\frac{1}{2}\right] + \left[\frac{1}{4} + \frac{1}{4}\right] + \left[\frac{1}{8} + \frac{1}{8} + \frac{1}{8} + \frac{1}{8}\right] + \left[\frac{1}{16} + \dots\right] + \dots = $$
	$$ = 1 + \frac{1}{2} + \frac{1}{2} + \dots \text{ - не ограничена сверху } \Rightarrow \sum \frac{1}{n} \text{ - расх. }.$$
	Значит по признаку сравнения исходный ряд расходится.
	\\\\
	Альтернатива: Из книги Кастрицы:
	$$ \begin{gathered} S_{2n} = \sum_{k=1}^{2n} \frac{1}{k} = S_n + \frac{1}{n+1} + \dots + \frac{1}{2n} > S_n + n \cdot \frac{1}{2n} = S_n + \frac{1}{2} \end{gathered} ,$$
	тогда при n $\to \infty$
	$$ S \ge S + \frac{1}{2} \quad - \text{противоречие}.$$
	\textbf{2560.}
	$$ \frac{1}{1001} + \frac{1}{2001} + \dots + \frac{1}{1000n+1} + \dots = \sum_{n=1}^{\infty} \frac{1}{1000n+1}.$$
	Поскольку
	$$ 1000n+1 \le 2000n \Rightarrow \frac{1}{1000n+1} \ge \frac{1}{2000n},$$
	то можем рассмотреть ряд
	$$ \sum_{n=1}^{\infty} \frac{1}{2000n} = \frac{1}{2000} \sum_{n=1}^{\infty} \frac{1}{n}.$$ Этот ряд расходится как гармонический. Следовательно по признаку сравнения исходный ряд расходится.
	\\\\
	\textbf{2561.}
	$$ 1 + \frac{2}{3} + \frac{3}{5} + \dots + \frac{n}{2n-1} + \dots = \sum_{n=1}^{\infty} \frac{n}{2n-1}.$$
	Рассмотрим $a_n$ член
	$$ \frac{n}{2n-1} = \frac{1}{2-\frac{1}{n}} \xrightarrow[n \to \infty]{} \frac{1}{2} \ne 0.$$ А тогда по необходимому условию данный ряд расходится.
	\\\\
	\textbf{2562.}
	$$ 1 + \frac{1}{3^2} +\dfrac{1}{5^2} + \dots + \frac{1}{(2n-1)^2} + \dots = \sum_{n=1}^{\infty} \frac{1}{(2n-1)^2} $$
	Возьмем ряд $$ \sum_{n=1}^{\infty} \frac{1}{n^2} $$ 
	Рассмотрим отношение:
	$$ \frac{n^2}{(2n-1)^2} = \frac{1}{4-\frac{4}{n}+\frac{1}{n^2}} \xrightarrow[n \to \infty]{} \frac{1}{4} = \text{const} $$
	А тогда по предельному признаку сравнения ряды $ \sum\limits_{n=1}^{\infty} \dfrac{1}{(2n-1)^2} $ и $ \sum\limits_{n=1}^{\infty} \dfrac{1}{n^2} $ сходятся или расходятся одновременно.
	Докажем сходимость $ \sum\limits_{n=1}^{\infty} \dfrac{1}{n^2} $. Для этого рассмотрим ряд $ \sum\limits_{n=2}^{\infty} \dfrac{1}{n(n-1)} $
	По аналогии с номером 2549
	$$ \frac{1}{n(n-1)} = \left[ \begin{gathered} A(n-1) + Bn = 1 \\ \begin{cases} A+B=0 \\ -A=1 \end{cases} \Rightarrow \begin{cases} A=-1 \\ B=1 \end{cases} \end{gathered} \right] = \frac{1}{n-1} - \frac{1}{n} \Rightarrow $$
	$$ S_k = \sum_{n=2}^{k} \left( \frac{1}{n-1} - \frac{1}{n} \right) = \left( \frac{1}{1} - \frac{1}{2} \right) + \left( \frac{1}{2} - \frac{1}{3} \right) + \dots + \left( \frac{1}{k-1} - \frac{1}{k} \right) = $$
	$$ = 1 - \frac{1}{k} \xrightarrow[k \to \infty]{} 1 \Rightarrow \text{ряд сходится} $$
	При этом
	$$ \frac{1}{n(n-1)} > \frac{1}{n^2} \text{, т.к. } n(n-1) < n^2 \quad \forall n \ge 2 $$
	а тогда ряд
	$$ \sum_{n=2}^{\infty} \frac{1}{n^2} \text{ сходится.} $$
	Причем, если добавить к нему единицу, то на сходимость это не повлияет:
	$$ 1 + \sum_{n=2}^{\infty} \frac{1}{n^2} = \sum_{n=1}^{\infty} \frac{1}{n^2}.$$
	Таким образом, исходный ряд сходится по предельному признаку сравнения.
	\\\\
	\textbf{2563.}
	$$ \frac{1}{1\sqrt{2}} + \frac{1}{2\sqrt{3}} + \frac{1}{3\sqrt{4}} + \dots + \frac{1}{n\sqrt{n+1}} + \dots = \sum_{n=1}^{\infty} \frac{1}{n\sqrt{n+1}}.$$
	Рассмотрим ряд
	$$ \sum_{n=1}^{\infty} \left( \frac{1}{\sqrt{n}} - \frac{1}{\sqrt{n+1}} \right) $$
	По аналогии с 2549
	$$ S_k = \left(\frac{1}{1} - \frac{1}{\sqrt{2}}\right) + \left(\frac{1}{\sqrt{2}} - \frac{1}{\sqrt{3}}\right) + \dots + \left(\frac{1}{\sqrt{k}} - \frac{1}{\sqrt{k+1}}\right) \xrightarrow[k \to \infty]{} 1 \Rightarrow \text{сходится.} $$
	Теперь сравним его с исходным рядом.
	Сперва преобразуем:
	$$ \frac{1}{\sqrt{n}} - \frac{1}{\sqrt{n+1}} = \frac{\sqrt{n+1}-\sqrt{n}}{\sqrt{n}\sqrt{n+1}} = \frac{n+1-n}{\sqrt{n}\sqrt{n+1}(\sqrt{n+1}+\sqrt{n})} = $$
	$$ = \frac{1}{\sqrt{n}\sqrt{n+1}(\sqrt{n+1}+\sqrt{n})} $$
	Разделим исходный ряд на получившийся:
	$$ \frac{\sqrt{n}\sqrt{n+1}(\sqrt{n+1}+\sqrt{n})}{n\sqrt{n+1}} = \sqrt{1+\frac{1}{n}} + 1 \xrightarrow[n \to \infty]{} 2 = \text{const} $$
	Таким образом, по предельному признаку сравнения исходный ряд сходится.
	\section{Признаки Коши и Даламбера}
	\begin{theorem}
		[Радикальный признак Коши]
		Пусть для ряда $\sum_{n=1}^{\infty} a_n$ существует предел $$L = \lim_{n \to \infty} \sqrt[n]{|a_n|}.$$
		\begin{itemize}
			\item Если $L < 1$, то ряд сходится абсолютно.
			\item Если $L > 1$, то ряд расходится.
			\item Если $L = 1$, то признак не дает ответа.
		\end{itemize}
	\end{theorem}
	\begin{theorem}
		[Признак Даламбера]
		Пусть для ряда $\sum_{n=1}^{\infty} a_n$ с ненулевыми членами существует предел $$L = \lim_{n \to \infty} \left| \frac{a_{n+1}}{a_n} \right|.$$
		\begin{itemize}
			\item Если $L < 1$, то ряд сходится абсолютно.
			\item Если $L > 1$, то ряд расходится.
			\item Если $L = 1$, то признак не даёт ответа.
		\end{itemize}
	\end{theorem}
	\section*{Практика}
	\textbf{2578.}
	$$\sum_{n=1}^{\infty} \frac{1000^n}{n!} $$
	Для исследования сходимости данного ряда воспользуемся признаком Даламбера.Обозначим общий член ряда как $$a_n = \frac{1000^n}{n!}.$$
	{Тогда следующий член ряда будет:}
	$$a_{n+1} = \frac{1000^{n+1}}{(n+1)!}.$$
	{Вычислим предел отношения следующего члена к предыдущему:}
	$$\frac{a_{n+1}}{a_n} = \frac{\frac{1000^{n+1}}{(n+1)!}}{\frac{1000^n}{n!}} =  \frac{1000^{n+1}}{(n+1)!} \cdot \frac{n!}{1000^n}  =  \frac{1000^n \cdot 1000}{(n+1) \cdot n!} \cdot \frac{n!}{1000^n} = \frac{1000}{n+1} \xrightarrow[n\to\infty]{} 0. $$
	{Так как полученный предел } $q = 0$, \text{ и } $q < 1$, {то, согласно признаку Даламбера, исходный ряд сходится.}
	\\\\
	\textbf{2579.}
	$$\sum_{n=1}^{\infty} \frac{(n!)^2}{(2n)!}.$$
	Применим признак Даламбера.
	Общий член ряда $$a_n = \frac{(n!)^2}{(2n)!}.$$
	Тогда следующий член ряда:
	$$ a_{n+1} = \frac{((n+1)!)^2}{(2(n+1))!} = \frac{((n+1)!)^2}{(2n+2)!} $$
	Найдём предел отношения $\frac{u_{n+1}}{u_n}$:
	$$ q = \lim_{n \to \infty} \frac{u_{n+1}}{u_n} = \lim_{n \to \infty} \frac{\frac{((n+1)!)^2}{(2n+2)!}}{\frac{(n!)^2}{(2n)!}} = \lim_{n \to \infty} \left( \frac{((n+1)!)^2}{(2n+2)!} \cdot \frac{(2n)!}{(n!)^2} \right) = $$
	Используя свойства факториала $(n+1)! = (n+1) \cdot n!$ и $(2n+2)! = (2n+2)(2n+1)(2n)!$, упростим выражение:
	$$ = \lim_{n \to \infty} \frac{((n+1) \cdot n!)^2}{(2n+2)(2n+1)(2n)!} \cdot \frac{(2n)!}{(n!)^2} = \lim_{n \to \infty} \frac{(n+1)^2 (n!)^2}{(2n+2)(2n+1)(2n)!} \cdot \frac{(2n)!}{(n!)^2} = $$
	После сокращения $(n!)^2$ и $(2n)!$ получаем:
	$$ = \lim_{n \to \infty} \frac{(n+1)^2}{(2n+2)(2n+1)} = \lim_{n \to \infty} \frac{n^2 + 2n + 1}{4n^2 + 6n + 2} = $$
	Разделим числитель и знаменатель на старшую степень $n^2$:
	$$ = \lim_{n \to \infty} \frac{1 + \frac{2}{n} + \frac{1}{n^2}}{4 + \frac{6}{n} + \frac{2}{n^2}} = \frac{1+0+0}{4+0+0} = \frac{1}{4}. $$
	Так как предел $q = \frac{1}{4} < 1$, то по признаку Даламбера исходный ряд сходится.
	\\\\
	\textbf{2580.}
	$$\sum_{n=1}^{\infty} \frac{n!}{n^n}.$$
	Применим признак Даламбера.
	Общий член ряда $$a_n = \frac{n!}{n^n}$$
	Тогда следующий член ряда:
	$$ a_{n+1} = \frac{(n+1)!}{(n+1)^{n+1}} $$
	Найдём предел отношения $\frac{a_{n+1}}{a_n}$:
	$$ q = \lim_{n \to \infty} \frac{a_{n+1}}{a_n} = \lim_{n \to \infty} \frac{\frac{(n+1)!}{(n+1)^{n+1}}}{\frac{n!}{n^n}} = \lim_{n \to \infty} \left( \frac{(n+1)!}{(n+1)^{n+1}} \cdot \frac{n^n}{n!} \right) = $$
	Используя свойства факториала $(n+1)! = (n+1) \cdot n!$ и степеней $(n+1)^{n+1} = (n+1) \cdot (n+1)^n$, упростим выражение:
	$$ = \lim_{n \to \infty} \left( \frac{(n+1) \cdot n!}{(n+1) \cdot (n+1)^n} \cdot \frac{n^n}{n!} \right) = $$
	После сокращения $(n+1)$ и $n!$ получаем:
	$$ = \lim_{n \to \infty} \frac{n^n}{(n+1)^n} = \lim_{n \to \infty} \left(\frac{n}{n+1}\right)^n = \lim_{n \to \infty} \left(\frac{1}{\frac{n+1}{n}}\right)^n = $$
	$$ = \lim_{n \to \infty} \frac{1}{\left(1+\frac{1}{n}\right)^n} = \frac{1}{\lim_{n \to \infty} \left(1+\frac{1}{n}\right)^n} = \frac{1}{e}. $$
	Здесь мы использовали второй замечательный предел: $\lim_{n \to \infty} \left(1+\frac{1}{n}\right)^n = e$.
	Так как предел $q = \frac{1}{e} < 1$ (поскольку $e \approx 2.718$), то по признаку Даламбера исходный ряд сходится.
	\\\\
	\textbf{2582.}
	$$\sum_{n=1}^{\infty} \frac{(n!)^2}{2^{n^2}}.$$
	Применим признак Даламбера.
	Общий член ряда $a_n = \frac{(n!)^2}{2^{n^2}}$.
	Тогда следующий член ряда:
	$$ a_{n+1} = \frac{((n+1)!)^2}{2^{(n+1)^2}} $$
	Найдём предел отношения $\frac{a_{n+1}}{a_n}$:
	$$ q = \lim_{n \to \infty} \frac{a_{n+1}}{a_n} = \lim_{n \to \infty} \frac{\frac{((n+1)!)^2}{2^{(n+1)^2}}}{\frac{(n!)^2}{2^{n^2}}} = \lim_{n \to \infty} \left( \frac{((n+1)!)^2}{2^{(n+1)^2}} \cdot \frac{2^{n^2}}{(n!)^2} \right) = $$
	Сгруппируем члены с факториалами и степенями:
	$$ = \lim_{n \to \infty} \left( \frac{((n+1)!)^2}{(n!)^2} \cdot \frac{2^{n^2}}{2^{(n+1)^2}} \right) = \lim_{n \to \infty} \left( \left(\frac{(n+1) \cdot n!}{n!}\right)^2 \cdot 2^{n^2 - (n+1)^2} \right) = $$
	Упростим выражение в показателе степени: $n^2 - (n+1)^2 = n^2 - (n^2 + 2n + 1) = -2n - 1$.
	$$ = \lim_{n \to \infty} \left( (n+1)^2 \cdot 2^{-2n-1} \right) = \lim_{n \to \infty} \frac{(n+1)^2}{2^{2n+1}} = 0. $$
	Предел равен нулю, так как экспоненциальная функция в знаменателе $2^{2n+1}$ растет быстрее любой степенной функции в числителе $(n+1)^2$.
	Так как предел $q = 0 < 1$, то по признаку Даламбера исходный ряд сходится.
	\\\\
	\textbf{2583.}
	$$ \frac{1000}{1} + \frac{1000 \cdot 1001}{1 \cdot 3} + \frac{1000 \cdot 1001 \cdot 1002}{1 \cdot 3 \cdot 5} + \dots $$
	Сначала запишем общий член ряда $a_n$.
	Числитель $n$-го члена представляет собой произведение $n$ чисел, начиная с 1000: $1000 \cdot 1001 \cdot \dots \cdot (1000 + n - 1)$.
	Знаменатель $n$-го члена представляет собой произведение первых $n$ нечётных чисел: $1 \cdot 3 \cdot 5 \cdot \dots \cdot (2n-1)$.
	Таким образом, общий член ряда имеет вид:
	$$ a_n = \frac{1000 \cdot 1001 \cdot \dots \cdot (1000 + n - 1)}{1 \cdot 3 \cdot 5 \cdot \dots \cdot (2n - 1)} $$
	Применим признак Даламбера. Для этого запишем следующий член ряда $a_{n+1}$:
	$$ a_{n+1} = \frac{1000 \cdot 1001 \cdot \dots \cdot (1000 + n - 1) \cdot (1000 + n)}{1 \cdot 3 \cdot 5 \cdot \dots \cdot (2n - 1) \cdot (2n + 1)} $$
	Найдём предел отношения $\frac{a_{n+1}}{a_n}$:
	$$ L = \lim_{n \to \infty} \frac{a_{n+1}}{a_n} = \lim_{n \to \infty} \frac{\frac{1000 \cdot \dots \cdot (1000 + n)}{1 \cdot \dots \cdot (2n + 1)}}{\frac{1000 \cdot \dots \cdot (1000 + n - 1)}{1 \cdot \dots \cdot (2n - 1)}} = $$
	Большинство множителей в числителе и знаменателе сокращаются:
	$$ = \lim_{n \to \infty} \frac{1000+n}{2n+1} = $$
	Чтобы найти предел, разделим числитель и знаменатель на $n$:
	$$ = \lim_{n \to \infty} \frac{\frac{1000}{n} + 1}{2 + \frac{1}{n}} = \frac{0 + 1}{2 + 0} = \frac{1}{2}. $$
	Так как предел $q = \frac{1}{2} < 1$, то по признаку Даламбера исходный ряд сходится.
	\\\\
	\textbf{2584.}
	$$\sum_{n=1}^{\infty} \frac{4 \cdot 7 \cdot 10 \cdots (3n+1)}{2 \cdot 6 \cdot 10 \cdots (4n-2)}$$
	Для исследования сходимости данного ряда воспользуемся признаком Даламбера. Обозначим общий член ряда как
	$$a_n = \frac{4 \cdot 7 \cdot 10 \cdots (3n+1)}{2 \cdot 6 \cdot 10 \cdots (4n-2)}.$$
	Тогда следующий член ряда будет:
	$$a_{n+1} = \frac{4 \cdot 7 \cdot 10 \cdots (3n+1)(3n+4)}{2 \cdot 6 \cdot 10 \cdots (4n-2)(4n+2)}.$$
	Вычислим предел отношения следующего члена к предыдущему:
	$$ \lim_{n \to \infty} \frac{a_{n+1}}{a_n} = \lim_{n \to \infty} \frac{\frac{4 \cdot 7 \cdots (3n+1)(3n+4)}{2 \cdot 6 \cdots (4n-2)(4n+2)}}{\frac{4 \cdot 7 \cdots (3n+1)}{2 \cdot 6 \cdots (4n-2)}} = \lim_{n \to \infty} \frac{3n+4}{4n+2} = \lim_{n \to \infty} \frac{3 + 4/n}{4 + 2/n} = \frac{3}{4}. $$
	Так как полученный предел $q = \frac{3}{4}$, и $q < 1$, то, согласно признаку Даламбера, исходный ряд сходится.
	\\\\
	\textbf{2585.}
	$$\sum_{n=1}^{\infty} (\sqrt{2} - \sqrt[3]{2})(\sqrt{2} - \sqrt[5]{2}) \cdots (\sqrt{2} - \sqrt[2n+1]{2})$$
	Для исследования сходимости данного ряда воспользуемся признаком Даламбера. Обозначим общий член ряда как
	$$a_n = (\sqrt{2} - \sqrt[3]{2})(\sqrt{2} - \sqrt[5]{2}) \cdots (\sqrt{2} - \sqrt[2n+1]{2}).$$
	Тогда следующий член ряда будет:
	$$a_{n+1} = (\sqrt{2} - \sqrt[3]{2}) \cdots (\sqrt{2} - \sqrt[2n+1]{2}) (\sqrt{2} - \sqrt[2n+3]{2}).$$
	Вычислим предел отношения следующего члена к предыдущему:
	$$ \lim_{n \to \infty} \frac{a_{n+1}}{a_n} = \lim_{n \to \infty} \frac{(\sqrt{2} - \sqrt[3]{2}) \cdots (\sqrt{2} - \sqrt[2n+1]{2}) (\sqrt{2} - \sqrt[2n+3]{2})}{(\sqrt{2} - \sqrt[3]{2}) \cdots (\sqrt{2} - \sqrt[2n+1]{2})} = \lim_{n \to \infty} (\sqrt{2} - \sqrt[2n+3]{2}). $$
	Поскольку $\lim_{n \to \infty} \sqrt[2n+3]{2} = \lim_{n \to \infty} 2^{\frac{1}{2n+3}} = 2^0 = 1$, предел равен:
	$$ \lim_{n \to \infty} (\sqrt{2} - \sqrt[2n+3]{2}) = \sqrt{2} - 1. $$
	Так как полученный предел $q = \sqrt{2} - 1 \approx 0.414$, и $q < 1$, то, согласно признаку Даламбера, исходный ряд сходится.
	\\\\
	\textbf{2586.}
	$$\sum_{n=1}^{\infty} \frac{n^2}{\left(2+\frac{1}{n}\right)^n}$$
	Для исследования сходимости данного ряда воспользуемся радикальным признаком Коши. Обозначим общий член ряда как
	$$a_n = \frac{n^2}{\left(2+\frac{1}{n}\right)^n}.$$
	Вычислим предел корня n-ой степени из общего члена ряда:
	$$ \lim_{n \to \infty} \sqrt[n]{a_n} = \lim_{n \to \infty} \sqrt[n]{\frac{n^2}{\left(2+\frac{1}{n}\right)^n}} = \lim_{n \to \infty} \frac{\sqrt[n]{n^2}}{2+\frac{1}{n}} = \lim_{n \to \infty} \frac{(\sqrt[n]{n})^2}{2+\frac{1}{n}}. $$
	Поскольку $\lim_{n \to \infty} \sqrt[n]{n} = 1$, предел равен:
	$$ \frac{1^2}{2+0} = \frac{1}{2}. $$
	Так как полученный предел $q = \frac{1}{2}$, и $q < 1$, то, согласно признаку Коши, исходный ряд сходится.
	\\\\
	\textbf{Доказательство.}
	$$\lim_{n \to \infty} \sqrt[n]{n} = 1$$
	Рассмотрим предел $L = \lim_{n \to \infty} n^{\frac{1}{n}}$.
	Для его вычисления воспользуемся свойством непрерывности логарифмической и показательной функций. Прологарифмируем выражение под знаком предела:
	$$ \ln\left(n^{\frac{1}{n}}\right) = \frac{1}{n} \ln(n) = \frac{\ln(n)}{n}. $$
	Теперь найдем предел этого выражения при $n \to \infty$. Мы имеем неопределенность вида $\left[\frac{\infty}{\infty}\right]$, поэтому можем применить правило Лопиталя.
	$$ \lim_{n \to \infty} \frac{\ln(n)}{n} = \lim_{n \to \infty} \frac{(\ln(n))'}{(n)'} = \lim_{n \to \infty} \frac{\frac{1}{n}}{1} = \lim_{n \to \infty} \frac{1}{n} = 0. $$
	Мы нашли, что предел логарифма исходного выражения равен 0. Чтобы найти исходный предел $L$, мы потенцируем полученный результат:
	$$ L = e^{\lim_{n \to \infty} \ln(n^{\frac{1}{n}})} = e^0 = 1. $$
	Таким образом, доказано, что
	$$ \lim_{n \to \infty} \sqrt[n]{n} = 1. $$
	\textbf{2587.}
	$$\sum_{n=1}^{\infty} \frac{n^{n+1/n}}{\left(n+\frac{1}{n}\right)^n}$$
	Для исследования сходимости данного ряда воспользуемся радикальным признаком Коши. Обозначим общий член ряда как
	$$a_n = \frac{n^{n+1/n}}{\left(n+\frac{1}{n}\right)^n}.$$
	Вычислим предел корня n-ой степени из общего члена ряда:
	$$ \lim_{n \to \infty} \sqrt[n]{a_n} = \lim_{n \to \infty} \sqrt[n]{\frac{n^{n+1/n}}{\left(n+\frac{1}{n}\right)^n}} = \lim_{n \to \infty} \frac{n^{(n+1/n) \cdot 1/n}}{n+\frac{1}{n}} = \lim_{n \to \infty} \frac{n^{1+1/n^2}}{n+\frac{1}{n}}. $$
	Разделив числитель и знаменатель на $n$, получим:
	$$ \lim_{n \to \infty} \frac{n \cdot n^{1/n^2}}{n(1+\frac{1}{n^2})} = \lim_{n \to \infty} \frac{\sqrt[n^2]{n}}{1+\frac{1}{n^2}} = \frac{1}{1} = 1. $$
	Так как полученный предел $q = 1$, то признак Коши не даёт ответа.
	Проверим выполнение необходимого признака сходимости: $\lim_{n \to \infty} a_n = 0$. Преобразуем общий член ряда:
	$$ a_n = \frac{n^n \cdot n^{1/n}}{\left(n\left(1+\frac{1}{n^2}\right)\right)^n} = \frac{n^n \cdot \sqrt[n]{n}}{n^n \left(1+\frac{1}{n^2}\right)^n} = \frac{\sqrt[n]{n}}{\left(1+\frac{1}{n^2}\right)^n}. $$
	Найдем его предел:
	$$ \lim_{n \to \infty} a_n = \frac{\lim_{n \to \infty} \sqrt[n]{n}}{\lim_{n \to \infty} \left(1+\frac{1}{n^2}\right)^n} = \frac{1}{\lim_{n \to \infty} \left[\left(1+\frac{1}{n^2}\right)^{n^2}\right]^{1/n}} = \frac{1}{e^0} = 1. $$
	Так как предел общего члена ряда не равен нулю ($\lim_{n \to \infty} a_n = 1 \neq 0$), необходимый признак сходимости не выполняется, следовательно, ряд расходится.
	\\\\
	\textbf{2588.}
	$$\sum_{n=2}^{\infty} \frac{1}{\sqrt[n]{\ln n}}$$
	Для доказательства расходимости ряда проверим выполнение необходимого признака сходимости. Обозначим общий член ряда как
	$$a_n = \frac{1}{\sqrt[n]{\ln n}}.$$
	Вычислим предел общего члена ряда при $n \to \infty$:
	$$ \lim_{n \to \infty} a_n = \lim_{n \to \infty} \frac{1}{(\ln n)^{1/n}}. $$
	Рассмотрим предел знаменателя $L = \lim_{n \to \infty} (\ln n)^{1/n}$. Это неопределенность вида $\left[\infty^0\right]$. Прологарифмируем его:
	$$ \lim_{n \to \infty} \ln\left((\ln n)^{1/n}\right) = \lim_{n \to \infty} \frac{\ln(\ln n)}{n}. $$
	Применяя правило Лопиталя для неопределенности вида $\left[\frac{\infty}{\infty}\right]$:
	$$ \lim_{n \to \infty} \frac{(\ln(\ln n))'}{(n)'} = \lim_{n \to \infty} \frac{\frac{1}{n \ln n}}{1} = 0. $$
	Следовательно, предел знаменателя равен $L = e^0 = 1$. Тогда предел общего члена ряда:
	$$ \lim_{n \to \infty} a_n = \frac{1}{1} = 1. $$
	Так как предел общего члена ряда не равен нулю ($\lim_{n \to \infty} a_n = 1 \neq 0$), необходимый признак сходимости не выполняется, следовательно, ряд расходится.
	\\\\
	\textbf{2589.}
	$$\sum_{n=1}^{\infty} \frac{n^{n-1}}{(2n^2+n+1)^{n+1/2}}$$
	Для исследования сходимости данного ряда воспользуемся радикальным признаком Коши. Обозначим общий член ряда как
	$$a_n = \frac{n^{n-1}}{(2n^2+n+1)^{n+1/2}}.$$
	Вычислим предел корня n-ой степени из общего члена ряда:
	$$ \lim_{n \to \infty} \sqrt[n]{a_n} = \lim_{n \to \infty} \sqrt[n]{\frac{n^{n-1}}{(2n^2+n+1)^{n+1/2}}} = \lim_{n \to \infty} \frac{n^{(n-1)/n}}{(2n^2+n+1)^{(n+1/2)/n}}. $$
	Упростим степени в числителе и знаменателе:
	$$ \lim_{n \to \infty} \frac{n^{1-1/n}}{(2n^2+n+1)^{1+1/(2n)}} = \lim_{n \to \infty} \frac{n \cdot n^{-1/n}}{(2n^2+n+1) \cdot (2n^2+n+1)^{1/(2n)}}. $$
	Разделим предел на произведение нескольких пределов:
	$$ \lim_{n \to \infty} \left(\frac{n}{2n^2+n+1}\right) \cdot \lim_{n \to \infty} \frac{1}{n^{1/n}} \cdot \lim_{n \to \infty} \frac{1}{(2n^2+n+1)^{1/(2n)}}. $$
	Вычислим каждый из них по отдельности:
	$$ \lim_{n \to \infty} \frac{n}{2n^2+n+1} = 0. $$
	$$ \lim_{n \to \infty} \frac{1}{\sqrt[n]{n}} = \frac{1}{1} = 1. $$
	$$ \lim_{n \to \infty} \frac{1}{(2n^2+n+1)^{1/(2n)}} = 1, \text{ поскольку это предел вида } \frac{1}{\infty^0} \text{ и } \lim_{n \to \infty} (\text{полином})^{1/n} = 1. $$
	Предел всего произведения равен:
	$$ q = 0 \cdot 1 \cdot 1 = 0. $$
	Так как полученный предел $q = 0$, и $q < 1$, то, согласно признаку Коши, исходный ряд сходится.
	\\\\
	\textbf{2589.1.}
	$$\sum_{n=1}^{\infty} \frac{n^5}{2^n + 3^n}$$
	Для исследования сходимости воспользуемся признаком сравнения. Оценим общий член ряда $a_n = \frac{n^5}{2^n + 3^n}$.
	Так как $2^n + 3^n > 3^n$, то
	$$ \frac{n^5}{2^n + 3^n} < \frac{n^5}{3^n}. $$
	Исследуем на сходимость ряд $\sum_{n=1}^{\infty} \frac{n^5}{3^n}$ с помощью радикального признака Коши.
	$$ q = \lim_{n \to \infty} \sqrt[n]{\frac{n^5}{3^n}} = \lim_{n \to \infty} \frac{\sqrt[n]{n^5}}{\sqrt[n]{3^n}} = \lim_{n \to \infty} \frac{(\sqrt[n]{n})^5}{3}. $$
	Поскольку $\lim_{n \to \infty} \sqrt[n]{n} = 1$, предел равен:
	$$ q = \frac{1^5}{3} = \frac{1}{3}. $$
	Так как $q < 1$, ряд $\sum_{n=1}^{\infty} \frac{n^5}{3^n}$ сходится. Следовательно, по признаку сравнения, исходный ряд $\sum_{n=1}^{\infty} \frac{n^5}{2^n + 3^n}$ также сходится.
	\\\\
	\textbf{2589.2.}
	$$\sum_{n=2}^{\infty} \left(\frac{n-1}{n+1}\right)^{n(n-1)}$$
	Для исследования сходимости воспользуемся радикальным признаком Коши. Обозначим общий член ряда как
	$$a_n = \left(\frac{n-1}{n+1}\right)^{n(n-1)}.$$
	Вычислим предел корня n-ой степени из общего члена ряда:
	$$ q = \lim_{n \to \infty} \sqrt[n]{a_n} = \lim_{n \to \infty} \sqrt[n]{\left(\frac{n-1}{n+1}\right)^{n(n-1)}} = \lim_{n \to \infty} \left(\frac{n-1}{n+1}\right)^{n-1}. $$
	Мы имеем неопределенность вида $\left[1^{\infty}\right]$. Преобразуем выражение, чтобы использовать второй замечательный предел:
	$$ \lim_{n \to \infty} \left(\frac{n+1-2}{n+1}\right)^{n-1} = \lim_{n \to \infty} \left(1-\frac{2}{n+1}\right)^{n-1} = \lim_{n \to \infty} \left(1+\frac{-2}{n+1}\right)^{(n+1)\frac{n-1}{n+1}}. $$
	Поскольку $\lim_{n \to \infty} \left(1+\frac{-2}{n+1}\right)^{n+1} = e^{-2}$ и $\lim_{n \to \infty} \frac{n-1}{n+1} = 1$, предел равен:
	$$ q = (e^{-2})^1 = e^{-2} = \frac{1}{e^2}. $$
	Так как полученный предел $q = \frac{1}{e^2} \approx \frac{1}{7.389} < 1$, то, согласно признаку Коши, исходный ряд сходится.
	\\\\
	\textbf{2595.}
	$$\sum_{n=1}^{\infty} \frac{2+(-1)^n}{2^n}$$
	Для исследования сходимости воспользуемся признаком сравнения. Оценим числитель общего члена $a_n = \frac{2+(-1)^n}{2^n}$.
	Так как $(-1)^n$ принимает значения $1$ (для четных $n$) и $-1$ (для нечетных $n$), то
	$$ 1 \le 2+(-1)^n \le 3. $$
	Следовательно, для общего члена ряда справедливо неравенство:
	$$ a_n = \frac{2+(-1)^n}{2^n} \le \frac{3}{2^n}. $$
	Рассмотрим ряд $\sum_{n=1}^{\infty} \frac{3}{2^n} = 3\sum_{n=1}^{\infty} \left(\frac{1}{2}\right)^n$. Это сходящийся геометрический ряд, так как его знаменатель $q = \frac{1}{2} < 1$.
	Поскольку члены исходного ряда меньше членов сходящегося ряда, то, согласно признаку сравнения, исходный ряд сходится.
	\\\\
	\textbf{2596.}
	$$\sum_{n=1}^{\infty} \frac{a \cos^2(n\pi/3)}{2^n}$$
	Для исследования сходимости воспользуемся признаком сравнения. Оценим общий член ряда $a_n = \frac{a \cos^2(n\pi/3)}{2^n}$.
	Так как функция косинуса ограничена, для ее квадрата справедливо неравенство:
	$$ 0 \le \cos^2(n\pi/3) \le 1. $$
	Тогда для модуля общего члена ряда (при $a \neq 0$) имеем:
	$$ |a_n| = \left|\frac{a \cos^2(n\pi/3)}{2^n}\right| = \frac{|a| \cos^2(n\pi/3)}{2^n} \le \frac{|a|}{2^n}. $$
	Рассмотрим ряд $\sum_{n=1}^{\infty} \frac{|a|}{2^n} = |a|\sum_{n=1}^{\infty} \left(\frac{1}{2}\right)^n$. Этот ряд является сходящимся геометрическим рядом со знаменателем $q = \frac{1}{2} < 1$.
	Поскольку исходный ряд сходится абсолютно (его члены по модулю меньше членов сходящегося ряда), он сходится.
	\section{Интегральный критерий Коши, степенной признак, тейлоровские разложения}
	Рассмотрим числовой ряд $\sumn a_n$, $a_n \geq 0$.
	Пусть функция $f(x)$ определена на $[1;+\infty)$ и $f(n) = a_n$, $n=1,2,\ldots$.
	\begin{theorem}
		[Интегральный критерий Коши]
		Если функция $f(x)$ положительна и убывает, то ряд $\sum_{k=1}^{\infty} a_k$ сходится тогда и только тогда, когда существует конечный предел
		\begin{equation}
			\lim_{A \to +\infty} \int_{1}^{A} f(x)dx.
		\end{equation}
	\end{theorem}
	\noindent
	Принято обозначать $\lim_{A \to +\infty} \int_{1}^{A} f(x)dx = \int_{1}^{+\infty} f(x)dx$. Интеграл $\int_{1}^{+\infty} f(x)dx$ называют \textbf{несобственным}. Если упомянутый предел конечен, то несобственный интеграл называют \textbf{сходящимся}. В противном случае интеграл \textbf{расходится}.
	\\\\
	\textbf{Задача.}
	$$\sum_{n=1}^{\infty} \frac{1}{n^\alpha}$$
	Для исследования сходимости данного обобщенного гармонического ряда воспользуемся интегральным признаком Коши.
	Рассмотрим функцию $f(x) = \frac{1}{x^\alpha}$. При $x \ge 1$ и $\alpha > 0$ эта функция является положительной, непрерывной и монотонно убывающей.
	Исследуем на сходимость несобственный интеграл:
	$$ \int_{1}^{+\infty} \frac{dx}{x^\alpha} = \lim_{A \to +\infty} \int_{1}^{A} x^{-\alpha} dx. $$
	Рассмотрим два случая.
	
	1. Пусть $\alpha \neq 1$.
	$$ \lim_{A \to +\infty} \frac{x^{1-\alpha}}{1-\alpha} \Big|_{1}^{A} = \lim_{A \to +\infty} \left( \frac{A^{1-\alpha}}{1-\alpha} - \frac{1}{1-\alpha} \right) = \frac{1}{1-\alpha} \lim_{A \to +\infty} (A^{1-\alpha} - 1). $$
	Этот предел конечен тогда и только тогда, когда $1-\alpha < 0$, то есть $\alpha > 1$. В этом случае $\lim_{A \to +\infty} A^{1-\alpha} = 0$, и интеграл равен $\frac{1}{\alpha-1}$.
	Если $\alpha < 1$, то $1-\alpha > 0$, и предел равен $+\infty$.
	
	2. Пусть $\alpha = 1$.
	$$ \int_{1}^{+\infty} \frac{dx}{x} = \lim_{A \to +\infty} \int_{1}^{A} \frac{dx}{x} = \lim_{A \to +\infty} \ln x\Big|_{1}^{A} = \lim_{A \to +\infty} (\ln A - \ln 1) = +\infty. $$
	Таким образом, несобственный интеграл сходится при $\alpha > 1$ и расходится при $\alpha \le 1$.
	Согласно интегральному признаку Коши, ряд $\sum_{n=1}^{\infty} \frac{1}{n^\alpha}$ сходится при $\alpha > 1$ и расходится при $\alpha \le 1$.
	\begin{theorem}
		Если $a_n\underset{n\to\infty}{\sim}\dfrac{M}{n^\alpha}$, $0<M<+\infty$, то ряд $\sumn$ сходится при $\alpha > 1$ и расходится при $\alpha \leq 1$.
	\end{theorem}
	\noindent
	\textbf{Задача.}
	$$\sum_{n=2}^{\infty} \frac{n}{(n^2+20)\sqrt{\ln(n+1)}}$$
	Для исследования сходимости применим интегральный признак Коши. Прямое интегрирование функции $f(x) = \frac{x}{(x^2+20)\sqrt{\ln(x+1)}}$ затруднительно. Поэтому сначала воспользуемся предельным признаком сравнения, чтобы упростить общий член ряда.
	\\\\
	Обозначим $a_n = \frac{n}{(n^2+20)\sqrt{\ln(n+1)}}$.
	Для больших $n$, $n^2+20 \sim n^2$ и $\ln(n+1) \sim \ln n$. Сравним исходный ряд с рядом, общий член которого $b_n = \frac{n}{n^2\sqrt{\ln n}} = \frac{1}{n\sqrt{\ln n}}$.
	\\\\
	Вычислим предел их отношения:
	$$ \lim_{n \to \infty} \frac{a_n}{b_n} = \lim_{n \to \infty} \frac{\frac{n}{(n^2+20)\sqrt{\ln(n+1)}}}{\frac{1}{n\sqrt{\ln n}}} = \lim_{n \to \infty} \frac{n^2 \sqrt{\ln n}}{(n^2+20)\sqrt{\ln(n+1)}} $$
	$$ = \lim_{n \to \infty} \left(\frac{n^2}{n^2+20}\right) \cdot \lim_{n \to \infty} \sqrt{\frac{\ln n}{\ln(n+1)}} = 1 \cdot \sqrt{1} = 1. $$
	Так как предел равен конечному числу, отличному от нуля, то ряды $\sum a_n$ и $\sum b_n$ сходятся или расходятся одновременно.
	\\\\
	Теперь исследуем сходимость ряда $\sum_{n=2}^{\infty} \frac{1}{n\sqrt{\ln n}}$ с помощью интегрального признака.
	Рассмотрим функцию $f(x) = \frac{1}{x\sqrt{\ln x}}$. Она положительна, непрерывна и монотонно убывает при $x \ge 2$. Вычислим несобственный интеграл:
	$$ \int_{2}^{+\infty} \frac{dx}{x\sqrt{\ln x}} = \lim_{A \to +\infty} \int_{2}^{A} \frac{dx}{x\sqrt{\ln x}}. $$
	Сделаем замену $u = \ln x$, тогда $du = \frac{dx}{x}$.
	$$ \int \frac{du}{\sqrt{u}} = \int u^{-1/2} du = 2u^{1/2} + C = 2\sqrt{\ln x} + C. $$
	Подставим пределы интегрирования:
	$$ \lim_{A \to +\infty} 2\sqrt{\ln x}\Big|_{2}^{A} = \lim_{A \to +\infty} (2\sqrt{\ln A} - 2\sqrt{\ln 2}) = +\infty. $$
	Так как несобственный интеграл расходится, то и ряд $\sum_{n=2}^{\infty} \frac{1}{n\sqrt{\ln n}}$ расходится.
	Следовательно, по предельному признаку сравнения, исходный ряд также расходится.
	\\\\
	\textbf{Задача.}
	$$\sum_{n=1}^{\infty} n^2 e^{-n^3}$$
	Для исследования сходимости данного ряда воспользуемся интегральным признаком Коши.
	Рассмотрим функцию $f(x) = x^2 e^{-x^3}$. При $x \ge 1$ эта функция положительна, непрерывна и монотонно убывает, так как ее производная $f'(x) = xe^{-x^3}(2-3x^3) < 0$ при $x \ge 1$.
	\\\\
	Вычислим несобственный интеграл:
	$$ \int_{1}^{+\infty} x^2 e^{-x^3} dx = \lim_{A \to +\infty} \int_{1}^{A} x^2 e^{-x^3} dx. $$
	Применим замену $u = -x^3$. Тогда $du = -3x^2 dx$, откуда $x^2 dx = -\frac{1}{3}du$.
	Новые пределы интегрирования: если $x=1$, то $u = -1$; если $x = A$, то $u = -A^3$.
	$$ \lim_{A \to +\infty} \int_{-1}^{-A^3} e^u \left(-\frac{1}{3}du\right) = \frac{1}{3} \lim_{A \to +\infty} \int_{-A^3}^{-1} e^u du = \frac{1}{3} \lim_{A \to +\infty} e^u\Big|_{-A^3}^{-1}=$$
	$$ = \frac{1}{3} \lim_{A \to +\infty} (e^{-1} - e^{-A^3}) = \frac{1}{3} \left(\frac{1}{e} - 0\right) = \frac{1}{3e}. $$
	Так как несобственный интеграл сходится (равен конечному числу), то, согласно интегральному признаку Коши, исходный ряд $\sum_{n=1}^{\infty} n^2 e^{-n^3}$ также сходится.
	\\\\
	\textbf{2619*.}
	$$\sum_{n=2}^{\infty} \frac{1}{n \ln^p n}$$
	Для исследования сходимости данного ряда в зависимости от параметра $p$ воспользуемся интегральным признаком Коши.
	Рассмотрим функцию $f(x) = \frac{1}{x \ln^p x}$. При $x \ge 2$ функция является положительной, непрерывной и монотонно убывающей. Все условия для применения признака выполнены.
	
	Исследуем на сходимость несобственный интеграл:
	$$ \int_{2}^{+\infty} \frac{dx}{x \ln^p x} = \lim_{A \to +\infty} \int_{2}^{A} \frac{dx}{x \ln^p x}. $$
	Применим замену $u = \ln x$, тогда $du = \frac{dx}{x}$.
	Новые пределы интегрирования: если $x=2$, то $u=\ln 2$; если $x=A$, то $u=\ln A$.
	$$ \lim_{A \to +\infty} \int_{\ln 2}^{\ln A} \frac{du}{u^p}. $$
	Рассмотрим два случая.
	
	1. Пусть $p \neq 1$.
	$$ \int_{\ln 2}^{\ln A} u^{-p} du = \left[ \frac{u^{1-p}}{1-p} \right]_{\ln 2}^{\ln A} = \frac{(\ln A)^{1-p} - (\ln 2)^{1-p}}{1-p}. $$
	Предел этого выражения при $A \to +\infty$ конечен тогда и только тогда, когда показатель степени у $\ln A$ отрицателен, то есть $1-p < 0$, что эквивалентно $p > 1$.
	При $p > 1$, $\lim_{A \to +\infty} (\ln A)^{1-p} = 0$, и интеграл сходится.
	При $p < 1$, $\lim_{A \to +\infty} (\ln A)^{1-p} = +\infty$, и интеграл расходится.
	
	2. Пусть $p = 1$.
	$$ \int_{\ln 2}^{\ln A} \frac{du}{u} = [\ln|u|]_{\ln 2}^{\ln A} = \ln(\ln A) - \ln(\ln 2). $$
	При $A \to +\infty$, $\ln(\ln A) \to +\infty$, следовательно, интеграл расходится.
	
	Таким образом, несобственный интеграл сходится при $p > 1$ и расходится при $p \le 1$.
	Согласно интегральному признаку Коши, ряд $\sum_{n=2}^{\infty} \frac{1}{n \ln^p n}$ сходится при $p > 1$ и расходится при $p \le 1$.
	\\\\
	\textbf{11.)}
	$$\sum_{n=1}^{\infty} \frac{2n+3}{3n^2+2n+1} \cdot \frac{1-\cos(1/n)}{\operatorname{ch}\frac{2}{\sqrt{n}}-1}$$
	Для исследования сходимости воспользуемся предельным признаком сравнения, заменив сомножители в общем члене ряда $a_n$ на эквивалентные им бесконечно малые.
	
	При $n \to \infty$:
	\begin{itemize}
		\item Дробь с многочленами эквивалентна отношению старших членов:
		$$ \frac{2n+3}{3n^2+2n+1} \sim \frac{2n}{3n^2} = \frac{2}{3n}. $$
		\item Используя эквивалентность $1-\cos x \sim \frac{x^2}{2}$ при $x \to 0$:
		$$ 1-\cos\left(\frac{1}{n}\right) \sim \frac{(1/n)^2}{2} = \frac{1}{2n^2}. $$
		\item Используя эквивалентность $\operatorname{ch} x - 1 \sim \frac{x^2}{2}$ при $x \to 0$:
		$$ \operatorname{ch}\left(\frac{2}{\sqrt{n}}\right) - 1 \sim \frac{(2/\sqrt{n})^2}{2} = \frac{4/n}{2} = \frac{2}{n}. $$
	\end{itemize}
	Собираем эквивалентное выражение для общего члена $a_n$:
	$$ a_n \sim \frac{2}{3n} \cdot \frac{1/(2n^2)}{2/n} = \frac{2}{3n} \cdot \frac{n}{4n^2} = \frac{2n}{12n^3} = \frac{1}{6n^2}. $$
	Сравним исходный ряд с рядом $\sum_{n=1}^{\infty} b_n$, где $b_n = \frac{1}{n^2}$.
	Ряд $\sum_{n=1}^{\infty} \frac{1}{n^2}$ является обобщенным гармоническим рядом (p-рядом) с показателем $p=2$. Так как $p > 1$, этот ряд сходится.
	Поскольку $\lim_{n \to \infty} \frac{a_n}{b_n} = \frac{1}{6}$ (конечен и не равен нулю), то исходный ряд ведет себя так же, как и ряд сравнения.
	Следовательно, исходный ряд сходится.
	\\\\
	\textbf{12.)}
	$$\sum_{n=2}^{\infty} \left(1-\frac{\ln n}{n}\right)^{2n}$$
	Для исследования сходимости воспользуемся предельным признаком сравнения. Обозначим общий член ряда как $a_n = \left(1-\frac{\ln n}{n}\right)^{2n}$.
	Преобразуем $a_n$, используя свойство $a^b = e^{b \ln a}$:
	$$ a_n = \exp\left(2n \ln\left(1-\frac{\ln n}{n}\right)\right). $$
	Поскольку $\lim_{n \to \infty} \frac{\ln n}{n} = 0$, мы можем использовать эквивалентность $\ln(1-x) \sim -x$ при $x \to 0$. Полагая $x = \frac{\ln n}{n}$, получаем:
	$$ \ln\left(1-\frac{\ln n}{n}\right) \sim -\frac{\ln n}{n}. $$
	Тогда показатель степени в экспоненте эквивалентен:
	$$ 2n \ln\left(1-\frac{\ln n}{n}\right) \sim 2n \left(-\frac{\ln n}{n}\right) = -2\ln n = \ln(n^{-2}) = \ln\left(\frac{1}{n^2}\right). $$
	Следовательно, общий член ряда $a_n$ эквивалентен:
	$$ a_n \sim \exp\left(\ln\left(\frac{1}{n^2}\right)\right) = \frac{1}{n^2}. $$
	Сравним исходный ряд с рядом $\sum_{n=2}^{\infty} b_n$, где $b_n = \frac{1}{n^2}$.
	Ряд $\sum_{n=2}^{\infty} \frac{1}{n^2}$ является сходящимся p-рядом ($p=2 > 1$).
	Так как $a_n \sim b_n$, то по предельному признаку сравнения исходный ряд также сходится.
	\\\\
	\textbf{2626.}
	$$\sum_{n=2}^{\infty} \frac{\sqrt{n+2}-\sqrt{n-2}}{n^\alpha}$$
	Для исследования сходимости воспользуемся предельным признаком сравнения. Упростим числитель, домножив на сопряженное выражение:
	$$ \sqrt{n+2}-\sqrt{n-2} = \frac{(\sqrt{n+2}-\sqrt{n-2})(\sqrt{n+2}+\sqrt{n-2})}{\sqrt{n+2}+\sqrt{n-2}} = \frac{(n+2)-(n-2)}{\sqrt{n+2}+\sqrt{n-2}} = \frac{4}{\sqrt{n+2}+\sqrt{n-2}}. $$
	При $n \to \infty$, знаменатель эквивалентен:
	$$ \sqrt{n+2}+\sqrt{n-2} \sim \sqrt{n}+\sqrt{n} = 2\sqrt{n}. $$
	Тогда общий член ряда $a_n$ эквивалентен:
	$$ a_n = \frac{4}{(\sqrt{n+2}+\sqrt{n-2})n^\alpha} \sim \frac{4}{2\sqrt{n} \cdot n^\alpha} = \frac{2}{n^{\alpha+1/2}}. $$
	Сравним исходный ряд с обобщенным гармоническим рядом $\sum \frac{1}{n^{\alpha+1/2}}$. Этот ряд сходится, если показатель степени больше 1:
	$$ \alpha + \frac{1}{2} > 1 \implies \alpha > \frac{1}{2}. $$
	Следовательно, исходный ряд сходится при $\alpha > 1/2$ и расходится при $\alpha \le 1/2$.
	\\\\
	\textbf{2627*.}
	$$\sum_{n=1}^{\infty} \left(\sqrt{n+a} - \sqrt[4]{n^2+n+b}\right)$$
	Для исследования сходимости преобразуем общий член ряда $a_n$, чтобы привести корни к одному показателю.
	$$ \sqrt{n+a} = \sqrt[4]{(n+a)^2} = \sqrt[4]{n^2+2an+a^2}. $$
	Тогда общий член ряда принимает вид:
	$$ a_n = \sqrt[4]{n^2+2an+a^2} - \sqrt[4]{n^2+n+b}. $$
	Теперь применим формулу разности $x-y = \frac{x^4-y^4}{x^3+x^2y+xy^2+y^3}$, домножив числитель и знаменатель на сопряженное выражение.
	$$ a_n = \frac{(n^2+2an+a^2) - (n^2+n+b)}{(\sqrt[4]{n^2+2an+a^2})^3 + \dots + (\sqrt[4]{n^2+n+b})^3}. $$
	Упростим числитель:
	$$ (n^2+2an+a^2) - (n^2+n+b) = (2a-1)n + (a^2-b). $$
	Теперь оценим поведение знаменателя при $n \to \infty$. Каждый из четырех членов в знаменателе эквивалентен $(\sqrt[4]{n^2})^3 = (n^{1/2})^3 = n^{3/2}$.
	Следовательно, знаменатель эквивалентен:
	$$ n^{3/2} + n^{3/2} + n^{3/2} + n^{3/2} = 4n^{3/2}. $$
	Таким образом, общий член ряда $a_n$ эквивалентен:
	$$ a_n \sim \frac{(2a-1)n + (a^2-b)}{4n^{3/2}}. $$
	Теперь рассмотрим два случая.
	\begin{itemize}
		\item \textbf{Случай 1: $a \neq 1/2$.}\\
		В этом случае старший член в числителе — это $(2a-1)n$. Тогда
		$$ a_n \sim \frac{(2a-1)n}{4n^{3/2}} = \frac{2a-1}{4n^{1/2}}. $$
		Ряд эквивалентен обобщенному гармоническому ряду $\sum \frac{1}{n^p}$ с показателем $p=1/2$. Так как $p \le 1$, ряд расходится.
		
		\item \textbf{Случай 2: $a = 1/2$.}\\
		В этом случае коэффициент при $n$ в числителе обращается в ноль: $2a-1 = 0$. Числитель становится константой: $a^2-b = (1/2)^2 - b = 1/4 - b$. Тогда
		$$ a_n \sim \frac{1/4-b}{4n^{3/2}}. $$
		Ряд эквивалентен обобщенному гармоническому ряду $\sum \frac{1}{n^p}$ с показателем $p=3/2$. Так как $p > 1$, ряд сходится для любого значения $b$. (Если $b=1/4$, то с некоторого номера члены ряда будут нулевыми, и ряд, очевидно, сходится).
	\end{itemize}
	\textbf{Вывод:} Ряд сходится, если $a=1/2$ (при любом $b$), и расходится, если $a \neq 1/2$.
	\\\\	
	\textbf{2628.}
	$$\sum_{n=1}^{\infty} \left(\operatorname{ctg}\frac{n\pi}{4n-2} - \sin\frac{n\pi}{2n+1}\right)$$
	Для исследования сходимости преобразуем аргументы тригонометрических функций.
	
	1. Аргумент котангенса:
	$$ \frac{n\pi}{4n-2} = \frac{\pi}{4}\left(1 + \frac{2}{4n-2}\right) = \frac{\pi}{4} + \frac{\pi}{4n-2}. $$
	Обозначим $x_n = \frac{\pi}{4n-2}$.
	
	2. Аргумент синуса:
	$$ \frac{n\pi}{2n+1} = \frac{\pi}{2}\left(1 - \frac{1}{2n+1}\right) = \frac{\pi}{2} - \frac{\pi}{2n+1}. $$
	Обозначим $y_n = \frac{\pi}{2n+1}$.
	
	Теперь преобразуем сами функции.
	Для котангенса используем свойство $\operatorname{ctg} z = 1/\operatorname{tg} z$ и формулу тангенса суммы $\operatorname{tg}(A+B) = \frac{\operatorname{tg} A + \operatorname{tg} B}{1 - \operatorname{tg} A \operatorname{tg} B}$:
	$$ \operatorname{ctg}\left(\frac{\pi}{4} + x_n\right) = \frac{1}{\operatorname{tg}\left(\frac{\pi}{4} + x_n\right)} = \frac{1 - \operatorname{tg}(\pi/4)\operatorname{tg}(x_n)}{\operatorname{tg}(\pi/4)+\operatorname{tg}(x_n)} = \frac{1-\operatorname{tg}(x_n)}{1+\operatorname{tg}(x_n)}. $$
	Для синуса используем формулу приведения:
	$$ \sin\left(\frac{\pi}{2} - y_n\right) = \cos(y_n). $$
	При $n \to \infty$, $x_n \to 0$ и $y_n \to 0$. Используем разложения в ряд Тейлора:
	$$ \operatorname{tg}(x_n) = x_n + O(x_n^3). $$
	$$ \cos(y_n) = 1 - \frac{y_n^2}{2} + O(y_n^4). $$
	Разложим выражение для котангенса, используя $(1+z)^{-1} \approx 1-z+z^2$:
	$$ \frac{1-\operatorname{tg}(x_n)}{1+\operatorname{tg}(x_n)} = (1-\operatorname{tg}(x_n))(1-\operatorname{tg}(x_n)+\operatorname{tg}^2(x_n)-\dots) \approx (1-x_n)(1-x_n) = 1 - 2x_n + x_n^2. $$
	Теперь соберем общий член ряда $a_n$:
	$$ a_n = \left(1 - 2x_n + O(x_n^2)\right) - \left(1 - \frac{y_n^2}{2} + O(y_n^4)\right) = -2x_n + \frac{y_n^2}{2} + O(x_n^2). $$
	Подставим обратно выражения для $x_n$ и $y_n$:
	$$ a_n = -2\left(\frac{\pi}{4n-2}\right) + \frac{1}{2}\left(\frac{\pi}{2n+1}\right)^2 + \dots = -\frac{\pi}{2n-1} + \frac{\pi^2}{2(2n+1)^2} + \dots $$
	При $n \to \infty$ главный член асимптотики $a_n$ определяется слагаемым с наименьшей степенью $n$ в знаменателе:
	$$ a_n \sim -\frac{\pi}{2n-1} \sim -\frac{\pi}{2n}. $$
	Сравним исходный ряд с рядом $\sum_{n=1}^{\infty} b_n$, где $b_n = \frac{1}{n}$. Ряд $\sum \frac{1}{n}$ является гармоническим и расходится.
	Поскольку $\lim_{n \to \infty} \frac{a_n}{b_n} = -\frac{\pi}{2}$ (конечен и не равен нулю), то исходный ряд ведет себя так же, как и гармонический ряд.
	Следовательно, исходный ряд расходится.
	\\\\
	\textbf{2629.}
	$$\sum_{n=1}^{\infty} \left(\frac{1}{\sqrt{n}} - \sqrt{\ln\frac{n+1}{n}}\right)$$
	Разложим в ряд Тейлора выражение под корнем:
	$$ \ln\frac{n+1}{n} = \ln\left(1+\frac{1}{n}\right) = \frac{1}{n} - \frac{1}{2n^2} + O\left(\frac{1}{n^3}\right). $$
	Теперь разложим корень из этого выражения:
	$$ \sqrt{\ln\left(1+\frac{1}{n}\right)} = \sqrt{\frac{1}{n} - \frac{1}{2n^2} + \dots} = \frac{1}{\sqrt{n}}\sqrt{1-\frac{1}{2n}+\dots}. $$
	Используем разложение $\sqrt{1-x} \approx 1-\frac{x}{2}$:
	$$ \frac{1}{\sqrt{n}}\left(1-\frac{1}{2}\frac{1}{2n} + O\left(\frac{1}{n^2}\right)\right) = \frac{1}{\sqrt{n}}\left(1-\frac{1}{4n}+\dots\right) = \frac{1}{\sqrt{n}} - \frac{1}{4n^{3/2}} + O\left(\frac{1}{n^{5/2}}\right). $$
	Подставим это в общий член ряда $a_n$:
	$$ a_n = \frac{1}{\sqrt{n}} - \left(\frac{1}{\sqrt{n}} - \frac{1}{4n^{3/2}} + \dots\right) = \frac{1}{4n^{3/2}} - \dots $$
	Главный член выражения $a_n$ равен $\frac{1}{4n^{3/2}}$. Ряд эквивалентен сходящемуся гармоническому ряду $\sum \frac{1}{n^{3/2}}$ (так как $p=3/2 > 1$), следовательно, исходный ряд сходится.
	\\\\
	\textbf{2630.}
	$$\sum_{n=1}^{\infty} \frac{\ln(n!)}{n^\alpha}$$
	Для исследования сходимости воспользуемся предельным признаком сравнения. Для нахождения асимптотики числителя при $n \to \infty$ применим формулу Стирлинга, которая является асимптотическим разложением для факториала.
	
	Формула Стирлинга:
	$$ n! \sim \sqrt{2\pi n} \left(\frac{n}{e}\right)^n. $$
	Прологарифмируем это выражение:
	$$ \ln(n!) \sim \ln\left(\sqrt{2\pi n} \left(\frac{n}{e}\right)^n\right) = \ln(\sqrt{2\pi n}) + \ln\left(\left(\frac{n}{e}\right)^n\right) $$
	$$ = \frac{1}{2}\ln(2\pi) + \frac{1}{2}\ln n + n(\ln n - \ln e) = n\ln n - n + \frac{1}{2}\ln n + \frac{1}{2}\ln(2\pi). $$
	При $n \to \infty$ главный член этой асимптотики — это $n\ln n$, так как он растет быстрее всех остальных слагаемых.
	$$ \ln(n!) \sim n\ln n. $$
	Теперь подставим это в общий член ряда $a_n$:
	$$ a_n = \frac{\ln(n!)}{n^\alpha} \sim \frac{n\ln n}{n^\alpha} = \frac{\ln n}{n^{\alpha-1}}. $$
	Сравним исходный ряд с рядом $\sum b_n$, где $b_n = \frac{\ln n}{n^{\alpha-1}}$.
	Этот ряд сходится, если показатель степени у $n$ в знаменателе строго больше 1, и расходится в противном случае.
	
	Исследуем сходимость ряда $\sum \frac{\ln n}{n^{\alpha-1}}$:
	\begin{itemize}
		\item Если $\alpha-1 > 1$, то есть $\alpha > 2$, ряд сходится. Это можно показать, сравнив его с рядом $\sum \frac{1}{n^p}$ где $1 < p < \alpha-1$.
		\item Если $\alpha-1 \le 1$, то есть $\alpha \le 2$, ряд расходится. Это можно показать, сравнив его с расходящимся рядом $\sum \frac{1}{n^{\alpha-1}}$.
	\end{itemize}
	Поскольку исходный ряд эквивалентен ряду $\sum \frac{\ln n}{n^{\alpha-1}}$, он имеет ту же область сходимости.
	
	\textbf{Вывод:} Ряд сходится при $\alpha > 2$ и расходится при $\alpha \le 2$.
	\\\\
	\textbf{Задача.}
	$$\text{Доказать расходимость ряда } \sum_{n=1}^{\infty} \left(\frac{n+1}{n+2}\right)^{\frac{n^2+n+1}{3n}}, \text{ используя необходимое условие.}$$
	
	Необходимое условие сходимости ряда $\sum_{n=1}^{\infty} a_n$ гласит, что предел его общего члена должен быть равен нулю:
	$$ \lim_{n \to \infty} a_n = 0. $$
	Если этот предел не равен нулю или не существует, то ряд расходится (этот вывод также называют признаком расходимости).
	
	Проверим выполнение этого условия для нашего ряда. Обозначим общий член ряда как
	$$ a_n = \left(\frac{n+1}{n+2}\right)^{\frac{n^2+n+1}{3n}}. $$
	Найдем предел $L = \lim_{n \to \infty} a_n$. Мы имеем неопределенность вида $\left[1^{\infty}\right]$, так как основание стремится к 1, а показатель — к $\infty$. Для ее раскрытия воспользуемся тождеством $a^b = e^{b \ln a}$:
	$$ L = \lim_{n \to \infty} \exp\left( \frac{n^2+n+1}{3n} \ln\left(\frac{n+1}{n+2}\right) \right). $$
	В силу непрерывности показательной функции, можно внести предел в показатель степени:
	$$ L = \exp\left( \lim_{n \to \infty} \frac{n^2+n+1}{3n} \ln\left(\frac{n+1}{n+2}\right) \right). $$
	Найдем предел показателя. Сначала преобразуем аргумент логарифма:
	$$ \frac{n+1}{n+2} = \frac{n+2-1}{n+2} = 1 - \frac{1}{n+2}. $$
	Используем известную эквивалентность $\ln(1+x) \sim x$ при $x \to 0$. Так как при $n \to \infty$ выражение $-\frac{1}{n+2} \to 0$, то
	$$ \ln\left(1 - \frac{1}{n+2}\right) \sim -\frac{1}{n+2}. $$
	Теперь вычислим предел показателя, заменяя логарифм на эквивалентное выражение:
	$$ \lim_{n \to \infty} \frac{n^2+n+1}{3n} \left(-\frac{1}{n+2}\right) = \lim_{n \to \infty} -\frac{n^2+n+1}{3n(n+2)} = \lim_{n \to \infty} -\frac{n^2+n+1}{3n^2+6n}. $$
	Предел этого отношения равен отношению коэффициентов при старших степенях ($n^2$):
	$$ -\frac{1}{3}. $$
	Мы нашли предел показателя. Теперь вернемся к исходному пределу $L$:
	$$ L = e^{-1/3} = \frac{1}{\sqrt[3]{e}}. $$
	
	\textbf{Вывод:}
	Предел общего члена ряда не равен нулю:
	$$ \lim_{n \to \infty} a_n = \frac{1}{\sqrt[3]{e}} \neq 0. $$
	Так как необходимое условие сходимости не выполняется, ряд расходится.
	\section{Признак Раабе и Гаусса}
	Признак Раабе — это более "тонкий" признак сходимости для знакоположительных рядов (рядов с положительными членами). Его применяют в тех случаях, когда более простой признак Даламбера не дает ответа, то есть когда предел отношения $\frac{a_{n+1}}{a_n}$ равен 1. Признак Раабе, по сути, анализирует, *насколько быстро* это отношение стремится к единице.
	\begin{theorem}
		[Признак Раабе]
			Пусть дан числовой ряд $\sum_{n=1}^{\infty} a_n$, где все члены $a_n > 0$.
		Рассмотрим предел:
		$$ R = \lim_{n \to \infty} n \left( \frac{a_n}{a_{n+1}} - 1 \right) $$
		Тогда:
		\begin{enumerate}
			\item Если $R > 1$, то ряд \textbf{сходится}.
			\item Если $R < 1$, то ряд \textbf{расходится}.
			\item Если $R = 1$, то признак \textbf{не дает ответа}.
		\end{enumerate}
	\end{theorem}
	\noindent
	Признак Гаусса предоставляет решающее заключение о сходимости или расходимости знакоположительного ряда в тех критических случаях, где более простые признаки (Даламбера, Раабе) оказываются бессильны. Он основан на представлении отношения $\frac{a_n}{a_{n+1}}$ в виде асимптотического разложения по степеням $\frac{1}{n}$.
	\begin{theorem}[Признак Гаусса]
		Пусть для ряда с положительными членами $\sum_{n=1}^{\infty} a_n$ отношение его соседних членов может быть представлено в виде:
		$$ \frac{a_n}{a_{n+1}} = \lambda + \frac{\mu}{n} + O\left(\frac{1}{n^{1+\epsilon}}\right) \quad \text{при } n \to \infty, $$
		где $\epsilon > 0$ — константа.
		Тогда:
		\begin{enumerate}
			\item Если $\lambda > 1$, то ряд \textbf{сходится}.
			\item Если $\lambda < 1$, то ряд \textbf{расходится}.
			\item Если $\lambda = 1$, то 
			\begin{enumerate}
				\item[а)] если $\mu > 1$, то ряд \textbf{сходится};
				\item[б)] если $\mu \le 1$, то ряд \textbf{расходится}.
			\end{enumerate}
		\end{enumerate}
	\end{theorem}
	\noindent\textbf{Задача.}
	Исследуем на сходимость ряд 
	$$\sum_{n=1}^{\infty} \left(\frac{(2n-1)!!}{(2n)!!}\right)^2.$$
	Возьмем общий член $$a_n = \left(\frac{1 \cdot 3 \cdot \ldots \cdot (2n-1)}{2 \cdot 4 \cdot \ldots \cdot (2n)}\right)^2.$$
	Тогда
	$$ \frac{a_{n+1}}{a_n} = \left(\frac{(2n+1)!! / (2n+2)!!}{(2n-1)!! / (2n)!!}\right)^2 = \left(\frac{2n+1}{2n+2}\right)^2 $$
	Предел этого отношения при $n \to \infty$ равен 1 (признак Даламбера не работает).
	Применим признак Раабе
	$$\lim_{n \to \infty} n \left( \frac{a_n}{a_{n+1}} - 1 \right) = \lim_{n \to \infty} n \left( \left(\frac{2n+2}{2n+1}\right)^2 - 1 \right)= $$
	$$ = \lim_{n \to \infty} n \left( \frac{(2n+2)^2 - (2n+1)^2}{(2n+1)^2} \right) = \lim_{n \to \infty} n \frac{4n^2+8n+4 - (4n^2+4n+1)}{(2n+1)^2} $$
	$$ = \lim_{n \to \infty} n \frac{4n+3}{4n^2+4n+1} = \lim_{n \to \infty} \frac{4n^2+3n}{4n^2+4n+1} = 1 $$
	Признак Раабе не дает ответа. Применим признак Гаусса. Нам нужно разложить отношение $\frac{a_n}{a_{n+1}}$:
	$$ \frac{a_n}{a_{n+1}} = \left(\frac{2n+2}{2n+1}\right)^2 = \left(1 + \frac{1}{2n+1}\right)^2 $$
	Разложим по формуле квадрата суммы:
	$$ = 1 + \frac{2}{2n+1} + \frac{1}{(2n+1)^2} $$
	Нам нужно разложение по степеням $1/n$, а не $1/(2n+1)$. Для этого преобразуем дробь:
	$$ \frac{1}{2n+1} = \frac{1}{2n(1 + 1/(2n))} = \frac{1}{2n} \left(1 + \frac{1}{2n}\right)^{-1} $$
	Используем разложение $(1+x)^{-1} = 1 - x + x^2 - \ldots$ для $x = \frac{1}{2n}$:
	$$ \frac{1}{2n+1} = \frac{1}{2n} \left(1 - \frac{1}{2n} + O\left(\frac{1}{n^2}\right)\right) = \frac{1}{2n} - \frac{1}{4n^2} + O\left(\frac{1}{n^3}\right) $$
	Теперь подставим это в наше выражение:
	$$ \frac{a_n}{a_{n+1}} = 1 + 2\left(\frac{1}{2n} - \frac{1}{4n^2} + \ldots\right) + \left(\frac{1}{2n+1}\right)^2 $$
	$$ = 1 + \frac{1}{n} - \frac{1}{2n^2} + O\left(\frac{1}{n^2}\right) $$
	Объединяя члены с $1/n^2$, получаем:
	$$ \frac{a_n}{a_{n+1}} = 1 + \frac{1}{n} + O\left(\frac{1}{n^2}\right).$$
	Таким образом, ряд расходится по признаку Гаусса.
	\\\\
	\textbf{2598.}
	$$ \sum_{n=1}^{\infty} \left(\frac{1 \cdot 3 \cdot \ldots \cdot (2n-1)}{2 \cdot 4 \cdot \ldots \cdot (2n)}\right)^p $$
	Для исследования сходимости данного ряда в зависимости от параметра $p$ воспользуемся признаком Раабе. Сначала необходимо убедиться, что более простой признак Даламбера не дает результата. Обозначим общий член ряда как
	$$ a_n = \left(\frac{(2n-1)!!}{(2n)!!}\right)^p. $$
	Тогда следующий член ряда будет:
	$$ a_{n+1} = \left(\frac{(2n+1)!!}{(2n+2)!!}\right)^p. $$
	Вычислим предел отношения следующего члена к предыдущему:
	$$ \lim_{n \to \infty} \frac{a_{n+1}}{a_n} = \lim_{n \to \infty} \frac{\left(\frac{(2n+1)!!}{(2n+2)!!}\right)^p}{\left(\frac{(2n-1)!!}{(2n)!!}\right)^p} = \lim_{n \to \infty} \left(\frac{(2n+1)!!}{(2n-1)!!} \cdot \frac{(2n)!!}{(2n+2)!!}\right)^p $$
	$$ = \lim_{n \to \infty} \left(\frac{2n+1}{2n+2}\right)^p = \left(\lim_{n \to \infty} \frac{2+1/n}{2+2/n}\right)^p = 1^p = 1. $$
	Так как предел равен 1, признак Даламбера не применим. Воспользуемся признаком Раабе и вычислим предел:
	$$ R = \lim_{n \to \infty} n \left( \frac{a_n}{a_{n+1}} - 1 \right). $$
	Обратное отношение равно:
	$$ \frac{a_n}{a_{n+1}} = \left(\frac{2n+2}{2n+1}\right)^p. $$
	Подставим его в формулу для $R$:
	$$ R = \lim_{n \to \infty} n \left[ \left(\frac{2n+2}{2n+1}\right)^p - 1 \right] = \lim_{n \to \infty} n \left[ \left(1 + \frac{1}{2n+1}\right)^p - 1 \right]. $$
	Используя известное асимптотическое разложение $(1+x)^\alpha \approx 1 + \alpha x$ при $x \to 0$, и полагая $x = \frac{1}{2n+1}$, получаем:
	$$ R = \lim_{n \to \infty} n \left[ \left(1 + p \cdot \frac{1}{2n+1} + O\left(\frac{1}{n^2}\right)\right) - 1 \right] = \lim_{n \to \infty} n \cdot \frac{p}{2n+1} = \lim_{n \to \infty} \frac{pn}{2n+1}. $$
	Вычислив предел, находим:
	$$ R = \lim_{n \to \infty} \frac{p}{2+1/n} = \frac{p}{2}. $$
	Согласно признаку Раабе, делаем вывод о сходимости ряда:
	\begin{enumerate}
		\item Если $R > 1$, то есть $\frac{p}{2} > 1$ или $p > 2$, ряд \textbf{сходится}.
		\item Если $R < 1$, то есть $\frac{p}{2} < 1$ или $p < 2$, ряд \textbf{расходится}.
		\item Если $R = 1$, то есть $p = 2$, признак Раабе не дает ответа.
	\end{enumerate}
	\textbf{2599.}
	$$ \frac{a}{b} + \frac{a(a+d)}{b(b+d)} + \frac{a(a+d)(a+2d)}{b(b+d)(b+2d)} + \cdots $$
	где $a, b, d$ — положительные числа.
	Для исследования сходимости данного ряда воспользуемся признаком Даламбера. Обозначим общий член ряда как
	$$ a_n = \frac{a(a+d)\cdots(a+(n-1)d)}{b(b+d)\cdots(b+(n-1)d)}. $$
	Тогда следующий член ряда будет:
	$$ a_{n+1} = \frac{a(a+d)\cdots(a+(n-1)d)(a+nd)}{b(b+d)\cdots(b+(n-1)d)(b+nd)}. $$
	Вычислим предел отношения следующего члена к предыдущему:
	$$ \lim_{n \to \infty} \frac{a_{n+1}}{a_n} = \lim_{n \to \infty} \frac{\frac{a(a+d)\cdots(a+nd)}{b(b+d)\cdots(b+nd)}}{\frac{a(a+d)\cdots(a+(n-1)d)}{b(b+d)\cdots(b+(n-1)d)}} = \lim_{n \to \infty} \frac{a+nd}{b+nd}. $$
	Разделив числитель и знаменатель на $n$, получим:
	$$ \lim_{n \to \infty} \frac{a/n+d}{b/n+d} = \frac{d}{d} = 1. $$
	Так как предел равен 1, признак Даламбера не дает ответа. Воспользуемся признаком Раабе и вычислим предел:
	$$ R = \lim_{n \to \infty} n \left( \frac{a_n}{a_{n+1}} - 1 \right). $$
	Обратное отношение равно:
	$$ \frac{a_n}{a_{n+1}} = \frac{b+nd}{a+nd}. $$
	Подставим его в формулу для $R$:
	$$ R = \lim_{n \to \infty} n \left( \frac{b+nd}{a+nd} - 1 \right) = \lim_{n \to \infty} n \left( \frac{b+nd - (a+nd)}{a+nd} \right) = \lim_{n \to \infty} n \left( \frac{b-a}{a+nd} \right). $$
	Вычислив предел, находим:
	$$ R = \lim_{n \to \infty} \frac{n(b-a)}{a+nd} = \lim_{n \to \infty} \frac{b-a}{a/n+d} = \frac{b-a}{d}. $$
	Согласно признаку Раабе, делаем вывод о сходимости ряда:
	\begin{enumerate}
		\item Если $R > 1$, то есть $\frac{b-a}{d} > 1$ или $b-a > d$, ряд \textbf{сходится}.
		\item Если $R < 1$, то есть $\frac{b-a}{d} < 1$ или $b-a < d$, ряд \textbf{расходится}.
		\item Если $R = 1$, то есть $b-a = d$, признак Раабе не дает ответа (в этом случае для исследования требуется признак Гаусса, который покажет, что ряд расходится).
	\end{enumerate}
	\textbf{2601.}
	$$ \sum_{n=1}^{\infty} \frac{\sqrt{n!}}{(2 + \sqrt 1)(2 + \sqrt 2) \cdots ( 2+ \sqrt n)} $$
	Для исследования сходимости данного ряда воспользуемся признаком Даламбера. Обозначим общий член ряда как
	$$ a_n = \frac{\sqrt{n!}}{(2 + \sqrt 1)(2 + \sqrt 2) \cdots ( 2+ \sqrt n)}. $$
	Тогда следующий член ряда будет:
	$$ a_{n+1} = \frac{\sqrt{(n+1)!}}{(2 + \sqrt 1)(2 + \sqrt 2) \cdots ( 2+ \sqrt n)(2 + \sqrt{n+1})}. $$
	Вычислим предел отношения следующего члена к предыдущему:
	$$ \lim_{n \to \infty} \frac{a_{n+1}}{a_n} = \lim_{n \to \infty} \frac{\frac{\sqrt{(n+1)!}}{(2 + \sqrt 1) \cdots (2 + \sqrt{n+1})}}{\frac{\sqrt{n!}}{(2 + \sqrt 1) \cdots (2 + \sqrt n)}} = \lim_{n \to \infty} \frac{\sqrt{(n+1)!}}{\sqrt{n!}} \cdot \frac{1}{2+\sqrt{n+1}}. $$
	Упрощая выражение, получаем:
	$$ \lim_{n \to \infty} \sqrt{\frac{(n+1)!}{n!}} \cdot \frac{1}{2+\sqrt{n+1}} = \lim_{n \to \infty} \frac{\sqrt{n+1}}{2+\sqrt{n+1}}. $$
	Разделив числитель и знаменатель на $\sqrt{n}$, находим:
	$$ \lim_{n \to \infty} \frac{\sqrt{1+1/n}}{2/\sqrt{n}+\sqrt{1+1/n}} = \frac{\sqrt{1}}{0+\sqrt{1}} = 1. $$
	Так как предел равен 1, признак Даламбера не применим. Воспользуемся признаком Раабе и вычислим предел:
	$$ R = \lim_{n \to \infty} n \left( \frac{a_n}{a_{n+1}} - 1 \right). $$
	Обратное отношение равно:
	$$ \frac{a_n}{a_{n+1}} = \frac{2+\sqrt{n+1}}{\sqrt{n+1}}. $$
	Подставим его в формулу для $R$:
	$$ R = \lim_{n \to \infty} n \left( \frac{2+\sqrt{n+1}}{\sqrt{n+1}} - 1 \right) = \lim_{n \to \infty} n \left( \frac{2+\sqrt{n+1}-\sqrt{n+1}}{\sqrt{n+1}} \right) = \lim_{n \to \infty} \frac{2n}{\sqrt{n+1}}. $$
	Вычислив предел, находим:
	$$ R = \lim_{n \to \infty} \frac{2n}{\sqrt{n}\sqrt{1+1/n}} = \lim_{n \to \infty} \frac{2\sqrt{n}}{\sqrt{1+1/n}} = \infty. $$
	Поскольку $R = \infty > 1$, исходный ряд \textbf{сходится}.
	\\\\
	\textbf{2602.}
	$$ \sum_{n=1}^{\infty} \frac{n! n^{-p}}{q(q+1)\cdots(q+n)}, \quad q > 0 $$
	Для исследования сходимости данного ряда в зависимости от параметра $p$ воспользуемся признаком Даламбера. Обозначим общий член ряда как
	$$ a_n = \frac{n! n^{-p}}{q(q+1)\cdots(q+n)}. $$
	Тогда следующий член ряда будет:
	$$ a_{n+1} = \frac{(n+1)! (n+1)^{-p}}{q(q+1)\cdots(q+n)(q+n+1)}. $$
	Вычислим предел отношения следующего члена к предыдущему:
	$$ \lim_{n \to \infty} \frac{a_{n+1}}{a_n} = \lim_{n \to \infty} \frac{(n+1)! (n+1)^{-p}}{q\cdots(q+n+1)} \cdot \frac{q\cdots(q+n)}{n! n^{-p}} $$
	$$ = \lim_{n \to \infty} \frac{(n+1)!}{n!} \cdot \frac{(n+1)^{-p}}{n^{-p}} \cdot \frac{1}{q+n+1} = \lim_{n \to \infty} (n+1) \left(\frac{n}{n+1}\right)^p \frac{1}{q+n+1}. $$
	Сгруппируем члены:
	$$ = \lim_{n \to \infty} \frac{n+1}{q+n+1} \cdot \left(\frac{n}{n+1}\right)^p. $$
	Предел первого множителя $\lim_{n \to \infty} \frac{n+1}{q+n+1} = 1$. Предел второго множителя $\lim_{n \to \infty} \left(\frac{n}{n+1}\right)^p = 1^p = 1$. Таким образом, итоговый предел равен $1 \cdot 1 = 1$.
	Так как предел равен 1, признак Даламбера не дает ответа. Воспользуемся признаком Раабе и вычислим предел:
	$$ R = \lim_{n \to \infty} n \left( \frac{a_n}{a_{n+1}} - 1 \right). $$
	Обратное отношение равно:
	$$ \frac{a_n}{a_{n+1}} = \frac{q+n+1}{n+1} \left(\frac{n+1}{n}\right)^p. $$
	Подставим его в формулу для $R$:
	$$ R = \lim_{n \to \infty} n \left[ \frac{q+n+1}{n+1} \left(1 + \frac{1}{n}\right)^p - 1 \right]. $$
	Используя асимптотические разложения при больших $n$:
	$$ \frac{q+n+1}{n+1} = 1 + \frac{q}{n+1} = 1+\frac{q}{n} + O\left(\frac{1}{n^2}\right) $$
	$$ \left(1 + \frac{1}{n}\right)^p = 1+\frac{p}{n} + O\left(\frac{1}{n^2}\right) $$
	Тогда произведение равно:
	$$ \frac{a_n}{a_{n+1}} = \left(1+\frac{q}{n} + O\left(\frac{1}{n^2}\right)\right)\left(1+\frac{p}{n} + O\left(\frac{1}{n^2}\right)\right) = 1 + \frac{p+q}{n} + O\left(\frac{1}{n^2}\right). $$
	Подставляя это в предел для $R$:
	$$ R = \lim_{n \to \infty} n \left[ \left(1 + \frac{p+q}{n} + O\left(\frac{1}{n^2}\right)\right) - 1 \right] = \lim_{n \to \infty} n \left[ \frac{p+q}{n} + O\left(\frac{1}{n^2}\right) \right] = p+q. $$
	Согласно признаку Раабе, делаем вывод о сходимости ряда:
	\begin{enumerate}
		\item Если $R > 1$, то есть $p+q > 1$, ряд \textbf{сходится}.
		\item Если $R < 1$, то есть $p+q < 1$, ряд \textbf{расходится}.
		\item Если $R = 1$, то есть $p+q = 1$, признак Раабе не дает ответа (в этом случае требуется признак Гаусса, который покажет, что ряд расходится).
	\end{enumerate}
	\textbf{Задача 5.2.}
	$$ \sum_{n=1}^{\infty} \frac{(2n-1)!!}{(2n)!! \sqrt{n}} $$
	*Примечание: в задаче опечатка, для корректного ряда первый член должен быть $(2n-1)!! / ((2n)!!\sqrt{n})$, а не $(2n+1)!! / ((2n+2)!!\sqrt{n})$. Исследуем исправленный ряд.*
	
	Для исследования сходимости данного ряда воспользуемся признаком Даламбера. Обозначим общий член ряда как
	$$ a_n = \frac{(2n-1)!!}{(2n)!! \sqrt{n}}. $$
	Тогда следующий член ряда будет:
	$$ a_{n+1} = \frac{(2n+1)!!}{(2n+2)!! \sqrt{n+1}}. $$
	Вычислим предел отношения следующего члена к предыдущему:
	$$ \lim_{n \to \infty} \frac{a_{n+1}}{a_n} = \lim_{n \to \infty} \frac{(2n+1)!!}{(2n+2)!! \sqrt{n+1}} \cdot \frac{(2n)!! \sqrt{n}}{(2n-1)!!} $$
	$$ = \lim_{n \to \infty} \frac{2n+1}{2n+2} \cdot \sqrt{\frac{n}{n+1}} = \left(\lim_{n \to \infty} \frac{2n+1}{2n+2}\right) \cdot \left(\lim_{n \to \infty} \sqrt{\frac{n}{n+1}}\right) = 1 \cdot 1 = 1. $$
	Так как предел равен 1, признак Даламбера не дает ответа. Воспользуемся признаком Раабе и вычислим предел:
	$$ R = \lim_{n \to \infty} n \left( \frac{a_n}{a_{n+1}} - 1 \right). $$
	Обратное отношение равно:
	$$ \frac{a_n}{a_{n+1}} = \frac{2n+2}{2n+1} \cdot \sqrt{\frac{n+1}{n}}. $$
	Подставим его в формулу для $R$:
	$$ R = \lim_{n \to \infty} n \left[ \frac{2n+2}{2n+1} \sqrt{1+\frac{1}{n}} - 1 \right]. $$
	Используя асимптотические разложения при больших $n$:
	$$ \frac{2n+2}{2n+1} = 1 + \frac{1}{2n+1} = 1 + \frac{1}{2n} - \frac{1}{4n^2} + O\left(\frac{1}{n^3}\right) $$
	$$ \sqrt{1+\frac{1}{n}} = \left(1+\frac{1}{n}\right)^{1/2} = 1 + \frac{1}{2n} - \frac{1}{8n^2} + O\left(\frac{1}{n^3}\right) $$
	Перемножим эти разложения:
	$$ \frac{a_n}{a_{n+1}} = \left(1 + \frac{1}{2n} - \frac{1}{4n^2}\right)\left(1 + \frac{1}{2n} - \frac{1}{8n^2}\right) + \ldots = 1 + \frac{1}{n} + \frac{1}{4n^2} - \frac{1}{4n^2} - \frac{1}{8n^2} + \ldots $$
	$$ = 1 + \frac{1}{n} - \frac{1}{8n^2} + O\left(\frac{1}{n^3}\right). $$
	Подставляя это в предел для $R$:
	$$ R = \lim_{n \to \infty} n \left[ \left(1 + \frac{1}{n} + O\left(\frac{1}{n^2}\right)\right) - 1 \right] = \lim_{n \to \infty} n \left[ \frac{1}{n} + O\left(\frac{1}{n^2}\right) \right] = 1. $$
	Согласно признаку Раабе, делаем вывод о сходимости ряда:
	\begin{enumerate}
		\item Если $R > 1$, ряд \textbf{сходится}.
		\item Если $R < 1$, ряд \textbf{расходится}.
		\item Если $R = 1$, признак Раабе \textbf{не дает ответа}.
	\end{enumerate}
	Поскольку $R=1$, признак Раабе не позволяет сделать вывод. Однако, полученное нами асимптотическое разложение $\frac{a_n}{a_{n+1}} = 1 + \frac{1}{n} - \frac{1}{8n^2} + \ldots$ имеет вид $1 + \frac{h}{n} + \ldots$ с коэффициентом $h=1$. Согласно признаку Гаусса, при $h \le 1$ ряд расходится. Следовательно, исходный ряд \textbf{расходится}.
	\\\\
	\textbf{Задача.}
	$$ \sum_{n=1}^{\infty} \frac{5 \cdot 8 \cdot \ldots \cdot (3n+2)}{(2n+3)!!} \left(\frac{2}{3}\right)^n $$
	Для исследования сходимости данного ряда воспользуемся признаком Даламбера. Обозначим общий член ряда как
	$$ a_n = \frac{5 \cdot 8 \cdot \ldots \cdot (3n+2)}{(2n+3)!!} \left(\frac{2}{3}\right)^n. $$
	Тогда следующий член ряда будет:
	$$ a_{n+1} = \frac{5 \cdot 8 \cdot \ldots \cdot (3n+2)(3n+5)}{(2n+5)!!} \left(\frac{2}{3}\right)^{n+1}. $$
	Вычислим предел отношения следующего члена к предыдущему:
	$$ \lim_{n \to \infty} \frac{a_{n+1}}{a_n} = \lim_{n \to \infty} \frac{\frac{5 \cdot \ldots \cdot (3n+5)}{(2n+5)!!} (\frac{2}{3})^{n+1}}{\frac{5 \cdot \ldots \cdot (3n+2)}{(2n+3)!!} (\frac{2}{3})^n} = \lim_{n \to \infty} \frac{3n+5}{2n+5} \cdot \frac{2}{3}. $$
	$$ = \lim_{n \to \infty} \frac{2(3n+5)}{3(2n+5)} = \lim_{n \to \infty} \frac{6n+10}{6n+15} = \frac{6}{6} = 1. $$
	Так как предел равен 1, признак Даламбера не дает ответа. Воспользуемся признаком Раабе и вычислим предел:
	$$ R = \lim_{n \to \infty} n \left( \frac{a_n}{a_{n+1}} - 1 \right). $$
	Обратное отношение равно:
	$$ \frac{a_n}{a_{n+1}} = \frac{3(2n+5)}{2(3n+5)} = \frac{6n+15}{6n+10}. $$
	Подставим его в формулу для $R$:
	$$ R = \lim_{n \to \infty} n \left( \frac{6n+15}{6n+10} - 1 \right) = \lim_{n \to \infty} n \left( \frac{6n+15 - (6n+10)}{6n+10} \right) = \lim_{n \to \infty} \frac{5n}{6n+10}. $$
	Вычислив предел, находим:
	$$ R = \lim_{n \to \infty} \frac{5}{6+10/n} = \frac{5}{6}. $$
	Согласно признаку Раабе, делаем вывод о сходимости ряда:
	\begin{enumerate}
		\item Если $R > 1$, ряд \textbf{сходится}.
		\item Если $R < 1$, ряд \textbf{расходится}.
		\item Если $R = 1$, признак не дает ответа.
	\end{enumerate}
	Поскольку $R = \frac{5}{6} < 1$, исходный ряд \textbf{расходится}.
	Конечно. Это классический пример, где необходим признак Гаусса, основанный на асимптотическом разложении с помощью формулы Стирлинга или, как вы просили, разложений Тейлора.
	\\\\
	\textbf{2600.}
	$$ \sum_{n=1}^{\infty} \frac{n! e^n}{n^{n+p}} $$
	Для исследования сходимости данного ряда в зависимости от параметра $p$ сначала найдем отношение $\frac{a_n}{a_{n+1}}$, так как признак Даламбера в пределе даст 1. Обозначим общий член ряда как
	$$ a_n = \frac{n! e^n}{n^{n+p}}. $$
	Тогда следующий член ряда будет:
	$$ a_{n+1} = \frac{(n+1)! e^{n+1}}{(n+1)^{n+1+p}}. $$
	Составим отношение, удобное для признака Гаусса:
	$$ \frac{a_n}{a_{n+1}} = \frac{n! e^n}{n^{n+p}} \cdot \frac{(n+1)^{n+p+1}}{(n+1)! e^{n+1}} = \frac{n!}{(n+1)!} \cdot \frac{e^n}{e^{n+1}} \cdot \frac{(n+1)^{n+p+1}}{n^{n+p}} $$
	$$ = \frac{1}{n+1} \cdot \frac{1}{e} \cdot \frac{(n+1)^{n+p} (n+1)}{n^{n+p}} = \frac{1}{e} \cdot \left(\frac{n+1}{n}\right)^{n+p} = \frac{1}{e} \left(1 + \frac{1}{n}\right)^{n+p}. $$
	Для того чтобы представить это выражение в виде $1 + \frac{h}{n} + \ldots$, воспользуемся разложением в ряд Тейлора. Удобнее всего сначала прологарифмировать выражение:
	$$ \ln\left(\frac{a_n}{a_{n+1}}\right) = \ln\left(\frac{1}{e} \left(1 + \frac{1}{n}\right)^{n+p}\right) = \ln\left(\frac{1}{e}\right) + \ln\left(\left(1 + \frac{1}{n}\right)^{n+p}\right) $$
	$$ = -1 + (n+p) \ln\left(1 + \frac{1}{n}\right). $$
	Теперь используем разложение Тейлора для логарифма: $\ln(1+x) = x - \frac{x^2}{2} + O(x^3)$. Полагая $x = \frac{1}{n}$, получаем:
	$$ \ln\left(\frac{a_n}{a_{n+1}}\right) = -1 + (n+p)\left(\frac{1}{n} - \frac{1}{2n^2} + O\left(\frac{1}{n^3}\right)\right) $$
	$$ = -1 + \left(n\left(\frac{1}{n} - \frac{1}{2n^2} + \ldots\right) + p\left(\frac{1}{n} - \frac{1}{2n^2} + \ldots\right)\right) $$
	$$ = -1 + \left(1 - \frac{1}{2n} + \ldots\right) + \left(\frac{p}{n} - \ldots\right) = \left(\frac{p}{n} - \frac{1}{2n}\right) + O\left(\frac{1}{n^2}\right) = \frac{p - 1/2}{n} + O\left(\frac{1}{n^2}\right). $$
	Мы получили разложение для логарифма. Теперь, чтобы найти разложение для самого отношения, воспользуемся разложением экспоненты $e^z = 1 + z + O(z^2)$. Полагая $z = \frac{p - 1/2}{n} + O\left(\frac{1}{n^2}\right)$:
	$$ \frac{a_n}{a_{n+1}} = \exp\left(\frac{p - 1/2}{n} + O\left(\frac{1}{n^2}\right)\right) = 1 + \frac{p - 1/2}{n} + O\left(\frac{1}{n^2}\right). $$
	Мы представили отношение в форме, необходимой для признака Гаусса:
	$$ \frac{a_n}{a_{n+1}} = 1 + \frac{h}{n} + O\left(\frac{1}{n^2}\right), \quad \text{где } h = p - \frac{1}{2}. $$
	Согласно признаку Гаусса, делаем вывод о сходимости ряда:
	\begin{enumerate}
		\item Если $h > 1$, то есть $p - \frac{1}{2} > 1 \implies p > \frac{3}{2}$, ряд \textbf{сходится}.
		\item Если $h \le 1$, то есть $p - \frac{1}{2} \le 1 \implies p \le \frac{3}{2}$, ряд \textbf{расходится}.
	\end{enumerate}
	\newpage
	\section{Функциональные пространства, норма и расстояние}
	
	\subsubsection{Напоминание: Норма и расстояние в $\mathbb{R}^n$}
	
	Прежде чем говорить о функциях, вспомним более простой объект — векторы в евклидовом пространстве $\mathbb{R}^n$.
	
	Вектор в $\mathbb{R}^n$ — это упорядоченный набор из $n$ действительных чисел: $\vec{v} = (v_1, v_2, \dots, v_n)$.
	
	\textbf{Норма} вектора (часто называемая евклидовой нормой) — это, по сути, его длина. Обозначается $\|\vec{v}\|$ и вычисляется по теореме Пифагора:
	$$ \|\vec{v}\| = \sqrt{v_1^2 + v_2^2 + \dots + v_n^2} $$
	Например, для вектора $\vec{v} = (3, 4)$ в $\mathbb{R}^2$, его норма (длина) равна $\|\vec{v}\| = \sqrt{3^2 + 4^2} = \sqrt{25} = 5$.
	
	\textbf{Расстояние} между двумя векторами $\vec{u}$ и $\vec{v}$ — это длина вектора их разности $\vec{u} - \vec{v}$.
	$$ d(\vec{u}, \vec{v}) = \|\vec{u} - \vec{v}\| = \sqrt{(u_1 - v_1)^2 + (u_2 - v_2)^2 + \dots + (u_n - v_n)^2} $$
	Это обычное расстояние по прямой между двумя точками в $n$-мерном пространстве.
	
	\subsubsection{Функциональное пространство E и норма-супремум}
	
	Теперь перейдем от векторов к функциям. Функциональное пространство — это множество функций, обладающее структурой векторного пространства (функции можно складывать и умножать на числа).
	
	Возьмем в качестве $E$ пространство всех \textbf{ограниченных} функций, определенных на некотором множестве $X$. То есть $f \in E$, если существует такое число $M > 0$, что $|f(x)| \le M$ для всех $x \in X$.
	
	В этом пространстве мы можем ввести норму, аналогичную длине вектора, но для функции. Одна из самых важных норм — это \textbf{супремум-норма} (или равномерная норма), обозначаемая $\|f\|_C$ или $\|f\|_{\infty}$.
	
	\textbf{Определение:} Нормой функции $f \in E$ называется
	$$ \|f\|_E = \sup_{x \in X} |f(x)| $$
	где $\sup$ (супремум) — это точная верхняя грань. Интуитивно, это "максимальная высота" или "максимальное отклонение" графика функции от оси $x=0$.
	
	\subsubsection{Взаимосвязь с пространством C[a,b]}
	
	Одним из важнейших примеров такого пространства является пространство $C[a,b]$ — множество всех \textbf{непрерывных} функций, определенных на замкнутом отрезке $[a,b]$.
	
	Каждая непрерывная на отрезке функция является ограниченной (согласно теореме Вейерштрасса), поэтому $C[a,b]$ является подпространством $E$, если в качестве $X$ взять отрезок $[a,b]$.
	
	Более того, для непрерывной функции на отрезке супремум всегда достигается, поэтому его можно заменить на максимум:
	$$ \|f\|_{C[a,b]} = \max_{x \in [a,b]} |f(x)| $$
	
	\subsubsection{Вычисление расстояния между функциями}
	
	Аналогично векторным пространствам, расстояние между двумя функциями $f(x)$ и $g(x)$ в пространстве $E$ определяется как норма их разности:
	$$ d(f, g) = \|f - g\|_E = \sup_{x \in [a,b]} |f(x) - g(x)| $$
	\textbf{Интуитивное понимание:} Это расстояние представляет собой \textbf{максимальное вертикальное расхождение} между графиками функций $f(x)$ и $g(x)$ на всей области определения. Представьте, что вы измеряете расстояние по вертикали между двумя кривыми в каждой точке $x$ и находите самое большое из этих расстояний.
	
	\begin{center}
		\textit{Пример: $f(x) = x^2$ и $g(x) = x$ на $[0,1]$.\\
			$d(f,g) = \max_{x \in [0,1]} |x^2 - x|$. Функция $h(x)=x-x^2$ достигает максимума в точке $x=1/2$, где $h(1/2) = 1/2 - 1/4 = 1/4$. Таким образом, расстояние между этими функциями в $C[0,1]$ равно $1/4$.}
	\end{center}
	
	\subsection{Сходимость функциональных последовательностей}
	
	Пусть у нас есть последовательность функций $\{f_n(x)\}_{n=1}^\infty$, где каждая функция $f_n(x)$ принадлежит пространству $E$. Мы хотим понять, что значит "последовательность функций сходится к некоторой функции $f(x)$".
	
	\textbf{Определение:} Последовательность $\{f_n(x)\}$ сходится \textbf{поточечно} к функции $f(x)$ на множестве $X$, если для \textbf{каждой фиксированной точки} $x_0 \in X$ числовая последовательность $\{f_n(x_0)\}$ сходится к числу $f(x_0)$.
	$$ \forall x_0 \in X: \quad \lim_{n \to \infty} f_n(x_0) = f(x_0) $$
	Функция $f(x)$, значения которой являются пределами в каждой точке, называется \textbf{предельной функцией}.
	
	\textbf{Ключевая мысль:} Сходимость в каждой точке рассматривается \textbf{независимо} от других точек. В одной точке сходимость может быть очень быстрой, а в другой — очень медленной.
	
	\textbf{Пример:} $f_n(x) = x^n$ на $[0,1]$.
	\begin{itemize}
		\item Если взять $x_0 = 1/2$, то последовательность $(1/2)^n \to 0$.
		\item Если взять $x_0 = 0.99$, то последовательность $(0.99)^n \to 0$ (хоть и медленнее).
		\item Если взять $x_0 = 1$, то последовательность $1^n = 1 \to 1$.
	\end{itemize}
	Предельная функция $f(x)$ является разрывной:
	$$ f(x) = \begin{cases} 0, & x \in [0, 1) \\ 1, & x=1 \end{cases} $$
	Это классический пример, где предел последовательности непрерывных функций не является непрерывной функцией.
	
	\textbf{Определение:} Последовательность $\{f_n(x)\}$ сходится \textbf{равномерно} к функции $f(x)$ на множестве $X$, если расстояние между $f_n$ и $f$, измеренное в супремум-норме, стремится к нулю.
	
	\textbf{Критерий равномерной сходимости:}
	$$ f_n \rightrightarrows f \iff \lim_{n \to \infty} \|f_n - f\|_E = 0 $$
	Раскрывая норму, это означает:
	$$ \lim_{n \to \infty} \left( \sup_{x \in X} |f_n(x) - f(x)| \right) = 0 $$
	
	\textbf{Сравнение и различие:}
	\begin{itemize}
		\item \textbf{Поточечная:} В каждой \textit{отдельной} точке $x_0$ график $f_n$ со временем попадает в $\varepsilon$-окрестность точки $f(x_0)$. Но для разных точек это может случиться при разных $n$.
		\item \textbf{Равномерная:} Весь график $f_n$ \textit{целиком} и \textit{одновременно} для всех $x$ попадает в $\varepsilon$-коридор вокруг $f(x)$. Скорость сходимости не зависит от точки $x$, она "равномерна" по всему множеству.
	\end{itemize}
	
	Вернемся к примеру $f_n(x) = x^n$ на $[0,1]$. Эта сходимость не является равномерной. Почему? Нарисуйте $\varepsilon$-коридор вокруг предельной функции $y=0$ (для $x \in [0,1)$), например, с $\varepsilon = 0.1$. Как бы велик ни был номер $n$, график $y=x^n$ всегда будет "выскакивать" из этого коридора при $x$, близких к 1. Например, для $x = \sqrt[n]{0.5}$, $f_n(x) = 0.5 > 0.1$. То есть, максимальное отклонение $\|f_n - f\|$ не стремится к нулю.

	
	\subsection{Исследование последовательностей на равномерную сходимость}
	
	\begin{enumerate}
		\item $f_n(x) = \frac{n}{nx + 4}, E = [1;5]$
		\begin{enumerate}
			\item \textbf{Точечный предел:} $$f(x) = \lim_{n \to \infty} \frac{n}{nx + 4} = \lim_{n \to \infty} \frac{1}{x + 4/n} = \frac{1}{x}.$$
			\item \textbf{Равномерная сходимость:}
			$$r_n(x) = |f_n(x) - f(x)| = \left|\frac{n}{nx + 4} - \frac{1}{x}\right| = \left|\frac{nx - (nx+4)}{x(nx+4)}\right| = \frac{4}{x(nx+4)}$$
			Для $x \in [1;5]$ функция $r_n(x)$ убывает, поэтому супремум достигается в точке $x=1$.
			$$\sup_{x \in [1;5]} r_n(x) = r_n(1) = \frac{4}{n+4}$$
			$$\lim_{n \to \infty} \sup_{x \in [1;5]} r_n(x) = \lim_{n \to \infty} \frac{4}{n+4} = 0$$
		\end{enumerate}
		\textbf{Вывод: Сходится равномерно.}
		
		\item $f_n(x) = n^2xe^{-nx}, E = [0;1]$
		\begin{enumerate}
			\item \textbf{Точечный предел:} Для $x=0$, $f_n(0)=0$. Для $x \in (0;1]$, $$f(x) = \lim_{n \to \infty} n^2xe^{-nx} = 0$$ Предельная функция $f(x)=0$.
			\item \textbf{Равномерная сходимость:}
			$r_n(x) = |n^2xe^{-nx}|$. Найдем максимум $r_n(x)$ на $[0,1]$.
			$$r_n'(x) = n^2e^{-nx}(1-nx) = 0 \Rightarrow x=1/n$$ Точка $x=1/n \in [0,1]$ для $n \ge 1$.
			$$\sup_{x \in [0,1]} r_n(x) = r_n(1/n) = n^2(1/n)e^{-1} = n/e$$
			$$\lim_{n \to \infty} \sup_{x \in [0,1]} r_n(x) = \lim_{n \to \infty} \frac{n}{e} = \infty \neq 0$$
		\end{enumerate}
		\textbf{Вывод: Сходится неравномерно.}
		
		\item $f_n(x) = \sqrt{16x^2 + \frac{1}{\ln n}}, E = \mathbb{R}$
		\begin{enumerate}
			\item \textbf{Точечный предел:} $$f(x) = \lim_{n \to \infty} \sqrt{16x^2 + \frac{1}{\ln n}} = \sqrt{16x^2} = 4|x|$$
			\item \textbf{Равномерная сходимость:}
			$$r_n(x) = \left|\sqrt{16x^2 + \frac{1}{\ln n}} - 4|x|\right| = \frac{(\sqrt{16x^2 + 1/\ln n} - 4|x|)(\sqrt{16x^2 + 1/\ln n} + 4|x|)}{\sqrt{16x^2 + 1/\ln n} + 4|x|}=$$
			$$ = \frac{1/\ln n}{\sqrt{16x^2 + 1/\ln n} + 4|x|}$$
			Супремум достигается при $x=0$.
			$$\sup_{x \in \mathbb{R}} r_n(x) = \frac{1/\ln n}{\sqrt{1/\ln n}} = \frac{1}{\sqrt{\ln n}}$$
			$$\lim_{n \to \infty} \sup_{x \in \mathbb{R}} r_n(x) = \lim_{n \to \infty} \frac{1}{\sqrt{\ln n}} = 0$$
		\end{enumerate}
		\textbf{Вывод: Сходится равномерно.}
		
		\item $f_n(x) = x^{n+2} - x^n, E=[0;1]$
		\begin{enumerate}
			\item \textbf{Точечный предел:} $$f(x) = \lim_{n \to \infty} (x^{n+2} - x^n) = \lim_{n \to \infty} x^n(x^2-1)$$
			Для $x \in [0;1)$, $f(x)=0$. Для $x=1$, $f(1)=0$. Итак, $f(x)=0$.
			\item \textbf{Равномерная сходимость:}
			$$r_n(x) = |x^n(x^2-1)| = x^n(1-x^2)$$ на $[0,1]$.
			$$r_n'(x) = nx^{n-1}(1-x^2) - 2x^{n+1} = x^{n-1}(n-nx^2-2x^2) = 0$$
			$$x^2 = n/(n+2) \Rightarrow x_n = \sqrt{n/(n+2)}$$
			$$\sup r_n(x) = r_n(x_n) = \left(\frac{n}{n+2}\right)^{n/2}\left(1-\frac{n}{n+2}\right) = \left(\frac{n}{n+2}\right)^{n/2}\frac{2}{n+2}$$
			$$\lim_{n \to \infty} \left(1-\frac{2}{n+2}\right)^{n/2}\frac{2}{n+2} = e^{-1} \cdot 0 = 0$$
		\end{enumerate}
		\textbf{Вывод: Сходится равномерно.}
		
	\end{enumerate}
	
\end{document}