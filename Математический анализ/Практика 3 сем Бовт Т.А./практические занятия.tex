\documentclass[a4paper, 12pt]{report}
\usepackage{cmap}
\usepackage{amssymb}
\usepackage{amsmath}
\usepackage{graphicx}
\usepackage{amsthm}
\usepackage{upgreek}
\usepackage{setspace}
\usepackage{empheq}
\usepackage{mathtools}
\setcounter{secnumdepth}{5}
\setcounter{tocdepth}{5}
\numberwithin{equation}{section}
\renewcommand{\theequation}{\arabic{equation}}
\usepackage[T2A]{fontenc}
\usepackage[utf8]{inputenc}
\usepackage[normalem]{ulem}
\usepackage{mathtext} % русские буквы в формулах
\usepackage[left=2cm,right=2cm, top=2cm,bottom=2cm,bindingoffset=0cm]{geometry}
\usepackage[english,russian]{babel}
\usepackage[unicode]{hyperref}
\newenvironment{Proof} % имя окружения
{\par\noindent{$\blacklozenge$}} % команды для \begin
{\hfill$\scriptstyle\square$}
\newcommand{\Rm}{\mathbb{R}}
\newcommand{\Cm}{\mathbb{C}}
\newcommand{\Z}{\mathbb{Z}}
\newcommand{\I}{\mathbb{I}}
\newcommand{\N}{\mathbb{N}}
\newcommand{\rank}{\operatorname{rank}}
\newcommand{\Ra}{\Rightarrow}
\newcommand{\ra}{\rightarrow}
\newcommand{\FI}{\Phi}
\newcommand{\Sp}{\text{Sp}}
\newcommand{\ol}{\overline}

\renewcommand{\leq}{\leqslant}
\renewcommand{\geq}{\geqslant}

\renewcommand{\alpha}{\upalpha}
\renewcommand{\beta}{\upbeta}
\renewcommand{\gamma}{\upgamma}
\renewcommand{\delta}{\updelta}
\renewcommand{\varphi}{\upvarphi}
\renewcommand{\phi}{\upvarphi}
\renewcommand{\tau}{\uptau}
\renewcommand{\theta}{\uptheta}
\renewcommand{\eta}{\upeta}
\renewcommand{\lambda}{\uplambda}
\renewcommand{\sigma}{\upsigma}
\renewcommand{\psi}{\uppsi}
\renewcommand{\mu}{\upmu}
\renewcommand{\omega}{\upomega}
\renewcommand{\xi}{\upxi}
\renewcommand{\epsilon}{\upvarepsilon}
\renewcommand{\rho}{\uprho}
\renewcommand{\varepsilon}{\upvarepsilon}

\renewcommand{\d}{\partial}
\renewcommand{\Re}{\operatorname{Re}}
\newcommand{\const}{\operatorname{const}}
\newcommand{\intx}{\int\limits_{x_0}^x}
\newcommand\Norm[1]{\left\| #1 \right\|}
\newcommand{\sumn}{\sum\limits_{n=1}^\infty}
\newcommand{\sumnk}{\sum\limits_{n=1}^k}
\newtheorem*{theorem}{Теорема}
\newtheorem*{cor}{Следствие}
\newtheorem*{lem}{Лемма}

\date{}
\begin{document}
	\newpage
	\section{Сходимость и расходимость числовых рядов}
	Пусть задача числовая последовательность $a_1, a_2,\ldots, a_n,\ldots \in \mathbb R$. 
	Сумма членов этой последовательности записывается в виде
	\begin{equation}
		a_1 + a_2 + \ldots + a_n + \ldots = \sumn a_n
	\end{equation}
	и называется \textbf{числовым рядом}.
	Сумма членов последовательности до $k$ называется \textbf{частной суммой ряда}
	\begin{equation}
		S_k = a_1 + a_2 + \ldots + a_k = \sumnk a_n.
	\end{equation}
	Рассмотрим последовательность частных сумм ряда
	\begin{equation*}
		S_1 = a_1,\ S_2 = a_1 + a_2,\ \ldots,\ S_k = a_1 + a_2 + \ldots + a_k,\ \ldots.
	\end{equation*}
	Если существует конечный предел последовательности частных сумм ряда
	\begin{equation*}
		\lim\limits_{k\to \infty}S_k = S,
	\end{equation*}
	то ряд называется \textbf{сходящимся}, иначе \textbf{расходящимся}.
	\begin{theorem}
		[Необходимое условие сходимости]
		Если числовой ряд $\sumn a_n$ сходится, то $$a_n \xrightarrow[n\to \infty]{}0.$$
	\end{theorem}
	\noindent\textbf{Следствие.} Если $a_n \not\xrightarrow[n\to \infty]{}0$, то числовой ряд $\sumn a_n$ расходится.
	\begin{theorem}
		[Критерий сходимости положительных числовых рядов ($a_n > 0$)]
		Числовой ряд $\sumn a_n$, $a_n > 0$ сходится тогда и только тогда, когда последовательность частных сумм $(S_n)_{n=1}^\infty$ ограничена, то есть $|S_n| \leq M$, $\forall n$, $M = \const$.
	\end{theorem}
	\begin{theorem}
		[Признаки сравнения]
		Пусть $a_n \geq 0$, $a_n \leq b_n$, $\forall n \in \mathbb N$. Тогда
		\begin{enumerate}
			\item Из сходимости $\sumn b_n$ следует сходимость $\sumn a_n$;
			\item Из расходимости $\sumn a_n$ следует расходимость $\sumn b_n$.
		\end{enumerate}
	\end{theorem}
	\begin{theorem}
		[Предельный признак сравнения]
		Пусть $b_n > 0$. Если \begin{equation}
			\lim\limits_{n\to\infty} \dfrac{a_n}{b_n} = l,\ 0 < l < +\infty,
		\end{equation}
		то $\sumn a_n$ и $\sumn b_n$ сходятся или расходятся одновременно.
	\end{theorem}
	\section*{Практика}
	\textbf{2546.}
	$$1 - \frac{1}{2} + \frac{1}{4} - \frac{1}{8} + \dots + \frac{(-1)^{n-1}}{2^{n-1}} + \dots$$ Перед нами бесконечно убывающая геометрическая прогрессия. Следовательно, используя формулу для такой прогрессии, сразу найдем сумму ряда
	$$\sum_{n=1}^{\infty} \frac{(-1)^{n-1}}{2^{n-1}} = \left[ \frac{b_1}{1-q} \right] = \frac{1}{1 - \left(-\frac{1}{2}\right)} = \frac{1}{\frac{3}{2}} = \frac{2}{3}.$$
	\textbf{2547.}
	$$\left(\frac{1}{2} + \frac{1}{3}\right) + \left(\frac{1}{2^2} + \frac{1}{3^2}\right) + \dots + \left(\frac{1}{2^n} + \frac{1}{3^n}\right) + \dots$$ 
	В данном случае имеем сумму двух бесконечно убывающих геометрических прогрессии:
	$$\sum_{n=1}^{\infty} \frac{1}{2^n} + \sum_{n=1}^{\infty} \frac{1}{3^n} = \left[ \frac{b_1}{1-q} \right] = \frac{\frac{1}{2}}{1-\frac{1}{2}} + \frac{\frac{1}{3}}{1-\frac{1}{3}} = 1 + \frac{1}{2} = \frac{3}{2}.$$
	\textbf{2549.}$$
	\frac{1}{1 \cdot 2} + \frac{1}{2 \cdot 3} + \frac{1}{3 \cdot 4} + \dots + \frac{1}{n(n+1)} + \dots = \sum_{n=1}^{\infty} \frac{1}{n(n+1)}
	$$
	Разложим $n$-ый член последовательности $a_n$ на простейшие дроби:
	$$\dfrac{A}{n} + \dfrac{B}{n+1} = \left[ A(n+1) + B \cdot n = 1 \Rightarrow \begin{cases} A+B=0 \\ A=1 \end{cases} \Rightarrow \begin{cases} A=1 \\ B=-1 \end{cases} \right] = \frac{1}{n} - \frac{1}{n+1}.
	$$ 
	$$\sum_{n=1}^{\infty} \left( \frac{1}{n} - \frac{1}{n+1} \right) = \left(1 - \frac{1}{2}\right)+\left(\frac{1}{2}-\frac{1}{3}\right)+\dots+\left(\frac{1}{n}-\frac{1}{n+1}\right) + \dots.
	$$ 
	Тогда рассмотрим частную сумму ряда
	$$
	S_k = \sum_{n=1}^{k} \left( \frac{1}{n} - \frac{1}{n+1} \right) = 1 - \frac{1}{k+1} \xrightarrow[k \to \infty]{} 1.
	$$ 
	То есть мы доказали, что существует конечный предел последовательности частных сумм. А значит по определению ряд сходится и его сумма равна 
	$$\sum_{n=1}^{\infty} \frac{1}{n(n+1)} = 1.$$
	\textbf{2550.}
	Обозначим исследуемый ряд
	$$\dfrac{1}{1\cdot 4} + \dfrac{1}{4\cdot 7} + \ldots + \frac{1}{(3n-2)(3n+1)} + \ldots = \sum_{n=1}^{\infty} \frac{1}{(3n-2)(3n+1)}.$$
	Разложим $a_n$ на сумму простейших дробей
	$$ \frac{A}{3n-2}+\frac{B}{3n+1} = \left[ \begin{gathered} A(3n+1) + B(3n-2) = 1 \\ \begin{cases} 3A+3B=0 \\ A-2B=1 \end{cases} \Rightarrow \begin{cases} A+B=0 \\ 3B=-1 \\ \end{cases} \Rightarrow \begin{cases} A=\frac{1}{3} \\ B=-\frac{1}{3} \end{cases} \end{gathered} \right] = \frac{1}{3(3n-2)} - \frac{1}{3(3n+1)} $$
	Тогда рассмотрим частную сумму ряда
	$$ S_k = \sum_{n=1}^{k} \left( \frac{1}{3(3n-2)} - \frac{1}{3(3n+1)} \right) = \left( \frac{1}{3} - \frac{1}{12} \right) + \left( \frac{1}{12} - \frac{1}{21} \right) + \dots $$
	$$ \dots + \left( \frac{1}{3(3k-2)} - \frac{1}{3(3k+1)} \right) = \frac{1}{3} - \frac{1}{3(3k+1)} \xrightarrow[k \to \infty]{} \frac{1}{3}.$$
	Последовательность часть ных сумм имеет конечный предел, следовательно ряд сходится его сумма равна
	$$ \sum_{n=1}^{\infty} \frac{1}{(3n-2)(3n+1)} = \frac{1}{3}.$$
	\textbf{2552.}
	$$ \sum_{n=1}^{\infty} (\sqrt{n+2} - 2\sqrt{n+1} + \sqrt{n})$$ 
	Распишем сумму по членам
	$$(\sqrt{3} - 2\sqrt{2} + 1) + (\sqrt{4} - 2\sqrt{3} + \sqrt{2}) + (\sqrt{5} - 2\sqrt{4} + \sqrt{3}) + (\sqrt{6} - 2\sqrt{5} + \sqrt{4}) + \dots $$ $$\ldots + (\sqrt{n+2} - 2\sqrt{n+1} + \sqrt{n}) + \dots $$
	Таким образом, частную сумму ряда можно записать в виде
	$$ S_k = 1 - \sqrt{2} + \sqrt{k+2} - \sqrt{k+1} = 1 - \sqrt{2} + \frac{1}{\sqrt{k+2} + \sqrt{k+1}} \xrightarrow[k \to \infty]{} 1-\sqrt{2}.$$
	Таким образом,
	$$ \sum_{n=1}^{\infty} (\sqrt{n+2} - 2\sqrt{n+1} + \sqrt{n}) = 1 - \sqrt{2}.$$
	\textbf{2556.}
	$$ 1-1+1-1+\dots = \sum_{n=1}^{\infty} (-1)^{n-1} $$
	$$ \lim_{n \to \infty} a_n = \lim_{n \to \infty} (-1)^{n-1} = \begin{cases} 1 \\ -1 \end{cases} \not\to 0, \text{предела не существ.} $$
	Тогда по необходимому условию сходимости числовой ряд расходится.
	\\\\
	\textbf{2557.}
	$$ 0.001 + \sqrt{0.001} + \sqrt[3]{0.001} + \dots = \sum_{n=1}^{\infty} (0.001)^{\frac{1}{n}}.$$
	Рассмотрим $n$-ый член последовательности
	$$ a_n = (0.001)^{\frac{1}{n}} \xrightarrow{n \to \infty} 1 \ne 0 \Rightarrow \text{расходится по необходимому условию сходимости.} $$
	\textbf{2558.}
	$$ \frac{1}{1!} + \frac{1}{2!} + \dots = \sum_{n=1}^{\infty} \frac{1}{n!} $$
	Рассмотрим частную сумму ряда
	$$ S_k = \frac{1}{1!} + \frac{1}{2!} + \dots + \frac{1}{k!} = \sum_{n=1}^{k} \frac{1}{n!}.$$
	Так как
	$$ S_{k+1} = S_k + \frac{1}{(k+1)!} \ge S_k,$$
	то последовательность монотонно возрастает. Нужно найти верхнюю грань.
	Из неравенства $$ n! \ge 2^{n-1} $$ следует, что
	$$ \frac{1}{k!} \le \frac{1}{2^{k-1}} \quad \left(n-1, \text{чтобы } \frac{1}{1!} \le \frac{1}{2^0}\right).$$
	Следовательно,
	$$ \sum_{k=1}^{n} \frac{1}{k!} \le \sum_{k=1}^{n} \frac{1}{2^{k-1}} = 1 + \frac{1}{2} + \dots + \frac{1}{2^{n-1}} = [\text{геом. прогр. }] = $$
	$$ = \left[ \frac{b_1(1-q^n)}{1-q} \right] = \frac{1 \cdot (1-\frac{1}{2^n})}{1-\frac{1}{2}} = 2 - \frac{1}{2^{n-1}} \le 2 \quad \forall n \Rightarrow |S_n| \le 2 \quad \forall n. $$
	Тогда по критерию сходимости исходный ряд сходится, так как последовательность частных сумм является ограниченной.
	\\\\
	\textbf{2559.}
	$$ 1 + \frac{1}{3} + \frac{1}{5} + \frac{1}{7} + \dots + \frac{1}{2n-1} + \dots = \sum_{n=1}^{\infty} \frac{1}{2n-1} $$
	Используем признак сравнения:
	$$ \sum_{n=1}^{\infty} \frac{1}{2n-1} \ge \sum_{n=1}^{\infty} \frac{1}{2n} = \frac{1}{2} \sum_{n=1}^{\infty} \frac{1}{n} $$
	Необходимо доказать расходимость гармонического ряда.
	Доказательство Орема:
	$$ \sum \frac{1}{n} = 1 + \left[\frac{1}{2}\right] + \left[\frac{1}{3} + \frac{1}{4}\right] + \left[\frac{1}{5} + \frac{1}{6} + \frac{1}{7} + \frac{1}{8}\right] + \dots $$
	$$ > 1 + \left[\frac{1}{2}\right] + \left[\frac{1}{4} + \frac{1}{4}\right] + \left[\frac{1}{8} + \frac{1}{8} + \frac{1}{8} + \frac{1}{8}\right] + \left[\frac{1}{16} + \dots\right] + \dots = $$
	$$ = 1 + \frac{1}{2} + \frac{1}{2} + \dots \text{ - не ограничена сверху } \Rightarrow \sum \frac{1}{n} \text{ - расх. }.$$
	Значит по признаку сравнения исходный ряд расходится.
	\\\\
	Альтернатива: Из книги Кастрицы:
	$$ \begin{gathered} S_{2n} = \sum_{k=1}^{2n} \frac{1}{k} = S_n + \frac{1}{n+1} + \dots + \frac{1}{2n} > S_n + n \cdot \frac{1}{2n} = S_n + \frac{1}{2} \end{gathered} ,$$
	тогда при n $\to \infty$
	$$ S \ge S + \frac{1}{2} \quad - \text{противоречие}.$$
	\textbf{2560.}
	$$ \frac{1}{1001} + \frac{1}{2001} + \dots + \frac{1}{1000n+1} + \dots = \sum_{n=1}^{\infty} \frac{1}{1000n+1}.$$
	Поскольку
	$$ 1000n+1 \le 2000n \Rightarrow \frac{1}{1000n+1} \ge \frac{1}{2000n},$$
	то можем рассмотреть ряд
	$$ \sum_{n=1}^{\infty} \frac{1}{2000n} = \frac{1}{2000} \sum_{n=1}^{\infty} \frac{1}{n}.$$ Этот ряд расходится как гармонический. Следовательно по признаку сравнения исходный ряд расходится.
	\\\\
	\textbf{2561.}
	$$ 1 + \frac{2}{3} + \frac{3}{5} + \dots + \frac{n}{2n-1} + \dots = \sum_{n=1}^{\infty} \frac{n}{2n-1}.$$
	Рассмотрим $a_n$ член
	$$ \frac{n}{2n-1} = \frac{1}{2-\frac{1}{n}} \xrightarrow[n \to \infty]{} \frac{1}{2} \ne 0.$$ А тогда по необходимому условию данный ряд расходится.
	\\\\
	\textbf{2562.}
	$$ 1 + \frac{1}{3^2} +\dfrac{1}{5^2} + \dots + \frac{1}{(2n-1)^2} + \dots = \sum_{n=1}^{\infty} \frac{1}{(2n-1)^2} $$
	Возьмем ряд $$ \sum_{n=1}^{\infty} \frac{1}{n^2} $$ 
	Рассмотрим отношение:
	$$ \frac{n^2}{(2n-1)^2} = \frac{1}{4-\frac{4}{n}+\frac{1}{n^2}} \xrightarrow[n \to \infty]{} \frac{1}{4} = \text{const} $$
	А тогда по предельному признаку сравнения ряды $ \sum\limits_{n=1}^{\infty} \dfrac{1}{(2n-1)^2} $ и $ \sum\limits_{n=1}^{\infty} \dfrac{1}{n^2} $ сходятся или расходятся одновременно.
	Докажем сходимость $ \sum\limits_{n=1}^{\infty} \dfrac{1}{n^2} $. Для этого рассмотрим ряд $ \sum\limits_{n=2}^{\infty} \dfrac{1}{n(n-1)} $
	По аналогии с номером 2549
	$$ \frac{1}{n(n-1)} = \left[ \begin{gathered} A(n-1) + Bn = 1 \\ \begin{cases} A+B=0 \\ -A=1 \end{cases} \Rightarrow \begin{cases} A=-1 \\ B=1 \end{cases} \end{gathered} \right] = \frac{1}{n-1} - \frac{1}{n} \Rightarrow $$
	$$ S_k = \sum_{n=2}^{k} \left( \frac{1}{n-1} - \frac{1}{n} \right) = \left( \frac{1}{1} - \frac{1}{2} \right) + \left( \frac{1}{2} - \frac{1}{3} \right) + \dots + \left( \frac{1}{k-1} - \frac{1}{k} \right) = $$
	$$ = 1 - \frac{1}{k} \xrightarrow[k \to \infty]{} 1 \Rightarrow \text{ряд сходится} $$
	При этом
	$$ \frac{1}{n(n-1)} > \frac{1}{n^2} \text{, т.к. } n(n-1) < n^2 \quad \forall n \ge 2 $$
	а тогда ряд
	$$ \sum_{n=2}^{\infty} \frac{1}{n^2} \text{ сходится.} $$
	Причем, если добавить к нему единицу, то на сходимость это не повлияет:
	$$ 1 + \sum_{n=2}^{\infty} \frac{1}{n^2} = \sum_{n=1}^{\infty} \frac{1}{n^2}.$$
	Таким образом, исходный ряд сходится по предельному признаку сравнения.
	\\\\
	\textbf{2563.}
	$$ \frac{1}{1\sqrt{2}} + \frac{1}{2\sqrt{3}} + \frac{1}{3\sqrt{4}} + \dots + \frac{1}{n\sqrt{n+1}} + \dots = \sum_{n=1}^{\infty} \frac{1}{n\sqrt{n+1}}.$$
	Рассмотрим ряд
	$$ \sum_{n=1}^{\infty} \left( \frac{1}{\sqrt{n}} - \frac{1}{\sqrt{n+1}} \right) $$
	По аналогии с 2549
	$$ S_k = \left(\frac{1}{1} - \frac{1}{\sqrt{2}}\right) + \left(\frac{1}{\sqrt{2}} - \frac{1}{\sqrt{3}}\right) + \dots + \left(\frac{1}{\sqrt{k}} - \frac{1}{\sqrt{k+1}}\right) \xrightarrow[k \to \infty]{} 1 \Rightarrow \text{сходится.} $$
	Теперь сравним его с исходным рядом.
	Сперва преобразуем:
	$$ \frac{1}{\sqrt{n}} - \frac{1}{\sqrt{n+1}} = \frac{\sqrt{n+1}-\sqrt{n}}{\sqrt{n}\sqrt{n+1}} = \frac{n+1-n}{\sqrt{n}\sqrt{n+1}(\sqrt{n+1}+\sqrt{n})} = $$
	$$ = \frac{1}{\sqrt{n}\sqrt{n+1}(\sqrt{n+1}+\sqrt{n})} $$
	Разделим исходный ряд на получившийся:
	$$ \frac{\sqrt{n}\sqrt{n+1}(\sqrt{n+1}+\sqrt{n})}{n\sqrt{n+1}} = \sqrt{1+\frac{1}{n}} + 1 \xrightarrow[n \to \infty]{} 2 = \text{const} $$
	Таким образом, по предельному признаку сравнения исходный ряд сходится.
	\section{Признаки Коши и Даламбера}
	\begin{theorem}
		[Радикальный признак Коши]
		Пусть для ряда $\sum_{n=1}^{\infty} a_n$ существует предел $$L = \lim_{n \to \infty} \sqrt[n]{|a_n|}.$$
		\begin{itemize}
			\item Если $L < 1$, то ряд сходится абсолютно.
			\item Если $L > 1$, то ряд расходится.
			\item Если $L = 1$, то признак не дает ответа.
		\end{itemize}
	\end{theorem}
	\begin{theorem}
		[Признак Даламбера]
		Пусть для ряда $\sum_{n=1}^{\infty} a_n$ с ненулевыми членами существует предел $$L = \lim_{n \to \infty} \left| \frac{a_{n+1}}{a_n} \right|.$$
		\begin{itemize}
			\item Если $L < 1$, то ряд сходится абсолютно.
			\item Если $L > 1$, то ряд расходится.
			\item Если $L = 1$, то признак не даёт ответа.
		\end{itemize}
	\end{theorem}
	\section*{Практика}
	\textbf{2578.}
	$$\sum_{n=1}^{\infty} \frac{1000^n}{n!} $$
	Для исследования сходимости данного ряда воспользуемся признаком Даламбера.Обозначим общий член ряда как $$a_n = \frac{1000^n}{n!}.$$
	{Тогда следующий член ряда будет:}
	$$a_{n+1} = \frac{1000^{n+1}}{(n+1)!}.$$
	{Вычислим предел отношения следующего члена к предыдущему:}
	$$\frac{a_{n+1}}{a_n} = \frac{\frac{1000^{n+1}}{(n+1)!}}{\frac{1000^n}{n!}} =  \frac{1000^{n+1}}{(n+1)!} \cdot \frac{n!}{1000^n}  =  \frac{1000^n \cdot 1000}{(n+1) \cdot n!} \cdot \frac{n!}{1000^n} = \frac{1000}{n+1} \xrightarrow[n\to\infty]{} 0. $$
	{Так как полученный предел } $q = 0$, \text{ и } $q < 1$, {то, согласно признаку Даламбера, исходный ряд сходится.}
	\\\\
	\textbf{2579.}
	$$\sum_{n=1}^{\infty} \frac{(n!)^2}{(2n)!}.$$
	Применим признак Даламбера.
	Общий член ряда $$a_n = \frac{(n!)^2}{(2n)!}.$$
	Тогда следующий член ряда:
	$$ a_{n+1} = \frac{((n+1)!)^2}{(2(n+1))!} = \frac{((n+1)!)^2}{(2n+2)!} $$
	Найдём предел отношения $\frac{u_{n+1}}{u_n}$:
	$$ q = \lim_{n \to \infty} \frac{u_{n+1}}{u_n} = \lim_{n \to \infty} \frac{\frac{((n+1)!)^2}{(2n+2)!}}{\frac{(n!)^2}{(2n)!}} = \lim_{n \to \infty} \left( \frac{((n+1)!)^2}{(2n+2)!} \cdot \frac{(2n)!}{(n!)^2} \right) = $$
	Используя свойства факториала $(n+1)! = (n+1) \cdot n!$ и $(2n+2)! = (2n+2)(2n+1)(2n)!$, упростим выражение:
	$$ = \lim_{n \to \infty} \frac{((n+1) \cdot n!)^2}{(2n+2)(2n+1)(2n)!} \cdot \frac{(2n)!}{(n!)^2} = \lim_{n \to \infty} \frac{(n+1)^2 (n!)^2}{(2n+2)(2n+1)(2n)!} \cdot \frac{(2n)!}{(n!)^2} = $$
	После сокращения $(n!)^2$ и $(2n)!$ получаем:
	$$ = \lim_{n \to \infty} \frac{(n+1)^2}{(2n+2)(2n+1)} = \lim_{n \to \infty} \frac{n^2 + 2n + 1}{4n^2 + 6n + 2} = $$
	Разделим числитель и знаменатель на старшую степень $n^2$:
	$$ = \lim_{n \to \infty} \frac{1 + \frac{2}{n} + \frac{1}{n^2}}{4 + \frac{6}{n} + \frac{2}{n^2}} = \frac{1+0+0}{4+0+0} = \frac{1}{4}. $$
	Так как предел $q = \frac{1}{4} < 1$, то по признаку Даламбера исходный ряд сходится.
	\\\\
	\textbf{2580.}
	$$\sum_{n=1}^{\infty} \frac{n!}{n^n}.$$
	Применим признак Даламбера.
	Общий член ряда $$a_n = \frac{n!}{n^n}$$
	Тогда следующий член ряда:
	$$ a_{n+1} = \frac{(n+1)!}{(n+1)^{n+1}} $$
	Найдём предел отношения $\frac{a_{n+1}}{a_n}$:
	$$ q = \lim_{n \to \infty} \frac{a_{n+1}}{a_n} = \lim_{n \to \infty} \frac{\frac{(n+1)!}{(n+1)^{n+1}}}{\frac{n!}{n^n}} = \lim_{n \to \infty} \left( \frac{(n+1)!}{(n+1)^{n+1}} \cdot \frac{n^n}{n!} \right) = $$
	Используя свойства факториала $(n+1)! = (n+1) \cdot n!$ и степеней $(n+1)^{n+1} = (n+1) \cdot (n+1)^n$, упростим выражение:
	$$ = \lim_{n \to \infty} \left( \frac{(n+1) \cdot n!}{(n+1) \cdot (n+1)^n} \cdot \frac{n^n}{n!} \right) = $$
	После сокращения $(n+1)$ и $n!$ получаем:
	$$ = \lim_{n \to \infty} \frac{n^n}{(n+1)^n} = \lim_{n \to \infty} \left(\frac{n}{n+1}\right)^n = \lim_{n \to \infty} \left(\frac{1}{\frac{n+1}{n}}\right)^n = $$
	$$ = \lim_{n \to \infty} \frac{1}{\left(1+\frac{1}{n}\right)^n} = \frac{1}{\lim_{n \to \infty} \left(1+\frac{1}{n}\right)^n} = \frac{1}{e}. $$
	Здесь мы использовали второй замечательный предел: $\lim_{n \to \infty} \left(1+\frac{1}{n}\right)^n = e$.
	Так как предел $q = \frac{1}{e} < 1$ (поскольку $e \approx 2.718$), то по признаку Даламбера исходный ряд сходится.
	\\\\
	\textbf{2582.}
	$$\sum_{n=1}^{\infty} \frac{(n!)^2}{2^{n^2}}.$$
	Применим признак Даламбера.
	Общий член ряда $a_n = \frac{(n!)^2}{2^{n^2}}$.
	Тогда следующий член ряда:
	$$ a_{n+1} = \frac{((n+1)!)^2}{2^{(n+1)^2}} $$
	Найдём предел отношения $\frac{a_{n+1}}{a_n}$:
	$$ q = \lim_{n \to \infty} \frac{a_{n+1}}{a_n} = \lim_{n \to \infty} \frac{\frac{((n+1)!)^2}{2^{(n+1)^2}}}{\frac{(n!)^2}{2^{n^2}}} = \lim_{n \to \infty} \left( \frac{((n+1)!)^2}{2^{(n+1)^2}} \cdot \frac{2^{n^2}}{(n!)^2} \right) = $$
	Сгруппируем члены с факториалами и степенями:
	$$ = \lim_{n \to \infty} \left( \frac{((n+1)!)^2}{(n!)^2} \cdot \frac{2^{n^2}}{2^{(n+1)^2}} \right) = \lim_{n \to \infty} \left( \left(\frac{(n+1) \cdot n!}{n!}\right)^2 \cdot 2^{n^2 - (n+1)^2} \right) = $$
	Упростим выражение в показателе степени: $n^2 - (n+1)^2 = n^2 - (n^2 + 2n + 1) = -2n - 1$.
	$$ = \lim_{n \to \infty} \left( (n+1)^2 \cdot 2^{-2n-1} \right) = \lim_{n \to \infty} \frac{(n+1)^2}{2^{2n+1}} = 0. $$
	Предел равен нулю, так как экспоненциальная функция в знаменателе $2^{2n+1}$ растет быстрее любой степенной функции в числителе $(n+1)^2$.
	Так как предел $q = 0 < 1$, то по признаку Даламбера исходный ряд сходится.
	\\\\
	\textbf{2583.}
	$$ \frac{1000}{1} + \frac{1000 \cdot 1001}{1 \cdot 3} + \frac{1000 \cdot 1001 \cdot 1002}{1 \cdot 3 \cdot 5} + \dots $$
	Сначала запишем общий член ряда $a_n$.
	Числитель $n$-го члена представляет собой произведение $n$ чисел, начиная с 1000: $1000 \cdot 1001 \cdot \dots \cdot (1000 + n - 1)$.
	Знаменатель $n$-го члена представляет собой произведение первых $n$ нечётных чисел: $1 \cdot 3 \cdot 5 \cdot \dots \cdot (2n-1)$.
	Таким образом, общий член ряда имеет вид:
	$$ a_n = \frac{1000 \cdot 1001 \cdot \dots \cdot (1000 + n - 1)}{1 \cdot 3 \cdot 5 \cdot \dots \cdot (2n - 1)} $$
	Применим признак Даламбера. Для этого запишем следующий член ряда $a_{n+1}$:
	$$ a_{n+1} = \frac{1000 \cdot 1001 \cdot \dots \cdot (1000 + n - 1) \cdot (1000 + n)}{1 \cdot 3 \cdot 5 \cdot \dots \cdot (2n - 1) \cdot (2n + 1)} $$
	Найдём предел отношения $\frac{a_{n+1}}{a_n}$:
	$$ L = \lim_{n \to \infty} \frac{a_{n+1}}{a_n} = \lim_{n \to \infty} \frac{\frac{1000 \cdot \dots \cdot (1000 + n)}{1 \cdot \dots \cdot (2n + 1)}}{\frac{1000 \cdot \dots \cdot (1000 + n - 1)}{1 \cdot \dots \cdot (2n - 1)}} = $$
	Большинство множителей в числителе и знаменателе сокращаются:
	$$ = \lim_{n \to \infty} \frac{1000+n}{2n+1} = $$
	Чтобы найти предел, разделим числитель и знаменатель на $n$:
	$$ = \lim_{n \to \infty} \frac{\frac{1000}{n} + 1}{2 + \frac{1}{n}} = \frac{0 + 1}{2 + 0} = \frac{1}{2}. $$
	Так как предел $q = \frac{1}{2} < 1$, то по признаку Даламбера исходный ряд сходится.
	\\\\
	\textbf{2584.}
	$$\sum_{n=1}^{\infty} \frac{4 \cdot 7 \cdot 10 \cdots (3n+1)}{2 \cdot 6 \cdot 10 \cdots (4n-2)}$$
	Для исследования сходимости данного ряда воспользуемся признаком Даламбера. Обозначим общий член ряда как
	$$a_n = \frac{4 \cdot 7 \cdot 10 \cdots (3n+1)}{2 \cdot 6 \cdot 10 \cdots (4n-2)}.$$
	Тогда следующий член ряда будет:
	$$a_{n+1} = \frac{4 \cdot 7 \cdot 10 \cdots (3n+1)(3n+4)}{2 \cdot 6 \cdot 10 \cdots (4n-2)(4n+2)}.$$
	Вычислим предел отношения следующего члена к предыдущему:
	$$ \lim_{n \to \infty} \frac{a_{n+1}}{a_n} = \lim_{n \to \infty} \frac{\frac{4 \cdot 7 \cdots (3n+1)(3n+4)}{2 \cdot 6 \cdots (4n-2)(4n+2)}}{\frac{4 \cdot 7 \cdots (3n+1)}{2 \cdot 6 \cdots (4n-2)}} = \lim_{n \to \infty} \frac{3n+4}{4n+2} = \lim_{n \to \infty} \frac{3 + 4/n}{4 + 2/n} = \frac{3}{4}. $$
	Так как полученный предел $q = \frac{3}{4}$, и $q < 1$, то, согласно признаку Даламбера, исходный ряд сходится.
	\\\\
	\textbf{2585.}
	$$\sum_{n=1}^{\infty} (\sqrt{2} - \sqrt[3]{2})(\sqrt{2} - \sqrt[5]{2}) \cdots (\sqrt{2} - \sqrt[2n+1]{2})$$
	Для исследования сходимости данного ряда воспользуемся признаком Даламбера. Обозначим общий член ряда как
	$$a_n = (\sqrt{2} - \sqrt[3]{2})(\sqrt{2} - \sqrt[5]{2}) \cdots (\sqrt{2} - \sqrt[2n+1]{2}).$$
	Тогда следующий член ряда будет:
	$$a_{n+1} = (\sqrt{2} - \sqrt[3]{2}) \cdots (\sqrt{2} - \sqrt[2n+1]{2}) (\sqrt{2} - \sqrt[2n+3]{2}).$$
	Вычислим предел отношения следующего члена к предыдущему:
	$$ \lim_{n \to \infty} \frac{a_{n+1}}{a_n} = \lim_{n \to \infty} \frac{(\sqrt{2} - \sqrt[3]{2}) \cdots (\sqrt{2} - \sqrt[2n+1]{2}) (\sqrt{2} - \sqrt[2n+3]{2})}{(\sqrt{2} - \sqrt[3]{2}) \cdots (\sqrt{2} - \sqrt[2n+1]{2})} = \lim_{n \to \infty} (\sqrt{2} - \sqrt[2n+3]{2}). $$
	Поскольку $\lim_{n \to \infty} \sqrt[2n+3]{2} = \lim_{n \to \infty} 2^{\frac{1}{2n+3}} = 2^0 = 1$, предел равен:
	$$ \lim_{n \to \infty} (\sqrt{2} - \sqrt[2n+3]{2}) = \sqrt{2} - 1. $$
	Так как полученный предел $q = \sqrt{2} - 1 \approx 0.414$, и $q < 1$, то, согласно признаку Даламбера, исходный ряд сходится.
	\\\\
	\textbf{2586.}
	$$\sum_{n=1}^{\infty} \frac{n^2}{\left(2+\frac{1}{n}\right)^n}$$
	Для исследования сходимости данного ряда воспользуемся радикальным признаком Коши. Обозначим общий член ряда как
	$$a_n = \frac{n^2}{\left(2+\frac{1}{n}\right)^n}.$$
	Вычислим предел корня n-ой степени из общего члена ряда:
	$$ \lim_{n \to \infty} \sqrt[n]{a_n} = \lim_{n \to \infty} \sqrt[n]{\frac{n^2}{\left(2+\frac{1}{n}\right)^n}} = \lim_{n \to \infty} \frac{\sqrt[n]{n^2}}{2+\frac{1}{n}} = \lim_{n \to \infty} \frac{(\sqrt[n]{n})^2}{2+\frac{1}{n}}. $$
	Поскольку $\lim_{n \to \infty} \sqrt[n]{n} = 1$, предел равен:
	$$ \frac{1^2}{2+0} = \frac{1}{2}. $$
	Так как полученный предел $q = \frac{1}{2}$, и $q < 1$, то, согласно признаку Коши, исходный ряд сходится.
	\\\\
	\textbf{Доказательство.}
	$$\lim_{n \to \infty} \sqrt[n]{n} = 1$$
	Рассмотрим предел $L = \lim_{n \to \infty} n^{\frac{1}{n}}$.
	Для его вычисления воспользуемся свойством непрерывности логарифмической и показательной функций. Прологарифмируем выражение под знаком предела:
	$$ \ln\left(n^{\frac{1}{n}}\right) = \frac{1}{n} \ln(n) = \frac{\ln(n)}{n}. $$
	Теперь найдем предел этого выражения при $n \to \infty$. Мы имеем неопределенность вида $\left[\frac{\infty}{\infty}\right]$, поэтому можем применить правило Лопиталя.
	$$ \lim_{n \to \infty} \frac{\ln(n)}{n} = \lim_{n \to \infty} \frac{(\ln(n))'}{(n)'} = \lim_{n \to \infty} \frac{\frac{1}{n}}{1} = \lim_{n \to \infty} \frac{1}{n} = 0. $$
	Мы нашли, что предел логарифма исходного выражения равен 0. Чтобы найти исходный предел $L$, мы потенцируем полученный результат:
	$$ L = e^{\lim_{n \to \infty} \ln(n^{\frac{1}{n}})} = e^0 = 1. $$
	Таким образом, доказано, что
	$$ \lim_{n \to \infty} \sqrt[n]{n} = 1. $$
	\textbf{2587.}
	$$\sum_{n=1}^{\infty} \frac{n^{n+1/n}}{\left(n+\frac{1}{n}\right)^n}$$
	Для исследования сходимости данного ряда воспользуемся радикальным признаком Коши. Обозначим общий член ряда как
	$$a_n = \frac{n^{n+1/n}}{\left(n+\frac{1}{n}\right)^n}.$$
	Вычислим предел корня n-ой степени из общего члена ряда:
	$$ \lim_{n \to \infty} \sqrt[n]{a_n} = \lim_{n \to \infty} \sqrt[n]{\frac{n^{n+1/n}}{\left(n+\frac{1}{n}\right)^n}} = \lim_{n \to \infty} \frac{n^{(n+1/n) \cdot 1/n}}{n+\frac{1}{n}} = \lim_{n \to \infty} \frac{n^{1+1/n^2}}{n+\frac{1}{n}}. $$
	Разделив числитель и знаменатель на $n$, получим:
	$$ \lim_{n \to \infty} \frac{n \cdot n^{1/n^2}}{n(1+\frac{1}{n^2})} = \lim_{n \to \infty} \frac{\sqrt[n^2]{n}}{1+\frac{1}{n^2}} = \frac{1}{1} = 1. $$
	Так как полученный предел $q = 1$, то признак Коши не даёт ответа.
	Проверим выполнение необходимого признака сходимости: $\lim_{n \to \infty} a_n = 0$. Преобразуем общий член ряда:
	$$ a_n = \frac{n^n \cdot n^{1/n}}{\left(n\left(1+\frac{1}{n^2}\right)\right)^n} = \frac{n^n \cdot \sqrt[n]{n}}{n^n \left(1+\frac{1}{n^2}\right)^n} = \frac{\sqrt[n]{n}}{\left(1+\frac{1}{n^2}\right)^n}. $$
	Найдем его предел:
	$$ \lim_{n \to \infty} a_n = \frac{\lim_{n \to \infty} \sqrt[n]{n}}{\lim_{n \to \infty} \left(1+\frac{1}{n^2}\right)^n} = \frac{1}{\lim_{n \to \infty} \left[\left(1+\frac{1}{n^2}\right)^{n^2}\right]^{1/n}} = \frac{1}{e^0} = 1. $$
	Так как предел общего члена ряда не равен нулю ($\lim_{n \to \infty} a_n = 1 \neq 0$), необходимый признак сходимости не выполняется, следовательно, ряд расходится.
	\\\\
	\textbf{2588.}
	$$\sum_{n=2}^{\infty} \frac{1}{\sqrt[n]{\ln n}}$$
	Для доказательства расходимости ряда проверим выполнение необходимого признака сходимости. Обозначим общий член ряда как
	$$a_n = \frac{1}{\sqrt[n]{\ln n}}.$$
	Вычислим предел общего члена ряда при $n \to \infty$:
	$$ \lim_{n \to \infty} a_n = \lim_{n \to \infty} \frac{1}{(\ln n)^{1/n}}. $$
	Рассмотрим предел знаменателя $L = \lim_{n \to \infty} (\ln n)^{1/n}$. Это неопределенность вида $\left[\infty^0\right]$. Прологарифмируем его:
	$$ \lim_{n \to \infty} \ln\left((\ln n)^{1/n}\right) = \lim_{n \to \infty} \frac{\ln(\ln n)}{n}. $$
	Применяя правило Лопиталя для неопределенности вида $\left[\frac{\infty}{\infty}\right]$:
	$$ \lim_{n \to \infty} \frac{(\ln(\ln n))'}{(n)'} = \lim_{n \to \infty} \frac{\frac{1}{n \ln n}}{1} = 0. $$
	Следовательно, предел знаменателя равен $L = e^0 = 1$. Тогда предел общего члена ряда:
	$$ \lim_{n \to \infty} a_n = \frac{1}{1} = 1. $$
	Так как предел общего члена ряда не равен нулю ($\lim_{n \to \infty} a_n = 1 \neq 0$), необходимый признак сходимости не выполняется, следовательно, ряд расходится.
	\\\\
	\textbf{2589.}
	$$\sum_{n=1}^{\infty} \frac{n^{n-1}}{(2n^2+n+1)^{n+1/2}}$$
	Для исследования сходимости данного ряда воспользуемся радикальным признаком Коши. Обозначим общий член ряда как
	$$a_n = \frac{n^{n-1}}{(2n^2+n+1)^{n+1/2}}.$$
	Вычислим предел корня n-ой степени из общего члена ряда:
	$$ \lim_{n \to \infty} \sqrt[n]{a_n} = \lim_{n \to \infty} \sqrt[n]{\frac{n^{n-1}}{(2n^2+n+1)^{n+1/2}}} = \lim_{n \to \infty} \frac{n^{(n-1)/n}}{(2n^2+n+1)^{(n+1/2)/n}}. $$
	Упростим степени в числителе и знаменателе:
	$$ \lim_{n \to \infty} \frac{n^{1-1/n}}{(2n^2+n+1)^{1+1/(2n)}} = \lim_{n \to \infty} \frac{n \cdot n^{-1/n}}{(2n^2+n+1) \cdot (2n^2+n+1)^{1/(2n)}}. $$
	Разделим предел на произведение нескольких пределов:
	$$ \lim_{n \to \infty} \left(\frac{n}{2n^2+n+1}\right) \cdot \lim_{n \to \infty} \frac{1}{n^{1/n}} \cdot \lim_{n \to \infty} \frac{1}{(2n^2+n+1)^{1/(2n)}}. $$
	Вычислим каждый из них по отдельности:
	$$ \lim_{n \to \infty} \frac{n}{2n^2+n+1} = 0. $$
	$$ \lim_{n \to \infty} \frac{1}{\sqrt[n]{n}} = \frac{1}{1} = 1. $$
	$$ \lim_{n \to \infty} \frac{1}{(2n^2+n+1)^{1/(2n)}} = 1, \text{ поскольку это предел вида } \frac{1}{\infty^0} \text{ и } \lim_{n \to \infty} (\text{полином})^{1/n} = 1. $$
	Предел всего произведения равен:
	$$ q = 0 \cdot 1 \cdot 1 = 0. $$
	Так как полученный предел $q = 0$, и $q < 1$, то, согласно признаку Коши, исходный ряд сходится.
	\\\\
	\textbf{2589.1.}
	$$\sum_{n=1}^{\infty} \frac{n^5}{2^n + 3^n}$$
	Для исследования сходимости воспользуемся признаком сравнения. Оценим общий член ряда $a_n = \frac{n^5}{2^n + 3^n}$.
	Так как $2^n + 3^n > 3^n$, то
	$$ \frac{n^5}{2^n + 3^n} < \frac{n^5}{3^n}. $$
	Исследуем на сходимость ряд $\sum_{n=1}^{\infty} \frac{n^5}{3^n}$ с помощью радикального признака Коши.
	$$ q = \lim_{n \to \infty} \sqrt[n]{\frac{n^5}{3^n}} = \lim_{n \to \infty} \frac{\sqrt[n]{n^5}}{\sqrt[n]{3^n}} = \lim_{n \to \infty} \frac{(\sqrt[n]{n})^5}{3}. $$
	Поскольку $\lim_{n \to \infty} \sqrt[n]{n} = 1$, предел равен:
	$$ q = \frac{1^5}{3} = \frac{1}{3}. $$
	Так как $q < 1$, ряд $\sum_{n=1}^{\infty} \frac{n^5}{3^n}$ сходится. Следовательно, по признаку сравнения, исходный ряд $\sum_{n=1}^{\infty} \frac{n^5}{2^n + 3^n}$ также сходится.
	\\\\
	\textbf{2589.2.}
	$$\sum_{n=2}^{\infty} \left(\frac{n-1}{n+1}\right)^{n(n-1)}$$
	Для исследования сходимости воспользуемся радикальным признаком Коши. Обозначим общий член ряда как
	$$a_n = \left(\frac{n-1}{n+1}\right)^{n(n-1)}.$$
	Вычислим предел корня n-ой степени из общего члена ряда:
	$$ q = \lim_{n \to \infty} \sqrt[n]{a_n} = \lim_{n \to \infty} \sqrt[n]{\left(\frac{n-1}{n+1}\right)^{n(n-1)}} = \lim_{n \to \infty} \left(\frac{n-1}{n+1}\right)^{n-1}. $$
	Мы имеем неопределенность вида $\left[1^{\infty}\right]$. Преобразуем выражение, чтобы использовать второй замечательный предел:
	$$ \lim_{n \to \infty} \left(\frac{n+1-2}{n+1}\right)^{n-1} = \lim_{n \to \infty} \left(1-\frac{2}{n+1}\right)^{n-1} = \lim_{n \to \infty} \left(1+\frac{-2}{n+1}\right)^{(n+1)\frac{n-1}{n+1}}. $$
	Поскольку $\lim_{n \to \infty} \left(1+\frac{-2}{n+1}\right)^{n+1} = e^{-2}$ и $\lim_{n \to \infty} \frac{n-1}{n+1} = 1$, предел равен:
	$$ q = (e^{-2})^1 = e^{-2} = \frac{1}{e^2}. $$
	Так как полученный предел $q = \frac{1}{e^2} \approx \frac{1}{7.389} < 1$, то, согласно признаку Коши, исходный ряд сходится.
	\\\\
	\textbf{2595.}
	$$\sum_{n=1}^{\infty} \frac{2+(-1)^n}{2^n}$$
	Для исследования сходимости воспользуемся признаком сравнения. Оценим числитель общего члена $a_n = \frac{2+(-1)^n}{2^n}$.
	Так как $(-1)^n$ принимает значения $1$ (для четных $n$) и $-1$ (для нечетных $n$), то
	$$ 1 \le 2+(-1)^n \le 3. $$
	Следовательно, для общего члена ряда справедливо неравенство:
	$$ a_n = \frac{2+(-1)^n}{2^n} \le \frac{3}{2^n}. $$
	Рассмотрим ряд $\sum_{n=1}^{\infty} \frac{3}{2^n} = 3\sum_{n=1}^{\infty} \left(\frac{1}{2}\right)^n$. Это сходящийся геометрический ряд, так как его знаменатель $q = \frac{1}{2} < 1$.
	Поскольку члены исходного ряда меньше членов сходящегося ряда, то, согласно признаку сравнения, исходный ряд сходится.
	\\\\
	\textbf{2596.}
	$$\sum_{n=1}^{\infty} \frac{a \cos^2(n\pi/3)}{2^n}$$
	Для исследования сходимости воспользуемся признаком сравнения. Оценим общий член ряда $a_n = \frac{a \cos^2(n\pi/3)}{2^n}$.
	Так как функция косинуса ограничена, для ее квадрата справедливо неравенство:
	$$ 0 \le \cos^2(n\pi/3) \le 1. $$
	Тогда для модуля общего члена ряда (при $a \neq 0$) имеем:
	$$ |a_n| = \left|\frac{a \cos^2(n\pi/3)}{2^n}\right| = \frac{|a| \cos^2(n\pi/3)}{2^n} \le \frac{|a|}{2^n}. $$
	Рассмотрим ряд $\sum_{n=1}^{\infty} \frac{|a|}{2^n} = |a|\sum_{n=1}^{\infty} \left(\frac{1}{2}\right)^n$. Этот ряд является сходящимся геометрическим рядом со знаменателем $q = \frac{1}{2} < 1$.
	Поскольку исходный ряд сходится абсолютно (его члены по модулю меньше членов сходящегося ряда), он сходится.
\end{document}