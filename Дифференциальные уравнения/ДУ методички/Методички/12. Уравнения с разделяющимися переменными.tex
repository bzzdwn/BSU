\documentclass[a4paper, 12pt]{article}
\usepackage{cmap}
\usepackage{amssymb}
\usepackage{amsmath}
\usepackage{graphicx}
\usepackage{amsthm}
\usepackage{upgreek}
\usepackage{setspace}
\usepackage[T2A]{fontenc}
\usepackage[utf8]{inputenc}
\usepackage[normalem]{ulem}
\usepackage{mathtext} % русские буквы в формулах
\usepackage[left=2cm,right=2cm, top=2cm,bottom=2cm,bindingoffset=0cm]{geometry}
\usepackage[english,russian]{babel}
\usepackage[unicode]{hyperref}
\newenvironment{Proof} % имя окружения
{\par\noindent{$\blacklozenge$}} % команды для \begin
{\hfill$\scriptstyle\boxtimes$}
\newcommand{\Rm}{\mathbb{R}}
\newcommand{\Cm}{\mathbb{C}}
\newcommand{\I}{\mathbb{I}}
\renewcommand{\phi}{\upvarphi}
\renewcommand{\varphi}{\upvarphi}
\renewcommand{\alpha}{\upalpha}
\renewcommand{\psi}{\uppsi}
\renewcommand{\tau}{\uptau}
\renewcommand{\mu}{\upmu}
\renewcommand{\omega}{\upomega}
\renewcommand{\d}{\partial}
\newcommand{\N}{\mathbb{N}}
\newcommand{\Ln}{L_n = D^n + a_{n-1}D^{n-1} + \ldots + a_1D + a_0D^0}
\begin{document}
	\section*{Уравнения с разделяющимися переменными.}
	$\bullet$\textit{ Уравнение вида $$P_1(x)Q_1(y)dx + P_2(x)Q_2(y)dy = 0\eqno (1)$$ называется \textbf{уравнением с разделяющимися переменными (УРП)}.}\\\\
	С помощью интегрирующего множителя $\mu(x,y) = \dfrac{1}{Q_1(y)P_2(x)}$, причем $Q_1(y)P_2(x) \ne 0$, уравнение (1) можем свести к уравнению с разделенными переменными $$\dfrac{P_1(x)}{P_2(x)}dx + \dfrac{Q_2(y)}{Q_1(y)}dy = 0.$$
	Уравнения же с разделенными переменными мы научились решать в прошлом уроке. \\\\
	\textbf{Пример 1.} Найти общее решение уравнения 
	$$(x^2-1)y^\prime +2xy^2 = 0.$$
	\textbf{Решение.} Сначала приведем уравнение к виду (1), домножив его на $dx$
	$$2xy^2dx + (x^2 - 1)dy = 0.$$
	Мы получили УРП. Найдем для него интегрирующий множитель (по сути мы должны домножить на него его таким образом, чтобы при $dx$ остались функции лишь с $x$, а при $dy$ --- с $y$). Так как из (1)
	$$P_1(x) = 2x,\quad Q_1(y) = y^2,\quad P_2(x) = x^2 - 1, \quad Q_2(x) = 1,$$
	то в нашем случае интегрирующий множитель равен $\mu(x,y) = \dfrac{1}{y^2(x^2 - 1)}.$ Домножим уравнение на этот интегрирующий множитель и получим $$\dfrac{2x}{x^2-1}dx + \dfrac{dy}{y^2} = 0.$$
	Данное уравнение является уравнением с разделенными переменными. Найдем его общее решение, приняв $x_0 = 0$, $y_0 = 1$
	\begin{multline*}
		\int\limits_0^x \dfrac{2x}{x^2-1}dx + \int\limits_1^y \dfrac{dy}{y^2} = \int\limits_0^x\dfrac{d(x^2 - 1)}{x^2 - 1} + \int\limits_1^y y^{-2}dy = \ln|x^2 - 1|\Big|_0^x - \dfrac{1}{y}\Big|_1^y = \\ =\ln (x^2 - 1) - \ln1 - \dfrac{1}{y} + 1 = C.
	\end{multline*}
Тогда общее решение исходного уравнения имеет вид $$\ln (x^2 - 1) - \dfrac{1}{y} = C.$$
\textbf{Ответ:} $\ln (x^2 - 1) - \dfrac{1}{y} = C.$\\\\
\textbf{Пример 2.} Найти общее решение уравнения 
$$x+2y = x^\prime.$$
\textbf{Решение.} Перенесем $x^\prime$ влево и домножим уравнение на $dy$, чтобы привести его к виду (1). Получим $$dx - (x+2y)dy = 0.$$
Однако данное уравнение все равно не соответствует виду (1), так как вид (1) предполагает, что функции от $x$ и $y$ умножаются друг на друга, а в нашем случае мы имеем сумму двух функций. Для того, чтобы это исправить сделаем замену, такую замену, чтобы пропала сумма. Например, пусть $$u = x + 2y.$$ При этом, проводя замену, мы должны поменять одну переменную ($x$ или $y$) на новую переменную $u$. Пусть это будет $x$, то есть $$x = u - 2y.$$
Отсюда $$dx = du - 2dy.$$
Подставим $x$ и $dx$ в уравнение и получим $$du - 2dy - udy = du - (u+2)dy = 0.$$
Теперь же перед нами УРП. Домножим его на интегрирующий множитель $\mu(u,y) = \dfrac{1}{u+2}$ и получим $$\dfrac{du}{u+2} - dy = 0.$$
Данное уравнение является уравнением с разделенными переменными, следовательно найдем его общее решение, взяв $u_0 = -1$, $y_0 =0$ $$
	\int\limits^u_{-1}\dfrac{du}{u+2} - \int\limits^y_{0} dy = \ln|u+2|\Big|_{-1}^u - y\Big|_0^y = \ln(u+2) - y = C.
$$
Применим обратную подстанову и получим $$\ln (x+2y + 2) - y = C.$$
\textbf{Ответ:} $\ln (x+2y + 2) - y = C.$\\\\
Таким образом, приводя ДУ к виду (1), мы получаем или сразу УРП, которое сводим к уравнению с разделенными переменными, или же уравнение, которое сводится к УРП с помощью замены.\\\\
Напоследок рассмотрим УРП с задачей Коши.\\\\
\textbf{Пример 3.} Решить задачу Коши $$\ctg xdy + (y-2)dx = 0, \quad x|_{y=-1} = 0.$$
\textbf{Решение.} Решение задачи Коши для УРП ничем не отличается от решения задачи Коши для УПД, так как начальные условия мы все равно подставляем только на этапе интегрирования.\\\\ Исходное уравнение уже является УРП. Домножим уравнение на интегрирующий множитель $\mu(x,y) = \dfrac{1}{(y-2)\ctg x}$, тогда $$\tg dx + \dfrac{dy}{y-2} = 0.$$ Полученное уравнение является уравнением с разделенными переменными. Следовательно, можем найти решение задачи Коши, подставив начальные условия: 
\begin{multline*}
	\int\limits_0^x\tg x dx + \int\limits_{-1}^y\dfrac{dy}{y-2} =	-\int\limits_0^x\dfrac{d(\cos x)}{\cos x} + \int\limits_{-1}^y\dfrac{d(y-2)}{y-2} =\\= \ln|\cos x|\Big|_{0}^x+ \ln|y-2|\Big|_{-1}^y = -\ln\cos x + \ln 1 + \ln (2-y) - \ln 3 = 0.
\end{multline*}
Отсюда $$ 2- 3\cos x = y.$$
\textbf{Ответ:} $ 2- 3\cos x = y.$\\\\
\textbf{Пример 4.} Решить задачу Коши $$xy^\prime = y\cdot (1 + \ln y - \ln x), \quad y|_{x=1} = e^2.$$
\textbf{Решение.} Попытаемся привести уравнение к виду (1). Для этого домножим его на $dx$ и перенесем всё в правую часть
$$xdy - y\cdot (1 + \ln y - \ln x) dx = 0.$$
Привести такое уравнение к уравнению с разделенными переменными мы не можем. Тогда сделаем замену, чтобы избавиться от разности $$ u = \ln y - \ln x = \ln \dfrac{y}{x}.$$
Продифференцируем это равенство и получим $$du = \dfrac{dy}{y} - \dfrac{dx}{x}.$$
Чтобы удобнее было подставить замену, разделим исходное уравнение на $xy$ $$\dfrac{dy}{y} - (1 + \ln y - \ln x)\dfrac{dx}{x} = 0.$$
Подставим нашу замену и получим $$du + \dfrac{dx}{x} - (1+u)\dfrac{dx}{x} = du - u\dfrac{dx}{x} = 0.$$
$$\dfrac{du}{u} - \dfrac{dx}{x} = 0.$$
Таким образом, мы привели наше сходное уравнение к уравнению с разделенными переменными. В этот раз мы не будем сразу решать задачу Коши, а сперва найдем общее решение уравнения. Возьмем в интегралах $u_0 = x_0 = 1$, тогда $$\int\limits_1^u\dfrac{du}{u} - \int\limits_1^x\dfrac{dx}{x} = \ln u - \ln x =\ln \dfrac{u}{x} =\ln\dfrac{\ln\frac{y}{x}}{x}= C.$$
Тогда $\ln\dfrac{y}{x} = Cx$ (так как $e^C$ можно заменить на $C$), следовательно $$y = xe^{Cx}.$$
Получили общее решение уравнения. Теперь из него найдем решение задачи Коши. Для этого подставим в общее решение начальные условия и найдем из них коэффициент $C$: $$e^2 = e^C\Rightarrow C = 2\Rightarrow y = xe^{2x}.$$
Получившееся уравнение и будет решением задачи Коши.\\\\
\textbf{Ответ:} $y = xe^{2x}.$
\end{document}