\documentclass[a4paper, 12pt]{article}
\usepackage{cmap}
\usepackage{amssymb}
\usepackage{amsmath}
\usepackage{graphicx}
\usepackage{amsthm}
\usepackage{upgreek}
\usepackage{setspace}
\usepackage[T2A]{fontenc}
\usepackage[utf8]{inputenc}
\usepackage[normalem]{ulem}
\usepackage{mathtext} % русские буквы в формулах
\usepackage[left=2cm,right=2cm, top=2cm,bottom=2cm,bindingoffset=0cm]{geometry}
\usepackage[english,russian]{babel}
\usepackage[unicode]{hyperref}
\newenvironment{Proof} % имя окружения
{\par\noindent{$\blacklozenge$}} % команды для \begin
{\hfill$\scriptstyle\boxtimes$}
\newcommand{\Rm}{\mathbb{R}}
\newcommand{\Cm}{\mathbb{C}}
\newcommand{\I}{\mathbb{I}}
\renewcommand{\varphi}{\upvarphi}
\renewcommand{\alpha}{\upalpha}
\renewcommand{\psi}{\uppsi}
\newcommand{\N}{\mathbb{N}}
\newcommand{\Ln}{L_n = D^n + a_{n-1}D^{n-1} + \ldots + a_1D + a_0D^0}
\begin{document}
	\section*{СтЛОУ. Общее решение СтЛОУ. Факторизация.}
	Первая тема, которую мы будем рассматривать в курсе изучения дифференциальных уравнений после знакомства с основными понятиями и простейшими уравнениями, касается однородных линейный уравнений с постоянными коэффициентами.\\\\
	$\bullet$ \textit{\textbf{Стационарным линейным дифференциальным однородным уравнением n-ого порядка (СтЛОУ-n)} называется уравнение вида $$D^nx + a_{n-1}D^{n-1}x + \ldots + a_1Dx + a_0D^0x = 0,\quad t\in \I,$$ где $a_i \in \Rm$ --- постоянные коэффициенты.}\\\\
	Обозначим через $L_n = D^n + a_{n-1}D^{n-1} + \ldots + a_1D + a_0$ линейный оператор дифференцирования. \\\\
	$\bullet$ \textit{Построим уравнение $\lambda^n + a_{n-1}\lambda^{n-1} + \ldots + a_1\lambda + a_0 = 0$ с коэффициентами оператора $L_n$, которое будем называть \textbf{характеристическим уравнением} $L_n$. Многочлен из левой части равенства будем называть \textbf{характеристическим}.}\\\\
	Теперь, когда мы разобрались с некоторыми определениями, рассмотрим первую задачку.\\\\
	\textbf{Пример 1.} Построить общее решение уравнения $$D^2x - Dx = 0.$$
	\textbf{Решение.} Для начала построим характеристическое уравнение для данного. Для этого заменим операторы дифференцирования числами $D^ix = \lambda^i$ со степенями, соответствующими порядкам операторов. Следовательно, имеем уравнение в виде $$\lambda^2 - \lambda = 0.$$ Корнями такого уравнения будут числа $\lambda_1 = 0$ c кратностью $k_1 = 1$ и $\lambda_2 = 1$ с кратностью $k_2 = 1$.\\\\ Теперь необходимо построить общее решение данного уравнения. Оно представляет собой сумму слагаемых $C_ie^{\lambda_i t}$, где $C_i$ --- постоянные. Причем в зависимости от кратности добавляются слагаемые $C_{i} t^j e^{\lambda_i t}$, где $j$ изменяется от $(k_i - 1)$ до $0$.\\\\
	Тогда, исходя из этого, общее решение нашего уравнения имеет вид $$x(t) = C_1e^{\lambda_1 t} + C_2e^{\lambda_2 t} = C_1 + C_2e^t.$$
	\textbf{Ответ: $x(t) = C_1 + C_2e^t.$}\\\\
	Сейчас узнаем, откуда берутся $C_ie^{\lambda_i t}$. Для этого разберемся, что вообще из себя представляет решение СтЛОУ. Для этого рассмотрим пример в общем виде для СтЛОУ-1. Пусть у нас есть уравнение $Dx + ax = 0$. Для того, чтобы найти его общее решение, сделаем преобразование:
	$$Dx + ax = 0\ \Big| \cdot e^{\int adt}= e^{at},$$
	$$Dxe^{at} + axe^{at} = 0,$$
	$$D(xe^{at}) = 0.$$
	Таким образом, мы привели наше СтЛОУ-1 к простейшему ДУ-1, решением которых мы уже занимались. Проинтегрируем полученное уравнение:
	$$xe^{at} = C\Rightarrow x = Ce^{-at}.$$
	Такой вид имеет общее решение СтЛОУ-1.\\\\
	Тогда, возвращаясь к рассмотренному выше примеру, проведем аналогичные преобразования и для уравнения $D^2x - Dx = 0$:
	$$D^2x - Dx = 0\ \Big| \cdot e^{\int-dt} = e^{-t},$$
	$$D^2xe^{-t} - Dxe^{-t} = 0,$$
	$$D(Dxe^{-t}) = 0,$$
	$$Dxe^{-t} = C_1\Rightarrow Dx = C_1e^t,$$
	$$x(t) = C_1e^t + C_2.$$
	Мы получили то же общее решение. То есть мы доказали, что общее решение полученное в примере действительно является общим решением исходного уравнения.\\\\
	\textbf{Пример 2.} Построить общее решение уравнения $$D^3x + 3D^2x + 3Dx + x = 0.$$
	\textbf{Решение.} Аналогично с предыдущим примером составим характеристическое уравнение и найдем его корни: $$\lambda^3 + 3\lambda^2 + 3\lambda + 1 = 0;$$
	$\lambda_1 = -1$, $k_1 = 3$. Теперь составим общее решение. В отличие от предыдущей задачи, кратность корня в данном уравнении больше 1. В таком случае, так как порядок уравнения $n = 3$, общее решение будет иметь следующий вид:
	$$x(t) = C_1t^{k_1 - 1}e^{\lambda_1 t} + C_2t^{k_1 - 2}e^{\lambda_1 t} + C_3t^{k_1 - 3}e^{\lambda_1 t} = C_1t^2e^{-t} + C_2te^{-t} + C_3e^{-t}.$$
	\textbf{Ответ: $x(t) = e^{-t}(C_1t^2 + C_2t + C_3).$}
	\\\\
	\textbf{Замечание:} \textit{Количество различных $C_i$ равно порядку уравнения. Благодаря этому можно проверять себя на правильность построенного общего решения}.\\\\
	\textbf{Пример 3.} Построить общее решение уравнения $$x'' - 2x' + 3x = 0.$$
	\textbf{Решение.} Найдем корни характеристического уравнения: $\lambda_1 = 1 + i\sqrt{2}$, $k_1 = 1$; $\lambda_2 = 1 - i\sqrt{2}$, $k_2 = 1$. Мы получили комплексные решения уравнения. То есть общее решение уравнения будет выглядеть как $$x(t) = C_1e^{1+i\sqrt{2}} + C_2e^{1-i\sqrt{2}}.$$ Но такая форма записи общего решения не предпочтительна, поэтому представим его над полем $\Rm$.\\\\ Применим формулу Эйлера: $$e^{(\lambda + i\mu)t} = e^{\lambda t}\cdot e^{i\mu t} = e^{\lambda t}(\cos(\mu t) + i \sin(\mu t)).$$ Тогда общее решение нашего уравнения будет иметь вид $$x(t) = C_1e^t(\cos(\sqrt{2}t) + i \sin(\sqrt{2}t)) + C_2e^t(\cos(\sqrt{2}t) - i \sin(\sqrt{2}t)).$$ Раскроем скобочки и запишем по-другому
	$$x(t) = (C_1e^t\cos(\sqrt{2}t) + C_2e^t\cos(\sqrt{2}t)) + (C_1e^ti \sin(\sqrt{2}t) - C_2e^ti \sin(\sqrt{2}t))$$ и сделаем замену: $C_1 + C_2 = \widetilde{C}_1$, $i\cdot(C_1 - C_2) = \widetilde{C}_2$. Мнимая единица уйдет к коэффициенту $\widetilde{C}_2$, так как в общем решении \textbf{не должно быть мнимой единицы}. Тогда наше решение будет выглядеть следующим образом:
	$$x(t) = e^t(\widetilde{C}_1\cos(\sqrt{2}t) + \widetilde{C}_2 \sin(\sqrt{2}t)).$$
	\textbf{Ответ:} $x(t) = e^t(\widetilde{C}_1\cos(\sqrt{2}t) + \widetilde{C}_2 \sin(\sqrt{2}t)).$\\\\
	Вспомним тему прошлого занятия, а именно начальные и граничные задачи.\\\\
	\textbf{Пример 4.} Решить уравнение с заданными граничными условиями $$D^2x + x = 0,\quad x|_{t=0} = 1,\quad x|_{t=\pi/2} = 1.$$
	\textbf{Решение.} Найдем корни характеристического уравнения: $\lambda_1 = i$, $k_1 = 1$; $\lambda_2 = -i$, $k_2 = 1$. Теперь составим общее решение уравнения:$$x(t) = \widetilde{C}_1\cos t + \widetilde{C}_2\sin t,\quad \widetilde{C}_1 = C_1 + C_2,\quad \widetilde{C}_2 = i\cdot (C_1 - C_2).$$
	Теперь подставим условия в последнее уравнение и получим два соответствующих равенства: $$1 = \widetilde{C}_1\cos0 + \widetilde{C}_2\sin0 = \widetilde{C}_1;$$
	$$1 = \widetilde{C}_1\cos\dfrac{\pi}{2} + \widetilde{C}_2\sin\dfrac{\pi}{2} = \widetilde{C}_2.$$
	Подставим полученные константы в общее решение и получим итоговое уравнение $$x(t)=\cos t + \sin t.$$
	\textbf{Ответ:} $x(t)=\cos t + \sin t.$\\\\
	\textbf{Замечание:} \textit{В дальнейших примерахх при комлексных корнях характеристического уравнения замена $\widetilde{C}_1 = C_1 + C_2$, $\widetilde{C}_2 = i\cdot (C_1 - C_2)$ будет опускаться. Однако подразумевается именно такая замена.}
	\subsubsection*{Факторизация.}
	$\bullet$ \textit{Линейный оператор можно разложить на множители $L_n = (D-\lambda_1D^0)(D-\lambda_2D^0)\ldots(D-\lambda_nD^0)$. Такое разложение называется \textbf{факторизацией} оператора.}\\\\
	\textbf{Пример 5.} Факторизовать линейный оператор:
	$$D^2x + 6Dx + 5x = 0.$$
	\textbf{Решение.} Сделаем замену $\lambda^i = D^ix$ и составим характеристическое уравнениe $$\lambda^2 + 6\lambda + 5 = 0.$$ Найдем корни характеристического уравнения: $\lambda_1 = -1$, $k_1 = 1$; $\lambda_2 = -5$, $k_2 = 1$. Теперь запишем характеристический многочлен в виде произведения многочленов первой степени $$\delta(\lambda) = (\lambda + 1)(\lambda + 5).$$ Применим обратную замену и получим факторизацию нашего линейного оператора: $$L_n = (D+D^0)(D + 5D^0).$$
	\textbf{Ответ:} $L_n = (D+D^0)(D + 5D^0).$\\\\
	\textbf{Замечание:} \textit{Проверить факторизацию на правильность можно путем раскрытия скобок и домножения оператора на $x$. Тогда, раскрыв скобочки из предыдущего уравнения, получим уравнение $$L_n x = D^2x + 6Dx + 5x,$$ которое и является левой частью исходного СтЛОУ.}\\\\
	Теперь скомбинируем условия из задач выше и рассмотрим задачу посложнее.\\\\
	\textbf{Пример 6.} Найти общее решение уравнения и факторизовать оператор $L_n$ над полями действительных и комплексных чисел $$D^5 x-6D^4 x+20D^3 x-60D^2 x+99Dx-54x = 0.$$
	\textbf{Решение.} Для начала составим характеристическое уравнение: $$\lambda^5 - 6\lambda^4 + 20\lambda^3 - 60\lambda^2 + 99 \lambda - 54 = 0.$$
	Подбором найдем корень $\lambda_1 = 1$, $k_1 = 1$. Тогда, разделив многочлен на $(\lambda - 1)$, найдем подбором корень $\lambda_2 = 2$, $k_2 = 1$. Продолжая рассуждения по аналогии, получим последний действительный корень $\lambda_3 = 3$, $k_3 = 1$. Тогда характеристический многочлен имеет вид $$\delta(\lambda) = (\lambda-1)(\lambda-2)(\lambda-3)(\lambda^2 + 9).$$
	Следовательно, над полем $\Rm$, сделав обратную замену, получим разложение $$L_n = (D-D^0)(D-2D^0)(D-3D^0)(D^2 + 9D^0).$$
	Теперь разложим крайнюю скобочку в правой части полученного уравнения над полем $\Cm$. Получим корни $\lambda_4 = 3i$, $k_4 = 1$; $\lambda_5 = -3i$, $k_5 = 1$. Тогда над полем $\Cm$ факторизация оператора имеет вид: 
	$$L_n = (D-D^0)(D-2D^0)(D-3D^0)(D - 3iD^0)(D+3iD^0).$$
	Мы нашли все корни характеристического уравнения, следовательно, можем построить общее решение уравнения: $$x(t) = C_1e^t + C_2e^{2t} + C_3e^{3t} + C_4\cos(3t) + C_5 \sin(3t).$$
	\section*{СтЛОУ. Базис пространства решений.}
	Рассмотрим уравнение $$L_nx = 0,\quad t\in \I\subseteq\Rm.$$
	Пространство решений этого уравнения является конечномерным векторным (линейным) пространством размерности $n$ ($n$ --- порядок уравнения). Как и любое векторное пространство, пространство решений СтЛОУ имеет базис.\\\\
	$\bullet$ \textit{Функции $\varphi_1(t),\ldots,\varphi_n(t)$ называются \textbf{линейно зависимыми}, если существуют числа $\alpha_1,\ldots, \alpha_n$ не обращающиеся в $0$ одновременно такие, что $\alpha_1\varphi_1(t) + \ldots + \alpha_n\varphi_n(t) = 0\ \forall t$. В противном случае функции называются \textbf{линейно независимыми}.}\\\\
	$\bullet$ \textit{Пусть $\varphi_1(t),\ldots,\varphi_n(t)$ --- это $(n-1)$ раз дифференцируемые функции. Определитель $$W(t) = \begin{vmatrix}
			\varphi_1(t) & \dots & \varphi_n(t)\\
			D\varphi_1(t) & \dots & D\varphi_n(t)\\
			\vdots & \ddots & \vdots\\
			D^{n-1}\varphi_1(t) & \dots & D^{n-1}\varphi_n(t)
	\end{vmatrix}$$ называется \textbf{Вронскианом} этих функций.}\\\\
	Для решения задач рассмотрим две теоремы:\\\\
	\textbf{Теоремы.}\begin{enumerate}
		\item \textit{Если функции линейно зависимы, то их $W(t) = 0$.}
		\item \textit{Если функции линейно независимы, то их $W(t) \ne 0$.}
	\end{enumerate}
	Как мы знаем из курса линейной алгебры, базис в пространстве образуют линейно независимые векторы, через которые выражается любой другой вектор. Аналогично базис пространства решений СтЛОУ образуют линейно независимые функции, через которые выражается любое другое решение. То есть общее решение СтЛОУ строится с помощью базиса по формуле $$x(t) = \sum_{i = 1}^{n}C_i\varphi_i(t),\quad C_i \in \Rm.$$
	\textbf{Пример 7.} Показать, что данные решения СтЛОУ-$n$ образуют базис пространства решений. Построить уравнение и записать для него общее решение:
	$$\varphi_1(t) = \sin t,\ \varphi_2(t) = \cos t,\ \varphi_3(t) = e^t.$$
	\textbf{Решение.} Построим Вронскиан этих функций и найдем, чему он равен:
	$$W(t) = \begin{vmatrix}
		\sin t & \cos t & e^t\\
		\cos t & -\sin t & e^t\\
		-\sin t & -\cos t & e^t\\
	\end{vmatrix} \sim \begin{vmatrix}
	\sin t & \cos t & e^t\\
	\cos t & -\sin t & e^t\\
	0 & 0 & 2e^t\\
\end{vmatrix} = -2e^t \ne 0.$$
Значит функции линейно независимые, а следовательно являются базисом пространства решений. Выразим общее решение СтЛОУ через эти функции:
$$x(t) = C_1\cos t + C_2\sin t + C_3e^t.$$
Тогда корни исходного уравнения $\lambda_1 = i$, $k_1 = 1$; $\lambda_2 = -i$, $k_2 = 1$; $\lambda_3 = 1$, $k_3 = 1$.
Построим факторизацию оператора $L_n$:
$$L_n = (D - iD^0)(D + iD^0)(D - D^0).$$
Раскроем скобочки, домножим на $x$ и получим СтЛОУ:
$$D^3x - D^2x + Dx - x = 0.$$
\textbf{Ответ:} $D^3x - D^2x + Dx - x = 0.$\\\\
$\bullet$ \textit{Базис пространства решений линейной однородной системы уравнений называется \textbf{фундаментальной системой решений}}.\\\\
То есть в качестве ФСР СтЛОУ можно взять зависящие от $t$ функции, стоящие при $C_i$.\\\\
Рассмотрим $n$ задач Коши:
$$\begin{cases}
	L_nx = 0,\\
	x|_{t=t_0} = 1,\\
	Dx|_{t=t_0} = 0,\\
	\dotfill\\
	D^{n-1}x|_{t=t_0} = 0;
\end{cases}\quad\begin{cases}
	L_nx = 0,\\
	x|_{t=t_0} = 0,\\
	Dx|_{t=t_0} = 1,\\
	\dotfill\\
	D^{n-1}x|_{t=t_0} = 0;
\end{cases}\dots\dots\dots\quad\begin{cases}
	L_nx = 0,\\
	x|_{t=t_0} = 0,\\
	Dx|_{t=t_0} = 0,\\
	\dotfill\\
	D^{n-1}x|_{t=t_0} = 1;
\end{cases}\eqno(1)$$
$\bullet$ \textit{Фундаментальная система решений, удовлетворяющая условиям $(1)$ называется \textbf{нормированной} при $t = t_0$.}\\\\
$\bullet$ \textit{Функция $\psi(t)$ называется \textbf{сдвигом} функции $\varphi(t)$ на $t=t_0$, если $\psi(t) = \varphi(t-t_0)$.}\\\\
\textbf{Следствие из теоремы о сдвиге.} \textit{Если $\varphi_0(t),\ldots,\varphi_{n-1}(t)$ --- ФСР нормированная при $t=0$, то $\varphi_0(t-t_0),\ldots,\varphi_{n-1}(t-t_0)$ --- ФСР нормированная при $t=t_0$.}\\\\
Стоит различать понятия ФСР и ФСР нормированной в точке $t_0$. Отличия же в них следующие: ФСР уравнения --- условно произвольная комбинация функций, являющихся решением, в то время как ФСР нормированная в точке $t_0$ выбирается таким образом, что при подстановке $\varphi_i(t_0)$, $i = \overline{0, n-1}$ она образует столбец $(1, 0, \ldots, 0)^T$. Иными словами, $$\varphi_0(t_0) = 1,\quad \varphi_1(t_0) = 0,\quad\ldots ,\quad \varphi_{n-1}(t_0) = 0.$$
Зачем нам нужна ФСР нормированная в точке? С ее помощью мы можем находить решение задачи Коши для стационарных линейных уравнений. Так что для решения задачи Коши $L_nx = 0$, $D^ix|_{t=t_0} = \xi_i$, $i = \overline{0, n-1}$ используется формула $$x(t) = \xi_0\varphi_0(t-t_0) + \ldots + \xi_{n-1}\varphi_{n-1}(t-t_0),$$
где $\varphi_0(t-t_0),\ldots,\varphi_{n-1}(t-t_0)$ --- ФСР нормированная в точке $t_0$.\\\\
На практике же вы будете использовать ФСР нормированную в точке $t_0$ только тогда, когда необходимо решить задачу Коши для СтЛНУ методом Коши (подробнее в следующем уроке).\\\\
\textbf{Пример 8.}  Построить ФСР нормированную в точке $t_0 = -1$ для уравнения $$D^2x -2Dx + x = 0.$$
\textbf{Решение.} Сперва найдем корни характеристического уравнения: $\lambda_1 = 1$, $k_1 = 2$. Далее строим общее решение уравнения: $$x(t) = C_1te^t + C_2e^t.$$
ФСР данного уравнения образуют функции $$\varphi_1(t) = te^t,\quad \varphi_2(t) = e^t.$$ Теперь найдем ФСР нормированную в точке $t_0$.\\\\
Для построения ФСР нормированной в точке мы сначала найдем коэффициенты $C_i$ при $t_0 = 0$ из системы (1), а затем для получившихся функций сделаем сдвиг на $t_0 = -1$.\\\\ Нам нужно найти $n-1$ производную от общего решения из задач (1). Поскольку порядок уравнения 2, то нам нужно построить только $Dx$. В нашем случае $$Dx(t) = C_1te^t + C_1e^t + C_2e^t.$$
Найдем значения полученных функций ($x(t)$ и $Dx(t)$) в точке $t = 0$: $$x(t)|_{t=0} = C_2;\quad Dx(t)|_{t=0} = C_1 + C_2.$$
Теперь мы переходим к решению $n-1$ задач Коши. Для этого нам необходимо составить и решить системы уравнений (1):
$$\begin{cases}
	L_nx = 0,\\
	x(t)|_{t=0} = 1,\\
	Dx(t)|_{t=0} = 0;
\end{cases} \begin{cases}
L_nx = 0,\\
x(t)|_{t=0} = 0,\\
Dx(t)|_{t=0} = 1;
\end{cases}$$ Отбросим первое равенство в обеих системах и подставим полученные значения:
$$\begin{cases}
	C_2 = 1,\\
	C_1 + C_2 = 0;
\end{cases}\begin{cases}
C_2 = 0,\\
C_1 + C_2 = 1.
\end{cases}$$ Решим системы методом Гаусса. Cоставим матрицу, где слева у нас коэффициенты при константах $C_i$ при $x$ (первая строка), $Dx$ (вторая строка), а справа единичная матрица (так как, отбросив первую строку в каждой системе из (1), в правой части равенств получается единичная матрица):
$$\begin{pmatrix}
	0 & 1 & \vline & 1 & 0\\
	1 & 1 & \vline & 0 & 1
\end{pmatrix}\sim \begin{pmatrix}
	1 & 0 & \vline & -1 & 1\\
	0 & 1 & \vline & 1 & 0
\end{pmatrix}$$
Теперь составим ФСР данного уравнения нормированную в точке $t_0 = 0$. Для этого определим функции $\varphi_0(t)$ и $\varphi_1(t)$ следующим образом. Берём первый столбец правой части полученной матрицы $\begin{pmatrix}
	-1\\1
\end{pmatrix}$ и подставляем его $i$-ую строку вместо $C_i$ в $x(t)$, то есть $C_1 = -1$, $C_2 = 1$ и $$\varphi_0(t) = -te^t + e^t;$$
аналогично берем второй столбец $\begin{pmatrix}
	1\\0
\end{pmatrix}$, подставляем $C_1 = 1$, $C_2 = 0$ и получаем функцию
$$\varphi_1(t) = te^t.$$
Теперь последнее действие: нам нужно получить ФСР нормированную в точке $t_0 = -1$. Для этого воспользуемся следствием из теоремы о сдвиге. То есть нам нужно найти значение функций $\varphi_i(t)$ при $t - t_0 = t + 1$. Тогда $$\varphi_0(t+1) = -(t+1)e^{t+1} + e^{t+1} = -te^{t+1};$$
$$\varphi_1(t+1) = (t+1)e^{t+1},$$
что и требовалось найти.\\\\
То есть система функций $$\varphi_0(t+1)= -te^{t+1},\quad\varphi_1(t+1) = (t+1)e^{t+1}$$ образует ФСР уравнения нормированную в точке $t_0 = -1$.\\\\
\textbf{Ответ:} $\varphi_0(t+1)= -te^{t+1}$, $\varphi_1(t+1) = (t+1)e^{t+1}$.\\\\
\textbf{Замечание:} \textit{Исходя из решения предыдущего примера, можно составить следующий алгоритм нахождения ФСР нормированную в точке:}\begin{enumerate}
	\item \textit{находим корни характеристического уравнения и составляем общее решение уравнения;}
	\item \textit{находим $n-1$ производную от $x(t)$;}
	\item \textit{находим значения всех полученных функций $D^ix(t)$ при $t=0$, $i = \overline{0, n-1}$;}
	\item \textit{составляем матрицу, где в левой части у нас коэффициенты при $C_i$, а в правой единичная матрица, затем приводим левую часть к единичной матрице;}
	\item \textit{полученные в правой части значения подставляем по столбцам соотвественно в $x(t)$ вместо $C_i$, обозначая полученные функции $\varphi_i(t)$;
	\item для нормирования в точке $t_0$ находим значение функции при $t-t_0$.}
\end{enumerate}
Рассмотрим аналогичный пример для уравнения большего порядка.\\\\
\textbf{Пример 9.} Построить ФСР нормированную в точке $t_0 = 2$ для уравнения $$D^4x-D^3x = 0.$$
\textbf{Решение.} Найдем корни характеристического уравнения: $\lambda_1 = 0$, $k_1 = 3$; $\lambda = 1$, $k_2 = 1$. Тогда общее решение имеет вид $$x(t) = C_1t^2 + C_2t + C_3 + C_4e^t.$$
Система функций $$t^2,\ t,\ 1,\ e^t$$ образует ФСР уравнения.
Теперь найдем ФСР нормированную в точке $t_0 = 0$. Для этого вычислим производные до 3-го порядка:
$$Dx(t) = 2C_1t + C_2 + C_4e^t;$$
$$D^2x(t) = 2C_1 + C_4e^t;$$
$$D^3x(t) = C_4e^t.$$
Подставляем в полученные уравнения $t = 0$, тогда
$$\begin{cases}
	x(t)|_{t=0} = C_3 + C_4,\\
	Dx(t)|_{t=0} = C_2 + C_4,\\
	D^2x(t)|_{t=0} = 2C_1 + C_4,\\
	D^3x(t)|_{t=0} = C_4.
\end{cases}$$
Составим системы из (1), отбросим из них первую строку и запишем в матричном виде:
$$\begin{pmatrix}
	0 & 0 & 1 & 1 & \vline & 1 & 0 & 0 & 0\\
	0 & 1 & 0 & 1 & \vline & 0 & 1 & 0 & 0\\
	2 & 0 & 0 & 1 & \vline & 0 & 0 & 1 & 0\\
	0 & 0 & 0 & 1 & \vline & 0 & 0 & 0 & 1
\end{pmatrix}.$$
Приведем матрицу слева к единичному виду и получим
$$\begin{pmatrix}
	1 & 0 & 0 & 0 & \vline & 0 & 0 & 1/2 & -1/2\\
	0 & 1 & 0 & 0 & \vline & 0 & 1 & 0 & -1\\
	0 & 0 & 1 & 0 & \vline & 1 & 0 & 0 & -1\\
	0 & 0 & 0 & 1 & \vline & 0 & 0 & 0 & 1
\end{pmatrix}.$$
Подставим матрицу справа по столбцам в $x(t)$ и получим функции:
$$\varphi_0(t) = 1,\quad \varphi_1(t) = t,\quad \varphi_2(t) = \dfrac{t^2}{2},\quad \varphi_3(t) = -\dfrac{t^2}{2} - t - 1 + e^t.$$
Данная система функций образует ФСР нормированную в точке $t_0 = 0$. Для того, чтобы она была нормированной в точке $t_0 = 2$, сделаем сдвиг на $t-2$:
$$\varphi_0(t-2) = 1,\quad \varphi_1(t-2) = t-2,\quad \varphi_2(t-2) = \dfrac{t^2}{2}-2t+2,\quad \varphi_3(t-2) = -\dfrac{t^2}{2} + t - 1 + e^{t-2}.$$
\textbf{Ответ:} $\varphi_0(t-2) = 1,\ \varphi_1(t-2) = t-2,\ \varphi_2(t-2) = \dfrac{t^2}{2}-2t+2,\ \varphi_3(t-2) = -\dfrac{t^2}{2} + t - 1 + e^{t-2}.$
\\\\
\textbf{Пример 10.} Построить ФСР нормированную в точке $t_0 = 3$ для уравнения $$D^2x + 9x = 0.$$
	\textbf{Решение.} Найдем корни характеристического уравнения: $\lambda_1 = 3i$, $k_1 = 1$; $\lambda_1 = -3i$, $k_2 = 1$. Тогда общее решение имеет вид $$x(t) = C_1\cos(3t) + C_2\sin(3t).$$
 Тогда $$Dx(t) = -3C_1\sin(3t) + 3C_2\cos(3t).$$ Находим значения в точке $t =0$:
	$$x(t)|_{t = 0} = C_1;\quad Dx(t)|_{t = 0} = 3C_2.$$ Составим матрицу и решим её методом Гаусса:
	$$\begin{pmatrix}
		1 & 0 & \vline & 1 & 0\\
		0 & 3 & \vline & 0 & 1
	\end{pmatrix}\sim \begin{pmatrix}
	1 & 0 & \vline & 1 & 0\\
	0 & 1 & \vline & 0 & \frac{1}{3}
\end{pmatrix}$$
 Подставим коэффициенты и получим функции, которые образуют ФСР нормированную в точке $t_0 = 0$: $$\varphi_0(t) = 1\cdot \cos(3t) + 0\cdot \sin (3t) = \cos(3t);$$ $$\varphi_1(t) = 0\cdot \cos(3t) + \dfrac{1}{3}\cdot \sin (3t) = \dfrac{1}{3}\sin(3t).$$
Нормируем в точке $t_0 = 3$ и получим
$$\varphi_0(t - 3) = \cos(3t - 9);$$
$$\varphi_1(t - 3) = \dfrac{1}{3}\sin(3t - 9),$$
что и требовалось найти.\\\\
\textbf{Ответ:} $\varphi_0(t - 3) = \cos(3(t - 3)),$
$\varphi_1(t - 3) = \dfrac{1}{3}\sin(3(t - 3)).$\\\\
Теперь немного переформулируем предыдущий пример.\\\\
\textbf{Пример 11.} Найти решение задачи Коши $x|_{t=0} = 3$, $Dx|_{t=0} = 3$ для уравнения
$$D^2x + 9x = 0.$$
\textbf{Решение.} Перепишем $x(t)$ и $Dx(t)$ из предыдущего примера
$$x(t) = C_1\cos(3t) + C_2\sin(3t),$$
$$Dx(t) = -3C_1\sin(3t) + 3C_2\cos(3t).$$
Рассмотрим два способа: один уже известный нам из прошлого урока, а второй с помощью ФСР нормированной в точке:
\begin{enumerate}
	\item Просто подставим условия задачи Коши и найдем коэффициенты $C_1$ и $C_2$:
	$$x(t)|_{t=0} = C_1 = 3,$$
	$$Dx(t)|_{t=0} = 3C_2 = 3 \Rightarrow C_2 = 1.$$
	Подставим найденные коэффициенты в $x(t)$ и получим решение задачи Коши:
	$$x(t) = 3\cos(3t) + \sin(3t).$$
	\item Возьмем найденную ранее ФСР нормированную в точке $t = 0$:
	$$\varphi_0(t) = \cos(3t),\quad \varphi_1(t) = \dfrac{\sin(3t)}{3}.$$
	Применим формулу $$x(t) = \xi_0\varphi_0(t-t_0) + \ldots + \xi_{n-1}\varphi_{n-1}(t-t_0),$$
	в нашем случае равную
	$$x(t) = \xi_0\varphi_0(t) + \xi_{1}\varphi_{1}(t).$$
	Тогда получим $$x(t) = 3\cos(3t) + \sin(3t).$$ 
\end{enumerate}
\textbf{Ответ:} $x(t) = 3\cos(3t) + \sin(3t).$\\\\
На этом урок окончен. Можете переходить к самостоятельному решению задач.
 \end{document}