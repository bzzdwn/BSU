\documentclass[a4paper, 12pt]{article}
\usepackage{cmap}
\usepackage{amssymb}
\usepackage{amsmath}
\usepackage{graphicx}
\usepackage{amsthm}
\usepackage{upgreek}
\usepackage{setspace}
\usepackage[T2A]{fontenc}
\usepackage[utf8]{inputenc}
\usepackage[normalem]{ulem}
\usepackage{mathtext} % русские буквы в формулах
\usepackage[left=2cm,right=2cm, top=2cm,bottom=2cm,bindingoffset=0cm]{geometry}
\usepackage[english,russian]{babel}
\usepackage[unicode]{hyperref}
\newenvironment{Proof} % имя окружения
{\par\noindent{}} % команды для \begin
{\hfill$\scriptstyle$}
\newcommand{\Rm}{\mathbb{R}}
\newcommand{\Cm}{\mathbb{C}}
\newcommand{\I}{\mathbb{I}}
\newcommand{\N}{\mathbb{N}}
\newtheorem*{thrm}{Теорема}
\renewcommand{\alpha}{\upalpha}
\renewcommand{\beta}{\upbeta}
\renewcommand{\gamma}{\upgamma}
\newcommand{\Ln}{L_n = D^n + a_{n-1}D^{n-1} + \ldots + a_1D + a_0D^0}
\begin{document}
	\section*{Основные понятия. Простейшие ДУ.}
	Этот урок является вводным. В нем мы рассмотрим несколько простейших базовых примеров, которые пригодятся нам для дальнейшего изучения дифференциальных уравнений.
	\subsection*{Основные понятия.}
	$\bullet$ \textit{\textbf{Обыкновенным дифференциальным уравнением (ОДУ)} называется выражение вида $$F(t, x, x', \dots, x^{(n)}) = 0,\eqno(1)$$ где $F$ --- некоторая функция $(n+2)$-ух переменных, определенная в некоторой области, $t$ --- независимая переменная, $x = x(t)$ --- неизвестная функция независимой переменной, $x',\dots, x^{(n)}$ --- производные функции $x(t)$, причем переменная $t$ и функции $F$ и $x$ действительны.}\\\\
	$\bullet$ \textit{Порядок старшей производной, присутствующей в уравнении, называется \textbf{поряком} уравнения.}\\\\
	\textbf{Пример.} Для уравнения $x''' - x'' - x = 0$ порядок $n=3$.\\\\
	$\bullet$ \textit{\textbf{Решением} уравнения $(1)$ называется функция, заданная и $n$ раз дифференцируемая на некотором промежутке $\I \subseteq \Rm$ и обращающая уравнение в верное равенство.}\\\\
	\textbf{Пример 1.} Показать, что функция $$x(t) = te^t + t^2 + 2$$ является решением уравнения $$x'' - x = 2e^t - t^2.$$
	\textbf{Решение.} Для доказательства необходимо решение в данное уравнение. Для этого найдем вторую производную от функции $x(t)$ и подставим в уравнение. Первая производная равна $$x' = e^t + te^t + 2t;$$
	тогда вторая производная
	$$x'' = 2e^t + te^t + 2.$$
	Подставим полученную производную и функцию $x(t)$ в уравнение и получим $$2e^t + te^t + 2 -  te^t - t^2 - 2 = 2e^t - t^2,$$
	что и требовалось показать.\\\\
	 $\bullet$ \textit{Совокупность решений уравнения  вида $x = \varphi(t, C_1,\dots,C_n)$, зависящая от $n$ произвольных постоянных $C_i$, называется \textbf{общим решением} уравнения.}\\\\
	 \textbf{Пример 2.} Показать, что функция $$x(t) = C_1t + \ln C_1$$ является общим решением уравнения $$x - tx' = \ln x'.$$
	 \textbf{Решение.} Аналогично с предыдущим примером находим производную $x'$ от функции $x(t)$:
	 $$x' = C_1$$
	и подставляем эту производную и функцию $x(t)$ в уравнение $$C_1t + \ln C_1 - C_1t = \ln C_1,$$ что и требовалось показать.\\\\
	В будущем аналогичными действиями можно проверять, правильное ли решение дифференциального уравнения было найдено.\\\\
	\textbf{Пример 3.} Построить дифференциальное уравнение наименьшего порядка путем исключения произвольных постоянных $$x(t) = C_1te^t + C_2e^t.$$
	\textbf{Решение.} По сути своей задание сводится к тому, что мы должны построить по данному нам общему решению. Очень важно учитывать, что количество констант равно наименьшему порядку получаемого дифференциального уравнения. Так как мы имеем 2 постоянные, то наменьший порядкок уравнения, для которого функция будет являться решением, равен 2. Следовательно, найдем вторую производную от $x(t)$:
	$$x' = C_1e^t + C_1te^t + C_2e^t;$$
	$$x'' = 2C_1e^t + C_1te^t + C_2e^t.$$
	Далее путем умножения на какие-то постоянные, сложения и прибавления или вычитания каких-то функций мы должны получить из $x''$ тождественный ноль (то есть нужно найти нетривиальную линейную комбинацию из производных, которая даст нам тождественный ноль):
	$$\alpha x'' + \beta x' + \gamma x = 0.$$
	\begin{multline*}
		\alpha (2C_1e^t + C_1te^t + C_2e^t) + \beta (C_1e^t + C_1te^t + C_2e^t) + \gamma (C_1te^t + C_2e^t) = \\=
		C_1e^t(2\alpha + \beta) + C_1te^t(\alpha + \beta + \gamma) + C_2e^t(\alpha + \beta + \gamma) = 0.
	\end{multline*}
	Отсюда
	$$\begin{cases}
		2\alpha + \beta = 0,\\
		\alpha + \beta + \gamma = 0
	\end{cases} \Longleftrightarrow \begin{cases}
		\beta = -2\alpha,\\
	\alpha = \gamma.
\end{cases}$$
Возьмем наименьшие ненулевые положительные коэффициенты: $\alpha = \gamma = 1$, $\beta = -2$. Таким образом получим уравнение $$x'' - 2x' + x  = 2C_1e^t + C_1te^t + C_2e^t -2 C_1e^t -2C_1te^t -2C_2e^t + C_1te^t + C_2e^t = 0.$$
	\textbf{Ответ:} $x'' - 2x' + x = 0$.
	\subsection*{Простейшие ДУ.}
	Пусть $D$ --- оператор дифференцирования, то есть $D:x\mapsto x'$. Тогда первую производную функции $x$ будем обозначать $Dx = x'$, вторую $D^2x = x''$ и так далее $D^i = x^{(i)}$. В течение 1-ой и 2-ой глав будем использовать такое обозначение.\\\\
	$\bullet$ \textit{\textbf{Простейшим} называется дифференциальное уравнение вида $$D^nx = f(t),\ t\in \mathbb{I},$$ где $\mathbb{I}\subseteq\mathbb{R}$, $f(t)$ --- непрерывная в $\mathbb{I}$ функция.}\\\\
	Общее решение простейшего ДУ первого порядка $Dx = f(t),\ t \in \mathbb{I}$ имеет вид $$x(t) = \int f(t)dt = \int\limits_{t_0}^tf(\uptau)d\uptau + C,$$ где $t_0 \in \mathbb{I}$, а $C$ --- произвольная постоянная.\\\\
	Общее решение уравнения $D^nx = f(t)$ имеет вид $$x(t) = \int\limits_{t_0}^{t}\dfrac{(t-\uptau)^{n-1}}{(n-1)!}f(\uptau)d\uptau + \sum\limits_{i=0}^{n-1}\widetilde{C}_it^i.\eqno(2)$$
	Найти общее решение функции уравнения $D^nx = f(t)$ обычно можно путем интегрирования уравнения $n$ раз. Формула (2) же используется зачастую для неберущихся интегралов.\\\\
	\textbf{Пример 4.} Найти общее решение уравнения $$D^3x = t^{-3},\quad \I = (0; +\infty).$$
	\textbf{Решение.} Как говорилось выше, необходимо $n = 3$ раза проинтегрировать наше уравнение:
	$$D^2x = \int t^{-3}dt = -\dfrac{1}{2t^2} + C_1;$$
	$$Dx = \int\Big(-\dfrac{1}{2t^2} + C_1\Big) dt = \dfrac{1}{2t} + C_1t + C_2;$$
	$$x = \int \Big(\dfrac{1}{2t} + C_1t + C_2\Big)dt = \dfrac{\ln|t|}{2} + C_1t^2 + C_2t + C_3.$$
	Таким образом, получили общее решение нашего уравнения, причем модуль в логарифме можно раскрыть, так как $t\in \I \Rightarrow t > 0$.\\\\
	\textbf{Ответ:} $x = \dfrac{\ln t}{2} + C_1t^2 + C_2t + C_3.$\\\\
	Проверить, правильное ли решение уравнения можно взяв третью прозводную от полученного решения.\\\\
	\textbf{Пример 5.} Найти общее решение $$D^2x = \dfrac{e^t}{t},\quad \I = (0; +\infty).$$
	\textbf{Решение.} Для нахождения общего решения так же, как и в прошлом примере, необходимо проинтегрировать $n=2$ раза данное уравнение. Однако функция $\int \dfrac{e^t}{t}dt$ не имеет первообразной в элементарных функциях, то есть интеграл от данной функции неберущийся. Следовательно, общее решение можно записать по формуле (2):
	$$x(t) = \int\limits_{t_0}^{t}\dfrac{(t-\uptau)\cdot e^{\uptau}}{1!\cdot \uptau}d\uptau + C_2t + C_1.$$
	\textbf{Ответ:} $x(t) = \int\limits_{t_0}^{t}\dfrac{(t-\uptau)\cdot e^{\uptau}}{\uptau}d\uptau + C_2t + C_1.$\\\\
	\subsection*{Начальные задачи. Граничные задачи. Задачи Коши.}
	$\bullet$ \textit{Дополнительные условия накладываемые на неизвестную функцию называются \textbf{начальными}, если они относятся к одному значению аргумента, и \textbf{граничными}, если относятся к разным значениям аргумента.}\\\\
	\textbf{Пример 6.} Решить начальную задачу:
	$$D^2x = 1,\ x|_{t=0} = 1,\ Dx|_{t = 1} = 0,\quad t \in \I = \Rm.$$
	\textbf{Решение.} Найдем общее решение уравнения. Для этого проинтегрируем его два раза:
	$$Dx(t) = \int dt = t + C_1,$$
	$$x(t) = \int (t + C_1) dt = \int tdt + C_1\int dt = \dfrac{t^2}{2} + C_1t + C_2.$$
	Подставим наши начальные условия. Возьмем условие $Dx|_{t=1} = Dx(1) = 0$. Тогда $$ 0 = 1 + C_1.$$
	Таким образом, $C_1 = -1$. Теперь подставим условие $x(0) = 1$:
	$$1 = 0 + 0 + C_2.$$
	Соответственно, $C_2 = 1$.\\
	Если бы уравнение имело больший порядок, то подставляли бы мы до тех пор, пока не нашли все $C_i$. Теперь все полученные $C_i$ необходимо подставить в общее решение $x(t)$.
	$$x(t) = \dfrac{t^2}{2} - t + 1.$$
	\textbf{Ответ:} $x(t) = \dfrac{t^2}{2} - t + 1.$\\\\
	\textbf{Пример 7.} Решить граничную задачу:
	$$D^2x = 1,\ x|_{t=0} = 4,\ x|_{t = 2} = 6,\quad t \in \I = \Rm.$$
	\textbf{Решение.} В предыдущем примере мы нашли общее решение уравнения. Оно имеет вид $$x(t) =\dfrac{t^2}{2} + C_1t + C_2.$$
	Аналогично начальным задачам, граничные задачи также решаются подстановкой. Однако сейчас нам не нужно задействовать производные от функции, все условия мы подставляем непосредственно в общее решение:
	$$\begin{cases}
		4 = 0 + 0 + C_2,\\
		6 = 2 + 2C_1 + C_2;
	\end{cases}\Rightarrow\begin{cases}
	C_2 = 4,\\
	6 = 2C_1 + 6;
\end{cases}\Rightarrow \begin{cases}
	C_2 = 4,\\
	C_1 =0;
\end{cases}$$ Полученные коэффициенты подставляем в общее решение и получаем $$x(t) = \dfrac{t^2}{2} + 4.$$
\textbf{Ответ:} $x(t) = \dfrac{t^2}{2} + 4.$\\\\
$\bullet$ \textit{Начальная задача вида $$\begin{cases} F(t, x, x', \dots, x^{(n)}) = 0,\\ x|_{t = t_0} = \xi_0, x'|_{t = t_0} = \xi_1, \dots, x^{(n-1)}|_{t = t_0} = \xi_{n-1};\end{cases}\quad t_0 \in \mathbb{I}$$ называется \textbf{задачей Коши}}.\\\\
То есть, исходя из определения, задача Коши --- начальная задача, где в каждом условии $t_0$ имеет одно и то же значение.\\\\
\textbf{Пример 8.} Решить задачу Коши:
$$D^3x = e^t,\ x|_{t=1} = 1,\ Dx|_{t = 1} = 0,\ D^2x|_{t = 1} = 1\quad t \in \I = \Rm.$$
\textbf{Решение.} Для начала найдем общее решение уравнения
$$D^2x = \int e^tdt = e^t + C_1;$$
$$Dx = \int (e^t + C_1) dt = e^t + C_1t+C_2;$$
$$x = \int (e^t + C_1t+C_2) dt = e^t + \dfrac{C_1t^2}{2} + C_2t+C_3.$$
Подставим начальные условия и составим систему уравнений
$$\begin{cases}
	1 = e + C_1 + C_2 + C_3,\\
	0 = e + C_1 + C_2,\\
	1 = e + C_1;
\end{cases}\Longleftrightarrow\begin{cases}
C_1 = 1-e,\\
C_2 = -1,\\
C_3 = 1.
\end{cases}$$
Полученные коэффициенты подставим в общее решение и получим
$$x(t) = e^t + \dfrac{(1-e)t^2}{2} - t + 1.$$
\textbf{Ответ:} $x(t) = e^t + \dfrac{(1-e)t^2}{2} - t + 1.$\\\\
Иногда в задачах может стоять вопрос о поиске решения \textbf{нулевой задачи Коши}. В таком случае для $x^{(i)}|_{t=t_0} = \xi_i$ его $\xi_i = 0$. То есть, к примеру, условие \begin{center}
	$"$найти решение нулевой задачи Коши при $t = 3"$
\end{center} будет эквивалентно условию
\begin{center}
	$"$найти решение задачи Коши при $x|_{t=3} = 0$, $Dx|_{t=3} = 0$, $D^2x|_{t=3} = 0"$.
\end{center} 
\end{document}