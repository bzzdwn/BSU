\documentclass[a4paper, 12pt]{article}
\usepackage{cmap}
\usepackage{amssymb}
\usepackage{amsmath}
\usepackage{graphicx}
\usepackage{amsthm}
\usepackage{upgreek}
\usepackage{setspace}
\usepackage[T2A]{fontenc}
\usepackage[utf8]{inputenc}
\usepackage[normalem]{ulem}
\usepackage{mathtext} % русские буквы в формулах
\usepackage[left=2cm,right=2cm, top=2cm,bottom=2cm,bindingoffset=0cm]{geometry}
\usepackage[english,russian]{babel}
\usepackage[unicode]{hyperref}
\newenvironment{Proof} % имя окружения
{\par\noindent{$\blacklozenge$}} % команды для \begin
{\hfill$\scriptstyle\boxtimes$}
\newcommand{\Rm}{\mathbb{R}}
\newcommand{\Cm}{\mathbb{C}}
\newcommand{\I}{\mathbb{I}}
\renewcommand{\phi}{\upvarphi}
\renewcommand{\varphi}{\upvarphi}
\renewcommand{\alpha}{\upalpha}
\renewcommand{\psi}{\uppsi}
\renewcommand{\tau}{\uptau}
\renewcommand{\mu}{\upmu}
\renewcommand{\omega}{\upomega}
\renewcommand{\d}{\partial}
\newcommand{\N}{\mathbb{N}}
\renewcommand{\leq}{\leqslant}
\renewcommand{\geq}{\geqslant}
\renewcommand{\alpha}{\upalpha}
\renewcommand{\beta}{\upbeta}
\renewcommand{\gamma}{\upgamma}
\renewcommand{\delta}{\updelta}
\renewcommand{\varphi}{\upvarphi}
\renewcommand{\tau}{\uptau}
\renewcommand{\lambda}{\uplambda}
\renewcommand{\psi}{\uppsi}
\renewcommand{\mu}{\upmu}
\renewcommand{\omega}{\upomega}
\renewcommand{\d}{\partial}
\renewcommand{\xi}{\upxi}
\renewcommand{\epsilon}{\upvarepsilon}
\newcommand{\Ln}{L_n = D^n + a_{n-1}D^{n-1} + \ldots + a_1D + a_0D^0}
\begin{document}
	\section*{Однородные уравнения.}
	$\bullet$ \textit{Уравнение $$P(x,y)dx + Q(x,y)dy = 0$$
	называется \textbf{однородным (ОУ)}, если $P(tx,ty) = t^k\cdot P(x,y)$ и $Q(tx,ty) = t^k\cdot  Q(x,y)$, $(tx,ty)\in D$.}\\\\
Алгоритм решения таких уравнений следующий: \begin{itemize}
	\item приводим уравнение к виду $y' = f\Big(\dfrac{y}{x}\Big)$ (к виду разрешенному относительно производной);
	\item делаем подстановку $y = ux$;
	\item отсюда $xu' = f(u) - u$;
	\item получившееся уравнение будет являться УРП и останется лишь проинтегрировать уравнение $$\dfrac{du}{f(u) - u} = \dfrac{dx}{x}.$$
\end{itemize}
\textbf{Пример 1.} Найти общий интеграл уравнения $$(x + 2y)dx - xdy = 0.$$
\textbf{Решение.} Проверим, является ли данное уравнение ОУ: $$P(x,y) = x+2y \quad \Rightarrow\quad P(tx,ty) = t(x + 2y),\qquad Q(x,y) = -x\quad \Rightarrow\quad Q(tx,ty) = -tx.$$
Следовательно, данное уравнение является ОУ. Приведем уравнение к виду разрешенному относительно производной от $y$:
$$y' = \dfrac{x + 2y}{x} = 1 + \dfrac{2y}{x}.$$
Применяем подстановку $y = ux$. Тогда $y'_x = u'_x x + u$ (или просто $y' = u'x + u$). Отсюда $u'x = y' - u$, а $y'$ мы только что выразили. Соответственно подставим $y'$ и получим $$xu' = 1 + 2u - u = 1 + u.$$
Перенесём всё в левую часть и домножим на $dx$:
$$xdu - (1+u)dx = 0.$$
А данное уравнение является УРП. Домножим его на интегрирующий множитель $\mu(x,u) = \dfrac{1}{x(1+u)}$ и получим $$\dfrac{du}{(1+u)} - \dfrac{dx}{x} = 0.$$
Найдем решение этого уравнения, как решение уравнения с разделенными переменными и получим $$\int\limits_{0}^u\dfrac{du}{(1+u)} - \int\limits_{1}^x\dfrac{dx}{x} = \ln |1+u|\Big|_{0}^u -\ln |x|\Big|_{1}^x  = \ln(1+u) - \ln x = \ln\dfrac{1+u}{x} = C.$$
Отсюда $$\dfrac{1 + u}{x} = C \quad \Rightarrow\quad 1+u = Cx \quad\Rightarrow\quad 1 + \dfrac{y}{x}= Cx \quad\Rightarrow\quad x + y = Cx^2.$$
\textbf{Ответ:} $x+y = Cx^2$.\\\\
Все такие уравнения решаются только по одному алгоритму, поэтому для закрепления рассмотрим ещё одно такое уравнение и будем двигаться дальше.\\\\
\textbf{Пример 2.} Найти общий интеграл уравнения $$xdy - \Big(y + x\tg \dfrac{y}{x}\Big)dx = 0.$$
\textbf{Решение.} Проверим, является ли уравнение однородным:
$$P(tx,ty) = tx,\qquad Q(tx,ty) = -t\Big(y + x\tg t\dfrac{y}{x}\Big).$$
То есть уравнение является однородным. Приведем его к виду разрешенному относительно производной $y'$:
$$y' = \dfrac{y}{x} + \tg \dfrac{y}{x}.$$
Применим подстановку $y = ux$. Тогда $$xu' = u + \tg u - u = \tg u.$$
Приведем это уравнение к УРП и найдем его полный интеграл:
$$xdu - \tg u dx = 0\quad\Rightarrow\quad = \dfrac{du}{\tg u} - \dfrac{dx}{x} = 0.$$
\begin{multline*}
	\int\limits_{\pi / 4}^{u}\dfrac{du}{\tg u} - \int\limits_{1}^{x}\dfrac{dx}{x} = \int\limits_{\pi / 4}^{u}\dfrac{(\cos u)du}{\sin u} - \ln |x|\Big|_{1}^x  =  \int\limits_{\pi / 4}^{u}\dfrac{d(\sin u)}{\sin u} - \ln x=\\ = \ln |\sin u|\Big|_{\pi / 4}^u - \ln x = \ln (\sin u) - \ln x = \ln \dfrac{\sin u}{x} = C.
\end{multline*}
Тогда $$\dfrac{\sin u}{x} = C\quad\Rightarrow\quad \sin u = Cx \quad \Rightarrow\quad \sin \dfrac{y}{x} = Cx.$$
\textbf{Ответ:} $\sin \dfrac{y}{x} = Cx$.\\\\
Далеко не все элементарные уравнения являются однородными. Но некоторые из них можно привести к однородным уравнениям.\\\\
Уравнение $$\varphi_1(a_1x + b_1 y + c_1)dx + \varphi_2(a_2x + b_2 y + c_2)dy = 0$$ или $$y '= \varphi\Big(\dfrac{a_2x + b_2 y + c_2}{a_1x + b_1 y + c_1}\Big),$$ где \begin{enumerate}
	\item $a_1b_2 \ne a_2b_1$, можно привести к ОУ с помощью подстановки $x = x_1 + \alpha$, $y = y_1 + \beta$, где $x_1, y_1$ --- функции, а $\alpha,\beta$ --- постоянные, удовлетворяющие СЛАУ $$\begin{cases}
		\alpha a_1 + \beta b_1 + c_1 = 0,\\
		\alpha a_2 + \beta b_2 + c_2 = 0.
	\end{cases}\eqno(1)$$
\item $a_1b_2 = a_2b_1$, можно привести к УРП с помощью подстановки $a_2 x + b_2 y = u$ (т.к. в таком случае система (1) несовместна).
\end{enumerate}
\textbf{Пример 3.} Проинтегрировать уравнение $$x-y-1 + (-x + y + 2)\cdot y' = 0.$$
\textbf{Решение.} Домножим уравнение на $dx$, тогда $$(x-y-1)dx + (-x + y + 2) dy = 0.$$
В нашем случае коэффициенты $a_1 = 1$, $b_1 = -1$, $a_2 = -1$, $b_2 = 1$. Тогда $a_1b_2 = a_2b_1$. Значит применим подстановку $$a_2x + b_2y = -x + y = u.$$
Отсюда же $du = dy - dx$, и пусть переменную $u$ возьмем вместо $y$. Тогда, подставляя в исходное уравнение, получим $$(-u-1)dx + (u+2)(du + dx) = 0.$$
Раскроем скобочки, тогда $$dx + (u+2)du = 0.$$
Полученное уравнение является уравнением с разделенными переменными. А его интегрировать мы уже умеем. Тогда его общий интегарл будет иметь вид $$x + \dfrac{u^2}{2} + 2u = C.$$
Сделаем обратную замену и домножим на 2:
$$2x + (y - x)^2 - 4y - 4x  = -2x - 4y + (y-x)^2 = C.$$
При желании можно раксрыть скобочки и попытаться собрать всё в полный квадрат.\\\\
\textbf{Ответ:} $-2x - 4y + (y-x)^2 = C.$\\\\
\textbf{Пример 4.} Проинтегрировать уравнение $$(2x - y -1)dx + (-x + 2y +1)dy = 0.$$
\textbf{Решение.} Введем подстановку $x = x_1 + \alpha$, $y = y_1 + \beta$, причем коэффициенты $\alpha$ и $\beta$ найдем из системы (1):
$$\begin{cases}
	2\alpha - \beta - 1 = 0,\\
	-\alpha + 2\beta +1 = 0
\end{cases}\Rightarrow\quad \begin{pmatrix}
2 & -1 & \vline & 1\\
-1 & 2 & \vline & -1
\end{pmatrix}\sim \begin{pmatrix}
1 & 0 & \vline & 1/3\\
0 & 1 & \vline & -1/3
\end{pmatrix}.$$
Отсюда $\alpha = 1/3$, $\beta = -1/3$. Тогда $x = x_1 + 1/3$, $y = y_1 -1/3$. Подставим в исходное уравнение (причем $dx = dx_1$, $dy = dy_1$)
$$(2x_1 -y_1)dx_1 + (-x_1 + 2y_1)dy_1 = 0.$$
Проверить, правильно ли были найдены $\alpha$ и $\beta$ можно так: исходные коэффициенты $c_1$ и $c_2$ в уравнении с подставленной заменой должны сократиться.\\\\
Получившееся уравнение является ОУ, а его интегрировать мы уже умеем. Для начала приведем уравнение к виду разрешенному относительно производной:
$$y_1' = \dfrac{y_1 - 2x_1}{2y_1 - x_1}.$$
Тогда сделаем замену $y_1 = ux_1$:
$$u'x_1 = \dfrac{ux_1 - 2x_1}{2ux_1 - x_1} - u = \dfrac{u - 2}{2u  -1} - u = \dfrac{-2u^2 + 2u - 2}{2u - 1}.$$
Получили УРП:
$$-\dfrac{1}{2}\cdot\dfrac{(2u-1)du}{u^2 - u +1} - \dfrac{dx_1}{x_1} = 0.$$
$$-\dfrac{1}{2}\int\limits_{u_0}^u\dfrac{(2u-1)du}{u^2 - u +1} - \int\limits_{x_{1_0}}^{x_1}\dfrac{dx_1}{x_1} = C.$$
$$-\dfrac{1}{2}\ln(u^2 - u + 1) - \ln x_1 = C.$$
Домноижм уравнение на -2 и воспользуемся свойствами логарифма$$\ln (x_1^2\cdot (u^2 - u + 1)) = C.$$
Так как $u = y_1/x_1$, то $$\ln(y_1^2 - y_1x_1 + x_1^2) = C \Rightarrow y_1^2 - y_1x_1 + x_1^2 = C.$$
Тогда, учитывая, что $x_1 = x-1/3$, $y_1 = y + 1/3$, сделаем обратную замену и получим общее решение исходного уравнения $$y^2 + y - xy - x + x^2 = C.$$
\textbf{Ответ:} $y^2 + y - xy - x + x^2 = C.$
\end{document}