\documentclass[a4paper, 12pt]{article}
\usepackage{cmap}
\usepackage{amssymb}
\usepackage{amsmath}
\usepackage{graphicx}
\usepackage{amsthm}
\usepackage{upgreek}
\usepackage{setspace}
\usepackage[T2A]{fontenc}
\usepackage[utf8]{inputenc}
\usepackage[normalem]{ulem}
\usepackage{mathtext} % русские буквы в формулах
\usepackage[left=2cm,right=2cm, top=2cm,bottom=2cm,bindingoffset=0cm]{geometry}
\usepackage[english,russian]{babel}
\usepackage[unicode]{hyperref}
\newenvironment{Proof} % имя окружения
{\par\noindent{$\blacklozenge$}} % команды для \begin
{\hfill$\scriptstyle\boxtimes$}
\newcommand{\Rm}{\mathbb{R}}
\newcommand{\Cm}{\mathbb{C}}
\newcommand{\I}{\mathbb{I}}
\renewcommand{\phi}{\upvarphi}
\renewcommand{\varphi}{\upvarphi}
\renewcommand{\alpha}{\upalpha}
\renewcommand{\psi}{\uppsi}
\renewcommand{\tau}{\uptau}
\renewcommand{\mu}{\upmu}
\renewcommand{\omega}{\upomega}
\renewcommand{\d}{\partial}
\newcommand{\N}{\mathbb{N}}
\renewcommand{\leq}{\leqslant}
\renewcommand{\geq}{\geqslant}
\renewcommand{\alpha}{\upalpha}
\renewcommand{\beta}{\upbeta}
\renewcommand{\gamma}{\upgamma}
\renewcommand{\delta}{\updelta}
\renewcommand{\varphi}{\upvarphi}
\renewcommand{\tau}{\uptau}
\renewcommand{\lambda}{\uplambda}
\renewcommand{\psi}{\uppsi}
\renewcommand{\mu}{\upmu}
\renewcommand{\omega}{\upomega}
\renewcommand{\d}{\partial}
\renewcommand{\xi}{\upxi}
\renewcommand{\epsilon}{\upvarepsilon}
\newcommand{\Ln}{L_n = D^n + a_{n-1}D^{n-1} + \ldots + a_1D + a_0D^0}
\begin{document}
	\section*{Приведение уравнений в общей форме к уравнениям в нормальной форме.}
	$\bullet$ \textit{\textbf{Обыкновенным дифференциальным уравнением в общей форме}, или \textbf{уравнением неразрешенным относительно производной (УНОП)} называется уравнение вида $$F(x,y,y') = 0,$$ где функция $F$ определена в некоторой области $D\subseteq \Rm^2$.}\\\\
	Исходя из названия, данный тип уравнений мы не можем разрешить известными нам методами. Однако такие уравнения мы можем свести к тем, с которыми сталкивались уже ранее, а именно к разрешенным относительно производной. Для этого рассмотрим 2 метода.\\\\
	\textbf{Первый метод.} Уравнение вида $$(y')^n + a_{n-1}(x,y)(y')^{n-1} + \ldots + a_1(x,y)y' + a_0(x,y) = 0$$ называется \textbf{алгебраическим относительно производной}. С помощью вынесения общих множителей, выделения полных квадратов, кубов и т.д. данные уравнения можно свести к нескольким разрешенным относительно производной уравнениям.\\\\
	\textbf{Второй метод.} Продифференцировав УНОП по $x$, мы получим $$\Big(F(x,y,y')\Big)'_x = F'_x + F'_y\cdot y' + F_{y'}\cdot y'' = 0.$$
	Данное уравнение является разрешенным относительно $y''$. Но в таком случае возникают некоторые нюансы в связи с повышением порядка уравнения. А именно, если все рещения исходного уравнения находились во множестве $\I_1$, то теперь они перейдут во множество $\I_2 \supseteq \I_1$, т.е. появятся новые решения, которые не входят в исходное множество. Поэтому их необходимо исключать.\\\\
	\textbf{Пример 1.} Найти полное решение уравнения $$(y')^2 - 2xy' = 8x^2.$$
	\textbf{Решение.} Преобразуем данное уравнение. Если мы прибавим к обеим частям уравнения $x^2$, то левую часть можно свернуть в полный квадрат:
	$$(y')^2 - 2xy' + x^2 = 9x^2\ \Rightarrow\ (y' - x)^2 = 9x^2.$$
	Перенесём правую часть равенства влево и получим разность квадратов:
	$$(y' - x)^2 - 9x^2 = 0 \ \Rightarrow\ (y' - x - 3x)(y' - x + 3x) = (y' - 4x)(y' + 2x)= 0.$$
	Мы получили произведение двух уравнений разрешенных относительно производной. Поэтому полным решеним исходного уравнения будет совокупность функций, которые являются решениями каждого из множителей.\\\\
	Рассмотрим уравнение $y' - 4x = 0$. Тогда $y' = 4x$ --- простейшее ДУ (т.к. $y = y(x)$). Проинтегрируем его $$y = \int\limits_{x_0}^x4\tau d\tau + C = 2x^2 + C.$$ 
	Теперь рассмотрим уравнение $y' + 2x=0$. Уравнение $y' = -2x$ также является простейшим. Аналогично проинтегрируем его $$y= \int\limits_{x_0}^x (-2)\tau d\tau + C = -x^2 + C.$$ 
	Тогда полным решением исходного уравнения будет совокупность двух функций $$\left[
	\begin{aligned}
		&y = 2x^2 + C,\\
		&y = -x^2 + C.
	\end{aligned}
	\right.$$
	\textbf{Ответ:} $\left[
	\begin{aligned}
		&y = 2x^2 + C,\\
		&y = -x^2 + C.
	\end{aligned}
	\right.$\\\\
	\textbf{Пример 2.} Найти полное решение уравнения $$(y')^3 - (x^2 + xy + y^2)((y')^2 - xyy') - x^3y^3 = 0.$$
	\textbf{Решение.} Преобразуем уравнение. Для начала раскроем скобки и получим $$(y')^3 - x^2y^2 + x^3yy' - xy(y')^2 + x^2y^2y' - y^2(y')^2 + xy^3y' - x^3y^3 = 0.$$
	Вынесем общие множители
	$$(y')^2(y'-xy) + x^2y^2(y'-xy) - y^2y'(y'-xy) - x^2y'(y'-xy) = 0.$$
	$$(y'-xy) = ((y')^2 + x^2y^2 - y^2y' - x^2y') = 0.$$
	Снова вынесем общие множители и получим
	$$(y'-xy)(y'-y^2)(y' - x^2) = 0.$$
	В итоге у нас получилось произведение уравнений разрешенных относительно $y'$.\\\\
	Рассмотрим уравнение $y'-xy = 0$. Это уравнение линейно относительно $y$ и его общее решение имеет вид $$y = Ce^{x^2/2}.$$
	Рассмотрим уравнение $y' - y^2 = 0.$ Это УРП. Проинтигреруем его $$\dfrac{dy}{y^2} -dx = 0.$$
	Тогда $-\dfrac1y - x  = C$. Отсюда общее решение $$y = -\dfrac{1}{x + C}.$$
	Рассмотрим уравнение $y' - x^2 = 0$. Это уравнение простейшее, следовательно его общее решение $$y = \dfrac{x^3}{3} + C.$$
	Таким образом, полным решеним исходного уравнения является совокупность функций $$\left[
	\begin{aligned}
		&y = Ce^{x^2/2},\\
		&y = -\dfrac{1}{x + C},\\
		&y = \dfrac{x^3}{3} + C.
	\end{aligned}
	\right.$$
	\textbf{Ответ:} $\left[
	\begin{aligned}
		&y = Ce^{x^2/2},\\
		&y = -\dfrac{1}{x + C},\\
		&y = \dfrac{x^3}{3} + C.
	\end{aligned}
	\right.$\\\\
	\textbf{Пример 3.} Найти полное решение уравнения $$(y')^2 - 4y = 0.$$
	\textbf{Решение.} Данное уравнение можно также преобразовать. Однако попробуем в этот раз воспользоваться вторым методом. Проинтегрируем данное уравнение по $x$. Тогда $$2y'\cdot y'' - 4y' = 0.$$
	Тогда $$2y'(y'' - 2) = 0.$$
	Получаем совокупность простейших уравнений $$\left[
	\begin{aligned}
		&y' = 0,\\
		&y'' = 2.
	\end{aligned}
	\right.\ \Rightarrow\ \left[
	\begin{aligned}
		&y = C,\\
		&y' = 2x + C_1.
	\end{aligned}
	\right.\ \Rightarrow\ \left[
	\begin{aligned}
		&y = C,\\
		&y = x^2 + C_1x + C_2.
	\end{aligned}
	\right.$$
	Однако данные уравнения не являются решениями исходного, так как мы расширили множество решений, повысив порядок уравнения. Поэтому выделим из совокупности те решения, которые удовлетворяю исходному уравнению.\\\\
	Возьмем $y = C$ и подставим в исходное уравнение
	$$0 - 4C = 0\ \Rightarrow\ C = 0.$$
	То есть из всех решений $y = C$ для исходного уравнения подходит только решение $$y = 0.$$
	Проделаем аналогичные действия с решением $y = x^2 + C_1x + C_2$. Подставим его в исходное уравнение $$4x^2 + C_1^2 + 4xC_1 - 4x^2 - 4xC_1 - 4C_2 = 0.$$
	Отсюда $$C_1^2 = 4C_2\ \Rightarrow\ C_1 = \pm 2\sqrt{C_2}.$$
	Возьмем $C_1 = 2\sqrt C_2$ и подставим в уравнение $y = x^2 + C_1x + C_2$. Тогда $$y = x^2 + 2\sqrt C_2x + C_2\ \Rightarrow\ y = (x + \sqrt C_2)^2.$$
	Причем $\sqrt C_2$ можно заменить на другую константу $C = \sqrt C_2$. Тогда получаем решение $$y= (x + C)^2.$$
	Теперь возьмем $C_1 = -2\sqrt C_2$. Также подставим в $y = x^2 + C_1x + C_2$ и получим $$y = (x - \sqrt C_2)^2.$$
	Здесь мы также можем заменить $C = -\sqrt C_2$, тем самым получив то же самое решение  $$y= (x + C)^2.$$
	В итоге полное решение исходного уравнения будет составлять совокупность функций $$\left[
	\begin{aligned}
		&y = 0,\\
		&y = (x + C)^2.
	\end{aligned}
	\right.$$
	\textbf{Ответ:} $\left[
	\begin{aligned}
		&y = 0,\\
		&y = (x + C)^2.
	\end{aligned}
	\right.$
\end{document}