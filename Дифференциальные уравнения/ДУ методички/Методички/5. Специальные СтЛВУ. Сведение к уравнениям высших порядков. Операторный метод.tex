\documentclass[a4paper, 12pt]{article}
\usepackage{cmap}
\usepackage{amssymb}
\usepackage{amsmath}
\usepackage{graphicx}
\usepackage{amsthm}
\usepackage{upgreek}
\usepackage{setspace}
\usepackage[T2A]{fontenc}
\usepackage[utf8]{inputenc}
\usepackage[normalem]{ulem}
\usepackage{mathtext} % русские буквы в формулах
\usepackage[left=2cm,right=2cm, top=2cm,bottom=2cm,bindingoffset=0cm]{geometry}
\usepackage[english,russian]{babel}
\newenvironment{Proof} % имя окружения
{\par\noindent{$\blacklozenge$}} % команды для \begin
{\hfill$\scriptstyle\boxtimes$}
\newcommand{\Rm}{\mathbb{R}}
\newcommand{\Cm}{\mathbb{C}}
\newcommand{\I}{\mathbb{I}}
\newcommand{\N}{\mathbb{N}}
\newtheorem*{theorem}{Теорема}
\newtheorem*{cor}{Следствие}
\newtheorem*{lem}{Лемма}
\newcommand{\Ln}{L_n = D^n + a_{n-1}D^{n-1} + \ldots + a_1D + a_0D^0}
\begin{document}
	\section*{СтЛВУ. Специальные СтЛВУ. Сведение к уравнениям высших  порядков. Операторный метод интегрирования.}
	Мы переходим к рассмотрению нового раздела "Стационарные линейные векторные уравнения", или "Системы стационарных линейных уравнений". И сразу же начнем с пары базовых определений.\\\\
	$\bullet$ \textit{\textbf{Системой ДУ} называется совокупность выражений вида
		$$F_i(t,x_1, Dx_1, \ldots, D^{m_1}x_1,\ldots,x_k,D^kx,\ldots,D^{m_k}x_k) = 0,$$ $$i = 1,\ldots,s,\quad t\in \I\subseteq \Rm,$$ где $F_i$ --- некоторая функция от своих  переменных.}\\\\
	Вынесем производные высших порядков и получим следующее.\\\\
	$\bullet$ \textit{Говорят, что система ДУ \textbf{имеет нормальную форму}, если она состоит из уравнений вида $$D^{m_i}x_i(t) = f_i(t,x_1,Dx_1,\ldots,D^{m_1-1}x_1, \ldots, x_k(t), Dx_k, \ldots, D^{m_k - 1}x_k),\quad i = 1,\ldots, k.$$}
	$\bullet$ \textit{Если функции $f_i$ являются линейными функциями от неизвестных функций $x_i(t)$ и их производных, то система называется \textbf{линейной}.}\\\\
	Однако удобнее рассматривать системы в более наглядном виде:
	$$\begin{cases}
		Dx_1=a_{11}(t)x_1(t) + \ldots + a_{1n}(t)x_n(t) + f_1(t),\\
		\dotfill\\
		Dx_n=a_{n1}(t)x_1(t) + \ldots + a_{nn}(t)x_n(t) + f_n(t);
	\end{cases}\quad t \in \I.\eqno(1)$$
	$\bullet$ \textit{\textbf{Решением} системы $(1)$ называется совокупность непрерывно дифференцируемых на $\I$ функций $x_1(t),\ldots, x_n(t)$, обращающих систему $(1)$ в верное равенство.}\\\\
	$\bullet$ \textit{Если коэффициенты $a_{ij}$ систем являются постоянными, то системы называются \textbf{стационарными линейными}.}\\\\
	$\bullet$ \textit{\textbf{Задачей Коши} для системы $(1)$ называется задача отыскания решения системы $(1)$, удовлетворяющего условиям
		$$x_1|_{t = t_0} = \xi_0, x_2|_{t = t_0} = \xi_1, \dots, x_n|_{t = t_0} = \xi_{n},\quad t_0 \in \I.$$}
	Обозначив $X(t) = \begin{pmatrix}
		x_1(t)\\\vdots\\x_n(t)
	\end{pmatrix}$, $DX(t) = \begin{pmatrix}
		Dx_1(t)\\\vdots\\Dx_n(t)
	\end{pmatrix}$, $A=(a_{ij})$, $f(t) = \begin{pmatrix}
		f_1(t)\\\vdots\\f_n(t)
	\end{pmatrix}$,  систему (1) можно записать в матричном виде $$DX = AX + f(t).\eqno(2)$$
	А начальные условия задачи Коши в виде $X|_{t=t_0} = \xi$, где $\xi = \begin{pmatrix}
		\xi_1\\\vdots\\\xi_n
	\end{pmatrix}$.\\\\
	$\bullet$ \textit{Уравнение $(2)$ называется \textbf{линейным стационарным векторным уравнением}.}\\\\
	В дальнейшем будем рассматривать уравения в виде (2).
	\subsection*{Специальные СтЛВУ.}
	Теперь, когда мы ввели все необходимые нам определения и обозначения, мы можем перейти к рассмотрению первого типа задач. Характеризуются они специальным видом матрицы $A$. Матрица $A$ в таких уравнениях обязательно имеет или диагональный вид, или треугольный вид. Решаются такие уравнения путем последовательного решения уравнений первого порядка относительно одной неизвестной.\\\\
	\textbf{Пример 1.} Найти общее решение уравнения вида $DX = AX + f(t)$, где $$A = \begin{pmatrix}
		2 & 0 & 0\\
		1 & 1 & 0\\
		1 & -1 & 3
	\end{pmatrix},\quad f(t)\equiv 0.$$
	\textbf{Решение.} Столбец неоднородности в нашем случае равен нулю, следовательно, уравнение имеет вид $DX = AX$. Перепишем в виде (1):
	$$\begin{cases}
		Dx_1 = 2x_1,\\
		Dx_2 = x_1 + x_2,\\
		Dx_3 = x_1 - x_2 + 3x_3;
	\end{cases}$$
	В первой строке системы СтЛОУ относительно одной неизвестной. Следовательно, мы можем найти общее решение для этого уравнения. Оно имеет вид $$x_1(t) = C_1e^{2t}.$$
	Теперь подставим полученную функцию в во вторую строку системы и получим 
	$$Dx_2 = C_1e^{2t} + x_2 \Longleftrightarrow Dx_2 - x_2 = C_1e^{2t}.$$
	Получили СтЛНУ-1. Найдем его общее решение методом Лагранжа:\\ $x_{2\text{oo}} = C_2e^t$, $x_{2\text{чн}} = u_2e^t$. Тогда $u_2'e^t = C_1e^{2t} \Rightarrow u_2 = C_1e^t.$ Получаем, что $x_{2\text{чн}} = C_1e^{2t}$, $$x_{2}(t) = C_2e^t + C_1e^{2t}.$$
	Полученные функции $x_1$ и $x_2$ подставим в последнее равенство системы и получим
	$$Dx_3 = C_1e^{2t} - C_2e^t - C_1e^{2t} +3x_3 \Longleftrightarrow Dx_3 - 3x_3 = -C_2e^t.$$
	Получили СтЛНУ-1. Найдем его общее решние методом Эйлера:
	$x_{3\text{oo}} = C_3e^{3t}$, $x_{3\text{чн}} = Ae^t$, $Dx_3 = Ae^t$. Тогда $-2Ae^t = -C_2e^t$, отсюда $A = \dfrac{C_2}{2}\Rightarrow x_{3\text{чн}} = \dfrac{C_2}{2}e^t.$ Получаем
	$$x_3(t) = C_3e^{3t} + \dfrac{C_2}{2}e^t.$$
	Таким образом, общее решение СтЛВУ имеет вид $$X = \begin{pmatrix}
		x_1\\x_2\\x_3
	\end{pmatrix} = \begin{pmatrix}
	C_1e^{2t}\\ C_2e^t + C_1e^{2t}\\C_3e^{3t} + \frac{C_2}{2}e^t
\end{pmatrix} = (C_1e^{2t}, C_2e^t + C_1e^{2t}, C_3e^{3t} + \frac{C_2}{2}e^t)^T.$$
\textbf{Ответ:} $X = (C_1e^{2t}, C_2e^t + C_1e^{2t}, C_3e^{3t} + \frac{C_2}{2}e^t)^T$.\\\\
\textbf{Пример 2.} Найти общее решение уравнения вида $DX = AX + f(t)$, где $$A = \begin{pmatrix}
	2 & -32 \\
	0 & 4
\end{pmatrix},\quad f(t)=\begin{pmatrix}
-4t - 1\\
-t^2
\end{pmatrix}.$$
\textbf{Решение.} Представим векторное уравнение в виде системы:
$$\begin{cases}
	Dx_1 = 2x_1 - 32x_2 - 4t - 1,\\
	Dx_2 = 4x_2 - t^2;
\end{cases}$$
В нижней строке системы располагается СтЛНУ-1. Найдем его общее решение методом Эйлера: $x_{2\text{oo}} = C_1e^{4t}$, $x_{2\text{чн}} = At^2 + Bt + C$, $Dx_2 = 2At + B$. Подставим и получим\\
$$2At + B - 4At^2 - 4Bt - 4C = -t^2.$$
Отсюда $A = \dfrac{1}{4}$, $B = \dfrac{1}{8}$, $C = \dfrac{1}{32}$. Следовательно, $$x_{2} = C_1e^{4t}+ \dfrac{t^2}{4} + \dfrac{t}{8} + \dfrac{1}{32}.$$
Подставим $x_2$ в верхнее уравнение системы и получим
$$Dx_1 - 2x_1 =  C_1e^{4t}+ 8t^2.$$
Решим уравнение также методом Эйлера:
$$x_{1\text{oo}} = C_2e^{2t},\ x_{1\text{чн}} = A_1e^{4t} + A_2t^2 + B_2t + C_2,\ Dx_1 = 4A_1e^{4t} + 2A_2t + B_2.$$
$$4A_1e^{4t} + 2A_2t + B_2 - 2A_1e^{4t} -2 A_2t^2 -2B_2t -2 C_2  =  C_1e^{4t}+ 8t^2.$$
Таким образом, $A_1 = \dfrac{C_1}{2}$, $A_2 = -4$, $B_2 = -4$, $C_2 = -2$. Тогда $$x_1 = C_2e^{2t} + \dfrac{C_1}{2} e^{4t} - 4t^2 - 4t - 2.$$ 
Тогда общее решение системы уравнений имеет вид $$X = \begin{pmatrix}
	C_2e^{2t} + \dfrac{C_1}{2} e^{4t} - 4t^2 - 4t - 2\\
	C_1e^{4t}+ \dfrac{t^2}{4} + \dfrac{t}{8} + \dfrac{1}{32}
\end{pmatrix} =(C_2e^{2t} + \dfrac{C_1}{2} e^{4t} - 4t^2 - 4t - 2,
C_1e^{4t}+ \dfrac{t^2}{4} + \dfrac{t}{8} + \dfrac{1}{32})^T. $$
\textbf{Ответ:} $X = (C_2e^{2t} + \dfrac{C_1}{2} e^{4t} - 4t^2 - 4t - 2,
C_1e^{4t}+ \dfrac{t^2}{4} + \dfrac{t}{8} + \dfrac{1}{32})^T.$
\begin{theorem}
	Задача Коши для действительного стационарного уравнения $$DX = AX + f(t),\quad X|_{t=t_0} = \xi\eqno(3.1.4)$$ с непрерывной на $\I$ векторной функцией $f(t)$ имеет единственное решение $\forall t_0 \in \I$, $\forall \xi~\in~\Rm_{n,1}.$
\end{theorem}
\textbf{Пример 3.} Решить задачу Коши для уравнения вида $DX = AX + f(t)$, где $$A = \begin{pmatrix}
	2 & 0 & 0\\
	0 & 8 & 0\\
	0 & 0 & 3
\end{pmatrix},\quad f(t)\equiv0,\quad t_0 = 2,\quad \xi =\begin{pmatrix}
1\\0\\2
\end{pmatrix}.$$
\textbf{Решение.} Решение задачи Коши для СтЛВУ аналогично решению задачи Коши для СтЛУ. Для начала найдем общее решение уравнения:
$$\begin{cases}
	Dx_1 = 2x_1,\\
	Dx_2 = 8x_2,\\
	Dx_3 = 3x_3;
\end{cases}$$
Получили систему из трех СтЛОУ-1. Тогда общее решение уравнения имеет вид
$$X = \begin{pmatrix}
	C_1e^{2t}\\
	C_2e^{8t}\\
	C_3e^{3t}
\end{pmatrix};$$
Остается лишь подставить значения столбца $X$ при $t = t_0 = 2$ и найти значения постоянных $C_i$
$$X|_{t=2} = \begin{pmatrix}
	C_1e^{4}\\
	C_2e^{16}\\
	C_3e^{6}
\end{pmatrix} = \begin{pmatrix}
1\\
0\\
2
\end{pmatrix}\Rightarrow\begin{cases}
C_1e^{4} = 1,\\
C_2e^{16} = 0,\\
C_3e^{6} = 2;
\end{cases}\Longleftrightarrow\begin{cases}
C_1 = e^{-4},\\
C_2 = 0,\\
C_3 = 2e^{-6};
\end{cases}$$
Тогда получаем решение задачи Коши
$$X = \begin{pmatrix}
	e^{2t-4}\\
	0\\
	2e^{3t-6}
\end{pmatrix} = (e^{2t-4}, 0, 2e^{3t-6})^T.$$
\textbf{Ответ:} $(e^{2t-4}, 0, 2e^{3t-6})^T$.\\\\
\textit{\textbf{Замечание:} Далее подробное описание нахождения частного решения линейного уравнения относительно одной переменной будет опускаться. Предполагается, что у читателя уже достаточно навыков, для самостоятельного вычисления.}
\subsection*{Сведение СтЛВУ к уравнениям высших порядков.}
Перейдем к рассмотрению следующего метода. Поскольку векторное стационарное уравнение можно записать в виде системы линейны уравнений, то и найти решение системы можно достаточно тривиальным способом --- выражением и подстановкой переменных. Таким образом, уравнения с несколькими неизвестными в системе можно привести к уравнениям высших порядком относительно одной неизвестной функции.\\\\
Алгоритм нахождения решения незамысловатый:\begin{itemize}
	\item дифференцируем первое уравнение по $x_1$;
	\item подставляем в него дифференциал из другого уравнения;
	\item из исходного первого уравнения выражаем какую-либо переменную $x_2,\ldots,x_n$ и подставляем в получившееся;
	\item если не получили СтЛУ относительно одной неизвестной, то повторяем действия;
	\item если получили СтЛУ, находим функцию $x_1$, подставляем ее в одно из остальных $n-1$ уравнений;
	\item повторяем действия, пока не найдем все $x_i$ переменные.
\end{itemize}
Необязательно начинать именно с первой строки как в алгоритме. Можно так же начать алгоритм с любой другой строки.\\\\
\textbf{Пример 4.} Путем сведения к уравнениям высших порядков разрешить систему:
$$\begin{cases}
	Dx_1 = -2x_1 + x_3,\\
	Dx_2 = 2x_2 ,\\
	Dx_3 = -x_1.
\end{cases}$$
\textbf{Решение.} Сначала рассмотрим всё уравнение и постараемся выделить части, которые мы сразу можем решить. В нашем случае это вторая строчка $Dx_2 = 2x_2$. Данное уравнение зависит лишь от одной переменной, и мы сразу можем сказать, что решением будет функция $$x_2 = C_1e^{2t}.$$
Тогда можем выбросить из исходной системы вторую строку и получить $$\begin{cases}
	Dx_1 = -2x_1 + x_3,\\
	Dx_3 = -x_1.
\end{cases}$$ Без преобразований найти решение в таком уравнении мы уже не сможем. Тогда воспользуемся методом сведения к уравнению высшего порядка. Продифференцируем первую строку: 
$$\begin{cases}
	D^2x_1 = -2Dx_1 + Dx_3,\\
	Dx_3 = -x_1.
\end{cases}$$
Теперь мы можем подставить вторую строку в первую и получить уравнение высшего порядка относительно одной переменной $$D^2x_1+2Dx_1 + x_1 = 0.$$
Найдем решение этого уравнения, оно имеет вид $$x_1 = C_2te^{-t} + C_3e^{-t}.$$
Осталь найти $x_3$. Для этого вернемся к системе вида $$\begin{cases}
	Dx_1 = -2x_1 + x_3,\\
	Dx_3 = -x_1,
\end{cases}$$
где из первой строки выразим переменную $x_3$:
$$x_3 = Dx_1 + 2x_1.$$
Функцию $x_1$ мы знаем. Следовательно, мы можем найти и функцию $Dx_1 = -C_2te^{-t} + C_2e^{-t} -C_3e^{-t}.$
Подставим эти функции и получим $$x_3 = -C_2te^{-t} + C_2e^{-t} -C_3e^{-t} + 2C_2te^{-t} + 2C_3e^{-t} = C_2te^{-t} + C_2e^{-t} + C_3e^{-t}.$$
Все неизвестные функции мы нашли. Тогда мы можем составить общее решение исходной системы уравнений $$X(t) = \begin{pmatrix}
	C_2te^{-t} + C_3e^{-t}\\
	C_1e^{2t}\\
	C_2te^{-t} + C_2e^{-t} + C_3e^{-t}
\end{pmatrix}.$$
Вообще говоря, мы могли бы для поиска $x_3$ использовать и последнюю строку исхондой системы $Dx_3 = -x_1$, однако тогда пришлось бы интегрировать $x_1$. Но тут кому как удобнее.\\\\
\textbf{Ответ:} $X(t) = \begin{pmatrix}
	C_2te^{-t} + C_3e^{-t}\\
	C_1e^{2t}\\
	C_2te^{-t} + C_2e^{-t} + C_3e^{-t}
\end{pmatrix}$.\\\\
\textbf{Пример 5.} Путем сведения к уравнениям высших порядков разрешить системы:
$$\begin{cases}
	Dx_1 = -2x_1 + 2x_2 + e^t,\\
	Dx_2 = x_1 -3 x_2 + t.
\end{cases}$$
\textbf{Решение.} В данном уравнении присутствует неоднородность. Однако ход решения остается аналогичным. Продифференцируем первое уравнение системы:
$$D^2x_1 = -2Dx_1 + 2Dx_2 + e^t.$$
Подставим $Dx_2$ из второго уравнения системы и получим
$$D^2x_1 = -2Dx_1 + 2x_1 - 6x_2 + 2t + e^t.$$
Выразим $x_2$ из первого равенства системы $Dx_1 = -2x_1 + 2x_2 + e^t$:
$$x_2 = \dfrac{Dx_1 + 2x_1 - e^t}{2}$$
и подставим его в полученное уравнение. Таким образом, получаем СтЛНУ-2, которое имеет вид
$$D^2x_1 = -2Dx_1 + 2x_1 - 3Dx_1 -6x_1 + 3e^t + 2t + e^t\Longleftrightarrow D^2x_1 + 5Dx_1 + 4x_1 = 4e^t + 2t.$$
С помощью метода Эйлера найдем частное решение уравнения: $x_{1\text{чн}} = A_1e^t + A_2t + B_2 = \dfrac{2}{5}e^t + \dfrac{t}{2} - \dfrac{5}{8}.$ Тогда общее решение уравнения имеет вид:
$$x_1 = C_1e^{-4t} + C_2e^{-t} + \dfrac{2}{5}e^t + \dfrac{t}{2} - \dfrac{5}{8}.$$
Теперь найдем $x_2$. Рассмотрим исходную систему. Для вычисления $x_2$ перспективнее брать первое уравнение, так как, если вычислять с помощью второго уравнения, придется искать решение еще одного СтЛНУ. В свою очередь, в первом уравнении нам необходимо лишь знать $Dx_1$, которое мы можем запросто найти:
$$Dx_1 = -4C_1e^{-4t} -C_2e^{-t} + \dfrac{2}{5}e^t + \dfrac{1}{2}.$$
Подставим $x_1$ и $Dx_1$ в выраженное ранее из первого уравнения $x_2$ и получим
$$x_2 = -C_1e^{-4t} + C_2e^{-t} + \dfrac{1}{10}e^t + \dfrac{t}{2} - \dfrac{3}{8}.$$
Таким образом, решением векторного уравнения является столбец $$X = \begin{pmatrix}
	C_1e^{-4t} + C_2e^{-t} + \dfrac{2}{5}e^t + \dfrac{t}{2} - \dfrac{5}{8}\\
	-C_1e^{-4t} + C_2e^{-t} + \dfrac{1}{10}e^t + \dfrac{t}{2} - \dfrac{3}{8}
\end{pmatrix}.$$
\textbf{Ответ:} $(C_1e^{-4t} + C_2e^{-t} + \dfrac{2}{5}e^t + \dfrac{t}{2} - \dfrac{5}{8},\
-C_1e^{-4t} + C_2e^{-t} + \dfrac{1}{10}e^t + \dfrac{t}{2} - \dfrac{3}{8})^T$.
\subsection*{Операторный метод интегрирования.}
Операторный метод напоминает напоминает метод Гаусса для решения линейный система алгебраических уравнений и по сути своей является его модификацией. Основан он на свойстве оператора дифференцирования $D^nD^k = D^{n+k}$.\\
Для интегрирования операторным методом необходимо \begin{itemize}
	\item представить систему в операторном виде (для наглядности, но не обязательно);
	\item записать матрицу, элементами которой являются операторы $D$ при соответствующих неизвестных;
	\item элементарными преобразованиями привести матрицу к треугольному виду;
	\item перейти обратно к системе уравнений и, начиная со строки, где все элементы нулевые кроме последнего или первого, аналогично первому методу искать решения $x_1,\ldots,x_n$.
\end{itemize} Таким образом, введем определение\\\\
$\bullet$ \textit{\textbf{Элементарными преобразованиями} матриц, составленных из оператора дифференцирования, являются}\begin{enumerate}
	\item \textit{домножение строки на постоянный ненулевой множитель;}
	\item \textit{прибавление к одной строки другой, домноженной на ненулевой постоянный множитель;}
	\item\textit{ перестановка двух строк.}
\end{enumerate} 
Для столбцов преобразования неверны. Если столбец неоднородности СтЛВУ ненулевой, то записывается \textbf{расширенная матрица}, элементами которой являются операторы дифференцированиая, а последний столбец --- столбец неоднородности исходного СтЛВУ.\\\\
Стоит подметить, что приведение операторной матрицы к треугольному виду напоминает преобразование полиномиальной матрицы. Поэтому для того, чтобы потренироваться в преобразованиях, можно обратиться к теме полиномиальных матриц в курсе линейной алгебры.\\\\
\textbf{Пример 6.} Используя операторный метод, проинтегрировать систему:
$$\begin{cases}
	2D^2x_2 + Dx_1 + 3Dx_2 + 3x_1 =0,\\
	Dx_2 + x_1 + x_2 = 0.
\end{cases}$$
\textbf{Решение.} Для начала запишем систему в операторном виде (вынесем переменные $x_i$ за скобки): 
$$\begin{cases}
	(D + 3D^0)x_1 + (2D^2 + 3D)x_2 =0,\\
	 D^0x_1 + (D + D^0)x_2= 0.
\end{cases}$$
Составим матрицу системы, состояющую из операторов дифференцирования (первый столбец --- операторы при $x_1$, второй --- при $x_2$ и так далее)
$$\begin{pmatrix}
	D + 3D^0 & 2D^2 + 3D\\
	D^0 & D + D^0
\end{pmatrix}$$
Данную матрицу требуется привести к треугольному виду:
\begin{center}
	$\begin{pmatrix}
		D + 3D^0 & 2D^2 + 3D\\
		D^0 & D + D^0
	\end{pmatrix}\sim$ [меняем 1-ую и 2-ую строки местами] $\sim \begin{pmatrix}
	D^0 & D + D^0\\
	D + 3D^0 & 2D^2 + 3D
\end{pmatrix}\sim$ [1-ую строку домножим на $-D-3D^0$ и прибавим ко второй] $\sim \begin{pmatrix}
D^0 & D + D^0\\
0 & D^2 - 4D
\end{pmatrix}.$
\end{center}
Перепишем обратно матрицу в виде системы:
$$\begin{cases}
	x_1 + Dx_2 + x_2 = 0,\\
	D^2x_2 - 4Dx_2 = 0
\end{cases}$$
Из второго уравнения найдем $x_2$: $$x_2 = C_1 + C_2e^{4t}.$$ Тогда $Dx_2 = 4C_2e^{4t}$. Подставим в первое уравнение и получим $$x_1 = -4C_2e^{4t} - C_1 - C_2e^{4t}= -C_1 - 5C_2e^{4t}.$$
И решением исходной системы является столбец 
$$X = \begin{pmatrix}
	-C_1 - 5C_2e^{4t}\\
	C_1 + C_2e^{4t}
\end{pmatrix}.$$
\textbf{Ответ:} $(-C_1 - 5C_2e^{4t},\
C_1 + C_2e^{4t})^T$.\\\\
\textbf{Пример 7.} Используя операторный метод, проинтегрировать систему:
$$\begin{cases}
	D^2x_1 + 2x_2 = 2e^t,\\
	4D^2x_2 + 2x_1 = t.
\end{cases}$$
\textbf{Решение.} Составим расширенную матрицу системы (учитывая неоднородность) и приведем ее к треугольному виду:
\begin{center}
	$\begin{pmatrix}
		D^2 & 2D^0 & \vline & 2e^{2t}\\
		2D^0 & 4D^2 & \vline & t
	\end{pmatrix}\sim$ [меняем 1-ую и 2-ую строки местами] $\sim \begin{pmatrix}
	2D^0 & 4D^2 & \vline & t\\
	D^2 & 2D^0 & \vline & 2e^{2t}
\end{pmatrix} \sim$ [домножим первую строку на $-\dfrac{D^2}{2}$ и прибавим ко второй] $\sim \begin{pmatrix}
2D^0 & 4D^2 & \vline & t\\
0 & -2D^4 + 2D^0 & \vline & 2e^{2t}
\end{pmatrix}\sim\begin{pmatrix}
2D^0 & 4D^2 & \vline & t\\
0 & D^4 - D^0 & \vline & e^{2t}
\end{pmatrix}.$
\end{center}
Стоит подметить, что когда мы домножаем строку на оператор дифференцирования однородность также домножается. Однако, в нашем случае, $-\dfrac{D^2t}{2} = 0$.\\
Вернемся к системе:
$$\begin{cases}
	2x_1 + 4D^2x_2 = t,\\
	D^4x_2 - x_2 = e^t
\end{cases}$$
Отсюда найдем методом Эйлера $x_2$: $$x_2 = C_1e^{-t} + C_2e^t + C_3cos(t) + C_4sin(t) + \dfrac{e^{2t}}{15}.$$
Тогда из первого уравнения получим $x_1 = \dfrac{t}{2}  - 2D^2x_2.$
Найдем $D^2x_2$:
$$Dx_2 = -C_1e^{-t} + C_2e^t - C_3sin(t) + C_4cos(t) + \dfrac{2e^{2t}}{15}.$$
$$D^2x_2 =C_1e^{-t} + C_2e^t - C_3cos(t) - C_4sin(t) + \dfrac{4e^{2t}}{15}.$$
Следовательно, $$x_1 = \dfrac{t}{2} - 2C_1e^{-t} -2 C_2e^t +2C_3cos(t) +2 C_4sin(t) - \dfrac{8e^{2t}}{15}.$$
И решение системы имеет вид
$$X = \begin{pmatrix}
	\dfrac{t}{2} - 2C_1e^{-t} -2 C_2e^t +2C_3cos(t) +2 C_4sin(t) - \dfrac{8e^{2t}}{15}\\
	C_1e^{-t} + C_2e^t + C_3cos(t) + C_4sin(t) + \dfrac{e^{2t}}{15}
\end{pmatrix}.$$
\textbf{Ответ:} $(\dfrac{t}{2} - 2C_1e^{-t} -2 C_2e^t +2C_3cos(t) +2 C_4sin(t) - \dfrac{8e^{2t}}{15},\ 
C_1e^{-t} + C_2e^t + C_3cos(t) + C_4sin(t) + ~\dfrac{e^{2t}}{15})^T$
\end{document}