\documentclass[a4paper, 12pt]{report}
\usepackage{cmap}
\usepackage{amssymb}
\usepackage{amsmath}
\usepackage{graphicx}
\usepackage{amsthm}
\usepackage{upgreek}
\usepackage{setspace}
\usepackage[T2A]{fontenc}
\usepackage[utf8]{inputenc}
\usepackage[normalem]{ulem}
\usepackage{mathtext} % русские буквы в формулах
\usepackage[left=2cm,right=2cm, top=2cm,bottom=2cm,bindingoffset=0cm]{geometry}
\usepackage[english,russian]{babel}
\usepackage[unicode]{hyperref}
\newenvironment{Proof} % имя окружения
{\par\noindent{$\blacklozenge$}} % команды для \begin
{\hfill$\scriptstyle\boxtimes$}
\newcommand{\Rm}{\mathbb{R}}
\newcommand{\Cm}{\mathbb{C}}
\newcommand{\Z}{\mathbb{Z}}
\newcommand{\I}{\mathbb{I}}
\newcommand{\N}{\mathbb{N}}
\newcommand{\Ra}{\Rightarrow}
\newcommand{\ra}{\rightarrow}
\newcommand{\FI}{\Phi}
\newcommand{\Sp}{\text{Sp}}
\renewcommand{\leq}{\leqslant}
\renewcommand{\geq}{\geqslant}
\renewcommand{\alpha}{\upalpha}
\renewcommand{\beta}{\upbeta}
\renewcommand{\gamma}{\upgamma}
\renewcommand{\delta}{\updelta}
\renewcommand{\varphi}{\upvarphi}
\renewcommand{\tau}{\uptau}
\renewcommand{\lambda}{\uplambda}
\renewcommand{\psi}{\uppsi}
\renewcommand{\mu}{\upmu}
\renewcommand{\omega}{\upomega}
\renewcommand{\d}{\partial}
\renewcommand{\xi}{\upxi}
\renewcommand{\epsilon}{\upvarepsilon}
\newcommand{\intx}{\int\limits_{x_0}^x}
\newcommand\Norm[1]{\left\| #1 \right\|}
\newcommand{\Ln}{L_n = D^n + a_{n-1}D^{n-1} + \ldots + a_1D + a_0D^0}
\newcommand{\KFunc}{\int\limits_{t_0}^{t}\varphi_{n-1}(t-\uptau)f(\uptau)d\uptau}
\newtheorem*{theorem}{Теорема}
\newtheorem*{cor}{Следствие}
\newtheorem*{lem}{Лемма}
\begin{document}
	\subsection*{Уравнения высших порядков. Понижение порядка уравнения.}
	Все рассмотренные в теме "Элементарные дифференциальные уравнения" имели первый порядок. Теперь же мы переходим к рассмотрению уравнений $n$-ого порядка.\\\\
	Рассмотрим уравнение вида $$F(x, y, y',\ldots, y^{(n)}) = 0,$$
	где $F$ --- функция определенная и непрерывная по всем своим аргументам в некоторой области $D \subseteq \Rm^{n+2}$. \\\\
	На самом деле, всё достаточно незамысловато. Имея уравнение $n$-ого порядка, мы всего навсего понижаем его порядок до тех пор, пока не получим такое уравнение, методы интегрирования которого мы уже знаем. И понижаем порядок мы с помощью замен независимой переменной или неизвестной функции. Рассмотрим эти замены.
	\begin{enumerate}
		\item \textbf{Уравнения, не содержащие функции $y$ и несколько первых производных этой функции.}\\\\
		Рассмотрим уравнение вида $$F(x,y^{(k)}, \ldots, y^{(n)}) = 0.$$
		Порядок уравнения может быть понижен заменой $$y^{(k)} = z(x) \Rightarrow F(x, z, \ldots, z^{(n-k)}) = 0.$$
		Если в результате рещения полученного уравнения получаем общее решение $$z(x) = \varphi(x, C_1, \ldots, C_{n-k}),$$
		то общее решение исходного уравнения является решением ПДУ $$y^{(k)} = \varphi(x, C_1, \ldots, C_{n-k}).$$
		\item \textbf{Уравнение, не содержащее независимой переменной.}\\\\Рассмотрим уравнение вида $$F(y,y',\ldots, y^{(n)}) = 0.$$
		Порядок уравнения можно понизить с помощью замены $$y' = z(y).$$
		Тогда  $$\begin{aligned}
			y' =& z(y),\\
			y'' =& z'\cdot y' = z'\cdot z,\\
			y''' =& (z''\cdot z + z'^2)z = z''\cdot z^2 + z'^2 \cdot z,\\
			\vdots
		\end{aligned}$$
		В результате замены получим уравнение вида $$\FI(y,z,\ldots, z^{(n-1)}) = 0.$$
		Если функция $z(y) = \varphi(y, C_1,\ldots, C_{n-1})$ --- общее решение этого уравнения, то, сделав обратную замену, получим УРП $$y' =  \varphi(y, C_1,\ldots, C_{n-1}).$$
		\item \textbf{Уравнения однородные относительно неизвестной функции и ее производных.}\\\\
		$\bullet$ \textit{Функция $F(x,y,y',\ldots, y^{(n)})$ называется \textbf{однородной} относительно $y,y',\ldots, y^{(n)}$, если $$F(x,py,py',\ldots, py^{(n)}) = p^m\cdot F(x,y,y',\ldots, y^{(n)}).$$ Соответственно уравнение $F(x,y,y',\ldots, y^{(n)}) = 0$ называется \textbf{однородным}.}\\\\
		Порядок однородного уравнения может быть понижен заменой $$y' = y\cdot z(x).$$
		Тогда $$y'' = z'y + zy' = z'y + z^2 y = (z' + z^2)\cdot y;$$
		$$y''' = (z'' + 2zz')\cdot y + (z' + z^2)\cdot y \cdot z = (z'' + 3zz' + z^3)\cdot y.$$
		Продолжая рассуждения аналогичным образом, получим, что $$y^{(k)} = (z,z',\ldots, z^{(k-1)}).$$
		Выполнив эту замену, получим $$F(x,y,\varphi_1(z)\cdot y, \varphi_2(z,z')\cdot y,\ldots, \varphi_n(z,z',\ldots, z^{(n-1)})\cdot y) = 0.$$
		Отдельно рассмотрев ситуацию $y= 0 $, можем сократить на $y$. Тогда получим новое уравнение с неизвестной функцией $z$ порядка $(n-1)$:$$F(x,1, \varphi_1(z), \varphi_2(z,z'),\ldots, \varphi_n(z,z',\ldots, z^{(n-1)})) = 0.$$ 
		\item \textbf{Обобщенное однородное уравнение.}\\\\
		$\bullet$ \textit{Уравнение $F(x, y, y',\ldots, y^{(n)}) = 0$ называется \textbf{обобщенным однородным уравнением}, если для функции $F$ выполняется соотношение} $$F(px, p^ky , p^{k-1}y',\ldots, p^{k-n}y^{(n)}) = p^m\cdot F(x, y, y',\ldots, y^{(n)}).$$
		Выполним замену неизвестной функции и независимой переменной следующим образом $$x = e^t,\quad y = z\cdot e^{kt}.$$
		Тогда $$y' = (z' + kz)\cdot e^{(k-1)t}.$$
$$
			y''  = \Big(z'' + (kz - 1)\cdot z' + k\cdot (k-1)\cdot z'\Big)\cdot e^{(k-2)t}.
$$
		$$y''' = (z''' + \ldots)\cdot e^{(k-3)t}.$$
		Подставив эти замены в уравнение, получим $$F(e^t, ze^{kt}, (z'+kz)e^{(k-1)t}, (z'' + \ldots)e^{(k-2)t},\ldots, (z^{(n)}+\ldots)e^{(k-n)t}) = 0.$$
		$$p^m \cdot F(1,z, z'+kz, z'' + \ldots, \ldots, z^{(n)}+\ldots) = 0.$$
		Получим уравнение, в котором не содержится независимой переменной. Порядок этого уравнения может быть понижен способом 3. Зачастую принято брать $k=1$.
		\item \textbf{Уравнение в точных производных}.\\\\
		$\bullet$ \textit{Уравнение $F(x, y, y',\ldots, y^{(n)}) = 0$ называется \textbf{уравнением в точных производных}, если $$\exists \FI(x,y,y',\ldots, y^{(n-1)}) : \dfrac{d \FI}{d x} = F.$$}
		Тогда уравнение $F(x, y, y',\ldots, y^{(n)}) = 0$ равносильно уравнению $$\dfrac{d \FI}{\d x}= 0 \Rightarrow \FI(x,y,y', \ldots, y^{(n-1)}) = C$$
		--- уравнение $(n-1)$-ого порядка.
		Иногда исходное уравнение не является уравнением в точных производных, но его можно получить, умножив его на некий множитель.
	\end{enumerate}
Разберем на каждый случай по одному примеру.\\\\
\textbf{Пример 1.} Проинтегрировать уравнение $$y'' (e^x + 1) + y' = 0.$$
\textbf{Решение.} Так как в данном уравнении отсутствует переменная $y$, то данное уравнение подходит под 1-ый тип. Введем замену $$y' = z(x),\ y'' = z'(x).$$
Подставим эту замену и получим УРП $$z'(e^x + 1) + z = 0.$$
Приведем его к более привычному виду $$\dfrac{dz}{z} + \dfrac{dx}{e^x + 1} = 0.$$
Проинтегрируем и получим $$\int\limits_{z_0}^z \dfrac{dz}{z} + \int\limits_{x_0}^x\dfrac{dx}{e^x + 1} = C_1.$$
$$\ln z + \ln \dfrac{e^x}{e^x + 1} = C_1.$$
Преобразуем его, чтобы выделить $z$. Тогда $$z = C_1\dfrac{e^x + 1}{e^x } = C_1 + C_1e^{-x}.$$
Сделаем обратную замену, то есть $$y' = C_1 + C_1e^{-x}.$$
Данное уравнение является простейшим. Проинтегрируем обе части по $x$ и получим общее решение исходного уравнения:
$$y = \int\limits_{x_0}^x (C_1 + C_1e^{-x})dx + C_2 = C_1x - C_1e^{-x} + C_2.$$
\textbf{Ответ:} $y = C_1x - C_1e^{-x} + C_2.$\\\\
\textbf{Замечание}. \textit{Также под первый тип могут попасть уравнения, в которых нет ни $x$, ни $y$, ни первых $k$ производных от $y$. Например, порядок уравнения $$y''\cdot y''' + 1 =0$$ можно понизить заменой $y'' = z(x)$. Тогда получим $$z\cdot z' + 1 = 0.$$ Далее интегрируем и делаем обратную замену.}\\\\
\textbf{Пример 2.} Проинтегрировать уравнение $$y'^2 + 2yy'' = 0.$$
\textbf{Решение.} Данное уравнение подходит ко 2-му типу. Тогда введем замену $$y' = z(y),\ y''= z'y' = z'z.$$
Подставим эту замену и получим $$z^2 +2z'zy = 0.$$
Запомним случай $$z = 0$$ и сократим уравнение на $z$. Тогда получим УРП $$z + 2z' y = 0.$$
$$\dfrac{dz}{z} + \dfrac{dy}{2y} = 0.$$
Отсюда $$\ln z + \dfrac{1}{2}\ln y = C_1.$$
$$z = \dfrac{C_1}{\sqrt{y}}.$$
Сделаем обратную замену и получим снова УРП $$y' = \dfrac{C_1}{\sqrt{y}}.$$
$$\sqrt{y}dy = C_1dx.$$
Тогда получим общее решение уравнения $$\dfrac{2}{3}y^{\frac32} = C_1x + C_2.$$
\textbf{Ответ:} $y = \Big(\dfrac{3}{2}(C_1x + C_2)\Big)^{\frac23}.$\\\\
\textbf{Пример 3.} Проинтегрировать уравнение $$yy'' - y'^2 = \dfrac{yy'}{1+x}.$$
\textbf{Решение.} В данном уравнении нет отсутствующих функций. Следовательно, необходимо проверить, подходит ли оно под 3 или 4 тип. Для начала
\end{document}
