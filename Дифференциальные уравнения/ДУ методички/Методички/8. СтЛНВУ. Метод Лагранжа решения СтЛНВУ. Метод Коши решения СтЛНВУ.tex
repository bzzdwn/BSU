\documentclass[a4paper, 12pt]{article}
\usepackage{cmap}
\usepackage{amssymb}
\usepackage{amsmath}
\usepackage{graphicx}
\usepackage{amsthm}
\usepackage{upgreek}
\usepackage{setspace}
\usepackage[T2A]{fontenc}
\usepackage[utf8]{inputenc}
\usepackage[normalem]{ulem}
\usepackage{mathtext} % русские буквы в формулах
\usepackage[left=2cm,right=2cm, top=2cm,bottom=2cm,bindingoffset=0cm]{geometry}
\usepackage[english,russian]{babel}
\newenvironment{Proof} % имя окружения
{\par\noindent{$\blacklozenge$}} % команды для \begin
{\hfill$\scriptstyle\boxtimes$}
\newcommand{\Rm}{\mathbb{R}}
\newcommand{\Cm}{\mathbb{C}}
\newcommand{\I}{\mathbb{I}}
\newcommand{\N}{\mathbb{N}}
\newtheorem*{theorem}{Теорема}
\newtheorem*{cor}{Следствие}
\newtheorem*{lem}{Лемма}
\newcommand{\FI}{\text{Ф}}
\renewcommand{\tau}{\uptau}
\newcommand{\Ln}{L_n = D^n + a_{n-1}D^{n-1} + \ldots + a_1D + a_0D^0}
\begin{document}
	\section*{СтЛНВУ. Метод Лагранжа решения СтЛНВУ. Метод Коши решения СтЛНВУ.}
	Рассмотрим уравнение $$DX = AX + f(t),\quad t \in \I\eqno(1)$$
	с непрерывной на $\I$ векторной функцией $f(t)$ и соответствующее ему однородное уравнение $$DX = AX.\eqno(2)$$
	Все решения неоднородного уравнения можно получить, прибавив к некоторому частному решению все решения соответствующего однородного уравнения, то есть $$X_\text{OH} = X_\text{OO} + X_\text{ЧН}.$$
	\textbf{Замечание.} \textit{Нахождение решения СтЛНВУ методом Лагранжа предусматривает умение находить решение СтЛВУ методом Эйлера. Нахождение решения СтЛНВУ методом Коши предусматривает умение находить решение СтЛВУ матричным методом. Если Вы пропустили, то лучше вернуться и повторить прошлые уроки, иначе могут возникнуть трудности с пониманием материала в данном уроке.}
	\subsection*{Метод Лагранжа решения СтЛНВУ.}
	Частное решение неоднородного уравнения может быть найдено \textbf{методом вариации произвольной постоянной (методом Лагранжа)}.\\\\
	Пусть $\FI(t) = [X_1(t),\ldots,X_n(t)]$ --- ФМ. Частное решение уравнения (1) будем искать в виде $$X(t) = \FI(t)\cdot U(t), \quad U(t) = \begin{pmatrix}
		u_1(t) \\ \vdots \\ u_n(t)
	\end{pmatrix}.$$
	Подставим функцию $X(t)$ в уравнение (1) и получим
	$$DXU + XDU = AX + f(t)\Rightarrow\FI(t)\cdot DU(t) = f(t).$$
	$\det \FI(t) \ne 0$ $\forall t \Rightarrow \exists \FI^{-1}(t) : DU(t) = \FI^{-1}(t) f(t)$. Тогда в качестве $U(t)$ можно выбрать $$U(t) = \int\limits_{t_0}^t\FI^{-1}(\uptau) f(\uptau)d\uptau,\quad t, t_0 \in \I.$$
	Следовательно, общее решение неоднородного уравнения имеет вид $$
	\boxed{X_{\text{он}} = \FI(t)\cdot C + \FI(t) \int\limits_{t_0}^t\FI^{-1}(\uptau) f(\uptau)d\uptau.}\eqno (3)$$
	\textbf{Пример 1.} Методом Лагранжа найти общее решение уравнения $DX = AX + f(t)$, $t \in \I = \Rm$, где $$A = \begin{pmatrix}
		0 & 1\\
		-2 & -3
	\end{pmatrix}, \quad f(t) =\begin{pmatrix}
	t\\
	-t
\end{pmatrix}.$$
\textbf{Решение.} Воспользуемся формулой (3). Для нахождения общего решения СтЛНВУ нам необходимо найти ФМ $\FI(t)$. Вычислим ее методом Эйлера.\\
Матрица $A$ имеет собственные значения $\lambda_1 = -2$, $\lambda_2 = -1$. Найдем собственные векторы, соответствующие этим собственным значениям и сразу же найдем решения уравнения:
$$A + 2E = \begin{pmatrix}
	2 & 1\\
	-2 & -1
\end{pmatrix}\Rightarrow x_1 = e^{-2t}(1,-2).$$
$$A + E = \begin{pmatrix}
	1 & 1\\
	-2 & -2
\end{pmatrix}\Rightarrow x_2 = e^{-t}(1,-1).$$
Тогда составим фундаментальную матрицу
$$\FI(t) = \begin{pmatrix}
	e^{-2t} & e^{-t}\\
	-2e^{-2t} & -e^{-t}
\end{pmatrix}.$$
Также матрицу $\FI(t)$ можно составить с помощью матричного метода:
$$\FI(t) = Se^{Jt} = \begin{pmatrix}
	1 & 1\\
	-2 & -1
\end{pmatrix}\begin{pmatrix}
e^{-2t} & 0\\
0 & e^{-t}
\end{pmatrix} = \begin{pmatrix}
e^{-2t} & e^{-t}\\
-2e^{-2t} & -e^{-t}
\end{pmatrix}.$$
Теперь мы можем найти общее решение соответствующего однородного СтЛВУ:
$$X_{\text{оо}}(t) = \FI(t)\cdot C = 
	\begin{pmatrix}
			e^{-2t} & e^{-t}\\
		-2e^{-2t} & -e^{-t}
	\end{pmatrix}\begin{pmatrix}
	C_1\\C_2
\end{pmatrix} = \begin{pmatrix}
C_1e^{-2t} + C_2e^{-t}\\
-2C_1e^{-2t} - C_2e^{-t}
\end{pmatrix}.$$
Теперь найдем частное решение СтЛНВУ по формуле $$X_{\text{чн}}(t) =\FI(t) \int\limits_{t_0}^t\FI^{-1}(\uptau) f(\uptau)d\uptau. $$
Для этого вычислим $\FI^{-1}(\uptau)f(\uptau)$:
$$\FI^{-1}(\uptau)f(\uptau) = \begin{pmatrix}
	-e^{2\uptau} & -e^{2\uptau}\\
	2e^{\uptau} & e^{\uptau}
\end{pmatrix}\begin{pmatrix}
\tau\\-\tau
\end{pmatrix} = \begin{pmatrix}
0 \\ \tau e^{\tau}
\end{pmatrix}.$$
Подставим в формулу частного решения. Тогда нам необходимо найти интеграл от этой матрицы по $\tau$. Вычислим его, проинтегрировав каждую строку полученного столбца по отдельности при $t_0 = 0$:
$$\int\limits_{0}^t \begin{pmatrix}
	0 \\ \tau e^{\tau}
\end{pmatrix}d\tau = \begin{pmatrix}
0\\
te^t - e^t + 1
\end{pmatrix}.$$
Теперь умножим фундаментальную матрицу на полученный столбец. Это и будет частное решение нашего СтЛНВУ:
$$X_{\text{чн}}(t) = \begin{pmatrix}
	e^{-2t} & e^{-t}\\
-2e^{-2t} & -e^{-t}
\end{pmatrix}\begin{pmatrix}
0\\
te^t - e^t + 1
\end{pmatrix} = \begin{pmatrix}
t - 1 + e^{-t}\\
-t + 1 - e^{-t}
\end{pmatrix}.$$
Осталось сложить общее и частное решения и получить общее решение исходного СтЛНВУ:
$$X(t) = \begin{pmatrix}
C_1e^{-2t} + C_2e^{-t}\\
-2C_1e^{-2t} - C_2e^{-t}
\end{pmatrix} + \begin{pmatrix}
t - 1 + e^{-t}\\
-t + 1 - e^{-t}
\end{pmatrix} = \begin{pmatrix}
C_1e^{-2t} + C_2e^{-t} + t - 1 + e^{-t}\\
-2C_1e^{-2t} - C_2e^{-t} -t + 1 - e^{-t}
\end{pmatrix}.$$
\textbf{Ответ:} $X(t) = \begin{pmatrix}
C_1e^{-2t} + C_2e^{-t} + t - 1 + e^{-t}\\
-2C_1e^{-2t} - C_2e^{-t} -t + 1 - e^{-t}
\end{pmatrix}.$\\\\
\textbf{Пример 2.} Методом Лагранжа найти общее решение уравнения $DX = AX + f(t)$, $t \in \I = \Rm$, где $$A = \begin{pmatrix}
	3 & 1 & -1\\
	0 & 2 & 0\\
	1 & 1 & 1
\end{pmatrix}, \quad f(t) =\begin{pmatrix}
	e^{2t}\\
	6te^{2t}\\
	e^{2t}
\end{pmatrix}.$$
\textbf{Решение}. Матрица $A$ имеет собственное значение $\lambda = 2$, $k = 3$. 
$$A - 2E = \begin{pmatrix}
	1 & 1 & -1\\
	0 & 0 & 0\\
	1 & 1 & -1
\end{pmatrix}\sim \begin{pmatrix}
1 & 1 & -1
\end{pmatrix}.$$
Следовательно, геометрическая кратность $r = 2$. Собственные векторы соответствующие собственному значению равны $(1,0,1)^T$, $(1,-1,0)^T$. Тогда линейно независимые решения равны, соответствующего однородного уравнения равны
$$x_1 = e^{2t}(1,0,1),\quad x_2 = e^{2t}(1,-1,0).$$
Дополним совокупность линейно независимым решением $x_3 = e^{2t}(\alpha_0 + \alpha_1t)$, где $\alpha_1 = (1,0,1)^T$, $\alpha_0 = (1,0,0)^T$. Тогда $$x_3 = e^{2t}(t+1, 0, t)^T.$$
Построим фундаментальную матрицу
$$\FI(t) = e^{2t}\begin{pmatrix}
	1 & 1 & t+1\\
	0 & -1 & 0\\
	1 & 0 & t
\end{pmatrix},\quad \FI^{-1}(t) = e^{-2t}\begin{pmatrix}
-t & -t & t+1\\
0 & -1 & 0\\
1 & 1 & -1
\end{pmatrix}.$$
Тогда общее решение соответствующего однородного уравнения имеет вид
$$X_{\text{oo}}(t) = e^{2t}\begin{pmatrix}
	C_1 + C_2 + C_3t + C_3\\
	-C_2\\
	C_1 + C_3t
\end{pmatrix}.$$
Перейдем к поиску частного решения. Вычислим $\FI^{-1}(\uptau)f(\uptau)$:
$$\FI^{-1}(\uptau)f(\uptau) = e^{-2\tau}\begin{pmatrix}
	-\tau & -\tau & \tau+1\\
	0 & -1 & 0\\
	1 & 1 & -1
\end{pmatrix}\begin{pmatrix}
	e^{2\tau}\\
	6\tau e^{2\tau}\\
	e^{2\tau}
\end{pmatrix} = \begin{pmatrix}
	-6\tau^2 + 1\\
	-6\tau\\
	6\tau
\end{pmatrix}.$$
Проинтегрируем полученный столбец, взяв $t_0 = 0$:
$$\int\limits_{0}^t\begin{pmatrix}
	-2\tau^2 + 1\\
	-2\tau\\
	2\tau
\end{pmatrix}d\tau = \begin{pmatrix}
-2t^3 + t\\
-3t^2\\
3t^2
\end{pmatrix}.$$
Домножим слева на фундаментальную матрицу и получим
$$X_{\text{чн}}(t) = e^{2t}\begin{pmatrix}
	1 & 1 & t+1\\
	0 & -1 & 0\\
	1 & 0 & t
\end{pmatrix}\begin{pmatrix}
-2t^3 + t\\
-3t^2\\
3t^2
\end{pmatrix} = e^{2t}\begin{pmatrix}
t^3 + t\\
3t^2\\
t^3 + t
\end{pmatrix}.$$
Сложим общее и частное решения и получим общее решение СтЛНВУ
$$X(t) =e^{2t}\begin{pmatrix}
	C_1 + C_2 + C_3t + C_3\\
	-C_2\\
	C_1 + C_3t
\end{pmatrix} + e^{2t}\begin{pmatrix}
t^3 + t\\
3t^2\\
t^3 + t
\end{pmatrix} = e^{2t}\begin{pmatrix}
C_1 + C_2 + C_3t + C_3 + t^3 + t\\
-C_2 + 3t^2\\
C_1 + C_3t + t^3 + t
\end{pmatrix}.$$
\textbf{Ответ:}$X(t) = e^{2t}\begin{pmatrix}
	C_1 + C_2 + C_3t + C_3 + t^3 + t\\
	-C_2 + 3t^2\\
	C_1 + C_3t + t^3 + t
\end{pmatrix}.$
\subsection*{Метод Коши решения СтЛНВУ.}
Используя формулу (3), найдем решение задачи Коши $DX = AX + f(t)$, $X|_{t=t_0} = \xi$. Для этого подставим в (3) начальные условия:
	$X|_{t=t_0} = \FI(t_0)\cdot C = \xi \Rightarrow C = \FI^{-1}(t_0)\xi.$
	Следовательно, решение задачи Коши имеет вид $$X(t) = \FI(t)\cdot \FI^{-1}(t_0)\xi + \FI(t)\int\limits_{t_0}^t\FI^{-1}(\uptau) f(\uptau)d\uptau.\eqno(4)$$
	Возьмем $\FI(t) = e^{At}$. Тогда решение задачи Коши примет вид $$X(t) = e^{A(t-t_0)}\xi + \int\limits_{t_0}^te^{A(t-\uptau)} f(\uptau)d\uptau.\eqno (5)$$
	Задача Коши для уравнения (1) с нулевыми начальными условиями ($\xi = 0$) имеет вид $$X(t) = \int\limits_{t_0}^te^{A(t-\uptau)} f(\uptau)d\uptau.$$
	Эта является частным решением уравнения (1).\\\\
	$\bullet$ \textit{Полученная функция называется \textbf{методом Коши} отыскания частного решения неоднородного уравнения.}\\\\
	Следовательно, общее решение уравнения (1) можно записать в виде $$\boxed{X_{\text{он}}(t) = e^{A(t-t_0)}C + \int\limits_{t_0}^te^{A(t-\uptau)} f(\uptau)d\uptau.}\eqno(6)$$
\textbf{Пример 3.} Методом Коши найти общее решение уравнения $DX = AX + f(t)$, $t \in \I$, где
$$A = \begin{pmatrix}
	2 & 0 & 0\\
	1 & 2 & 0\\
	0 & 0 & -1
\end{pmatrix},\quad f(t)=\begin{pmatrix}
0\\ e^{2t} \\ 1
\end{pmatrix}.$$
\textbf{Решение.} Для нахождения решения воспользуемся формулой (6). И для нахождения общего решения, и для частного нам необходимо найти матричную экспоненту $e^{A(t-t_0)}$. Матрица $A$ имеет собственные значения $\lambda_1 = -1$, $k_1 = 1$; $\lambda_2 = 2$, $k_2=2$.\\
Рассмотрим $\lambda_1 = -1$:
$$A + E = \begin{pmatrix}
	3 & 0 & 0\\
	1 & 3 & 0\\
	0 & 0 & 0
\end{pmatrix}\sim \begin{pmatrix}
1 & 0 &\vline & 0\\
0 & 1 & \vline & 0
\end{pmatrix}.$$
Тогда данному собственному значению соответствует собственный вектор $$a_1(0,0,1).$$
Рассмотрим $\lambda_2 = 2$:
$$A - 2E = \begin{pmatrix}
	0 & 0 & 0\\
	1 & 0 & 0\\
	0 & 0 & -3
\end{pmatrix}\sim \begin{pmatrix}
1 & \vline & 0 & \vline & 0\\
0 & \vline & 0 & \vline & 1
\end{pmatrix}.$$
Таким образом, геометрическая кратность $r = n - rank(A-2E) = 1$. Найдем собственный вектор, соответствующий данному собственному значению $$a_2(0,1,0).$$
Найдем присоединенный к нему вектор $$(A-2E\ |\ a_2) = \begin{pmatrix}
	0 & 0 & 0 & \vline & 0\\
	1 & 0 & 0 & \vline & 1\\
	0 & 0 & -3 & \vline & 0
\end{pmatrix}\Rightarrow a_3(1,0,0).$$
Теперь можем построить трансформирующую матрицу и матричную экспоненту ЖНФ:
$$S = S^{-1} = \begin{pmatrix}
	0 & 0 & 1\\
	0 & 1 & 0\\
	1 & 0 & 0
\end{pmatrix},\quad e^{Jt} = \begin{pmatrix}
e^{-t} & 0 & 0\\
0 & e^{2t} & te^{2t}\\
0 & 0 & e^{2t}
\end{pmatrix}.$$
$$e^{A(t-t_0)} =Se^{J(t-t_0)}S^{-1}= \begin{pmatrix}
	e^{2(t-t_0)} & 0 & 0\\
	(t-t_0) e^{2(t-t_0)} & e^{2(t-t_0)} & 0\\
	0 & 0 & e^{-(t-t_0)}
\end{pmatrix}.$$
Теперь можем найти общее решение соответствующего СтЛОВУ, приняв $t_0 = 0$: $$X_{\text{oo}}(t) = \begin{pmatrix}
	e^{2t} & 0 & 0\\
	t e^{2t} & e^{2t} & 0\\
	0 & 0 & e^{-t}
\end{pmatrix}\begin{pmatrix}
C_1\\C_2\\C_3
\end{pmatrix} = \begin{pmatrix}
C_1e^{2t}\\
C_1te^{2t} + C_2e^{2t}\\
C_3e^{-t}
\end{pmatrix}.$$
Но общее решение можно искать и другим способом. Например, необязательно умножать $Se^{Jt}$ на $S^{-1}$:
$$X(t) = Se^{Jt}C = \begin{pmatrix}
	0 & 0 & 1\\
	0 & 1 & 0\\
	1 & 0 & 0
\end{pmatrix}\begin{pmatrix}
e^{-t} & 0 & 0\\
0 & e^{2t} & te^{2t}\\
0 & 0 & e^{2t}
\end{pmatrix}\begin{pmatrix}
C_1\\C_2\\C_3
\end{pmatrix}= \begin{pmatrix}
C_3e^{2t}\\
C_3te^{2t} + C_2e^{2t}\\
C_1e^{-t}
\end{pmatrix}.$$
Такое решение отличается от полученного ранее, однако также является общим. На мой взгляд, разница между двумя этими вариантами в данном случае невелика, так как мы все равно должны вычислять матрицу $S^{-1}$ для нахождения частного решения.\\\\
Осталось найти частное решение $\int\limits_{t_0}^te^{A(t-\uptau)}f(\uptau)d\uptau.$
Так как $t \in \I = \Rm$, можем взять $t_0 = 0$. Найдем $e^{A(t-\uptau)}f(\uptau)$:
$$e^{A(t-\uptau)}f(\uptau) =  \begin{pmatrix}
	e^{2(t-\uptau)} & 0 & 0\\
	(t-\uptau) e^{2(t-\uptau)} & e^{2(t-\uptau)} & 0\\
	0 & 0 & e^{-(t-\uptau)}
\end{pmatrix}\begin{pmatrix}
0\\e^{2\uptau}\\1
\end{pmatrix} = \begin{pmatrix}
0\\e^{2t}\\e^{-(t-\uptau)}
\end{pmatrix}.$$
Полученный столбец необходимо проинтегрировать. Для этого проинтегрируем каждую строку по отдельности:
$$\int\limits_{0}^t\begin{pmatrix}
	0\\e^{2t}\\e^{-(t-\uptau)}
\end{pmatrix}d\uptau = \begin{pmatrix}
0\\te^{2t}\\1-e^{-t}
\end{pmatrix}.$$
Полученный столбец является столбцом частных решений. Остается лишь сложить полученные общие решения и частные решения (причем в качестве общего решения возьмем столбец $Se^{Jt}S^{-1}C$) и получить матрицу, которая является общим решением СтЛВНУ:
$$X_{\text{oн}}(t) = \begin{pmatrix}
	C_1e^{2t}\\
	C_1te^{2t} + C_2e^{2t} + te^{2t}\\
	C_3e^{-t} + 1 - e^{-t}
\end{pmatrix}.$$
\textbf{Ответ:} $X_{\text{oн}}(t) = \begin{pmatrix}
	C_1e^{2t}\\
	C_1te^{2t} + C_2e^{2t} + te^{2t}\\
	C_3e^{-t} + 1 - e^{-t}
\end{pmatrix}.$\\\\
\textbf{Пример 4.} Методом Коши найти общее решение уравнения $DX = AX + f(t)$, $t \in \I$, где
$$A = \begin{pmatrix}
	1 & 0 & 1\\
	0 & 1 & 1\\
	1 & -1 & 1
\end{pmatrix},\quad f(t)=\begin{pmatrix}
	1\\ 1\\ e^{t}
\end{pmatrix}.$$
\textbf{Решение.} Матрица $A$ имеет собственное значение $\lambda = 1$, $k = 3$. 
$$A - E = \begin{pmatrix}
	0 & 0 & 1\\
	0 & 0 & 1\\
	1 & -1 & 0
\end{pmatrix}\sim \begin{pmatrix}
1 & \vline & -1 & \vline& 0\\
0 & \vline & 0 & \vline& 1
\end{pmatrix}.$$
Следовательно, геометрическая кратность $r = 1$, и собственному значению соответствует собственный вектор $$a_1(1,1,0).$$
Найдем присоединенные векторы:
$$(A-E\ |\ a_1) = \begin{pmatrix}
		0 & 0 & 1 & \vline & 1\\
	0 & 0 & 1 & \vline & 1\\
	1 & -1 & 0 & \vline & 0
\end{pmatrix}\sim \begin{pmatrix}
0 & 0 & 1  & \vline & 1\\
1 & -1 & 0  & \vline & 0
\end{pmatrix}\Rightarrow a_2(0,0,1).$$
$$(A-E\ |\ a_2) = \begin{pmatrix}
	0 & 0 & 1 & \vline & 0\\
	0 & 0 & 1 & \vline & 0\\
	1 & -1 & 0 & \vline & 1
\end{pmatrix}\sim \begin{pmatrix}
	0 & 0 & 1  & \vline & 0\\
	1 & -1 & 0  & \vline & 1
\end{pmatrix}\Rightarrow a_3(1,0,0).$$
Построим трансформирующую матрицу и матричную экспоненту ЖНФ:
$$S = \begin{pmatrix}
	1 & 0 & 1\\
	1 & 0 & 0\\
	0 & 1 & 0
\end{pmatrix},\quad e^{Jt} = e^t\begin{pmatrix}
1 & t & \frac{t^2}{2}\\
0 & 1 & t\\
0 & 0 & 1
\end{pmatrix},\quad S^{-1} = \begin{pmatrix}
0 & 1 & 0\\
0 & 0 & 1\\
1 & -1 & 0
\end{pmatrix}.$$
Тогда $$e^{A(t-t_0)} = e^t\begin{pmatrix}
	\frac{(t-t_0)^2}{2} + 1 & \frac{(t-t_0)^2}{2} & (t-t_0)\\
	\frac{(t-t_0)^2}{2} & 1 - \frac{(t-t_0)^2}{2} & (t-t_0)\\
	(t-t_0) & -(t-t_0) & 1
\end{pmatrix}.$$
Примем $t_0 = 0$ и найдем общее решение СтЛВУ
$$X_{\text{oo}}(t) = e^t\begin{pmatrix}
	\frac{t^2}{2} + 1 & \frac{t^2}{2} &t\\
	\frac{t^2}{2} & 1 - \frac{t^2}{2} & t\\
	t & -t & 1
\end{pmatrix}\begin{pmatrix}
C_1\\C_2\\C_3
\end{pmatrix} = e^t\begin{pmatrix}
\frac{1}{2}C_1t^2 + C_1 + \frac{1}{2}C_2t^2 + C_3t\\
\frac{1}{2}C_1t^2 + C_2 - \frac{1}{2}C_2t^2 + C_3t\\
C_1t - C_2t + C_3
\end{pmatrix}.$$
Переходим к нахождению частного решения. Аналогично берем $t_0 =0$ и ищем $e^{A(t-\uptau)}f(\uptau)$:
$$e^{A(t-\uptau)}f(\uptau) =e^{t-\uptau}
	\begin{pmatrix}
		\frac{(t-\uptau)^2}{2} + 1 & \frac{(t-\uptau)^2}{2} & (t-\uptau)\\
		\frac{(t-\uptau)^2}{2} & 1 - \frac{(t-\uptau)^2}{2} & (t-\uptau)\\
		(t-\uptau) & -(t-\uptau) & 1
	\end{pmatrix}\begin{pmatrix}
	1\\-1\\e^{\uptau}
\end{pmatrix} = \begin{pmatrix}
e^{t-\uptau} + (t-\uptau)e^t\\
e^{t-\uptau} + (t-\uptau)e^t\\
2e^{t-\uptau}(t-\uptau) + e^{t}\\
\end{pmatrix}.$$
Проинтегрируем полученный столбец, взяв $t_0 = 0$:
$$X_{\text{чн}}(t) = \int\limits_{0}^{t} \begin{pmatrix}
	e^{t-\uptau} + (t-\uptau)e^t\\
	e^{t-\uptau} + (t-\uptau)e^t\\
	2e^{t-\uptau}(t-\uptau) + e^{t}\\
\end{pmatrix} d\uptau = \begin{pmatrix}
1/2t^2e^t + e^t - 1\\
1/2t^2e^t + e^t - 1\\
3te^t - 2e^t + 2
\end{pmatrix}.$$
Сложим общее и частное решения и получим общее решение СтЛНВУ
$$X_{\text{oн}}(t) = \begin{pmatrix}
	1/2C_1t^2 + C_1 + 1/2C_2t^2 + C_3t + 1/2t^2e^t + e^t - 1\\
	1/2C_1t^2 + C_2 - 1/2C_2t^2 + C_3t + 1/2t^2e^t + e^t - 1\\
	C_1t - C_2t + C_3 + 3te^t - 2e^t + 2
\end{pmatrix}.$$
\textbf{Ответ:} $X_{\text{oн}}(t) = \begin{pmatrix}
	1/2C_1t^2 + C_1 + 1/2C_2t^2 + C_3t + 1/2t^2e^t + e^t - 1\\
	1/2C_1t^2 + C_2 - 1/2C_2t^2 + C_3t + 1/2t^2e^t + e^t - 1\\
	C_1t - C_2t + C_3 + 3te^t - 2e^t + 2
\end{pmatrix}.$\\\\
\textbf{Пример 5.} Методом Коши решить задачу Коши $DX = AX + f(t)$, $t \in \I$, $X_{t=t_0} = \xi$, где
$$A = \begin{pmatrix}
5 & 4 & 6\\
4 & 5 & 6\\
-4 & -4 & -5
\end{pmatrix},\quad f(t)=\begin{pmatrix}
1\\-1\\0
\end{pmatrix},\quad t_0 = 1,\quad \xi = \begin{pmatrix}
0\\\frac{1}{2}\\0
\end{pmatrix}.$$
\textbf{Решение.} Вспользуемся формулой (5). Метод решения аналогичен предудыщим примерам, но после вычисления $e^{A(t-t_0)}$ мы не находим общее решение, а сразу подставляем начальные условия.\\\\
Матрица $A$ имеет собственные значения $\lambda_1 = 1$, $k_1 = 2$; $\lambda_2 = 3$, $k_2 = 1$. Рассмотрим $\lambda_1$:
$$A - E = \begin{pmatrix}
	4 & 4 & 6\\
	4 & 4 & 6\\
	-4 & -4 & -6
\end{pmatrix}\sim \begin{pmatrix}
2 & 2 & 3
\end{pmatrix}.$$
Следовательно, геометрическая кратность $r_1 = 2$, а собственные вектора соответствующие собственному значению равны $$a_1(1, -1, 0),\quad a_2(3,0,-2).$$
Рассмотрим $\lambda_2$:
$$A - 3E = \begin{pmatrix}
	2 & 4 & 6\\
	4 & 2 & 6\\
	-4 & -4 & -2
\end{pmatrix}\sim \begin{pmatrix}
1 & 0 & 1\\
0 & 1 & 1
\end{pmatrix}\Rightarrow a_3(1,1,-1).$$
Вычислим матричную экспоненту:
$$e^{A(t-t_0)} = \begin{pmatrix}
	1 & 3 & 1\\
	-1 & 0 & 1\\
	0 & -2 & -1
\end{pmatrix}\begin{pmatrix}
e^{t-t_0} & 0 & 0\\
0 & e^{t-t_0} & 0\\
0 & 0 & e^{3(t-t_0)}
\end{pmatrix}\begin{pmatrix}
2 & 1 & 3\\
-1 & -1 & -2\\
2 & 2 & 3
\end{pmatrix}=$$ $$=\begin{pmatrix}
2e^{3(t-t_0)} - e^{(t-t_0)} & 2e^{3(t-t_0)} - 2e^{(t-t_0)} & 3e^{3(t-t_0)} - 3 e^{(t-t_0)}\\
2e^{3(t-t_0)} - 2e^{(t-t_0)} & 2e^{3(t-t_0)} - e^{(t-t_0)} & 3e^{3(t-t_0)} - 3e^{(t-t_0)}\\
2e^{(t-t_0)} - 2e^{3(t-t_0)} & 2e^{(t-t_0)} - 2e^{3(t-t_0)}& 4e^{(t-t_0)} - 3e^{3(t-t_0)} 
\end{pmatrix}.$$
Вычислим $e^{A(t-t_0)}\xi$:
$$e^{A(t-t_0)}\xi = \begin{pmatrix}
	2e^{3(t-1)} - e^{(t-1)} & 2e^{3(t-1)} - 2e^{(t-1)} & 3e^{3(t-1)} - 3 e^{(t-1)}\\
	2e^{3(t-1)} - 2e^{(t-1)} & 2e^{3(t-1)} - e^{(t-1)} & 3e^{3(t-1)} - 3e^{(t-1)}\\
	2e^{(t-1)} - 2e^{3(t-1)} & 2e^{(t-1)} - 2e^{3(t-1)}& 4e^{(t-1)} - 3e^{3(t-1)} 
\end{pmatrix}\begin{pmatrix}
0\\\frac{1}{2}\\0
\end{pmatrix}=\begin{pmatrix}
e^{3(t-1)} - e^{t-1}\\
e^{3(t-1)} - 1/2e^{t-1}\\
 e^{t-1} - e^{3(t-1)}
\end{pmatrix}.$$
Теперь найдем частное решение. Вычисляется оно абсолютно аналогично прошлым примерам. Только в интеграле принимаем $t_0 = 1$ в соответствии с нашими начальными условиями.
$$e^{A(t-\uptau)}f(\uptau) = \begin{pmatrix}
	2e^{3(t-\uptau)} - e^{(t-\uptau)} & 2e^{3(t-\uptau)} - 2e^{(t-\uptau)} & 3e^{3(t-\uptau)} - 3 e^{(t-\uptau)}\\
	2e^{3(t-\uptau)} - 2e^{(t-\uptau)} & 2e^{3(t-\uptau)} - e^{(t-\uptau)} & 3e^{3(t-\uptau)} - 3e^{(t-\uptau)}\\
	2e^{(t-\uptau)} - 2e^{3(t-\uptau)} & 2e^{(t-\uptau)} - 2e^{3(t-\uptau)}& 4e^{(t-\uptau)} - 3e^{3(t-\uptau)} 
\end{pmatrix}\begin{pmatrix}1\\-1\\0
\end{pmatrix} = \begin{pmatrix}
e^{t-\uptau}\\
-e^{t-\uptau}\\
0
\end{pmatrix}.$$
Тогда 
$$X_{\text{чн}}(t) = \int\limits_{1}^{t} \begin{pmatrix}
	e^{t-\uptau}\\
	-e^{t-\uptau}\\
	0
\end{pmatrix} d\uptau = \begin{pmatrix}
e^{t-1} - 1\\
1 - e^{t-1}\\
0
\end{pmatrix}.$$
Полученные решения как всегда складываем и получим решение исходной задачи Коши:
$$X(t) = \begin{pmatrix}
	e^{3(t-1)} - e^{t-1}\\
	e^{3(t-1)} - 1/2e^{t-1}\\
	e^{t-1} - e^{3(t-1)}
\end{pmatrix} + \begin{pmatrix}
e^{t-1} - 1\\
1 - e^{t-1}\\
0
\end{pmatrix} = \begin{pmatrix}
	e^{3(t-1)} - 1\\
	e^{3(t-1)} - 3/2e^{t-1} + 1\\
	e^{t-1} - e^{3(t-1)}
\end{pmatrix}.$$
\textbf{Ответ:} $X(t) = \begin{pmatrix}
	e^{3(t-1)} - 1\\
	e^{3(t-1)} - 3/2e^{t-1} + 1\\
	e^{t-1} - e^{3(t-1)}
\end{pmatrix}.$\\\\
\textbf{Замечание.} \textit{Искать решение задачи Коши мы также можем методом Лагранжа по формуле $(4)$, но, в отличие от метода Лагранжа, вычисляя в методе Коши $e^{A(t-t_0)}$, мы сразу находим ФСР нормированную в точке $t=t_0$. Поэтому после нахождения $e^{A(t-t_0)}$ мы сразу же можем подставить начальные условия.} 
\subsection*{Задачи для самостоятельного решения.}
Методом Лагранжа найти общее решение уравнения $DX = AX + f(t)$, $t \in \I = \Rm$, где \begin{enumerate}
	\item $$A = \begin{pmatrix}
		0 & 1\\
		-1 & 0
	\end{pmatrix},\quad f(t) = \begin{pmatrix}
	\tg^2 - 1\\
	\tg t
\end{pmatrix},\quad \I = (-\pi/2; \pi/2);$$
\item $$A = \begin{pmatrix}
	-2 & 2\\
	-1 & 1
\end{pmatrix},\quad f(t) = \begin{pmatrix}
e^{-2t}\\6e^{6t}
\end{pmatrix},\quad \I = \Rm;$$
\item $$A = \begin{pmatrix}
	0 & 1\\
	-2 & 3
\end{pmatrix},\quad f(t) = \begin{pmatrix}
	0\\1/(e^t + 1)
\end{pmatrix},\quad \I = \Rm;$$
\item $$A = \begin{pmatrix}
	3 & 1 & -1\\
	0 & 2 & 0\\
	1 & 1 & 1
\end{pmatrix}, \quad f(t) =\begin{pmatrix}
	e^{2t}\cos t\\
		e^{2t}\sin t\\
	e^{2t}
\end{pmatrix},\quad \I = \Rm;$$
\item $$A = \begin{pmatrix}
	3 & -1 & 1\\
	-1 & 5 & -1\\
	1 & -1 & 3
\end{pmatrix}, \quad f(t) =\begin{pmatrix}
	e^{2t}\\
	e^{5t}\\
	1
\end{pmatrix},\quad \I = \Rm.$$
\end{enumerate}
Методом Коши найти общее решение уравнения $DX = AX + f(t)$, $t \in \I$, где
$$A = \begin{pmatrix}
	0 & 1 & 0\\
	-1 & 0 & 0\\
	1 & 0 & 0
\end{pmatrix},$$
\begin{enumerate}
	\item $ f(t)=(0, e^t, 0)^T;$
	\item $f(t) = (\cos t, \sin t, 1)^T$.
\end{enumerate}
Методом Коши решить задачу Коши $DX = AX + f(t)$, $t \in \I$, $X_{t=t_0} = \xi$, где
$$A = \begin{pmatrix}
	3 & -1 & -1\\
	12 &-3 & -12\\
	-4 & 1 & 6
\end{pmatrix},$$
\begin{enumerate}
	\item $f(t) = (t, 0 ,1)^T, t_0 = 2, \xi = (1,0,0)^T;$
	\item $f(t) = (e^{-t}, e^{2t} ,0)^T, t_0 = 0, \xi = (1,0,0)^T;$
\end{enumerate}
\end{document}