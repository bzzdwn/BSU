\documentclass[a4paper, 12pt]{report}
\usepackage{cmap}
\usepackage{amssymb}
\usepackage{amsmath}
\usepackage{graphicx}
\usepackage{amsthm}
\usepackage{upgreek}
\usepackage{setspace}
\usepackage[T2A]{fontenc}
\usepackage[utf8]{inputenc}
\usepackage[normalem]{ulem}
\usepackage{mathtext} % русские буквы в формулах
\usepackage[left=2cm,right=2cm, top=2cm,bottom=2cm,bindingoffset=0cm]{geometry}
\usepackage[english,russian]{babel}
\usepackage[unicode]{hyperref}
\newenvironment{Proof} % имя окружения
{\par\noindent{$\blacklozenge$}} % команды для \begin
{\hfill$\scriptstyle\boxtimes$}
\newcommand{\Rm}{\mathbb{R}}
\newcommand{\Cm}{\mathbb{C}}
\newcommand{\Z}{\mathbb{Z}}
\newcommand{\I}{\mathbb{I}}
\newcommand{\N}{\mathbb{N}}
\newcommand{\rank}{\operatorname{rank}}
\newcommand{\Ra}{\Rightarrow}
\newcommand{\ra}{\rightarrow}
\newcommand{\FI}{\Phi}
\newcommand{\Sp}{\text{Sp}}
\renewcommand{\leq}{\leqslant}
\renewcommand{\geq}{\geqslant}
\renewcommand{\alpha}{\upalpha}
\renewcommand{\beta}{\upbeta}
\renewcommand{\gamma}{\upgamma}
\renewcommand{\delta}{\updelta}
\renewcommand{\varphi}{\upvarphi}
\renewcommand{\phi}{\upvarphi}
\renewcommand{\tau}{\uptau}
\renewcommand{\lambda}{\uplambda}
\renewcommand{\psi}{\uppsi}
\renewcommand{\mu}{\upmu}
\renewcommand{\omega}{\upomega}
\renewcommand{\d}{\partial}
\renewcommand{\xi}{\upxi}
\renewcommand{\epsilon}{\upvarepsilon}
\newcommand{\intx}{\int\limits_{x_0}^x}
\newcommand\Norm[1]{\left\| #1 \right\|}
\newcommand{\sumk}{\sum\limits_{k=0}^\infty}
\newcommand{\sumi}{\sum\limits_{i=0}^\infty}
\newtheorem*{theorem}{Теорема}
\newtheorem*{cor}{Следствие}
\newtheorem*{lem}{Лемма}
\title{\textbf{\Huge{Численные методы}}\\Конспект по 3 курсу 
	специальности «прикладная математика»\\(лектор А. М. Будник)}
\date{}
\begin{document}
	\maketitle
	\tableofcontents{}
	\newpage
	\chapter{Методы решения нелинейных уравнений.}
	В данной главе будут рассмотрены некоторые методы решения нелинейных уравнений и систем уравнений. Рассмотрим случай одного нелинейного уравнения.
	\section{Постановка задачи.}
	Пусть задана функция $f(x)$ действительного переменного $x \in \Rm$. Требуется найти корни уравнения $$f(x) = 0,\eqno(1)$$
	или, что то же самое, нули функции $f(x)$. 
	Выясним, является ли задача корректно поставленной. Для ответа на вопрос существования и единственности решения введем теорему из математического анализа.
	\begin{theorem}
		Если функция $f(x)$ непрерывна на отрезке $[a,b]$ и принимает на его концах значения разных знаков, то на этом отрезке существует по крайней мере один корень уравнения $f(x) = 0$.
		Если при этом функция $f(x)$ будет монотонной на отрезке $[a,b]$, то она может иметь только один корень.
	\end{theorem}
	\begin{Proof}
		Без доказательства.
	\end{Proof}\\\\
	Выясним условие устойчивости для рассматриваемой задачи. Как правило, в качестве входных данных мы имеем функцию $f(x)$, заданную в виде функциональной формы. Поэтому понятие устойчивости здесь отпадает.
	\\\\
	Нелинейное уравнение в зависимости от вида функции $f(x)$ можно разделить на два класса:
	\begin{enumerate}
		\item алгебраические;
		\item трансцендентные.
	\end{enumerate}
	В первом классе функция $f(x)$ содержит только алгебраические функции. Например, полином $P_n(x)$ является целой алгебраической функцией. Ко второму классу относятся все другие функции, которые содержат выражения тригонометрические, показательные, логарифмические и так далее.\\\\
	Методы решения нелинейных уравнений делятся на прямые и итерационные. Мы будем рассматривать лишь итерационные методы. \\\\
	Задача нахождения корней уравнения (1) обычно решается в два этапа:
	\begin{enumerate}
		\item отделение корней;\\\\
		На этом этапе изучается расположение корней в общем случае на комплексной плоскости, проводится их разделение, т.е. выделяются области, содержащие только один корень. Кроме того изучается вопрос о кратности корней. Находятся некоторые начальные приближения $x^0$ для точного решения.
		\item построение метода.
		\\\\
		На этом этапе, используя заданное начальное приближение, строится итерационный процесс, позволяющий уточнить значение отыскиваемого корня до некоторой заданной точности $\epsilon$. Т.е., зная $x^0$, строим последовательность $x^k \xrightarrow[k\to\infty]{\epsilon}x^*$.
	\end{enumerate}
	В заключение этого параграфа запишем несколько соображений, касающихся первого этапа. По отделению корней мы можем выделить несколько способов нахождения начального приближения:
	\begin{itemize}
		\item из физических соображений;
		\item графический способ;
		\item построение таблицы значений функции $f(x)$ на заданной сетке узлов;
		\item метод деления отрезка пополам (метод дихотомии, метод бисекции).
	\end{itemize}
	Метод деления отрезка пополам заключается в том, что мы берем отрезок $[a,b]$ и смотрим, чтобы на этом отрезке функция меняла знак. Затем делим отрезок пополам, берем точку $c : a<c<b$ и в зависимости от того, где меняется знак, переходим к следующему отрезку и так далее. В итоге мы придем к тому, что отрезок получится меньше $\epsilon$, то есть мы и получим искомый корень. Число делений отрезка пополам $N \geq \log_2 \dfrac{b-a}{\epsilon}$.
	\section{Метод простой итерации решения нелинейного уравнения.}
	Применение метода требует предварительного приведения уравнения $f(x) = 0$ к каноническому виду 
	$$x = \varphi(x), \eqno(1)$$ где $\varphi(x)$ --- это заданная функция. Метод простой итерации будет иметь следующий вид: $$x^{k+1} = \varphi(x^k),\ k = 0,1,2,\ldots,\eqno(2)$$
	где $x^k$ --- последовательность, начинающаяся с $x^0$, которая должна сходится к точному решению. Область изменения аргумента $x$ на числовой оси обозначим через $X$, а через $Y$ обозначим область значений функции $y = \varphi(x)$. Тогда функцию $\varphi(x)$ можно рассматривать как оператор, преобразующий $X$ в $Y$: $$\varphi : X \to Y.$$
	Таким образом, нам нужно найти такие точки области $X$, которые при преобразовании оператором $\varphi$ переходят сами в себя, то есть точки остающиеся неподвижными при преобразовании $X$ в $Y$. Значит решения уравнения $(1)$ --- это точки, остающиеся неподвижными при преобразовании $X$ в $Y$. Геометрически это можно изобразить следующим образом: $$\includegraphics[scale=0.4]{img1.png}$$
	Таким образом, после процедуры отделения корней мы находим начальное приближение $x^0$ в окрестности корня $x^*$. И по найденному начальному приближению по формуле (2) строится итерационная последовательность, которая и называется \textbf{методом простой итерации.}\\\\
	Мы должны обеспечить сходимость этого итерационного процесса. Сформулируем и докажем для этого теорему.
	\begin{theorem}
		[о сходимости метода простой итерации]
		Пусть выполняются следующие условия:\begin{enumerate}
			\item функция $\varphi(x)$ определена на отрезке $$|x - x^0| \leq \delta,\eqno(3)$$ непрерывна на нем и удовлетворяет условию Липшица с постоянным коэффициентом меньше единицы, то есть $\forall x, \widetilde{x}$ $$|\varphi(x) - \varphi(\widetilde{x})| \leq q |x - \widetilde{x}| ,\quad 0 \leq q < 1;\eqno(4)$$
			\item для начального приближения $x^0$ верно неравенство $$|x^0 - \varphi(x^0)| \leq m;$$
			\item числа $\delta, q, m$ удовлетворяют условию $$\dfrac{m}{1-q}\leq \delta. \eqno(5)$$
		\end{enumerate}
		Тогда \begin{enumerate}
			\item уравнение $(1)$ в области $(3)$ имеет решение;
			\item последовательность $x^k$ построенная по правилу $(2)$ принадлежит отрезку $[x^0 - \delta, x^0 + \delta]$, является сходящейся и ее предел удовлетворяет уравнению $(1)$: $$x^k \xrightarrow[k\to \infty]{} x^*;$$
			\item скорость сходимости $x^k$ к $x^*$ оценивается неравенством $$|x^* - x^k| \leq \dfrac{m}{1- q}q^k,\ k = 1,2,\ldots \eqno(6)$$
		\end{enumerate}
		Также эта теорема может называется \textbf{методом сжимающих отображений}.
	\end{theorem}
	\begin{Proof}
		Докажем второй пункт, т.е. принадлежность.
		Методом математической индукции покажем, что при всех значениях $k=1,2,\ldots$ приближения $x^k \in [x^0 - \delta, x^0 + \delta]$ и для них верно неравенство $$|x^{k+1} - x^k| \leq mq^k.\eqno(7)$$
		При $k=0$ имеем $x^1 = \varphi(x^0)$, а $x^1$ всегда может быть найден, поскольку $\varphi$ определена в $x^0$. Кроме того $$|x^1 - x^0| = |\varphi(x^0) - x^0| \leq m,$$ т.е. формула (7) справедлива. Докажем, что $x^1$ находится не дальше, чем $m$ от $x^0$: $$m \leq \dfrac{m}{1-q}\leq \delta,$$ отсюда следует, что $x^1 \in [x^0 - \delta, x^0 + \delta]$.
		\\\\
		Пусть данное предположение справедливо при $x^0,x^1,\ldots, x^k \in [x^0 - \delta, x^0 + \delta]$ и $$|x^{n+1} - x^n| \leq mq^n,\ n = 0,1,\ldots, k-1.$$
		По предположению $x^k \in [x^0 - \delta, x^0 + \delta]$, следовательно, $x^{k+1} = \varphi(x^k)$ может быть вычислено. По сделанному допущению справедливо $$|x^k - x^{k-1}| \leq m q ^{k-1}.$$
		Теперь рассмотрим неравенство для $k+1$-ой итерации: $$|x^{k+1} - x^k| = |\varphi(x^k) - \varphi(x^{k-1})|\leq q|x^k - x^{k-1}| \leq mq^k.$$
		Осталось проверить $x^k \in  [x^0 - \delta, x^0 + \delta]$. Рассмотрим разность $$|x^{k+1} - x^0| = \Big|(x^{k+1} - x^k) + (x^k - x^{k-1}) + \ldots + (x^1 - x^0)\Big|\leq mq^k + mq^{k-1} +\ldots + m.$$
		Легко видеть, что эта сумма легко подсчитывается, как сумма геометрической прогрессии, и равна $$\dfrac{m - mq^{k+1}}{1-q} < \dfrac{m}{1-q} \leq \delta.$$
		Итак, мы доказали, что $x^{k+1}$ принадлежит отрезку (3). \\\\
		Докажем сходимость последовательности. Для этого покажем, что для последовательности выполняется условие Больцано-Коши $$|x^{k+p} - x^k| = \Big|(x^{k+p} - x^{k+p-1}) + (x^{k+p-1} - x^{k+p-2}) + \ldots + (x^{k+1} - x^k)\Big|\leq \dfrac{m}{1-q}q^k.$$
		Так как оценка не зависит от $p$ и учитывая то, что $0\leq q < 1$, можно утверждать, что признак сходимости для последовательности $x^k$ выполняется, а значит существует предел этой последовательности $$\exists \lim\limits_{k\to\infty}x^k = x^*.$$
		Нужно доказать, что $x^* \in [x^0 - \delta; x^0 + \delta]$ и $x^*$ удовлетворяет формуле (1). Это следует из того, что все $x^k$ принадлежат этому отрезку, то есть и предел находится в этом отрезке. Для доказательства второго в формуле (2) устремим $k\to\infty$: $$x^* = \varphi(x^*).$$
		Ввиду непрерывности функции, $x^*$ является решением искомого уравнения, т.е. уравнение (1) превращается в тождество.\\\\
		Последнее, что нужно доказать, --- оценка из пункта 3. Для получения неравенства (6) достаточно в соотношении $$|x^{k+p} - x^k| \leq \dfrac{m}{1-q}q^k.$$ устремить $p \to \infty$. То есть $$|x^* - x^k| \leq \dfrac{m}{1-q}q^k,$$ что и является искомой оценкой.
	\end{Proof}\\\\
	\textbf{Замечания.}\begin{enumerate}
		\item На всяком множестве точек, где для функции $\varphi(x)$ выполняется условие $$|\phi(x) - \phi(y)| <|x-y|,\ x\ne y$$ уравнение (1) может иметь не более одного решения.
		\item Пользуясь оценкой (6), можно получить априорное количество итераций, необходимое для получения приближенного решения с заданной точностью $$k \geq \dfrac{\lg \frac{\epsilon(1-q)}{m}}{\lg q}.$$
		\item Для построения сходящегося метода простой итерации в практических вычислениях условие 1 теоремы о сходимости метода простой итерации обычно заменяется более строгим требованием, а именно для всех $x$ из отрезка $|x - x^0| \leq \delta$ функция $\varphi(x)$ имеет непрерывную первую производную $\varphi'(x)$ такую, что $$|\varphi'(x)|<1 \quad \forall x \in [x_0 - \delta; x_0 + \delta].$$
		Более того, если $0\leq\varphi'(x)<1$, то поведение последовательных приближений будет монотонным. Если $-1<\varphi'(x)\leq 0$, то поведение итерационной последовательности будет колебательным.\\\\
		Геометрический смысл метода простой итерации продемонстрируем на графике:
		$$
			\includegraphics[scale=0.5]{img2}
		$$
		В свою очередь, при $\varphi'(x) > 1$ процесс расходится, это можно увидеть из графика 
		$$
			\includegraphics[scale=0.5]{img4}
		$$
		\item Так как сходимость метода простых итераций возможна при сжимающем отображении, то условие $|\varphi'(x)|<1$ является определяющим при приведении исходного уравнения к каноническому виду. \\\\
		Наиболее универсальными способом приведения к каноническому виду является преобразование $$x = \underbrace{x+f(x)}_{\varphi(x)},$$
		но нам необходимо выполнение условия $|\varphi'(x)|<1$. Поэтому мы вводим параметр $\psi(x)$, выбираемый таким образом, чтобы обеспечить сходимость: $$x = \underbrace{x+\psi(x)f(x)}_{\varphi(x)},$$ Параметр $\psi(x)$ должен быть непрерывным и  $\psi(x^*)\ne 0.$ Самый простейший вариант --- взять постоянную функцию $\psi(x)=\operatorname{const}$ и подобрать эту константу из условия $|\varphi'(x)|<1$.
		\item Поведение последовательности приближений мы будем исследовать, изучая величину $$\epsilon_k = x^* - x^k$$ --- \textbf{погрешность приближенного решения на $k$-ой итерации}. Из этого соотношения легко видеть, что $$x^k = x^* - \epsilon_k$$ и подставим это в формулу (2). Тогда $$x^* - \epsilon_{k+1} = \varphi(x^* - \epsilon_k).$$
		Предполагая, что функция $\varphi(x)$ имеет непрерывную производную в окрестности точек $x_k$ и $x_{k+1}$, разложим правую часть в ряд Тейлора в окрестности $x^*$:
		$$x^* - \epsilon_{k+1} = \varphi(x^*) - \varphi'(x^*)\epsilon_k + O(\epsilon_k^2).$$ Такое разложение возможно при условии, что функция $\varphi(x)$ дифференцируема и при предположении достаточной малости $\epsilon_k$, чтобы мы могли отбросить остальные члены. Учитывая $x^* = \varphi(x^*)$ и отбрасывая достаточно малые слагаемые $O(\epsilon_k^2)$, получим $$\epsilon_{k+1} \approx \varphi'(x^*)\epsilon_k.\eqno(8)$$
		Формула (8) дает ответ о скорости сходимости метода простой итерации. То есть погрешность на каждой итерации уменьшается по сравнению с предыдущей в величину $\varphi'(x^*)$. Таким образом, \begin{enumerate}
			\item нам нужно обеспечить $|\varphi'|<1$, чтобы $\epsilon_{k+1} < \epsilon_k$;
			\item сходимость метода осуществляется по закону геометрической прогрессии со знаменателем $q = \varphi'$.
		\end{enumerate}
	\end{enumerate}
	\section{Метод Ньютона решения нелинейного уравнения.}
	Рассмотрим уравнение $$f(x) = 0,\eqno(1)$$ где $f(x)$ достаточно гладкая функция вещественного переменного. Предположим, что для точного решения $x^*$ каким-либо образом задано начальное приближение $x^0$. Для построения метода рассмотрим погрешность $\epsilon_0 = x^* - x^0$. В предположении, что $\epsilon_0$ достаточно малая по модулю величина, подставим в уравнение (1) $x^*$ вместо $x$, тогда $$f(x^0 + \epsilon_0)=0.$$
	Разложим это выражение в ряд Тейлора в окрестности точки $x^0$:$$f(x^0 + \epsilon_0) = f(x^0) + \epsilon_0 f'(x^0) + O(\epsilon_0^2) = 0.$$
	Теперь отбросим слагаемое $O(\epsilon_0^2)$ и получим в рамках отброшенной величины получим приближенное уравнение $$f(x^0) + \epsilon_0 f'(x^0)\approx 0.$$
	Решая это уравнение относительно $\epsilon_0$, получим $$\epsilon_0 \approx -\dfrac{f(x^0)}{f'(x_0)}.$$
	Тогда выразим из $x^* = x^0 + \epsilon_0$ и учитывая, что равенство приближенное, получим $$x^* \approx x^0 - \dfrac{f(x^0)}{f'(x_0)}.$$
	В итоге, повторяя описанную процедуру, мы можем построить итерационную формулу, которая носит название \textbf{метода Ньютона} $$x^{k+1} = x^k - \dfrac{f(x^k)}{f'(x^k)},\quad k = 0,1,\ldots;\quad x_0\eqno(2)$$
	(добавка $x_0$ означает, что начальное приближение задано).
	Иногда этот метод называют \textbf{методом касательных}. Это название следует из геометрического смысла.
	Если рассмотреть уравнение кривой $y = f(x)$, то в точке $x^k$ касательная к ней задается уравнением $$y - f(x^k) = f'(x^k) (x-x^k).$$ Находим точку пересечения касательной с осью $Ox$, полагая $y=0$, и тогда $$x= x^k - \dfrac{f(x^k)}{f'(x^k)}.$$ Таким образом строим приближение $x^{k+1}$ и так далее:
	$$
		\includegraphics[scale=0.5]{img3}
	$$
	То есть мы приближаемся к корню по последовательности касательных прямых.\\\\
	Выясним, какова скорость сходимости этого метода. С помощью подстановки получим формулу для скорости сходимости $$\epsilon_{k+1} = \dfrac{\epsilon_k f'(x^* - \epsilon_k) + f(x^* - \epsilon_k)}{f'(x^* - \epsilon_k)}.$$ 
	Для того, чтобы получить ответ на вопрос, какова скорость сходимости, необходимо сделать несколько преобразований данного выражения. Воспользуемся тем, что мы можем разложить функции в ряд Тейлора в окрестности $x^*$: $$f(x^* - \epsilon_k) = f(x^*) - \epsilon_kf'(x^*) + \dfrac{1}{2}\epsilon_k^2 f''(x^*) + O(\epsilon_k^3),$$
	$$f'(x^* - \epsilon_k) = f'(x^*) - \epsilon_kf''(x^*) + \dfrac{1}{2}\epsilon_k^2 f'''(x^*) + O(\epsilon_k^3).$$
	В итоге после подстановки мы получим формулу $$\epsilon_{k+1} =- \dfrac{1}{2} \dfrac{f''(x^*)}{f'(x^*)}\epsilon_k^2 + O(\epsilon_k^3).$$
	Отбросив величину более высокого порядка, чем $\epsilon_k^2$, мы получим приближенное равенство $$\epsilon_{k+1} \approx - \dfrac{1}{2} \dfrac{f''(x^*)}{f'(x^*)}\epsilon_k^2=\alpha\epsilon_k^2. \eqno(3)$$
	Формула (3) доказывает, что при $|\alpha|<1$ последовательность $x^k$ построенная по формуле (2) обладает квадратичной сходимостью.
	\begin{theorem}
		[о сходимости метода Ньютона]
		Пусть выполняются следующие условия:
		\begin{enumerate}
			\item Функция $f(x)$ определена и дважды непрерывно дифференцируема на отрезке $$s_0 = [x^0; x^0 + 2h_0],\quad h_0 =- \dfrac{f(x^0)}{f'(x^0)}.$$
			При этом на концах отрезка $f(x)f'(x)\ne 0$.
			\item Для начального приближения $x^0$ выполняется неравенство $$2|h_0|M \leq |f'(x_0)|,\quad M = \underset{x\in s_0}{\max}|f''(x)|.$$
		\end{enumerate}
		Тогда справедливы следующие утверждения: \begin{enumerate}
			\item Внутри отрезка $s_0$ уравнение $f(x) = 0$ имеет корень $x^*$ и при этом этот корень единственный.
			\item Последовательность приближений $x^k$, $k=1,2,\ldots$ может быть построена по формуле $(2)$ с заданным приближением $x^0$.
			\item Последовательность $x^k$ сходится к корню $x^*$, то есть $x^k \xrightarrow[k\to\infty]{}x^*$.
			\item Скорость сходимости характеризуется неравенством $$|x^* - x^{k+1}|\leq |x^{k+1} - x^k|\leq \dfrac{M}{2|f'(x^*)|}\cdot (x^k-x^{k-1})^2,\quad k=0,1,2,\ldots\eqno(4)$$
		\end{enumerate}
	\end{theorem}
	\begin{Proof}
		Сначала докажем утверждение 2, т.е., что последовательность приближений $x^k$ может быть построена. Будем доказывать по индукции. По условию 1 теоремы первый член последовательности (2) можно построить $$x^1 = x^0 - \dfrac{f(x_0)}{f'(x_0)},\quad f'(x_0) \ne 0.$$
		Чтобы доказать возможность построения $x^2$, докажем, что $x^1 \in s_0$ и $f'(x^1)\ne 0$. Учитывая тот факт, что $$x_1  = x_0 + h_0,$$
		получим тот факт, что $x^1$ является серединой отрезка $s_0$. Далее рассмотрим следующее выражение выражение, пользуясь вторым условием, теоремы $$|f'(x^1) - f'(x^0)| = \Big|\int\limits_{x^0}^{x^1} f''(x_0)dx\Big|\leq M|x^1 - x^0| = M | h_0|\leq \dfrac{|f'(x^0)|}{2}.$$
		Теперь рассмотрим $$|f'(x^1)| = \big|f'(x^0) - (f'(x^0) - f'(x^1))\big|\geq |f'(x_0)| - |f'(x^0) - f'(x^1)|\geq |f'(x^0)| - \dfrac{|f'(x^0)|}{2} = \dfrac{|f'(x_0)|}{2} \ne 0.$$
		Таким образом, $f'(x^1)\ne 0$, а значит $x^2$ может быть построено. Тогда $$x^2 = x^1 + h_1,\quad h_1 = -\dfrac{f(x^1)}{f'(x^1)}.$$
		И так далее все $x^k$ могут быть вычислены.\\\\
		 Рассмотрим, как себя ведут отрезки для того, чтобы доказать сходимость итерационного процесса. Наряду с отрезком $s_0$ рассмотрим отрезок $$s_1 = [x^1; x^1 + 2h_1].$$
		Середина этого отрезка --- это $x^2$. Покажем, что $s_1 \subset s_0$. Для этого нам нужно показать, что $h_1 < h_0$. Оценим величину $h_1$. Для этого используем разложение в ряд Тейлора: $$|f(x^1)| = \Big|f(x^0) + h_0f'(x^0) + \dfrac{h_0^2}{2}f''(x^0 + \theta h_0)\Big| = \Big|\dfrac{h_0^2}{2}f''(x^0 + \theta h_0)\Big|\leq \dfrac{h_0^2}{2}M.$$
		$$|h_1| = \Big|-\dfrac{f(x^1)}{f'(x^1)}\Big|\leq \dfrac{h_0^2}{2}\dfrac{M}{|f'(x^1)|}\leq \dfrac{h_0^2}{2}\dfrac{2M}{|f'(x^0)|} = h_0\dfrac{M}{|f'(x^0)|}\leq \dfrac{|h_0|}{2}.$$
		Итак $2|h_1| \leq |h_0|$, следовательно, $$x^1 + 2h_1 = x^0 + h_0 + 2h_1 \leq x^0 + 2h_0 \in s_0.$$
		Отсюда следует, что $s_1 \subset s_0$.\\\\
		Далее мы можем показать по индуктивному предположению, что на отрезке $s_1$ итерация $x^1$ будет удовлетворять условиям 1 и 2 теоремы. Обе части неравенства $|h_1|\leq \dfrac{|h_0|}{2}$ домножим на $\dfrac{2M}{|f'(x^1)|}$, тогда $$\dfrac{2M}{|f'(x^1)|}|h_1|\leq \dfrac{2|h_0| M}{2|f'(x^1)|}$$
		Воспользуемся ранее произведенными оценками:
		$$2|h_0| M \leq |f'(x^0)|,\quad 2|f'(x^1)|\geq \dfrac{1}{2}|f'(x^0)|$$
		Тогда $$\dfrac{2|h_0| M}{2|f'(x^1)|} \leq 1\Rightarrow 2|h_1| M \leq |f'(x^1)|.$$
		Таким образом, на отрезке $s_1$ функция $f(x)$ удовлетворяет условиям теоремы 1 и 2. Теперь по индукции очевидна возможность построения последовательности $x^{k+1}$ по формуле (2). При этом $x^{k+1}$ является серединой отрезка $$s_k = [x^k; x^k + 2h_k], \quad h_k = -\dfrac{f(x^k)}{f'(x^k)}.$$
		А отрезок $s_k\subset s_{k-1}$ и не превосходит половины длины $s_{k-1}$. Кроме того, выполняется неравенство, являющегося оценкой половины длины отрезка $$|h_k| \leq \dfrac{h_{k-1}^2M}{2|f'(x^k)|}.$$
		То есть мы доказали утверждение 2.\\\\
		Докажем утверждения 3 и 1. Так как мы построили последовательность вложенных отрезков $$s_k \subset s_{k-1}\subset \ldots \subset s_1 \subset s_0,$$ длины которых с ростом $k$ стремятся к нулю, то, таким образом, эти отрезки стягиваются в точку. А следовательно последовательность $x^{k+1}$, элементы которой являются серединами этих отрезков, также является сходящейся к некоторому значению $x^*$. Отсюда $$x^{k+1}\xrightarrow[k\to\infty]{} x^*,$$
		но существование предела еще не означает, что это нужный нам предел. Покажем, что $x^*$ -- это корень уравнения (1). Для этого в формуле (2) перейдем к пределу при $k\to\infty$: $$x^* = x^* - \dfrac{f(x^*)}{f'(x^*)},$$ но дробь нужно рассмотреть отдельно. Для того, чтобы перейти к пределу в $f(x^k)$, мы должны доказать, что $$\lim\limits_{k\to\infty}f(x^k) = f(\lim\limits_{k\to\infty}x^k),$$
		этот переход возможен в силу непрерывности функции $f$ и в силу того, что $f'(x^k)\ne 0$ $\forall k$. Тогда записанная нами формула будет верна. А из этой формулы можно сделать вывод, что $$f(x^*) =0.$$
		Теперь докажем единственность этого корня $x^*$. Для этого предположим, что $M > 0$ (случай $M = 0$ мы рассматриваем, иначе функция будет линейной, а в таком случае на первой же итерации мы получим точное решение). По условию теоремы $$f'(x^0) \ne 0,\quad f'(x^0 + 2h_0) \ne 0.$$
		Учитывая этот факт, мы можем утверждать, что $$f'(x) \ne 0,\quad \forall x \in s_0,$$
		действительно докажем это. Для этого рассмотрим любую точку отрезка $x \in s_0$: $$|f'(x) - f'(x^0)|  = \Big|\int\limits_{x^0}^x f''(t)dt\Big|\leq M|x-x^0| < M \cdot 2|h_0|\leq |f'(x^0)|.$$
		Теперь мы можем оценить величину $\forall x \in s_0$ $$|f'(x)| = |f'(x^0) - (f'(x^0) - f'(x))| \geq |f'(x^0)| - |f'(x^0) - f'(x)|>  |f'(x^0)| - |f'(x_0)| = 0.$$
		То есть $f'(x)\ne 0$ в любой точке отрезка $s_0$. Этот факт говорит о том, что $f(x)$ строго монотонна на $s_0$. Следовательно, уравнение (1) имеет не более одного корня.\\\\
		Докажем утверждение 4. По доказанным ранее утверждениям $x^{k+1}$ --- это середина отрезка $s_k$ длиной $2|h_k|$ и $x^* \in s_k$. Тогда можно рассмотреть $$|x^* - x^{k+1}| \leq |h_k|\leq \dfrac{h_{k-1}^k M}{2 |f'(x^1)|},\ k=0,1,\ldots$$
		Отсюда и следует формула (4).
	\end{Proof}\\\\
	\textbf{Замечания.}\begin{enumerate}
		\item Из оценки (4) можно получить априорную оценку количества итераций, необходимых для достижения заданной точности $\epsilon$ (доказать самостоятельно)
		$$ k \geq \log_2 \dfrac{\ln(\alpha \epsilon)}{\ln(\alpha |x^1 - x^0|)},\quad \alpha = \underset{x\in s_0}{\max}\Big|\dfrac{f''(x)}{2f'(x)}\Big|.$$
		\item Если в окрестности корня производная $f'(x)$ сохраняет знак и монотонна, то приближение $x_k$ построенное по формуле (2) сходится с одной стороны.
	\end{enumerate}
	\section{Видоизменения метода Ньютона и метода простой итерации.}
	\subsection{Модификации метода Ньютона.}
	Все видоизменения связаны с тем, что мы хотим упростить формулу метода Ньютона и уменьшить количество арифметических операций, а для этого будем пытаться заменить вычисление производной вычислением другой более простой функции.
	\subsubsection{Метод Ньютона с постоянной производной.}
	Формула этого метода имеет следующий вид $$x^{k+1} = x^k - \dfrac{f(x^k)}{f'(x^0)},\ k=0,1,\ldots,\quad x^0.$$
	Это видоизменение напрямую связано с уменьшением количества арифметических операции, поскольку мы отказываемся от вычисления последовательности $f'(x^k)$. Таким образом, с точки зрения количества операций метод простой итерации и метод Ньютона становятся сравнимы между собой.\\\\ Геометрически это означает, что, выбрав $x^0$, мы движемся по касательной. Найдя $x^1$, мы будем двигаться из точки $x^1$ по той же касательной, т.е. все касательные будут параллельны касательной в точке, которая является начальным приближением к корню.
	$$
		\includegraphics[scale=0.5]{img5}
	$$
	Но скорость сходимости данного метода ухудшится. Легко видеть, что погрешность на каждой итерации будет меняться по следующему закону $$\epsilon_{k+1} = \epsilon_k - \dfrac{f(x^* - \epsilon_k)}{f'(x^0)}.$$
	Проделав необходимые вычисления, связанные с разложением функции в окрестности $x^*$, можно получить $$\epsilon_{k+1}\approx\Big(1 - \dfrac{f'(x^*)}{f'(x^0)}\Big)\epsilon_k.\eqno(2)$$
	Исходя из вида формулы (2), мы можем утверждать, что такая модификация имеет линейную скорость сходимости.
	\subsubsection{Метод секущих.}
	Возьмем за основу формулу производной $$f'(x^k)\approx \dfrac{f(x^k) -f(x^{k-1})}{x^{k} - x^{k-1}},\ k = 1,2,\ldots.$$
	И, подставляя в формулу Ньютона, мы получим следующую формулу $$x^{k+1} = x^k - f(x^k)\dfrac{x^k - x^{k-1}}{f(x^k) - f(x^{k-1})},\ k = 1,2,\ldots;\ x^0\eqno(3)$$
	Однако мы должны знать не только $x^0$, но и $x^1$, поэтому метод секущих двухшаговый.\\\\
	Геометрически мы выбираем два приближения $x^0$ и $x^1$ и через две эти точки мы проводим прямую, и она является не касательной, а секущей. Таким образом, при пересечении секущей с осью $Ox$ мы получаем точку $x^2$. Проводим через $x^1$ и $x^2$ следующую секущую, получаем точку $x^3$ и так далее.
	$$
		\includegraphics[scale=0.5]{img6}
	$$
Количество операций в этом случае сравнимо с количеством операций метода Ньютона с постоянной производной. Но при этом мы выигрываем в скорости, покажем это. Мы имеем следующее уравнение для погрешности:
$$\epsilon_{k+1} = \epsilon_k - \dfrac{(\epsilon_k - \epsilon_{k+1})f(x^* - \epsilon_k)}{f(x^* - \epsilon_k) - f(x^* - \epsilon_{k-1})}.$$
После выделения главной части из формулы и приведения подобных слагаемых, мы получим соотношение между погрешностями $$\epsilon_{k+1}\approx -\dfrac{1}{2} \dfrac{f''(x^*)}{f'(x^*)}\epsilon_k\epsilon_{k-1}\eqno(4)$$
Таким образом, она выше чем линейная, но ниже, чем квадратичная. Для уточнения необходимо преобразовать данную величину. Соотношение на $k+1$ и $k$ итерациях может быть оценено как $$\epsilon_{k+1}\approx C\epsilon_k^\alpha,\quad \alpha = \dfrac{1+\sqrt5}{2}.$$
\subsubsection{Метод хорд.}
Формула метода хорд имеет вид $$x^{k+1} = x^k - f(x^k)\dfrac{x^k - x^0}{f(x^k) - f(x^0)},\ k=1,2,\ldots;\ x^0, x^1\eqno(5)$$
Для подсчетов нам нужно два приближения, но сам метод одношаговый.\\\\
Геометрически мы строим хорды, проходящие через точку $f(x^0)$ и $f(x^k)$ на каждой итерации. Точка пересечения этой хорды с осью $Ox$ приводит нас к новому приближению $x^{k+1}$:
$$
	\includegraphics[scale=0.5]{img7}
$$
 В количестве операций мы не выигрываем. Можно показать, что погрешность в данном случае будет иметь вид $$\epsilon_{k+1} \approx -\dfrac{1}{2}\dfrac{f''(x^*)}{f'(x^*)}\epsilon_0\epsilon_k\eqno(6)$$
Отсюда можно сделать вывод, что метод хорд сходится по закону геометрической прогрессии, а значит по линейному закону, но знаменатель прогрессии будет зависеть от $\epsilon_0$. При достаточно хорошем начальном приближении этот метод может сходиться быстрее, чем остальные методы. Практически обычно метод хорд используется для того, чтобы сузить область, где находится корень.
\subsection{Модификации метода простой итерации.}
Все модификации сводятся к тому, что мы хотим повысить скорость сходимости метода.
\subsubsection{Метод Стеффенсена.}
Метод Стеффенсена основывается на том, что мы укажем способ вычисления $x^{k+1}$ через $x^k$ таким образом, чтобы обеспечить квадратичную скорость сходимости. Для увеличения скорости сходимости в данном методе используется преобразование Эйткена. Суть его состоит в том, что, если имеется сходящаяся последовательность чисел $s_0, s_1,\ldots, s_n,\ldots$, которая сходится к числу $s$, и при этом мы знаем, что характер сходимости носит вид $$s_n = s + Aq^n,\quad A = \operatorname{const}, q < 1,$$ то есть сходимость по закону геометрической прогрессии со знаменателем $q$. Тогда закон Эйткена позволяет сразу получить значение искомого предела по формуле Эйткена, построив последовательность $$\sigma_0,\sigma_1,\ldots,\sigma_n, \quad \sigma_n = s = \dfrac{s_{n+1}s_{n-1} - s_n^2}{s_{n+1} - 2s_n + s_{n-1}} = \lim\limits_{n\to\infty} s_n.\eqno(7)$$
Мы будем использовать эту формулу для того, чтобы сразу найти нужный нам предел в методе простой итерации. \\\\
Пусть мы имеем $x^0$. Берем приближения $$x^1 = \varphi(x^0), \quad x^2 = \varphi(x^1) = \varphi(\varphi(x^0)).$$
Тогда, используя формулу (7), мы можем при $n=1$ получить $$\sigma_1 = \dfrac{x^0x^2 - (x^1)^2}{x^2 - 2x^1 +x^0} = \dfrac{x^0 \varphi(\varphi(x^0)) - (\varphi(x^0))^2}{\varphi(\varphi(x^0)) - 2 \varphi(x^0) + x^0}.$$
Заменим в этой формуле соответствующим образом индексы. В итоге получается итерационная формула, которая получила название \textbf{метода Стеффенсена}
$$x^{k+1} = \dfrac{x^k \varphi(\varphi(x^k)) - (\varphi(x^k))^2}{\varphi(\varphi(x^k)) - 2 \varphi(x^k) + x^k},\ k=0,1,\ldots;\ x^0.\eqno(8)$$
	\end{document}