\documentclass[14pt, a4paper]{report}

\usepackage[english,russian]{babel}
\usepackage{cite}
\usepackage{setspace}
\usepackage[unicode, hidelinks]{hyperref}
\usepackage[utf8]{inputenc}
\usepackage{setspace}
\usepackage{xcolor}
\usepackage{amsmath}
\usepackage{wasysym}
\usepackage{latexsym}
\usepackage{indentfirst}
\usepackage{mathtools}
\usepackage{microtype}
\definecolor{linkcolor}{rgb}{0.0,0.0,0.0}
\definecolor{urlcolor}{HTML}{799B03}
\hypersetup{pdfstartview=FitH, linkcolor=linkcolor,urlcolor=urlcolor, colorlinks=false}
\usepackage[14pt]{extsizes}
\usepackage{linegoal}
\usepackage{enumitem}
\usepackage{amsmath}  % Для теорем и других математических структур
\usepackage{amsthm}   % Для работы с теоремами
\usepackage[
    left=30mm,
    top=20mm,
    right=10mm,
    bottom=20mm
]{geometry}
\usepackage{graphicx}
\usepackage{svg}
\graphicspath{ {./src} }
\usepackage{subfigure}

\usepackage{afterpage}
\usepackage{titlesec}
\usepackage{float}
\usepackage{listings}
\usepackage{csquotes}
\usepackage{mathrsfs}
\usepackage{amssymb}
\usepackage{caption}
%\usepackage[outputdir=output]{mintTed}
%\usepackage{minted}
\usepackage{stackengine}
\usepackage{parskip}
% figure caption type changed
\setstretch{1.8}
\captionsetup{labelsep=space}
\addto\captionsrussian{\renewcommand{\figurename}{\bfseries Рисунок}}
% Настройка заголовков для таблиц
\captionsetup[table]{labelsep=space}
\addto\captionsrussian{\renewcommand{\tablename}{\bfseries Таблица}}
% chapter settings
\titleformat{\chapter}[display]
{\centering\large\bfseries}{\chaptertitlename\ \thechapter}{0pt}{\large}   
\titlespacing*{\chapter}{5pt}{-20pt}{18pt}
\titlespacing*{\section}{0pt}{18pt}{18pt} 
\titlespacing*{\subsection}{0pt}{18pt}{18pt} 
\titlespacing*{\subsubsection}{0pt}{5pt}{5pt} 
% section settings
\titleformat{\section}[block]
{\raggedright\large\bfseries}
{\hspace{1cm}\thesection\ }{0pt}{}
\titleformat{\subsection}[block]
{\raggedright\large\bfseries}
{\hspace{1cm}\thesubsection\ }{0pt}{}
\titleformat{\subsubsection}
{\raggedright\normalfont\normalsize\bfseries} % Шрифт для подподсекции
{\thesubsubsection\hspace{0.5em}} % Номер подподсекции с пробелом
  {0pt} % Отступ от номера
 {\hspace{1.25cm}}
%\titleformat{\section}[block]
  %{\large\bfseries}
  %{\hspace{1.25cm}\thesection\ }{0pt}{}
  %\titleformat{\subsection}[block]
  %{\large\bfseries}
 % {\hspace{1.25cm}\thesubsection\ }{0pt}{}
 %\titleformat{\subsubsection}
 % {\normalfont\normalsize\bfseries} % Шрифт для подподсекции
 % {\thesubsubsection\hspace{0.5em}} % Номер подподсекции с пробелом
 % {0pt} % Отступ от номера
  %{\hspace{1.25cm}}
% settings for chapters and sections
\addto\captionsrussian{% Replace "english" with the language you use
  \renewcommand{\contentsname}%
    {\centering \large ОГЛАВЛЕНИЕ}%
  \renewcommand{\chaptername}{ГЛАВА}
  \renewcommand{\chaptertitlename}{ГЛАВА}
  \renewcommand{\bibname}{\large СПИСОК ИСПОЛЬЗОВАННОЙ ЛИТЕРАТУРЫ}
}
\usepackage{multirow}
\newcommand{\specialcell}[2][c]{%
  \begin{tabular}[#1]{@{}c@{}}#2\end{tabular}}
\setcounter{tocdepth}{2}
\renewcommand{\baselinestretch}{1.28571428571}
\setlength{\parskip}{0pt}
\parindent=1.25cm
\usepackage{lipsum}
\usepackage{booktabs}
\usepackage{tablefootnote}
\makeatletter
\renewcommand{\@biblabel}[1]{#1.}
\makeatother
\newcommand\Norm[1]{\left\| #1 \right\|}
\newcommand{\dif}{\mathrm{d}}
\newcommand{\Rm}{\mathbb{R}}
\newcommand{\Cm}{\mathbb{C}}
\newcommand{\Z}{\mathbb{Z}}
\newcommand{\I}{\mathbb{I}}
\newcommand{\N}{\mathbb{N}}
\newcommand{\E}{\mathbf{E}}
\newcommand{\rank}{\operatorname{rank}}
\newcommand{\Ra}{\Rightarrow}
\newcommand{\ra}{\rightarrow}
\newcommand{\FI}{\Phi}
\newcommand{\Sp}{\text{Sp}}
\renewcommand{\d}{\partial}
\newcommand{\gt}{\textgreater}
\newcommand{\lt}{\textless}
\newtheorem*{theorem}{Теорема}
\newtheorem*{prim}{\hspace*{\parindent}Примечание}
\newtheorem*{zam}{Замечание}
\newtheorem*{cor}{Следствие}
\newtheorem*{lem}{Лемма}
\numberwithin{equation}{section}
\title{Статистический анализ моделей распространения заболеваний}
\author{Гут Валерия Александрова}
\begin{document}
\setcounter{page}{1}
%\setstretch{1.0}
\thispagestyle{empty}
\newgeometry{
	left=30mm,
    top=20mm,
    right=10mm,
    bottom=20mm
}
\begin{center}
\textls[-27]{\textbf{\MakeUppercase{министерство образования республики беларусь}}}\\
\textls[-27]{\textbf{\MakeUppercase{белорусский государственный университет}}}\\
\textls[-27]{\textbf{\MakeUppercase{факультет прикладной математики и информатики}}}\\
\textbf{Кафедра математического моделирования и анализа данных}\\
\vspace{2cm}
БАЖАНОВА\\
\vspace{0.1cm}
Надежда Дмитриевна\\
\vspace{2cm}
\bf
\textls[-27]{\textbf{\MakeUppercase{статистический анализ моделей}}}\\
\textls[-27]{\textbf{\MakeUppercase{распространения заболеваний}}}\\
\vspace{1.6cm}
\rm Дипломная работа
\vspace{2.1cm}
\end{center}
\begin{tabular}{ll}
\hspace{9.5cm}
&Научный руководитель:\\
&зав. кафедрой ММАД,\\
&доктор экономических наук,\\
&доцент В. И. Малюгин\\[2cm]
\end{tabular}\\
\setlength{\parindent}{0mm}
\indent Допущен к защите\par
\textquote{\underline{\hspace{8mm}}} \underline{\hspace{3cm}} 2025 г.\par
Зав. кафедрой математического моделирования и анализа данных\par
доктор экономических наук, доцент В.~И.~Малюгин\par
\vspace{2.5cm}
\setlength{\parindent}{10mm}
\begin{center}
    Mинск, 2025
\end{center}
%\clearpage
%\restoregeometry
%\thispagestyle{empty}
%\include{chapters/task}
\setcounter{page}{3}
\tableofcontents
\normalsize
%\include{chapters/abstr}
%\chapter*{��������}
\addcontentsline {toc}{chapter}{��������}

������� �������������� ������ � ������� ����������� �������� ������� �������� � 1760 ����, ����� �������� ������������� ��������� �������� ���������� �� ����. ����� �������� ������, � 1840 ����, ������ ���� ����������� ���������� ������������� ��� ������� ���������� ���������� �� ������ �� ���� � ������ � ������ �� 1837�1839 ����. ������� ���� ������ ��� ���������������� ������ �������, ������� � 1906 ���� ����������� ������ <<�������������� ������ � �������� ������: ������ ��������>>, ��� ����������� ������������������ ������, ��������� ������������� �������. � ���� �� ������ ����� � ���� ������ ��������� �������������� ������ ��� �������� ��������������� ��������, ��� ��������� ����� ������ ��������� ��������� ������� ���� � ����������� ������� � ������������ �������. �� ������������, � ����� ����� �����, �������, ������ � ������� � ������������, �������� ������ ��� ������������ �������� ��������������� ������������� ��������.

� ���� ������� ������� ��� ������������� <<����� ����������� ����>>, �������� �������� ���������� ����� ������� �������� ��������������� ������������ ������������ ������������� � ��������������. ������ ������� � ����������� ��������� ��������� ������ ��� �������� ���������� SIR-�������, ����������� �������� �������������, �������������� � ������������� � ������� ������ ���������������� � ���������� ���������.

SIR-������ (Susceptible-Infected-Recovered), ����������� �������� ��������, ����� ������� ��� ��������������� �������������, ��� ��� ���������� �������� ������ �������������� �������� � ����� ���� ��������� �� ���� �������� �������������� ��������� � �������������� ������. ��� ������ ����� ������ ��� �������� �� ������������������ �������� � ��������������, ��������� ��������� ��������� �������� ������ � ������ ����������, ����������� ��� ������������������ ���������.

� ������ 1920-� ����� ����� ��������, ��� ��� ��������� �����������, ��������, �����, ������������� ������� ����� �������������. � 1926 ���� ���������� ��������� �������������� ������ SIR-������ ��� ������� ������������ �������� ������ � �������, ���� ��� ������ ���� �������� �� �����. ������������ ������� �� �������� ������������� ������� ������� ������������ ��������, � ����� ������ ���� � ������, � ������� ����������� ������������ �������������. ��� ��������� ������� ��� ��������� ������� <<���������-������������ ������>>.

������ ��� � 1889 ���� ���������� ����������� �.\,�.~����� ���������� ���������� ������ ��������������� ��������, ����������� �������� ����������, ������������ ������ ����--������. ��� ������ ����� ��������� ��������� ������ ������ ����� � ������ ������ �� �������������� ������� � �������������. � ���������� ��� ����������� ����������� ��������� ������ ����� � ������� ������ ������������� ���������. ���������� ������� ������ ����� ��� ����������� � 1989 ����, ����� ���� ��� �������� ����� �� ����������� �������������� �������������.

������ �� �������������� SIR-������, �������������� ���������� � 1949 ����, ����� ������������ ������ ��� �������� �������������� ������� ��������. ������� ������������ ����� � ������ ����������� ��������� ���������� ������ ��������� ��������� � ���� �������.

� 1957 ���� ������� ��������� ���� �� ������ ���������������� ������� �������� �� ������ ��������� � ������� �����������. � ��� �� ���� �������� ������ ��������������� �������� � ���������������� ��������� $6 \times 6$ ����� � ����������� ������� ������������� �������������. � 1971 ���� ���� � ������ ����������� ������ ����������-��������������� ������ ��������������� �����������. ������ ��� ���� ������ ����� ���������� � ����, ��������� �� ������� ������ � �������������� ��������.

� 1980-� ����� �������� �������� �������� �������������� ������� ��� ������ ������������� ��� ������ � �����, ������� �������� ������������. ����� ������ ������������ ��� �� �����������������, ��� � �� ������������� ��������. ������� �������� �������������� ���������� � 80--90-� ����� ��������� ������ ������������ �������������� ������, ��� ������������� ������� � ����������-���������������� ������������� �� ����� XX � XXI �����.

� �������~1 ����� ����������� ������� ������������� SIR-������ � � ����� ������������ ����������. ����� ������� ������ ������ ������� ������� ���������������� ���������, ������� � ������ SIR-������, ��� �������������, ��� � ���������. ����� ����� �������� ������������� �������� ������������� SIR-������� � ������� �������� �������������� �����������. �����, ��� ������� �����������, ����� ������������ ������������ ���������������� SIR-������, ����������� ���������� ��������������� �������� � ������������.

� �������~2 �� �������� ����������-��������������� SIR-������ ��� ����� ������� ������� ������������� �������������, ������� ��������� ��������� ��������� �������������� ��������, �������� �� ��������������� �������. � �������� ���������� ������������ ����� ����������� ����������������� ������, ����������� �������� ����� ����������-��������������� �������, �� ������������ ������� �������������� �������������� �� ���� ��������� ���������.

� �������~3 � �������� ������� ����� ����������� ��������������� ������ � ��������. ������ ��� � ������ � ������� �� ������ ����������� �������� ���� �������������� ������� �������������� ����������, ������� �����. �� ������ ������ ������������� �������������� �������� ���� ����������� ��� ������� � ���������� � ���������� �� ���� �������� ����������, � ����� ���������� ����� 50\,\% ���������� ���� �� 18 ���. ��������� ��������� ������ �� �������� ������ � ��������, �� �������� ������������� ������, ��� �������� �������� ������� �� ������������.

������ ������ ������������ ����������� ������������� SIR-������� � �� ����������� ��� ������������� ��������� ��������� ��������������� ��������. ����� ����������� ��������� ������� ������� ������ ������������� ����������, ��� ������ ������ ����������� ��� ��� �������� ������������ ���������, ��� � ��������� ����������� � ���������� ����������.

%\include{chapters/1chapter}
%\include{chapters/2chapter}
%\include{chapters/3chapter}
%\include{chapters/4chapter}
%\include{chapters/conclusion}
\bibliography{refs}
\bibliographystyle{plain}
\begin{thebibliography}{}
\addcontentsline{toc}{chapter}{СПИСОК ИСПОЛЬЗОВАННОЙ ЛИТЕРАТУРЫ}
\bibitem{1} Национальный банк Республики Беларусь. Мониторинг предприятий реального сектора экономики Республики Беларусь. [Электронный ресурс]. --- \url{https://www.nbrb.by/publications/monitoringpredpriyatij/mp_2021_4.pdf} --- Дата доступа: 05.03.2025.
\bibitem{2} Национальный статистический комитет Республики Беларусь. Валовый внутренний продукт и методы его расчёта. [Электронный ресурс]. --- \url{https://www.belstat.gov.by/upload-belstat/upload-belstat-word/Metod_pologenija/VVP_27_02_2018.doc} --- Дата доступа: 06.03.2025.
\bibitem{3} Харин, Ю. С. Теория вероятностей, математическая и прикладная статистика / Ю. С. Харин, Н. М. Зуев, Е. Е. Жук -- Минск : БГУ, 2011.
\bibitem{4} Интерактивная информационно-аналитическая система распространения официальной статистической информации. [Электронный ресурс]. --- \url{https://dataportal.belstat.gov.by/osids/home-page} --- Дата доступа: 17.02.2025.
\bibitem{5} Национальный статистический комитет Республики Беларусь. Календарь пользователя. [Электронный ресурс]. --- \url{https://www.belstat.gov.by/calendar/} --- Дата доступа: 17.02.2025.
\bibitem{6} Национальный статистически комитет Республики Беларусь. Основные понятия системы национальных счетов. [Электронный ресурс]. --- \url{https://www.belstat.gov.by/upload-belstat/upload-belstat-word/Methodology/m7_sns_29_07_2017.doc} --- Дата доступа: 06.03.2025.
\end{thebibliography}
%\include{chapters/prilog}
%\include{back/part1}
\end{document}

\grid