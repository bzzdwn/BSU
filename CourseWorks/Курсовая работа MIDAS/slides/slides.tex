\documentclass[notheorems]{beamer}
\usepackage{SlideStyle}


\titlegraphic{\vspace*{-7cm}
    \parbox[c]{3cm}{\includegraphics[height=.7cm]{bsulogo}}
    \hspace*{1cm}%
    \parbox[c]{2cm}{\includegraphics[height=0.6cm]{FPMIlogo_new}}
    \hspace*{1cm}%
    \vspace*{3cm}
}

\title[Модели по данным разной частоты]{\Large МОДЕЛИ ПО ДАННЫМ РАЗНОЙ ЧАСТОТЫ И ИХ ПРИМЕНЕНИЯ В ЗАДАЧАХ ПРОГНОЗИРОВАНИЯ ВРЕМЕННЫХ РЯДОВ}


\author[Т. А. Бовт]{Бовт Тимофей Анатольевич}

\institute[]{Научный руководитель: В.И. Малюгин}


\date[]{}%{\scriptsize \structure{2017-2018}}


\begin{document}

\begin{frame}[plain]
  \titlepage
\end{frame}


%--------------------------------------------------------------------------------------
\begin{frame}{Постановка решаемой задачи по данным разной частоты}

\begin{itemize}

\item Кратко описать объект исследования .... \vfill

\item Цели работы  \vfill

\item (не более 3 строк)\vfill

\item ....

\end{itemize}

\end{frame}

%--------------------------------------------------------------------------

\begin{frame}
	{Проблема прогнозирования по данным разной частоты}
	Обычно все часто применяемые регрессионные модели машинного обучения работают с данными, заданными в одной частоте. Но некоторые данные из сферы экономики, как правило, формируются в квартальных представлениях. Параллельно с этим какие-либо объясняющие факторы могут быть собраны с более высокой частотой, будь то ежемесячные, еженедельные или ежедневные представления.
	
	\textbf{Популярные способы решения этой проблемы:}
	\begin{itemize}
		\item наивное приведение данных более высокой частоты к нужной нам более низкой частоте, иначе говоря, агрегация данных более высокой частоты;
		\item специальные подходы для заполнения пропущенных значений.
	\end{itemize}
\end{frame}
%--------------------------------------------------------------------------
\begin{frame}
	\frametitle{Модели по данным разной частоты}
	Чтобы ввести модель Mixed Data Sampling (MiDaS) регрессии, введем обозначения:
	\begin{itemize}
		\item $t = 1,\ldots, T$ --- единицы времени;
		\item $y_{t}^{(q)}$ --- эндогенная квартальная переменная;
		\item $x^{(m)}_{t}$ --- экзогенная месячная переменная;
		\item $\varepsilon_t^{(m)}$ --- белый шум;
		\item $\beta_0, \beta_1 \in \mathbb R$ --- свободные переменные;
		\item $B(L^{1/m}, \Theta) = \sum\limits_{j=0}^{K} B(j, \Theta) L^{j/m},$ где $L^{j/m}x_t^{(m)} = x_{(t-j)/m}^{(m)}$ --- лаговый оператор.
	\end{itemize}
	Тогда базовая MiDaS модель может быть сформулирована в виде
	\begin{equation}
		y_t^{(q)} = \beta_0 + \beta_1B(L^{1/m}, \Theta) x_t^{(m)} + \varepsilon_t^{(m)}.
	\end{equation}
	Также можем записать это уравнение в виде
	\begin{equation}
		y_t^{(q)} = \beta_0 + \beta_1 \sum_{j=0}^{K} B(j,\Theta) x_{(t-j)/m}^{(m)} + \varepsilon_t^{(m)}.
	\end{equation}
\end{frame}
%--------------------------------------------------------------------------
\begin{frame}
	\frametitle{Лаговые многочлены}
	Базовые MiDaS модели отличаются между собой в зависимости от выбора лагового оператора $B(L^{1/m}, \Theta) = \sum\limits_{j=0}^{K} B(j, \Theta) L^{j/m}.$ Фактически задание этого оператора определяет
	способ агрегации данных высокой частоты в ряд более низкой частоты.
	Наиболее распространенными являются следующие виды функции лаговых коэффициентов:
	\begin{itemize}
		\item экспоненциальные лаги Алмона
		$$B(j, \Theta) = \dfrac{e^{\Theta_1 j + \ldots \Theta_n j^n}}{\sum_{s=0}^{K}e^{\Theta_1 s + \ldots \Theta_n s^n}};$$
		\item бета лаги
		$$B(j, \Theta_1, \Theta_2) = \dfrac{f(\frac j K, \Theta_1;\Theta_2)}{\sum_{s=0}^{K}f(\frac s K, \Theta_1;\Theta_2)},$$
		где $$f(x, \Theta_1, \Theta_2) = \dfrac{x^{a-1}(1-x)^{b-1}\Gamma(\Theta_1 + \Theta_2)}{\Gamma(\Theta_1)\Gamma(\Theta_2)};$$
	\end{itemize}
\end{frame}

%--------------------------------------------------------------------------

\end{document} 