\documentclass[11pt]{article}
\usepackage[T2A]{fontenc}
\usepackage[utf8]{inputenc}
\usepackage[english,russian]{babel}

    \usepackage[breakable]{tcolorbox}
    \usepackage{parskip} % Stop auto-indenting (to mimic markdown behaviour)
    

    % Basic figure setup, for now with no caption control since it's done
    % automatically by Pandoc (which extracts ![](path) syntax from Markdown).
    \usepackage{graphicx}
    % Maintain compatibility with old templates. Remove in nbconvert 6.0
    \let\Oldincludegraphics\includegraphics
    % Ensure that by default, figures have no caption (until we provide a
    % proper Figure object with a Caption API and a way to capture that
    % in the conversion process - todo).
    \usepackage{caption}
    \DeclareCaptionFormat{nocaption}{}
    \captionsetup{format=nocaption,aboveskip=0pt,belowskip=0pt}

    \usepackage{float}
    \floatplacement{figure}{H} % forces figures to be placed at the correct location
    \usepackage{xcolor} % Allow colors to be defined
    \usepackage{enumerate} % Needed for markdown enumerations to work
    \usepackage{geometry} % Used to adjust the document margins
    \usepackage{amsmath} % Equations
    \usepackage{amssymb} % Equations
    \usepackage{textcomp} % defines textquotesingle
    % Hack from http://tex.stackexchange.com/a/47451/13684:
    \AtBeginDocument{%
        \def\PYZsq{\textquotesingle}% Upright quotes in Pygmentized code
    }
    \usepackage{upquote} % Upright quotes for verbatim code
    \usepackage{eurosym} % defines \euro

    \usepackage{iftex}
    \ifPDFTeX
        \usepackage[T1]{fontenc}
        \IfFileExists{alphabeta.sty}{
              \usepackage{alphabeta}
          }{
              \usepackage[mathletters]{ucs}
              \usepackage[utf8x]{inputenc}
          }
    \else
        \usepackage{fontspec}
        \usepackage{unicode-math}
    \fi

    \usepackage{fancyvrb} % verbatim replacement that allows latex
    \usepackage{grffile} % extends the file name processing of package graphics
                         % to support a larger range
    \makeatletter % fix for old versions of grffile with XeLaTeX
    \@ifpackagelater{grffile}{2019/11/01}
    {
      % Do nothing on new versions
    }
    {
      \def\Gread@@xetex#1{%
        \IfFileExists{"\Gin@base".bb}%
        {\Gread@eps{\Gin@base.bb}}%
        {\Gread@@xetex@aux#1}%
      }
    }
    \makeatother
    \usepackage[Export]{adjustbox} % Used to constrain images to a maximum size
    \adjustboxset{max size={0.9\linewidth}{0.9\paperheight}}

    % The hyperref package gives us a pdf with properly built
    % internal navigation ('pdf bookmarks' for the table of contents,
    % internal cross-reference links, web links for URLs, etc.)
    \usepackage{hyperref}
    % The default LaTeX title has an obnoxious amount of whitespace. By default,
    % titling removes some of it. It also provides customization options.
    \usepackage{titling}
    \usepackage{longtable} % longtable support required by pandoc >1.10
    \usepackage{booktabs}  % table support for pandoc > 1.12.2
    \usepackage{array}     % table support for pandoc >= 2.11.3
    \usepackage{calc}      % table minipage width calculation for pandoc >= 2.11.1
    \usepackage[inline]{enumitem} % IRkernel/repr support (it uses the enumerate* environment)
    \usepackage[normalem]{ulem} % ulem is needed to support strikethroughs (\sout)
                                % normalem makes italics be italics, not underlines
    \usepackage{mathrsfs}
    

    
    % Colors for the hyperref package
    \definecolor{urlcolor}{rgb}{0,.145,.698}
    \definecolor{linkcolor}{rgb}{.71,0.21,0.01}
    \definecolor{citecolor}{rgb}{.12,.54,.11}

    % ANSI colors
    \definecolor{ansi-black}{HTML}{3E424D}
    \definecolor{ansi-black-intense}{HTML}{282C36}
    \definecolor{ansi-red}{HTML}{E75C58}
    \definecolor{ansi-red-intense}{HTML}{B22B31}
    \definecolor{ansi-green}{HTML}{00A250}
    \definecolor{ansi-green-intense}{HTML}{007427}
    \definecolor{ansi-yellow}{HTML}{DDB62B}
    \definecolor{ansi-yellow-intense}{HTML}{B27D12}
    \definecolor{ansi-blue}{HTML}{208FFB}
    \definecolor{ansi-blue-intense}{HTML}{0065CA}
    \definecolor{ansi-magenta}{HTML}{D160C4}
    \definecolor{ansi-magenta-intense}{HTML}{A03196}
    \definecolor{ansi-cyan}{HTML}{60C6C8}
    \definecolor{ansi-cyan-intense}{HTML}{258F8F}
    \definecolor{ansi-white}{HTML}{C5C1B4}
    \definecolor{ansi-white-intense}{HTML}{A1A6B2}
    \definecolor{ansi-default-inverse-fg}{HTML}{FFFFFF}
    \definecolor{ansi-default-inverse-bg}{HTML}{000000}

    % common color for the border for error outputs.
    \definecolor{outerrorbackground}{HTML}{FFDFDF}

    % commands and environments needed by pandoc snippets
    % extracted from the output of `pandoc -s`
    \providecommand{\tightlist}{%
      \setlength{\itemsep}{0pt}\setlength{\parskip}{0pt}}
    \DefineVerbatimEnvironment{Highlighting}{Verbatim}{commandchars=\\\{\}}
    % Add ',fontsize=\small' for more characters per line
    \newenvironment{Shaded}{}{}
    \newcommand{\KeywordTok}[1]{\textcolor[rgb]{0.00,0.44,0.13}{\textbf{{#1}}}}
    \newcommand{\DataTypeTok}[1]{\textcolor[rgb]{0.56,0.13,0.00}{{#1}}}
    \newcommand{\DecValTok}[1]{\textcolor[rgb]{0.25,0.63,0.44}{{#1}}}
    \newcommand{\BaseNTok}[1]{\textcolor[rgb]{0.25,0.63,0.44}{{#1}}}
    \newcommand{\FloatTok}[1]{\textcolor[rgb]{0.25,0.63,0.44}{{#1}}}
    \newcommand{\CharTok}[1]{\textcolor[rgb]{0.25,0.44,0.63}{{#1}}}
    \newcommand{\StringTok}[1]{\textcolor[rgb]{0.25,0.44,0.63}{{#1}}}
    \newcommand{\CommentTok}[1]{\textcolor[rgb]{0.38,0.63,0.69}{\textit{{#1}}}}
    \newcommand{\OtherTok}[1]{\textcolor[rgb]{0.00,0.44,0.13}{{#1}}}
    \newcommand{\AlertTok}[1]{\textcolor[rgb]{1.00,0.00,0.00}{\textbf{{#1}}}}
    \newcommand{\FunctionTok}[1]{\textcolor[rgb]{0.02,0.16,0.49}{{#1}}}
    \newcommand{\RegionMarkerTok}[1]{{#1}}
    \newcommand{\ErrorTok}[1]{\textcolor[rgb]{1.00,0.00,0.00}{\textbf{{#1}}}}
    \newcommand{\NormalTok}[1]{{#1}}

    % Additional commands for more recent versions of Pandoc
    \newcommand{\ConstantTok}[1]{\textcolor[rgb]{0.53,0.00,0.00}{{#1}}}
    \newcommand{\SpecialCharTok}[1]{\textcolor[rgb]{0.25,0.44,0.63}{{#1}}}
    \newcommand{\VerbatimStringTok}[1]{\textcolor[rgb]{0.25,0.44,0.63}{{#1}}}
    \newcommand{\SpecialStringTok}[1]{\textcolor[rgb]{0.73,0.40,0.53}{{#1}}}
    \newcommand{\ImportTok}[1]{{#1}}
    \newcommand{\DocumentationTok}[1]{\textcolor[rgb]{0.73,0.13,0.13}{\textit{{#1}}}}
    \newcommand{\AnnotationTok}[1]{\textcolor[rgb]{0.38,0.63,0.69}{\textbf{\textit{{#1}}}}}
    \newcommand{\CommentVarTok}[1]{\textcolor[rgb]{0.38,0.63,0.69}{\textbf{\textit{{#1}}}}}
    \newcommand{\VariableTok}[1]{\textcolor[rgb]{0.10,0.09,0.49}{{#1}}}
    \newcommand{\ControlFlowTok}[1]{\textcolor[rgb]{0.00,0.44,0.13}{\textbf{{#1}}}}
    \newcommand{\OperatorTok}[1]{\textcolor[rgb]{0.40,0.40,0.40}{{#1}}}
    \newcommand{\BuiltInTok}[1]{{#1}}
    \newcommand{\ExtensionTok}[1]{{#1}}
    \newcommand{\PreprocessorTok}[1]{\textcolor[rgb]{0.74,0.48,0.00}{{#1}}}
    \newcommand{\AttributeTok}[1]{\textcolor[rgb]{0.49,0.56,0.16}{{#1}}}
    \newcommand{\InformationTok}[1]{\textcolor[rgb]{0.38,0.63,0.69}{\textbf{\textit{{#1}}}}}
    \newcommand{\WarningTok}[1]{\textcolor[rgb]{0.38,0.63,0.69}{\textbf{\textit{{#1}}}}}


    % Define a nice break command that doesn't care if a line doesn't already
    % exist.
    \def\br{\hspace*{\fill} \\* }
    % Math Jax compatibility definitions
    \def\gt{>}
    \def\lt{<}
    \let\Oldtex\TeX
    \let\Oldlatex\LaTeX
    \renewcommand{\TeX}{\textrm{\Oldtex}}
    \renewcommand{\LaTeX}{\textrm{\Oldlatex}}
    % Document parameters
    % Document title
    \title{11}
    
    
    
    
    
% Pygments definitions
\makeatletter
\def\PY@reset{\let\PY@it=\relax \let\PY@bf=\relax%
    \let\PY@ul=\relax \let\PY@tc=\relax%
    \let\PY@bc=\relax \let\PY@ff=\relax}
\def\PY@tok#1{\csname PY@tok@#1\endcsname}
\def\PY@toks#1+{\ifx\relax#1\empty\else%
    \PY@tok{#1}\expandafter\PY@toks\fi}
\def\PY@do#1{\PY@bc{\PY@tc{\PY@ul{%
    \PY@it{\PY@bf{\PY@ff{#1}}}}}}}
\def\PY#1#2{\PY@reset\PY@toks#1+\relax+\PY@do{#2}}

\@namedef{PY@tok@w}{\def\PY@tc##1{\textcolor[rgb]{0.73,0.73,0.73}{##1}}}
\@namedef{PY@tok@c}{\let\PY@it=\textit\def\PY@tc##1{\textcolor[rgb]{0.24,0.48,0.48}{##1}}}
\@namedef{PY@tok@cp}{\def\PY@tc##1{\textcolor[rgb]{0.61,0.40,0.00}{##1}}}
\@namedef{PY@tok@k}{\let\PY@bf=\textbf\def\PY@tc##1{\textcolor[rgb]{0.00,0.50,0.00}{##1}}}
\@namedef{PY@tok@kp}{\def\PY@tc##1{\textcolor[rgb]{0.00,0.50,0.00}{##1}}}
\@namedef{PY@tok@kt}{\def\PY@tc##1{\textcolor[rgb]{0.69,0.00,0.25}{##1}}}
\@namedef{PY@tok@o}{\def\PY@tc##1{\textcolor[rgb]{0.40,0.40,0.40}{##1}}}
\@namedef{PY@tok@ow}{\let\PY@bf=\textbf\def\PY@tc##1{\textcolor[rgb]{0.67,0.13,1.00}{##1}}}
\@namedef{PY@tok@nb}{\def\PY@tc##1{\textcolor[rgb]{0.00,0.50,0.00}{##1}}}
\@namedef{PY@tok@nf}{\def\PY@tc##1{\textcolor[rgb]{0.00,0.00,1.00}{##1}}}
\@namedef{PY@tok@nc}{\let\PY@bf=\textbf\def\PY@tc##1{\textcolor[rgb]{0.00,0.00,1.00}{##1}}}
\@namedef{PY@tok@nn}{\let\PY@bf=\textbf\def\PY@tc##1{\textcolor[rgb]{0.00,0.00,1.00}{##1}}}
\@namedef{PY@tok@ne}{\let\PY@bf=\textbf\def\PY@tc##1{\textcolor[rgb]{0.80,0.25,0.22}{##1}}}
\@namedef{PY@tok@nv}{\def\PY@tc##1{\textcolor[rgb]{0.10,0.09,0.49}{##1}}}
\@namedef{PY@tok@no}{\def\PY@tc##1{\textcolor[rgb]{0.53,0.00,0.00}{##1}}}
\@namedef{PY@tok@nl}{\def\PY@tc##1{\textcolor[rgb]{0.46,0.46,0.00}{##1}}}
\@namedef{PY@tok@ni}{\let\PY@bf=\textbf\def\PY@tc##1{\textcolor[rgb]{0.44,0.44,0.44}{##1}}}
\@namedef{PY@tok@na}{\def\PY@tc##1{\textcolor[rgb]{0.41,0.47,0.13}{##1}}}
\@namedef{PY@tok@nt}{\let\PY@bf=\textbf\def\PY@tc##1{\textcolor[rgb]{0.00,0.50,0.00}{##1}}}
\@namedef{PY@tok@nd}{\def\PY@tc##1{\textcolor[rgb]{0.67,0.13,1.00}{##1}}}
\@namedef{PY@tok@s}{\def\PY@tc##1{\textcolor[rgb]{0.73,0.13,0.13}{##1}}}
\@namedef{PY@tok@sd}{\let\PY@it=\textit\def\PY@tc##1{\textcolor[rgb]{0.73,0.13,0.13}{##1}}}
\@namedef{PY@tok@si}{\let\PY@bf=\textbf\def\PY@tc##1{\textcolor[rgb]{0.64,0.35,0.47}{##1}}}
\@namedef{PY@tok@se}{\let\PY@bf=\textbf\def\PY@tc##1{\textcolor[rgb]{0.67,0.36,0.12}{##1}}}
\@namedef{PY@tok@sr}{\def\PY@tc##1{\textcolor[rgb]{0.64,0.35,0.47}{##1}}}
\@namedef{PY@tok@ss}{\def\PY@tc##1{\textcolor[rgb]{0.10,0.09,0.49}{##1}}}
\@namedef{PY@tok@sx}{\def\PY@tc##1{\textcolor[rgb]{0.00,0.50,0.00}{##1}}}
\@namedef{PY@tok@m}{\def\PY@tc##1{\textcolor[rgb]{0.40,0.40,0.40}{##1}}}
\@namedef{PY@tok@gh}{\let\PY@bf=\textbf\def\PY@tc##1{\textcolor[rgb]{0.00,0.00,0.50}{##1}}}
\@namedef{PY@tok@gu}{\let\PY@bf=\textbf\def\PY@tc##1{\textcolor[rgb]{0.50,0.00,0.50}{##1}}}
\@namedef{PY@tok@gd}{\def\PY@tc##1{\textcolor[rgb]{0.63,0.00,0.00}{##1}}}
\@namedef{PY@tok@gi}{\def\PY@tc##1{\textcolor[rgb]{0.00,0.52,0.00}{##1}}}
\@namedef{PY@tok@gr}{\def\PY@tc##1{\textcolor[rgb]{0.89,0.00,0.00}{##1}}}
\@namedef{PY@tok@ge}{\let\PY@it=\textit}
\@namedef{PY@tok@gs}{\let\PY@bf=\textbf}
\@namedef{PY@tok@gp}{\let\PY@bf=\textbf\def\PY@tc##1{\textcolor[rgb]{0.00,0.00,0.50}{##1}}}
\@namedef{PY@tok@go}{\def\PY@tc##1{\textcolor[rgb]{0.44,0.44,0.44}{##1}}}
\@namedef{PY@tok@gt}{\def\PY@tc##1{\textcolor[rgb]{0.00,0.27,0.87}{##1}}}
\@namedef{PY@tok@err}{\def\PY@bc##1{{\setlength{\fboxsep}{\string -\fboxrule}\fcolorbox[rgb]{1.00,0.00,0.00}{1,1,1}{\strut ##1}}}}
\@namedef{PY@tok@kc}{\let\PY@bf=\textbf\def\PY@tc##1{\textcolor[rgb]{0.00,0.50,0.00}{##1}}}
\@namedef{PY@tok@kd}{\let\PY@bf=\textbf\def\PY@tc##1{\textcolor[rgb]{0.00,0.50,0.00}{##1}}}
\@namedef{PY@tok@kn}{\let\PY@bf=\textbf\def\PY@tc##1{\textcolor[rgb]{0.00,0.50,0.00}{##1}}}
\@namedef{PY@tok@kr}{\let\PY@bf=\textbf\def\PY@tc##1{\textcolor[rgb]{0.00,0.50,0.00}{##1}}}
\@namedef{PY@tok@bp}{\def\PY@tc##1{\textcolor[rgb]{0.00,0.50,0.00}{##1}}}
\@namedef{PY@tok@fm}{\def\PY@tc##1{\textcolor[rgb]{0.00,0.00,1.00}{##1}}}
\@namedef{PY@tok@vc}{\def\PY@tc##1{\textcolor[rgb]{0.10,0.09,0.49}{##1}}}
\@namedef{PY@tok@vg}{\def\PY@tc##1{\textcolor[rgb]{0.10,0.09,0.49}{##1}}}
\@namedef{PY@tok@vi}{\def\PY@tc##1{\textcolor[rgb]{0.10,0.09,0.49}{##1}}}
\@namedef{PY@tok@vm}{\def\PY@tc##1{\textcolor[rgb]{0.10,0.09,0.49}{##1}}}
\@namedef{PY@tok@sa}{\def\PY@tc##1{\textcolor[rgb]{0.73,0.13,0.13}{##1}}}
\@namedef{PY@tok@sb}{\def\PY@tc##1{\textcolor[rgb]{0.73,0.13,0.13}{##1}}}
\@namedef{PY@tok@sc}{\def\PY@tc##1{\textcolor[rgb]{0.73,0.13,0.13}{##1}}}
\@namedef{PY@tok@dl}{\def\PY@tc##1{\textcolor[rgb]{0.73,0.13,0.13}{##1}}}
\@namedef{PY@tok@s2}{\def\PY@tc##1{\textcolor[rgb]{0.73,0.13,0.13}{##1}}}
\@namedef{PY@tok@sh}{\def\PY@tc##1{\textcolor[rgb]{0.73,0.13,0.13}{##1}}}
\@namedef{PY@tok@s1}{\def\PY@tc##1{\textcolor[rgb]{0.73,0.13,0.13}{##1}}}
\@namedef{PY@tok@mb}{\def\PY@tc##1{\textcolor[rgb]{0.40,0.40,0.40}{##1}}}
\@namedef{PY@tok@mf}{\def\PY@tc##1{\textcolor[rgb]{0.40,0.40,0.40}{##1}}}
\@namedef{PY@tok@mh}{\def\PY@tc##1{\textcolor[rgb]{0.40,0.40,0.40}{##1}}}
\@namedef{PY@tok@mi}{\def\PY@tc##1{\textcolor[rgb]{0.40,0.40,0.40}{##1}}}
\@namedef{PY@tok@il}{\def\PY@tc##1{\textcolor[rgb]{0.40,0.40,0.40}{##1}}}
\@namedef{PY@tok@mo}{\def\PY@tc##1{\textcolor[rgb]{0.40,0.40,0.40}{##1}}}
\@namedef{PY@tok@ch}{\let\PY@it=\textit\def\PY@tc##1{\textcolor[rgb]{0.24,0.48,0.48}{##1}}}
\@namedef{PY@tok@cm}{\let\PY@it=\textit\def\PY@tc##1{\textcolor[rgb]{0.24,0.48,0.48}{##1}}}
\@namedef{PY@tok@cpf}{\let\PY@it=\textit\def\PY@tc##1{\textcolor[rgb]{0.24,0.48,0.48}{##1}}}
\@namedef{PY@tok@c1}{\let\PY@it=\textit\def\PY@tc##1{\textcolor[rgb]{0.24,0.48,0.48}{##1}}}
\@namedef{PY@tok@cs}{\let\PY@it=\textit\def\PY@tc##1{\textcolor[rgb]{0.24,0.48,0.48}{##1}}}

\def\PYZbs{\char`\\}
\def\PYZus{\char`\_}
\def\PYZob{\char`\{}
\def\PYZcb{\char`\}}
\def\PYZca{\char`\^}
\def\PYZam{\char`\&}
\def\PYZlt{\char`\<}
\def\PYZgt{\char`\>}
\def\PYZsh{\char`\#}
\def\PYZpc{\char`\%}
\def\PYZdl{\char`\$}
\def\PYZhy{\char`\-}
\def\PYZsq{\char`\'}
\def\PYZdq{\char`\"}
\def\PYZti{\char`\~}
% for compatibility with earlier versions
\def\PYZat{@}
\def\PYZlb{[}
\def\PYZrb{]}
\makeatother


    % For linebreaks inside Verbatim environment from package fancyvrb.
    \makeatletter
        \newbox\Wrappedcontinuationbox
        \newbox\Wrappedvisiblespacebox
        \newcommand*\Wrappedvisiblespace {\textcolor{red}{\textvisiblespace}}
        \newcommand*\Wrappedcontinuationsymbol {\textcolor{red}{\llap{\tiny$\m@th\hookrightarrow$}}}
        \newcommand*\Wrappedcontinuationindent {3ex }
        \newcommand*\Wrappedafterbreak {\kern\Wrappedcontinuationindent\copy\Wrappedcontinuationbox}
        % Take advantage of the already applied Pygments mark-up to insert
        % potential linebreaks for TeX processing.
        %        {, <, #, %, $, ' and ": go to next line.
        %        _, }, ^, &, >, - and ~: stay at end of broken line.
        % Use of \textquotesingle for straight quote.
        \newcommand*\Wrappedbreaksatspecials {%
            \def\PYGZus{\discretionary{\char`\_}{\Wrappedafterbreak}{\char`\_}}%
            \def\PYGZob{\discretionary{}{\Wrappedafterbreak\char`\{}{\char`\{}}%
            \def\PYGZcb{\discretionary{\char`\}}{\Wrappedafterbreak}{\char`\}}}%
            \def\PYGZca{\discretionary{\char`\^}{\Wrappedafterbreak}{\char`\^}}%
            \def\PYGZam{\discretionary{\char`\&}{\Wrappedafterbreak}{\char`\&}}%
            \def\PYGZlt{\discretionary{}{\Wrappedafterbreak\char`\<}{\char`\<}}%
            \def\PYGZgt{\discretionary{\char`\>}{\Wrappedafterbreak}{\char`\>}}%
            \def\PYGZsh{\discretionary{}{\Wrappedafterbreak\char`\#}{\char`\#}}%
            \def\PYGZpc{\discretionary{}{\Wrappedafterbreak\char`\%}{\char`\%}}%
            \def\PYGZdl{\discretionary{}{\Wrappedafterbreak\char`\$}{\char`\$}}%
            \def\PYGZhy{\discretionary{\char`\-}{\Wrappedafterbreak}{\char`\-}}%
            \def\PYGZsq{\discretionary{}{\Wrappedafterbreak\textquotesingle}{\textquotesingle}}%
            \def\PYGZdq{\discretionary{}{\Wrappedafterbreak\char`\"}{\char`\"}}%
            \def\PYGZti{\discretionary{\char`\~}{\Wrappedafterbreak}{\char`\~}}%
        }
        % Some characters . , ; ? ! / are not pygmentized.
        % This macro makes them "active" and they will insert potential linebreaks
        \newcommand*\Wrappedbreaksatpunct {%
            \lccode`\~`\.\lowercase{\def~}{\discretionary{\hbox{\char`\.}}{\Wrappedafterbreak}{\hbox{\char`\.}}}%
            \lccode`\~`\,\lowercase{\def~}{\discretionary{\hbox{\char`\,}}{\Wrappedafterbreak}{\hbox{\char`\,}}}%
            \lccode`\~`\;\lowercase{\def~}{\discretionary{\hbox{\char`\;}}{\Wrappedafterbreak}{\hbox{\char`\;}}}%
            \lccode`\~`\:\lowercase{\def~}{\discretionary{\hbox{\char`\:}}{\Wrappedafterbreak}{\hbox{\char`\:}}}%
            \lccode`\~`\?\lowercase{\def~}{\discretionary{\hbox{\char`\?}}{\Wrappedafterbreak}{\hbox{\char`\?}}}%
            \lccode`\~`\!\lowercase{\def~}{\discretionary{\hbox{\char`\!}}{\Wrappedafterbreak}{\hbox{\char`\!}}}%
            \lccode`\~`\/\lowercase{\def~}{\discretionary{\hbox{\char`\/}}{\Wrappedafterbreak}{\hbox{\char`\/}}}%
            \catcode`\.\active
            \catcode`\,\active
            \catcode`\;\active
            \catcode`\:\active
            \catcode`\?\active
            \catcode`\!\active
            \catcode`\/\active
            \lccode`\~`\~
        }
    \makeatother

    \let\OriginalVerbatim=\Verbatim
    \makeatletter
    \renewcommand{\Verbatim}[1][1]{%
        %\parskip\z@skip
        \sbox\Wrappedcontinuationbox {\Wrappedcontinuationsymbol}%
        \sbox\Wrappedvisiblespacebox {\FV@SetupFont\Wrappedvisiblespace}%
        \def\FancyVerbFormatLine ##1{\hsize\linewidth
            \vtop{\raggedright\hyphenpenalty\z@\exhyphenpenalty\z@
                \doublehyphendemerits\z@\finalhyphendemerits\z@
                \strut ##1\strut}%
        }%
        % If the linebreak is at a space, the latter will be displayed as visible
        % space at end of first line, and a continuation symbol starts next line.
        % Stretch/shrink are however usually zero for typewriter font.
        \def\FV@Space {%
            \nobreak\hskip\z@ plus\fontdimen3\font minus\fontdimen4\font
            \discretionary{\copy\Wrappedvisiblespacebox}{\Wrappedafterbreak}
            {\kern\fontdimen2\font}%
        }%

        % Allow breaks at special characters using \PYG... macros.
        \Wrappedbreaksatspecials
        % Breaks at punctuation characters . , ; ? ! and / need catcode=\active
        \OriginalVerbatim[#1,codes*=\Wrappedbreaksatpunct]%
    }
    \makeatother

    % Exact colors from NB
    \definecolor{incolor}{HTML}{303F9F}
    \definecolor{outcolor}{HTML}{D84315}
    \definecolor{cellborder}{HTML}{CFCFCF}
    \definecolor{cellbackground}{HTML}{F7F7F7}

    % prompt
    \makeatletter
    \newcommand{\boxspacing}{\kern\kvtcb@left@rule\kern\kvtcb@boxsep}
    \makeatother
    \newcommand{\prompt}[4]{
        {\ttfamily\llap{{\color{#2}[#3]:\hspace{3pt}#4}}\vspace{-\baselineskip}}
    }
    

    
    % Prevent overflowing lines due to hard-to-break entities
    \sloppy
    % Setup hyperref package
    \hypersetup{
      breaklinks=true,  % so long urls are correctly broken across lines
      colorlinks=true,
      urlcolor=urlcolor,
      linkcolor=linkcolor,
      citecolor=citecolor,
      }
    % Slightly bigger margins than the latex defaults
    
    \geometry{verbose,tmargin=1in,bmargin=1in,lmargin=1in,rmargin=1in}
    
    

\begin{document}
    
    \begin{titlepage}
    	\begin{center}
    		\textsc{МИНИСТЕРСТВО ОБРАЗОВАНИЯ РЕСПУБЛИКИ БЕЛАРУСЬ БЕЛОРУССКИЙ ГОСУДАРСТВЕННЫЙ УНИВЕРСИТЕТ
    			\\[5mm]
    			ФАКУЛЬТЕТ ПРИКЛАДНОЙ МАТЕМАТИКИ И ИНФОРМАТИКИ\\[2mm]
    			Кафедра информационных систем управления
    		}
    		
    		\vfill
    		
    		\textbf{Отчет по лабораторной работе №11\\
    			Вариант 2
    			\\[26mm]
    		}
    	\end{center}
    	
    	\hfill
    	\begin{minipage}{.5\textwidth}
    		\begin{flushright}
    			Бовта Тимофея Анатольевича\\
    			студента 3 курса\\
    			специальности «прикладная математика»\\[5mm]
    			
    			Преподаватель:\\[2mm] 
    			Д. Ю. Кваша\\
    		\end{flushright}
    	\end{minipage}%
    	\vfill
    	\begin{center}
    		Минск, 2024\ г.
    	\end{center}
    \end{titlepage}
    
    \section{Постановка
задачи}\label{ux43fux43eux441ux442ux430ux43dux43eux432ux43aux430-ux437ux430ux434ux430ux447ux438}

    Сформулируйте заданную проблему как проблему удовлетворения ограничений.
Определите переменные, домены и ограничения. Создайте модель. Решите
задачу OR-Tools.

    \begin{tcolorbox}[breakable, size=fbox, boxrule=1pt, pad at break*=1mm,colback=cellbackground, colframe=cellborder]
\prompt{In}{incolor}{1}{\boxspacing}
\begin{Verbatim}[commandchars=\\\{\}]
\PY{k+kn}{from} \PY{n+nn}{ortools}\PY{n+nn}{.}\PY{n+nn}{sat}\PY{n+nn}{.}\PY{n+nn}{python} \PY{k+kn}{import} \PY{n}{cp\PYZus{}model}
\end{Verbatim}
\end{tcolorbox}

    \section{Общее
условие}\label{ux43eux431ux449ux435ux435-ux443ux441ux43bux43eux432ux438ux435}

    У вас есть сумка, которая может нести 20 кг. У вас есть набор вещей,
которые вы хотите взять с собой, и их вес:

    \begin{tcolorbox}[breakable, size=fbox, boxrule=1pt, pad at break*=1mm,colback=cellbackground, colframe=cellborder]
\prompt{In}{incolor}{2}{\boxspacing}
\begin{Verbatim}[commandchars=\\\{\}]
\PY{n}{items} \PY{o}{=} \PY{p}{[}\PY{l+s+s1}{\PYZsq{}}\PY{l+s+s1}{book}\PY{l+s+s1}{\PYZsq{}}\PY{p}{,} \PY{l+s+s1}{\PYZsq{}}\PY{l+s+s1}{jacket}\PY{l+s+s1}{\PYZsq{}}\PY{p}{,} \PY{l+s+s1}{\PYZsq{}}\PY{l+s+s1}{washbag}\PY{l+s+s1}{\PYZsq{}}\PY{p}{,} \PY{l+s+s1}{\PYZsq{}}\PY{l+s+s1}{computer}\PY{l+s+s1}{\PYZsq{}}\PY{p}{,} \PY{l+s+s1}{\PYZsq{}}\PY{l+s+s1}{boots}\PY{l+s+s1}{\PYZsq{}}\PY{p}{,} \PY{l+s+s1}{\PYZsq{}}\PY{l+s+s1}{alarmclock}\PY{l+s+s1}{\PYZsq{}}\PY{p}{,} \PY{l+s+s1}{\PYZsq{}}\PY{l+s+s1}{anorak}\PY{l+s+s1}{\PYZsq{}}\PY{p}{,} \PY{l+s+s1}{\PYZsq{}}\PY{l+s+s1}{food}\PY{l+s+s1}{\PYZsq{}}\PY{p}{]}
\PY{n}{weight} \PY{o}{=} \PY{p}{[}\PY{l+m+mi}{2}\PY{p}{,} \PY{l+m+mi}{4}\PY{p}{,} \PY{l+m+mi}{3}\PY{p}{,} \PY{l+m+mi}{8}\PY{p}{,} \PY{l+m+mi}{7}\PY{p}{,} \PY{l+m+mi}{1}\PY{p}{,} \PY{l+m+mi}{2}\PY{p}{,} \PY{l+m+mi}{6}\PY{p}{]}
\end{Verbatim}
\end{tcolorbox}

    Сформулируем поставленную задачу как задачу целочисленного линейного
программирования.

Каждый предмет обозначим отдельной переменной:

\begin{itemize}
\item
  \(x_1 = \{\text{book}\}\);
\item
  \(x_2 = \{\text{jacket}\}\);
\item
  \(x_3 = \{\text{washbag}\}\);
\item
  \(x_4 = \{\text{computer}\}\);
\item
  \(x_5 = \{\text{boots}\}\);
\item
  \(x_6 = \{\text{alarmclock}\}\);
\item
  \(x_7 = \{\text{anorak}\}\);
\item
  \(x_8 = \{\text{food}\}\).
\end{itemize}

Также поставим условия \[x_i \in \{0, 1\},\] которые означают, что,
поскольку у нас по одной единице каждой из вещей, то вещь мы можем либо
взять, либо не взять.

Каждая вещь обладает своим весом, причем унести мы можем не больше
\(20\) кг. Это позволяет сформулировать ограничения:

\[2x_1 + 4x_2 + 3x_3 + 8 x_4 + 7x_5 + x_6 + 2x_7 + 6x_8 \leq 20.\]

    \section{Первое
подзадание}\label{ux43fux435ux440ux432ux43eux435-ux43fux43eux434ux437ux430ux434ux430ux43dux438ux435}

    \subsection{Условие}\label{ux443ux441ux43bux43eux432ux438ux435}

    Эти предметы имеют для вас определенную ценность:

    \begin{tcolorbox}[breakable, size=fbox, boxrule=1pt, pad at break*=1mm,colback=cellbackground, colframe=cellborder]
\prompt{In}{incolor}{3}{\boxspacing}
\begin{Verbatim}[commandchars=\\\{\}]
\PY{n}{value} \PY{o}{=} \PY{p}{[}\PY{l+m+mi}{6}\PY{p}{,} \PY{l+m+mi}{10}\PY{p}{,} \PY{l+m+mi}{8}\PY{p}{,} \PY{l+m+mi}{25}\PY{p}{,} \PY{l+m+mi}{22}\PY{p}{,} \PY{l+m+mi}{4}\PY{p}{,} \PY{l+m+mi}{5}\PY{p}{,} \PY{l+m+mi}{20}\PY{p}{]}
\end{Verbatim}
\end{tcolorbox}

    Упакуйте предметы, которые вы можете носить в своей сумке, которые
принесут вам максимально возможную общую ценность.

    Нам нужно максимизировать ценность, соответственно это позволяет нам
сформулировать целевую функцию для задачи ЦЛП:
\[6x_1 + 10x_2 + 8 x_3 + 25 x_4 + 22 x_5 + 4 x_6 + 5 x_7 + 20 x_8 \to \max.\]

В итоге, собрав все введенные формулы, мы имеем задачу ЦЛП вида:
\[6x_1 + 10x_2 + 8 x_3 + 25 x_4 + 22 x_5 + 4 x_6 + 5 x_7 + 20 x_8 \to \max,\]
\[2x_1 + 4x_2 + 3x_3 + 8 x_4 + 7x_5 + x_6 + 2x_7 + 6x_8 \leq 20,\]
\[x_i \in \{0,1\}.\]

Далее мы можем найти ее решение, применяя библиотеку OR-Tools.

    \subsection{Решение в
OR-Tools}\label{ux440ux435ux448ux435ux43dux438ux435-ux432-or-tools}

    Составим функцию, применяя библиотеку OR-Tools, которая будет находить
максимальное значение для задачи целочисленного программирования

    \begin{tcolorbox}[breakable, size=fbox, boxrule=1pt, pad at break*=1mm,colback=cellbackground, colframe=cellborder]
\prompt{In}{incolor}{4}{\boxspacing}
\begin{Verbatim}[commandchars=\\\{\}]
\PY{k}{def} \PY{n+nf}{maximize\PYZus{}value}\PY{p}{(}\PY{n}{items}\PY{p}{,} \PY{n}{weight}\PY{p}{,} \PY{n}{value}\PY{p}{)}\PY{p}{:}
    \PY{c+c1}{\PYZsh{} Создание модели задачи}
    \PY{n}{model} \PY{o}{=} \PY{n}{cp\PYZus{}model}\PY{o}{.}\PY{n}{CpModel}\PY{p}{(}\PY{p}{)}
    
    \PY{c+c1}{\PYZsh{} Создаем булевы переменные}
    \PY{n}{x} \PY{o}{=} \PY{p}{[}\PY{n}{model}\PY{o}{.}\PY{n}{NewBoolVar}\PY{p}{(}\PY{l+s+sa}{f}\PY{l+s+s1}{\PYZsq{}}\PY{l+s+s1}{x[}\PY{l+s+si}{\PYZob{}}\PY{n}{i}\PY{l+s+si}{\PYZcb{}}\PY{l+s+s1}{]}\PY{l+s+s1}{\PYZsq{}}\PY{p}{)} \PY{k}{for} \PY{n}{i} \PY{o+ow}{in} \PY{n+nb}{range}\PY{p}{(}\PY{n+nb}{len}\PY{p}{(}\PY{n}{items}\PY{p}{)}\PY{p}{)}\PY{p}{]}

    \PY{c+c1}{\PYZsh{} Ограничения на вес}
    \PY{n}{model}\PY{o}{.}\PY{n}{Add}\PY{p}{(}\PY{n+nb}{sum}\PY{p}{(}\PY{n}{weight}\PY{p}{[}\PY{n}{i}\PY{p}{]} \PY{o}{*} \PY{n}{x}\PY{p}{[}\PY{n}{i}\PY{p}{]} \PY{k}{for} \PY{n}{i} \PY{o+ow}{in} \PY{n+nb}{range}\PY{p}{(}\PY{n+nb}{len}\PY{p}{(}\PY{n}{items}\PY{p}{)}\PY{p}{)}\PY{p}{)} \PY{o}{\PYZlt{}}\PY{o}{=} \PY{l+m+mi}{20}\PY{p}{)}

    \PY{c+c1}{\PYZsh{} Целевая функция}
    \PY{n}{objective} \PY{o}{=} \PY{n+nb}{sum}\PY{p}{(}\PY{n}{value}\PY{p}{[}\PY{n}{i}\PY{p}{]} \PY{o}{*} \PY{n}{x}\PY{p}{[}\PY{n}{i}\PY{p}{]} \PY{k}{for} \PY{n}{i} \PY{o+ow}{in} \PY{n+nb}{range}\PY{p}{(}\PY{n+nb}{len}\PY{p}{(}\PY{n}{items}\PY{p}{)}\PY{p}{)}\PY{p}{)}
    \PY{n}{model}\PY{o}{.}\PY{n}{Maximize}\PY{p}{(}\PY{n}{objective}\PY{p}{)}
    
    \PY{c+c1}{\PYZsh{} Отыскание решения для модели солвером}
    \PY{n}{solver} \PY{o}{=} \PY{n}{cp\PYZus{}model}\PY{o}{.}\PY{n}{CpSolver}\PY{p}{(}\PY{p}{)}
    \PY{n}{status} \PY{o}{=} \PY{n}{solver}\PY{o}{.}\PY{n}{Solve}\PY{p}{(}\PY{n}{model}\PY{p}{)}

    \PY{k}{if} \PY{n}{status} \PY{o}{==} \PY{n}{cp\PYZus{}model}\PY{o}{.}\PY{n}{OPTIMAL}\PY{p}{:}
        \PY{n}{selected\PYZus{}items} \PY{o}{=} \PY{p}{[}\PY{n}{items}\PY{p}{[}\PY{n}{i}\PY{p}{]} \PY{k}{for} \PY{n}{i} \PY{o+ow}{in} \PY{n+nb}{range}\PY{p}{(}\PY{n+nb}{len}\PY{p}{(}\PY{n}{items}\PY{p}{)}\PY{p}{)} \PY{k}{if} \PY{n}{solver}\PY{o}{.}\PY{n}{Value}\PY{p}{(}\PY{n}{x}\PY{p}{[}\PY{n}{i}\PY{p}{]}\PY{p}{)} \PY{o}{==} \PY{l+m+mi}{1}\PY{p}{]}
        \PY{n}{total\PYZus{}value} \PY{o}{=} \PY{n}{solver}\PY{o}{.}\PY{n}{ObjectiveValue}\PY{p}{(}\PY{p}{)}

        \PY{k}{return} \PY{n}{selected\PYZus{}items}\PY{p}{,} \PY{n}{total\PYZus{}value}

    \PY{k}{return} \PY{p}{[}\PY{p}{]}\PY{p}{,} \PY{l+m+mi}{0}
\end{Verbatim}
\end{tcolorbox}

    Подставим в данную функцию известные нам значения и получим решение
задачи:

    \begin{tcolorbox}[breakable, size=fbox, boxrule=1pt, pad at break*=1mm,colback=cellbackground, colframe=cellborder]
\prompt{In}{incolor}{5}{\boxspacing}
\begin{Verbatim}[commandchars=\\\{\}]
\PY{n}{selected\PYZus{}items}\PY{p}{,} \PY{n}{total\PYZus{}value} \PY{o}{=} \PY{n}{maximize\PYZus{}value}\PY{p}{(}\PY{n}{items}\PY{p}{,} \PY{n}{weight}\PY{p}{,} \PY{n}{value}\PY{p}{)}
\PY{n+nb}{print}\PY{p}{(}\PY{l+s+s2}{\PYZdq{}}\PY{l+s+s2}{Selected items:}\PY{l+s+s2}{\PYZdq{}}\PY{p}{,} \PY{n}{selected\PYZus{}items}\PY{p}{)}
\PY{n+nb}{print}\PY{p}{(}\PY{l+s+s2}{\PYZdq{}}\PY{l+s+s2}{Total value:}\PY{l+s+s2}{\PYZdq{}}\PY{p}{,} \PY{n}{total\PYZus{}value}\PY{p}{)}
\end{Verbatim}
\end{tcolorbox}

    \begin{Verbatim}[commandchars=\\\{\}]
Selected items: ['book', 'washbag', 'computer', 'alarmclock', 'food']
Total value: 63.0
    \end{Verbatim}

    \section{Второе
подзадание}\label{ux432ux442ux43eux440ux43eux435-ux43fux43eux434ux437ux430ux434ux430ux43dux438ux435}

    Рюкзак также имеет ограниченную вместимость, а общий объем предметов,
которые он может поместить внутрь, составляет 2000 см2. Каждый предмет
имеет не только вес, но и объем:

    \begin{tcolorbox}[breakable, size=fbox, boxrule=1pt, pad at break*=1mm,colback=cellbackground, colframe=cellborder]
\prompt{In}{incolor}{6}{\boxspacing}
\begin{Verbatim}[commandchars=\\\{\}]
\PY{n}{volume} \PY{o}{=} \PY{p}{[}\PY{l+m+mi}{250}\PY{p}{,} \PY{l+m+mi}{500}\PY{p}{,} \PY{l+m+mi}{300}\PY{p}{,} \PY{l+m+mi}{250}\PY{p}{,} \PY{l+m+mi}{650}\PY{p}{,} \PY{l+m+mi}{130}\PY{p}{,} \PY{l+m+mi}{150}\PY{p}{,} \PY{l+m+mi}{600}\PY{p}{]} 
\end{Verbatim}
\end{tcolorbox}

    Найдите лучшее решение, как и в первой подзадаче, но общий объем
предметов в рюкзаке не может быть больше вместимости рюкзака.

    Данное условие позволяет нам сфорулировать еще одно ограничение на
поставленную задачу:

\[250x_1 + 500x_2 + 300x_3 + 250x_4 + 650x_5 + 130x_6 + 150x_7 + 600 x_8 \leq 2000.\]

Таким образом, задача ЦЛП переформулируется и будет иметь следующий вид:
\[6x_1 + 10x_2 + 8 x_3 + 25 x_4 + 22 x_5 + 4 x_6 + 5 x_7 + 20 x_8 \to \max,\]
\[2x_1 + 4x_2 + 3x_3 + 8 x_4 + 7x_5 + x_6 + 2x_7 + 6x_8 \leq 20,\]
\[250x_1 + 500x_2 + 300x_3 + 250x_4 + 650x_5 + 130x_6 + 150x_7 + 600 x_8 \leq 2000,\]
\[x_i \in \{0,1\}.\]

    \subsection{Решение в
OR-Tools}\label{ux440ux435ux448ux435ux43dux438ux435-ux432-or-tools}

    Скопируем функцию из предыдущего подзадания, но добавим в нее новые
ограничения:

    \begin{tcolorbox}[breakable, size=fbox, boxrule=1pt, pad at break*=1mm,colback=cellbackground, colframe=cellborder]
\prompt{In}{incolor}{7}{\boxspacing}
\begin{Verbatim}[commandchars=\\\{\}]
\PY{k}{def} \PY{n+nf}{maximize\PYZus{}value}\PY{p}{(}\PY{n}{items}\PY{p}{,} \PY{n}{weight}\PY{p}{,} \PY{n}{value}\PY{p}{,} \PY{n}{volume}\PY{p}{)}\PY{p}{:}
    \PY{c+c1}{\PYZsh{} Создание модели задачи}
    \PY{n}{model} \PY{o}{=} \PY{n}{cp\PYZus{}model}\PY{o}{.}\PY{n}{CpModel}\PY{p}{(}\PY{p}{)}
    
    \PY{c+c1}{\PYZsh{} Создаем булевы переменные}
    \PY{n}{x} \PY{o}{=} \PY{p}{[}\PY{n}{model}\PY{o}{.}\PY{n}{NewBoolVar}\PY{p}{(}\PY{l+s+sa}{f}\PY{l+s+s1}{\PYZsq{}}\PY{l+s+s1}{x[}\PY{l+s+si}{\PYZob{}}\PY{n}{i}\PY{l+s+si}{\PYZcb{}}\PY{l+s+s1}{]}\PY{l+s+s1}{\PYZsq{}}\PY{p}{)} \PY{k}{for} \PY{n}{i} \PY{o+ow}{in} \PY{n+nb}{range}\PY{p}{(}\PY{n+nb}{len}\PY{p}{(}\PY{n}{items}\PY{p}{)}\PY{p}{)}\PY{p}{]}

    \PY{c+c1}{\PYZsh{} Ограничения на вес}
    \PY{n}{model}\PY{o}{.}\PY{n}{Add}\PY{p}{(}\PY{n+nb}{sum}\PY{p}{(}\PY{n}{weight}\PY{p}{[}\PY{n}{i}\PY{p}{]} \PY{o}{*} \PY{n}{x}\PY{p}{[}\PY{n}{i}\PY{p}{]} \PY{k}{for} \PY{n}{i} \PY{o+ow}{in} \PY{n+nb}{range}\PY{p}{(}\PY{n+nb}{len}\PY{p}{(}\PY{n}{items}\PY{p}{)}\PY{p}{)}\PY{p}{)} \PY{o}{\PYZlt{}}\PY{o}{=} \PY{l+m+mi}{20}\PY{p}{)}
    
    \PY{c+c1}{\PYZsh{} Ограничения на объем}
    \PY{n}{model}\PY{o}{.}\PY{n}{Add}\PY{p}{(}\PY{n+nb}{sum}\PY{p}{(}\PY{n}{volume}\PY{p}{[}\PY{n}{i}\PY{p}{]} \PY{o}{*} \PY{n}{x}\PY{p}{[}\PY{n}{i}\PY{p}{]} \PY{k}{for} \PY{n}{i} \PY{o+ow}{in} \PY{n+nb}{range}\PY{p}{(}\PY{n+nb}{len}\PY{p}{(}\PY{n}{items}\PY{p}{)}\PY{p}{)}\PY{p}{)} \PY{o}{\PYZlt{}}\PY{o}{=} \PY{l+m+mi}{2000}\PY{p}{)}

    \PY{c+c1}{\PYZsh{} Целевая функция}
    \PY{n}{objective} \PY{o}{=} \PY{n+nb}{sum}\PY{p}{(}\PY{n}{value}\PY{p}{[}\PY{n}{i}\PY{p}{]} \PY{o}{*} \PY{n}{x}\PY{p}{[}\PY{n}{i}\PY{p}{]} \PY{k}{for} \PY{n}{i} \PY{o+ow}{in} \PY{n+nb}{range}\PY{p}{(}\PY{n+nb}{len}\PY{p}{(}\PY{n}{items}\PY{p}{)}\PY{p}{)}\PY{p}{)}
    \PY{n}{model}\PY{o}{.}\PY{n}{Maximize}\PY{p}{(}\PY{n}{objective}\PY{p}{)}
    
    \PY{c+c1}{\PYZsh{} Отыскание решения для модели солвером}
    \PY{n}{solver} \PY{o}{=} \PY{n}{cp\PYZus{}model}\PY{o}{.}\PY{n}{CpSolver}\PY{p}{(}\PY{p}{)}
    \PY{n}{status} \PY{o}{=} \PY{n}{solver}\PY{o}{.}\PY{n}{Solve}\PY{p}{(}\PY{n}{model}\PY{p}{)}

    \PY{k}{if} \PY{n}{status} \PY{o}{==} \PY{n}{cp\PYZus{}model}\PY{o}{.}\PY{n}{OPTIMAL}\PY{p}{:}
        \PY{n}{selected\PYZus{}items} \PY{o}{=} \PY{p}{[}\PY{n}{items}\PY{p}{[}\PY{n}{i}\PY{p}{]} \PY{k}{for} \PY{n}{i} \PY{o+ow}{in} \PY{n+nb}{range}\PY{p}{(}\PY{n+nb}{len}\PY{p}{(}\PY{n}{items}\PY{p}{)}\PY{p}{)} \PY{k}{if} \PY{n}{solver}\PY{o}{.}\PY{n}{Value}\PY{p}{(}\PY{n}{x}\PY{p}{[}\PY{n}{i}\PY{p}{]}\PY{p}{)} \PY{o}{==} \PY{l+m+mi}{1}\PY{p}{]}
        \PY{n}{total\PYZus{}value} \PY{o}{=} \PY{n}{solver}\PY{o}{.}\PY{n}{ObjectiveValue}\PY{p}{(}\PY{p}{)}

        \PY{k}{return} \PY{n}{selected\PYZus{}items}\PY{p}{,} \PY{n}{total\PYZus{}value}

    \PY{k}{return} \PY{p}{[}\PY{p}{]}\PY{p}{,} \PY{l+m+mi}{0}
\end{Verbatim}
\end{tcolorbox}

    \begin{tcolorbox}[breakable, size=fbox, boxrule=1pt, pad at break*=1mm,colback=cellbackground, colframe=cellborder]
\prompt{In}{incolor}{8}{\boxspacing}
\begin{Verbatim}[commandchars=\\\{\}]
\PY{n}{selected\PYZus{}items}\PY{p}{,} \PY{n}{total\PYZus{}value} \PY{o}{=} \PY{n}{maximize\PYZus{}value}\PY{p}{(}\PY{n}{items}\PY{p}{,} \PY{n}{weight}\PY{p}{,} \PY{n}{value}\PY{p}{,} \PY{n}{volume}\PY{p}{)}
\PY{n+nb}{print}\PY{p}{(}\PY{l+s+s2}{\PYZdq{}}\PY{l+s+s2}{Selected items:}\PY{l+s+s2}{\PYZdq{}}\PY{p}{,} \PY{n}{selected\PYZus{}items}\PY{p}{)}
\PY{n+nb}{print}\PY{p}{(}\PY{l+s+s2}{\PYZdq{}}\PY{l+s+s2}{Total value:}\PY{l+s+s2}{\PYZdq{}}\PY{p}{,} \PY{n}{total\PYZus{}value}\PY{p}{)}
\end{Verbatim}
\end{tcolorbox}

    \begin{Verbatim}[commandchars=\\\{\}]
Selected items: ['book', 'washbag', 'computer', 'alarmclock', 'food']
Total value: 63.0
    \end{Verbatim}

    Таким образом, введение дополнительных условий не повлияло на решение
задачи ЦЛП.

    \section{Третье
подзадание}\label{ux442ux440ux435ux442ux44cux435-ux43fux43eux434ux437ux430ux434ux430ux43dux438ux435}

    \subsection{Условие}\label{ux443ux441ux43bux43eux432ux438ux435}

    Некоторые вещи в сочетании стоят больше, а некоторые меньше. Вот
дополнительный (или уменьшенный) балл, который вы получаете за каждую
пару:

    \begin{tcolorbox}[breakable, size=fbox, boxrule=1pt, pad at break*=1mm,colback=cellbackground, colframe=cellborder]
\prompt{In}{incolor}{9}{\boxspacing}
\begin{Verbatim}[commandchars=\\\{\}]
\PY{n}{extra\PYZus{}value} \PY{o}{=} \PY{p}{[}    \PY{p}{[}\PY{l+m+mi}{0}\PY{p}{,} \PY{l+m+mi}{0}\PY{p}{,} \PY{l+m+mi}{0}\PY{p}{,} \PY{o}{\PYZhy{}}\PY{l+m+mi}{5}\PY{p}{,} \PY{l+m+mi}{0}\PY{p}{,} \PY{l+m+mi}{0}\PY{p}{,} \PY{l+m+mi}{0}\PY{p}{,} \PY{l+m+mi}{0}\PY{p}{]}\PY{p}{,}
                   \PY{p}{[}\PY{l+m+mi}{0}\PY{p}{,} \PY{l+m+mi}{0}\PY{p}{,} \PY{l+m+mi}{0}\PY{p}{,} \PY{l+m+mi}{0}\PY{p}{,} \PY{l+m+mi}{3}\PY{p}{,} \PY{l+m+mi}{0}\PY{p}{,} \PY{o}{\PYZhy{}}\PY{l+m+mi}{2}\PY{p}{,} \PY{l+m+mi}{0}\PY{p}{]}\PY{p}{,}
                   \PY{p}{[}\PY{l+m+mi}{0}\PY{p}{,} \PY{l+m+mi}{0}\PY{p}{,} \PY{l+m+mi}{0}\PY{p}{,} \PY{l+m+mi}{0}\PY{p}{,} \PY{l+m+mi}{0}\PY{p}{,} \PY{l+m+mi}{0}\PY{p}{,} \PY{l+m+mi}{0}\PY{p}{,} \PY{l+m+mi}{0}\PY{p}{]}\PY{p}{,}
                   \PY{p}{[}\PY{o}{\PYZhy{}}\PY{l+m+mi}{5}\PY{p}{,} \PY{l+m+mi}{0}\PY{p}{,} \PY{l+m+mi}{0}\PY{p}{,} \PY{l+m+mi}{0}\PY{p}{,} \PY{l+m+mi}{0}\PY{p}{,} \PY{o}{\PYZhy{}}\PY{l+m+mi}{2}\PY{p}{,} \PY{l+m+mi}{0}\PY{p}{,} \PY{l+m+mi}{0}\PY{p}{]}\PY{p}{,}
                   \PY{p}{[}\PY{l+m+mi}{0}\PY{p}{,} \PY{l+m+mi}{3}\PY{p}{,} \PY{l+m+mi}{0}\PY{p}{,} \PY{l+m+mi}{0}\PY{p}{,} \PY{l+m+mi}{0}\PY{p}{,} \PY{l+m+mi}{0}\PY{p}{,} \PY{l+m+mi}{0}\PY{p}{,} \PY{l+m+mi}{0}\PY{p}{]}\PY{p}{,}
                   \PY{p}{[}\PY{l+m+mi}{0}\PY{p}{,} \PY{l+m+mi}{0}\PY{p}{,} \PY{l+m+mi}{0}\PY{p}{,} \PY{o}{\PYZhy{}}\PY{l+m+mi}{2}\PY{p}{,} \PY{l+m+mi}{0}\PY{p}{,} \PY{l+m+mi}{0}\PY{p}{,} \PY{l+m+mi}{0}\PY{p}{,} \PY{l+m+mi}{0}\PY{p}{]}\PY{p}{,}
                   \PY{p}{[}\PY{l+m+mi}{0}\PY{p}{,} \PY{o}{\PYZhy{}}\PY{l+m+mi}{2}\PY{p}{,} \PY{l+m+mi}{0}\PY{p}{,} \PY{l+m+mi}{0}\PY{p}{,} \PY{l+m+mi}{0}\PY{p}{,} \PY{l+m+mi}{0}\PY{p}{,} \PY{l+m+mi}{0}\PY{p}{,} \PY{l+m+mi}{0}\PY{p}{]}\PY{p}{,}
                   \PY{p}{[}\PY{l+m+mi}{0}\PY{p}{,} \PY{l+m+mi}{0}\PY{p}{,} \PY{l+m+mi}{0}\PY{p}{,} \PY{l+m+mi}{0}\PY{p}{,} \PY{l+m+mi}{0}\PY{p}{,} \PY{l+m+mi}{0}\PY{p}{,} \PY{l+m+mi}{0}\PY{p}{,} \PY{l+m+mi}{0}\PY{p}{]}    \PY{p}{]}
\end{Verbatim}
\end{tcolorbox}

    Если extra\_value{[}i1, i2{]} = 3 и предметы i1 и i2 находятся в вашей
сумке, то общая стоимость вашей сумки увеличивается на 3. Естественно,
если extra\_value{[}i1, i2{]} = -2, то она уменьшается на 2. Расширьте
свою модель для первой подзадачи, чтобы максимизировать общую сумму с
измененными значениями.

    Эти дополнительные условия позволяют переформулировать целевую функцию.
Но для этого введем некоторые обозначения:

\begin{itemize}
\item
  \(x = (x_1,\ldots, x_8)^T\) -- вектор всех предметов,
  \(x_i \in \{0, 1\}\);
\item
  \(\mathcal V_{extra}\) -- матрица дополнительного балла за каждую пару
  данная по условию;
\item
  \(\omega\) -- вектор весов данный в общем условии;
\item
  \(v\) -- вектор ценности данный в первом подзадании;
\item
  \(\nu \) -- вектор ценности данный во втором подзадании.
\end{itemize}

В итоге задача ЦЛП станет иметь вид
\[v^T x + \mathcal V_{extra} (x^Tx) \to \max,\] \[\omega^T x \leq 20,\]
\[\nu  ^T x \leq 2000,\] \[x_i \in \{0,1\}.\]

    \subsection{Решение в
OR-Tools}\label{ux440ux435ux448ux435ux43dux438ux435-ux432-or-tools}

    Возьмем функцию из второго подзадания и добавим к целевой функции новое
условие:

    \begin{tcolorbox}[breakable, size=fbox, boxrule=1pt, pad at break*=1mm,colback=cellbackground, colframe=cellborder]
\prompt{In}{incolor}{10}{\boxspacing}
\begin{Verbatim}[commandchars=\\\{\}]
\PY{k}{def} \PY{n+nf}{maximize\PYZus{}value}\PY{p}{(}\PY{n}{items}\PY{p}{,} \PY{n}{weight}\PY{p}{,} \PY{n}{value}\PY{p}{,} \PY{n}{volume}\PY{p}{,} \PY{n}{extra\PYZus{}value}\PY{p}{)}\PY{p}{:}
    \PY{c+c1}{\PYZsh{} Создание модели задачи}
    \PY{n}{model} \PY{o}{=} \PY{n}{cp\PYZus{}model}\PY{o}{.}\PY{n}{CpModel}\PY{p}{(}\PY{p}{)}
    
    \PY{c+c1}{\PYZsh{} Создаем булевы переменные}
    \PY{n}{x} \PY{o}{=} \PY{p}{[}\PY{n}{model}\PY{o}{.}\PY{n}{NewBoolVar}\PY{p}{(}\PY{l+s+sa}{f}\PY{l+s+s1}{\PYZsq{}}\PY{l+s+s1}{x[}\PY{l+s+si}{\PYZob{}}\PY{n}{i}\PY{l+s+si}{\PYZcb{}}\PY{l+s+s1}{]}\PY{l+s+s1}{\PYZsq{}}\PY{p}{)} \PY{k}{for} \PY{n}{i} \PY{o+ow}{in} \PY{n+nb}{range}\PY{p}{(}\PY{n+nb}{len}\PY{p}{(}\PY{n}{items}\PY{p}{)}\PY{p}{)}\PY{p}{]}

    \PY{c+c1}{\PYZsh{} Ограничения на вес}
    \PY{n}{model}\PY{o}{.}\PY{n}{Add}\PY{p}{(}\PY{n+nb}{sum}\PY{p}{(}\PY{n}{weight}\PY{p}{[}\PY{n}{i}\PY{p}{]} \PY{o}{*} \PY{n}{x}\PY{p}{[}\PY{n}{i}\PY{p}{]} \PY{k}{for} \PY{n}{i} \PY{o+ow}{in} \PY{n+nb}{range}\PY{p}{(}\PY{n+nb}{len}\PY{p}{(}\PY{n}{items}\PY{p}{)}\PY{p}{)}\PY{p}{)} \PY{o}{\PYZlt{}}\PY{o}{=} \PY{l+m+mi}{20}\PY{p}{)}
    
    \PY{c+c1}{\PYZsh{} Ограничения на объем}
    \PY{n}{model}\PY{o}{.}\PY{n}{Add}\PY{p}{(}\PY{n+nb}{sum}\PY{p}{(}\PY{n}{volume}\PY{p}{[}\PY{n}{i}\PY{p}{]} \PY{o}{*} \PY{n}{x}\PY{p}{[}\PY{n}{i}\PY{p}{]} \PY{k}{for} \PY{n}{i} \PY{o+ow}{in} \PY{n+nb}{range}\PY{p}{(}\PY{n+nb}{len}\PY{p}{(}\PY{n}{items}\PY{p}{)}\PY{p}{)}\PY{p}{)} \PY{o}{\PYZlt{}}\PY{o}{=} \PY{l+m+mi}{2000}\PY{p}{)}

    \PY{c+c1}{\PYZsh{} Целевая функция}
    \PY{n}{objective} \PY{o}{=} \PY{n+nb}{sum}\PY{p}{(}\PY{n}{value}\PY{p}{[}\PY{n}{i}\PY{p}{]} \PY{o}{*} \PY{n}{x}\PY{p}{[}\PY{n}{i}\PY{p}{]} \PY{k}{for} \PY{n}{i} \PY{o+ow}{in} \PY{n+nb}{range}\PY{p}{(}\PY{n+nb}{len}\PY{p}{(}\PY{n}{items}\PY{p}{)}\PY{p}{)}\PY{p}{)}
    \PY{k}{for} \PY{n}{i1} \PY{o+ow}{in} \PY{n+nb}{range}\PY{p}{(}\PY{n+nb}{len}\PY{p}{(}\PY{n}{items}\PY{p}{)}\PY{p}{)}\PY{p}{:}
        \PY{k}{for} \PY{n}{i2} \PY{o+ow}{in} \PY{n+nb}{range}\PY{p}{(}\PY{n+nb}{len}\PY{p}{(}\PY{n}{items}\PY{p}{)}\PY{p}{)}\PY{p}{:}
            \PY{n}{var} \PY{o}{=} \PY{n}{model}\PY{o}{.}\PY{n}{NewBoolVar}\PY{p}{(}\PY{l+s+s1}{\PYZsq{}}\PY{l+s+s1}{\PYZsq{}}\PY{p}{)}
            \PY{n}{model}\PY{o}{.}\PY{n}{AddBoolAnd}\PY{p}{(}\PY{p}{[}\PY{n}{x}\PY{p}{[}\PY{n}{i1}\PY{p}{]}\PY{p}{,} \PY{n}{x}\PY{p}{[}\PY{n}{i2}\PY{p}{]}\PY{p}{]}\PY{p}{)}\PY{o}{.}\PY{n}{OnlyEnforceIf}\PY{p}{(}\PY{n}{var}\PY{p}{)}
            \PY{n}{objective} \PY{o}{+}\PY{o}{=} \PY{n}{extra\PYZus{}value}\PY{p}{[}\PY{n}{i1}\PY{p}{]}\PY{p}{[}\PY{n}{i2}\PY{p}{]} \PY{o}{*} \PY{n}{var}
    \PY{n}{model}\PY{o}{.}\PY{n}{Maximize}\PY{p}{(}\PY{n}{objective}\PY{p}{)}
    
    \PY{c+c1}{\PYZsh{} Отыскание решения для модели солвером}
    \PY{n}{solver} \PY{o}{=} \PY{n}{cp\PYZus{}model}\PY{o}{.}\PY{n}{CpSolver}\PY{p}{(}\PY{p}{)}
    \PY{n}{status} \PY{o}{=} \PY{n}{solver}\PY{o}{.}\PY{n}{Solve}\PY{p}{(}\PY{n}{model}\PY{p}{)}

    \PY{k}{if} \PY{n}{status} \PY{o}{==} \PY{n}{cp\PYZus{}model}\PY{o}{.}\PY{n}{OPTIMAL}\PY{p}{:}
        \PY{n}{selected\PYZus{}items} \PY{o}{=} \PY{p}{[}\PY{n}{items}\PY{p}{[}\PY{n}{i}\PY{p}{]} \PY{k}{for} \PY{n}{i} \PY{o+ow}{in} \PY{n+nb}{range}\PY{p}{(}\PY{n+nb}{len}\PY{p}{(}\PY{n}{items}\PY{p}{)}\PY{p}{)} \PY{k}{if} \PY{n}{solver}\PY{o}{.}\PY{n}{Value}\PY{p}{(}\PY{n}{x}\PY{p}{[}\PY{n}{i}\PY{p}{]}\PY{p}{)} \PY{o}{==} \PY{l+m+mi}{1}\PY{p}{]}
        \PY{n}{total\PYZus{}value} \PY{o}{=} \PY{n}{solver}\PY{o}{.}\PY{n}{ObjectiveValue}\PY{p}{(}\PY{p}{)}

        \PY{k}{return} \PY{n}{selected\PYZus{}items}\PY{p}{,} \PY{n}{total\PYZus{}value}

    \PY{k}{return} \PY{p}{[}\PY{p}{]}\PY{p}{,} \PY{l+m+mi}{0}
\end{Verbatim}
\end{tcolorbox}

    \begin{tcolorbox}[breakable, size=fbox, boxrule=1pt, pad at break*=1mm,colback=cellbackground, colframe=cellborder]
\prompt{In}{incolor}{11}{\boxspacing}
\begin{Verbatim}[commandchars=\\\{\}]
\PY{n}{selected\PYZus{}items}\PY{p}{,} \PY{n}{total\PYZus{}value} \PY{o}{=} \PY{n}{maximize\PYZus{}value}\PY{p}{(}\PY{n}{items}\PY{p}{,} \PY{n}{weight}\PY{p}{,} \PY{n}{value}\PY{p}{,} \PY{n}{volume}\PY{p}{,} \PY{n}{extra\PYZus{}value}\PY{p}{)}
\PY{n+nb}{print}\PY{p}{(}\PY{l+s+s2}{\PYZdq{}}\PY{l+s+s2}{Selected items:}\PY{l+s+s2}{\PYZdq{}}\PY{p}{,} \PY{n}{selected\PYZus{}items}\PY{p}{)}
\PY{n+nb}{print}\PY{p}{(}\PY{l+s+s2}{\PYZdq{}}\PY{l+s+s2}{Total value:}\PY{l+s+s2}{\PYZdq{}}\PY{p}{,} \PY{n}{total\PYZus{}value}\PY{p}{)}
\end{Verbatim}
\end{tcolorbox}

    \begin{Verbatim}[commandchars=\\\{\}]
Selected items: ['jacket', 'computer', 'boots', 'alarmclock']
Total value: 67.0
    \end{Verbatim}

    Таким образом, результат изменился по сравнению с предыдущем
подзаданием.

В итоге, собрав все условия в единую задачу, мы получили решение:
\[\max\Big(v^T x + \mathcal V_{extra} (x^Tx) \Big) = 67,\quad x = (0,1,0,1,1,1,0,0)^T.\]


    % Add a bibliography block to the postdoc
    
    
    
\end{document}
