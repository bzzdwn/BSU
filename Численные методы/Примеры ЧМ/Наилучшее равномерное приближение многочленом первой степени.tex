\documentclass[a4paper, 12pt]{article}
\usepackage{cmap}
\usepackage{amssymb}
\usepackage{amsmath}
\usepackage{graphicx}
\usepackage{amsthm}
\usepackage{upgreek}
\usepackage{setspace}
\usepackage{color}
\usepackage{pgfplots}
\pgfplotsset{compat=1.9}
\usepackage[T2A]{fontenc}
\usepackage[utf8]{inputenc}
\usepackage[normalem]{ulem}
\usepackage{mathtext} % русские буквы в формулах
\usepackage[left=2cm,right=2cm, top=2cm,bottom=2cm,bindingoffset=0cm]{geometry}
\usepackage[english,russian]{babel}
\usepackage[unicode]{hyperref}
\newenvironment{Proof} % имя окружения
{\par\noindent{$\blacklozenge$}} % команды для \begin
{\hfill$\scriptstyle\boxtimes$}
\newcommand{\Rm}{\mathbb{R}}
\newcommand{\Cm}{\mathbb{C}}
\newcommand{\Z}{\mathbb{Z}}
\newcommand{\I}{\mathbb{I}}
\newcommand{\N}{\mathbb{N}}
\newcommand{\rank}{\operatorname{rank}}
\newcommand{\Ra}{\Rightarrow}
\newcommand{\ra}{\rightarrow}
\newcommand{\FI}{\Phi}
\newcommand{\Sp}{\text{Sp}}
\renewcommand{\leq}{\leqslant}
\renewcommand{\geq}{\geqslant}
\renewcommand{\alpha}{\upalpha}
\renewcommand{\beta}{\upbeta}
\renewcommand{\gamma}{\upgamma}
\renewcommand{\delta}{\updelta}
\renewcommand{\varphi}{\upvarphi}
\renewcommand{\phi}{\upvarphi}
\renewcommand{\tau}{\uptau}
\renewcommand{\lambda}{\uplambda}
\renewcommand{\psi}{\uppsi}
\renewcommand{\mu}{\upmu}
\renewcommand{\omega}{\upomega}
\renewcommand{\d}{\partial}
\renewcommand{\xi}{\upxi}
\renewcommand{\epsilon}{\upvarepsilon}
\newcommand{\intx}{\int\limits_{x_0}^x}
\newcommand\Norm[1]{\left\| #1 \right\|}
\newcommand{\sumk}{\sum\limits_{k=0}^\infty}
\newcommand{\sumi}{\sum\limits_{i=0}^\infty}
\newtheorem*{theorem}{Теорема}
\newtheorem*{cor}{Следствие}
\newtheorem*{lem}{Лемма}
\begin{document}
	\section*{Наилучшее равномерное приближение многочленом первой степени}
	\subsubsection*{Условие}
	Построить наилучшее равномерное приближение функции $f(x) = 2^x$, $x\in [-1,1]$ с помощью многочлена первой степени. Найти наилучшее приближение.
	\subsubsection*{Алгоритм решения}
	Все последующие действия справедливы лишь при предположении, что исходная функция выпуклая (по свойствам степенной функции).\\\\
	Для построения наилучшего равномерного приближения многочленом первой степени понадобятся следующие формулы:
	\begin{enumerate}
		\item многочлен наилучшего равномерного приближения в общем виде \begin{eqnarray}
		P_1(x) = c_0 + c_1x
		\end{eqnarray}
		\item необходимое и достаточное условие существования и единственности многочлена \begin{eqnarray}
			f(x_i) - P_n(x_i) = (-1)^i\alpha\Delta,\quad \Delta = \Norm{f(x) - P_n(x)},\quad i = 0,\ldots,n+1,
		\end{eqnarray}
		где $\alpha=1$ или $\alpha = -1$, а $x_i$ --- точки чебышевского альтернанса.
	\end{enumerate}
	Также необходимо определить точки чебышевского альтернанса $x_0,x_1,x_2$ (точки, в которых задана исходная функция, но которые находятся дальше всего от приближающего многочлена). Две из них (первую и последнюю) мы можем задать на концах:
	$$\begin{cases}
		x_0 = -1,\\
		x_2 = 1.
	\end{cases}$$
	Для оставшейся точки мы сформулируем условие следующим образом. Вследствие выпуклости функция $f(x) - P_n(x)$ может иметь только одну внутреннюю точку экстремума. Эту точку и возьмем в качестве оставшейся точки альтернанса. То есть, если функция $f(x)$ дифференцируема, то  \begin{eqnarray}
	f'(x_1) - P_1'(x_1) = 0.
	\end{eqnarray}
	Таким образом, имея 3 условия из (2) и условие (3), составляем систему:
	\begin{eqnarray}
		\begin{cases}
			f(x_0) - P_1(x_0) = \alpha\Delta,\\
			f(x_1) - P_1(x_1) = -\alpha\Delta,\\
			f(x_2) - P_1(x_2) = \alpha\Delta,\\
			f'(x_1) - P_1'(x_1) = 0.
		\end{cases}
	\end{eqnarray}
	Подставим известные нам значения: $$\begin{cases}
		f(-1) - (c_0 + c_1 \cdot (-1)) = \alpha\Delta,\\
		f(x_1) - (c_0 + c_1\cdot x_1) = -\alpha\Delta,\\
		f(1) - (c_0 + c_1 \cdot 1) = \alpha\Delta,\\
		f'(x_1) - c_1 = 0.
	\end{cases}\Rightarrow \begin{cases}
	\dfrac12 - (c_0 - c_1) = \alpha\Delta,\\
	2^{x_1} - (c_0 + c_1\cdot x_1) = -\alpha\Delta,\\
	2 - (c_0 + c_1) = \alpha\Delta,\\
	2^{x_1}\ln2 - c_1 = 0.
	\end{cases}$$
	Вычислим $c_1$, отняв от третьего уравнения первое:
	$$\dfrac{3}{2} - 2c_1 = 0 \Rightarrow c_1 = \dfrac34.$$
	Вычислим $x_1$, подставив в последнее уравнение значение $c_1$:
	$$x_1 = \log_2\dfrac{3}{4\ln 2}\approx 0.11373.$$
	Сложим второе и третье уравнение, чтобы найти $c_0$:
	$$\dfrac{3}{4\ln2} + 2 - 2c_0 - \dfrac34 -\dfrac34 \log_2\dfrac{3}{4\ln 2} = 0 \Rightarrow c_0 = \dfrac{3}{8\ln 2} + \dfrac58 - \dfrac{3}{8}\log_2\dfrac{3}{4\ln 2}\approx 1.12336.$$
	Остается найти $\alpha\Delta$. Мы можем найти это значение как из 1, так и из 3 уравнения. К примеру, возьмем третье уравнение:
	$$\alpha\Delta = 2 - \dfrac{3}{8\ln 2} - \dfrac58 + \dfrac{3}{8}\log_2\dfrac{3}{4\ln 2}\approx 0.87664.$$
	Соответственно $\alpha = 1$, $\Delta \approx 0.87664$ и многочлен наилучшего равномерного приближения имеет вид $$P_1(x) = 0.75x + 1.12336.$$ 
	Графически это будет выглядеть следующим образом:
	\begin{center}\begin{tikzpicture}
		\begin{axis}[
			title = Function Approximation,
			legend pos = north west,
			xlabel = {$x$},
			ylabel = {$y$},
			minor tick num = 2,
			xmin = -1,
			xmax = 1,
			grid = major
			]
			\legend{$2^x$, $P_1(x)$}
			\addplot[blue] {2^x};
			\addplot[orange] {0.75*x + 1.12336};
		\end{axis}
	\end{tikzpicture}\end{center}
\end{document}