\documentclass[a4paper, 12pt]{article}
\usepackage{cmap}
\usepackage{amssymb}
\usepackage{amsmath}
\usepackage{graphicx}
\usepackage{amsthm}
\usepackage{upgreek}
\usepackage{setspace}
\usepackage{color}
\usepackage[T2A]{fontenc}
\usepackage[utf8]{inputenc}
\usepackage[normalem]{ulem}
\usepackage{mathtext} % русские буквы в формулах
\usepackage[left=2cm,right=2cm, top=2cm,bottom=2cm,bindingoffset=0cm]{geometry}
\usepackage[english,russian]{babel}
\usepackage[unicode]{hyperref}
\newenvironment{Proof} % имя окружения
{\par\noindent{$\blacklozenge$}} % команды для \begin
{\hfill$\scriptstyle\boxtimes$}
\newcommand{\Rm}{\mathbb{R}}
\newcommand{\Cm}{\mathbb{C}}
\newcommand{\Z}{\mathbb{Z}}
\newcommand{\I}{\mathbb{I}}
\newcommand{\N}{\mathbb{N}}
\newcommand{\rank}{\operatorname{rank}}
\newcommand{\Ra}{\Rightarrow}
\newcommand{\ra}{\rightarrow}
\newcommand{\FI}{\Phi}
\newcommand{\Sp}{\text{Sp}}
\renewcommand{\leq}{\leqslant}
\renewcommand{\geq}{\geqslant}
\renewcommand{\alpha}{\upalpha}
\renewcommand{\beta}{\upbeta}
\renewcommand{\gamma}{\upgamma}
\renewcommand{\delta}{\updelta}
\renewcommand{\varphi}{\upvarphi}
\renewcommand{\phi}{\upvarphi}
\renewcommand{\tau}{\uptau}
\renewcommand{\lambda}{\uplambda}
\renewcommand{\psi}{\uppsi}
\renewcommand{\mu}{\upmu}
\renewcommand{\omega}{\upomega}
\renewcommand{\d}{\partial}
\renewcommand{\xi}{\upxi}
\renewcommand{\epsilon}{\upvarepsilon}
\newcommand{\intx}{\int\limits_{x_0}^x}
\newcommand\Norm[1]{\left\| #1 \right\|}
\newcommand{\sumk}{\sum\limits_{k=0}^\infty}
\newcommand{\sumi}{\sum\limits_{i=0}^\infty}
\newtheorem*{theorem}{Теорема}
\newtheorem*{cor}{Следствие}
\newtheorem*{lem}{Лемма}
\begin{document}
	\section*{Многочлен Лагранжа при равноотстоящих узлах}
	\subsubsection*{Условие}
	Построить интерполяционный многочлен в форме Лагранжа для сетки равноотстоящих узлов. 
	\subsubsection*{Алгоритм решения}
	Пусть функция $f(x)$ задана таблично в $n$ узлах $x_i$, которые являются равноотстоящими, то есть $$x_i = x_0 + ih,\ h>0,\quad i = 0,1,\ldots,n.$$
	Тогда интерполяционный многочлен будет иметь степень $n$.\\\\
	Интерполяционный многочлен Лагранжа записывается в общем виде \begin{eqnarray}
	P_n(x) = \sum_{k=0}^{n}l_k(x) f(x_k),\quad l_k(x) = \dfrac{\prod\limits_{j=0, j\ne k}^n (x-x_j)}{\prod\limits_{j=0, j\ne k}^n (x_k-x_j)}.
	\end{eqnarray}
	Тогда, поскольку узлы равноотстоящие, имеем \begin{itemize}
		\item $x - x_j = x - x_0 - jh$;
		\item $x_k - x_j = x_0 - kh - x_0 + jh = h(k-j)$.
	\end{itemize}
	Отсюда $$l_k(x) = \dfrac{\prod\limits_{j=0, j\ne k}^n (x - x_0 - jh)}{\prod\limits_{j=0, j\ne k}^n h(k-j)} = \dfrac{1}{h^n}\cdot \dfrac{\prod\limits_{j=0, j\ne k}^n (x - x_0 - jh)}{\prod\limits_{j=0, j\ne k}^n (k-j)}.$$
	Введем замену $t = \dfrac{x-x_0}{h}$.
	Отсюда \begin{multline*}
		l_k(x) = l_k(x_0 + th) = \dfrac{1}{h^n}\cdot \dfrac{\prod\limits_{j=0, j\ne k}^n (th - jh)}{\prod\limits_{j=0, j\ne k}^n (k-j)}=\prod\limits_{j=0, j\ne k}^n\dfrac{(t - j)}{(k-j)}=\\=\dfrac{t(t-1)\ldots (t-n)}{t-k}\cdot \dfrac{(-1)^{n-k}}{k!(n-k)!} = (-1)^{n-k} C^k_n\dfrac{1}{t-k}\cdot \dfrac{t(t-1)\ldots (t-n)}{n!}.
	\end{multline*}
	Подставим это в выражение (1), тогда $$P_n(x) = (-1)^n\dfrac{t(t-1)\ldots (t-n)}{n!}\sum_{k=0}^{n}(-1)^{-k}C^k_n\dfrac{1}{t-k}f(x_k).$$
	Недостатком данной формулы является факториальная сложность числителя и знаменателя, что делает вычисления достаточно трудоемкими. Поэтому при равноотстоящих узлах принято использовать интерполяционный многочлен в форме Ньютона.
\end{document}