\documentclass[a4paper, 12pt]{article}
\usepackage{cmap}
\usepackage{amssymb}
\usepackage{amsmath}
\usepackage{graphicx}
\usepackage{amsthm}
\usepackage{upgreek}
\usepackage{setspace}
\usepackage{color}
\usepackage{pgfplots}
\usepackage{moreverb}
\pgfplotsset{compat=1.9}
\usepackage[T2A]{fontenc}
\usepackage[utf8]{inputenc}
\usepackage[normalem]{ulem}
\usepackage{mathtext} % русские буквы в формулах
\usepackage[left=2cm,right=2cm, top=2cm,bottom=2cm,bindingoffset=0cm]{geometry}
\usepackage[english,russian]{babel}
\usepackage[unicode]{hyperref}
\newenvironment{Proof} % имя окружения
{\par\noindent{$\blacklozenge$}} % команды для \begin
{\hfill$\scriptstyle\boxtimes$}
\newcommand{\Rm}{\mathbb{R}}
\newcommand{\Cm}{\mathbb{C}}
\newcommand{\Z}{\mathbb{Z}}
\newcommand{\I}{\mathbb{I}}
\newcommand{\N}{\mathbb{N}}
\newcommand{\rank}{\operatorname{rank}}
\newcommand{\Ra}{\Rightarrow}
\newcommand{\ra}{\rightarrow}
\newcommand{\FI}{\Phi}
\newcommand{\Sp}{\text{Sp}}
\renewcommand{\leq}{\leqslant}
\renewcommand{\geq}{\geqslant}
\renewcommand{\alpha}{\upalpha}
\renewcommand{\beta}{\upbeta}
\renewcommand{\gamma}{\upgamma}
\renewcommand{\delta}{\updelta}
\renewcommand{\varphi}{\upvarphi}
\renewcommand{\phi}{\upvarphi}
\renewcommand{\tau}{\uptau}
\renewcommand{\lambda}{\uplambda}
\renewcommand{\psi}{\uppsi}
\renewcommand{\mu}{\upmu}
\renewcommand{\omega}{\upomega}
\renewcommand{\d}{\partial}
\renewcommand{\xi}{\upxi}
\renewcommand{\epsilon}{\upvarepsilon}
\newcommand{\intx}{\int\limits_{x_0}^x}
\newcommand\Norm[1]{\left\| #1 \right\|}
\newcommand{\sumk}{\sum\limits_{k=0}^\infty}
\newcommand{\sumi}{\sum\limits_{i=0}^\infty}
\newtheorem*{theorem}{Теорема}
\newtheorem*{cor}{Следствие}
\newtheorem*{lem}{Лемма}
\begin{document}
	\section*{Численное интегрирование}
	\subsubsection*{Условия}
	\begin{enumerate}
		\item Определить алгебраическую степень точности указанной квадратурной формулы
		$$\int\limits_a^b f(x)dx \approx \dfrac{b-a}{4}\left(f(a) + 3f\left(\dfrac{a+2b}{3}\right)\right)$$ (\hyperlink{t1}{Решение})
		\item Построить квадратурную формулу максимально возможной алгебраической степени точности вида $$I(f) = \int\limits_{-1}^{1}\rho (x) f(x)dx \approx A_0 f(x_0) + A_1 f(x_1),$$
		приняв $\rho(x) = 1$. (\hyperlink{t2}{Решение})
		\item Построить квадратурную формулу максимально возможной степени точности вида $$\int\limits_0^1 \dfrac{f(x)}{\sqrt{1-x}} \approx A_0 f(x_0) + A_1 f(x_1).$$ (\hyperlink{t3}{Решение})
		\item Определить алгебраическую степень точности указанной квадратурной формулы $$\int\limits_0^{+\infty} e^{-x} f(x)dx\approx \dfrac{2+\sqrt2}{4} f(2-\sqrt2) + \dfrac{2-\sqrt2}{4}f(2+\sqrt2).$$ (\hyperlink{t4}{Решение})
	\end{enumerate}
	
	\newpage
	\subsubsection*{Решения}
	\begin{enumerate}
		\item \hypertarget{t1}{}
		Для построения для решения понадобятся следующие соотношения. Если \begin{eqnarray}
			\begin{cases}
				\int\limits_a^b \rho(x) x^idx = \sum\limits_{k=0}^{n}A_kx^i_k,\quad i=\overline{0,m},\\
				\int\limits_a^b \rho(x) x^{m+1}dx \ne \sum\limits_{k=0}^{n}A_kx^i_k;
			\end{cases}
		\end{eqnarray}
		то в этом случае говорят, что квадратурная формула \textbf{имеет алгебраическую степень точности равную \textit{m}}.\\\\
		Таким образом, для решения задачи необходимо строить по одному уравнению из соотношений (1) до тех пор, пока мы не получим неравенство.\\\\
		Теперь обратим внимание на составляющие элементы формулы (1), а именно $\rho(x), A_k, x_k$. Квадратурная формула записывается в общем случае в виде $$I(f) = \int\limits_a^b\rho(x)f(x)dx \approx A_0f(x_0) + A_1 f(x_1).$$
		Из условия следует, что $\rho(x) = 1$. Также из условия можно сделать вывод, что $$A_0 = \dfrac{b-a}{4},\quad A_1 = 3\cdot \dfrac{b-a}{4};\quad x_0 = a,\quad x_1 = \dfrac{a+2b}{3}.$$
		Начнем записывать соотношения из (1). Возьмем $i=0$, тогда $$x^0 : \int\limits_a^b dx = A_0 + A_1.$$
		Вычислим интеграл и подставим значения $A_k, x_k$ $$x^0 : b-a \overset{?}{=} \dfrac{b-a}{4} + 3\cdot \dfrac{b-a}{4} =4\cdot \dfrac{ b-a}{4}.$$
		Равенство выполняется. Далее по аналогии берем $i=1$:
		$$x^1 : \int\limits_a^bx dx = A_0x_0 + A_1x_1.$$
		Вычисляем значение интеграла и подставляем неизвестные:
		$$x^1 : \dfrac{b^2-a^2}{2} \overset{?}{=} \dfrac{b-a}{4}\left(a + 3\cdot \dfrac{a+2b}{3}\right) = 2\dfrac{b^2-a^2}{4}.$$
		Равенство выполняется. Берем $i=2$:
		$$x^2 : \int\limits_a^bx^2 dx = A_0x_0^2 + A_1x_1^2.$$
		$$x^2 : \dfrac{b^3-a^3}{3} \overset{?}{=} \dfrac{b-a}{4}\left(a^2 + 3\cdot \dfrac{(a+2b)^2}{9}\right) = \dfrac{b-a}{4}\cdot\left( \dfrac{4a^2 + 4ab + 4b^2}{3}\right) = \dfrac{b^3-a^3}{3}.$$
		Равенство выполняется. Берем $i=3$:
		$$x^3 : \int\limits_a^bx^3 dx = A_0x_0^3 + A_1x_1^3.$$
		$$x^2 : \dfrac{b^4-a^4}{4} \overset{?}{=} \dfrac{b-a}{4}\left(a^3 + 3\cdot \dfrac{(a+2b)^3}{27}\right) = \dfrac{b-a}{4}\cdot\left( \dfrac{10a^3+6a^2b + 12ab^2 + 8b^3}{9}\right).$$
		Отсюда уже видно, что равенство не выполняется, так как справа мы не получим как минимум $-\dfrac14 a^4$.\\\\
		Таким образом, берем последнюю степень, на которой выполнялось равенство. Это $i=2$. Тогда же и АСТ = 2.
		
		\newpage
		\item 
		\hypertarget{t2}{}
		Для построения квадратурной формулы понадобятся соотношения \begin{eqnarray}
			\begin{cases}
				\int\limits_a^b \rho(x) x^idx = \sum\limits_{k=0}^{n}A_kx^i_k,\quad i=\overline{0,m},\\
				\int\limits_a^b \rho(x) x^{m+1}dx \ne \sum\limits_{k=0}^{n}A_kx^i_k;
			\end{cases}
		\end{eqnarray}
		причем в этом случае говорят, что квадратурная формула \textbf{имеет алгебраическую степень точности равную \textit{m}}.\\\\
		Соответственно для построения квадратурной формулы необходимо построить систему из соотношений (1), по которой мы найдем неизвестные. Затем найти следующее выражение, при котором равенство нарушается.\\\\
		Из условия следует, что неизвестными значениями являются $A_0, A_1$, $x_0, x_1$. Итого 4 неизвестных. Для их отыскания следует построить систему из 4-ех уравнений по формуле (1). Тогда в соответствии со степенями $x^i$ получаем систему $$\begin{cases}
			x^0 : \int\limits_{-1}^1 dx = A_0 + A_1,\\
			x^1 : \int\limits_{-1}^1 xdx = A_0x_0 + A_1x_1,\\
			x^2 : \int\limits_{-1}^1 x^2dx = A_0x_0^2 + A_1x_1^2,\\
			x^3 : \int\limits_{-1}^1 x^3dx = A_0x_0^3 + A_1x_1^3.\\
		\end{cases}$$ 
		Поменяем местами значения относительно равенства, заодно вычислив значения интегралов:
		$$\begin{cases}
			A_0 + A_1 = 2,\\
			A_0x_0 + A_1x_1 = 0,\\
			A_0x_0^2 + A_1x_1^2 = \frac23,\\
			A_0x_0^3 + A_1x_1^3 = 0.\\
		\end{cases}$$ 
		Итого имеем нелинейную систему из 4-ех уравнений. Для ее решения будем применять эвристики. Заметим, что можно домножить 2-ое уравнение на $x_1^3$ и вычесть из него 4-ое:
		$$A_1x_1(x_1^2 - x_0^2) = 0.$$
		Причем ни $A_1$, ни $x_1$ не обращаются в ноль, иначе тогда либо $A_0=0$, либо $x_0$ = 0. Значит $$(x_1-x_0)(x_1+x_0) = 0.$$
		В предположении, что узлы не совпадают, имеем $$x_1 = -x_0.$$
		Тогда подставляем это во 2-ое уравнение и имеем 
		$$-A_0x_1 + A_1x_1 = x_1(A_1 - A_0) = 0.$$
		Отсюда $$A_0 = A_1.$$
		А из первого уравнения тогда будет следовать, что $$A_0 = A_1 = 1.$$
		Подставим это в третье уравнение и получим $$x_0^2 + x_1^2 = \dfrac23.$$
		Учитывая, что $x_1 = -x_0$, а тогда $x_1^2 = x_0^2$. А значит $$x_0^2 = x_1^2 = \dfrac{1}{3}\Rightarrow x_0 = -\dfrac{1}{\sqrt3},\ x_1 = \dfrac{1}{\sqrt3}.$$
		В итоге система решена и $$\begin{cases}
			A_0 = A_1 = 1,\\
			x_0 = -\dfrac{1}{\sqrt3},\\
			x_1 = \dfrac{1}{\sqrt3}.
		\end{cases}$$
		Таким образом, система имеет единственное решение. Отсюда можно сделать вывод, что \textbf{для всей квадратурной формулы АСТ $\geq 3$.}\\\\
		Теперь нужно показать, что АСТ $= 3$. Для этого необходимо добавить еще одно уравнение и посмотреть, будет ли выполняться равенство:
		$$x^4 : \int\limits_{-1}^1 x^4dx = A_0x_0^4 + A_1x_1^4.$$
		Вычисляя значение интеграла, получим $$A_0x_0^4 + A_1x_1^4 = \dfrac25.$$
		Подставим теперь известные нам значения $A_0,A_1$, $x_0,x_1$:
		$$\dfrac{1}{9} + \dfrac{1}{9} = \dfrac29\ne \dfrac25.$$ 
		Равенство не выполняется, а значит АСТ = 3 является максимально возможностей АСТ.\\\\
		Сама квадратурная формула будет иметь в таком случае вид
		$$I = \int\limits_{-1}^{1} f(x)dx \approx f\left(-\dfrac{1}{\sqrt3}\right) + f\left(\dfrac{1}{\sqrt3}\right).$$
		
		\newpage
		\item 
		\hypertarget{t3}{}
		Для построения квадратурной формулы понадобятся соотношения \begin{eqnarray}
			\begin{cases}
				\int\limits_a^b \rho(x) x^idx = \sum\limits_{k=0}^{n}A_kx^i_k,\quad i=\overline{0,m},\\
				\int\limits_a^b \rho(x) x^{m+1}dx \ne \sum\limits_{k=0}^{n}A_kx^i_k;
			\end{cases}
		\end{eqnarray}
		причем в этом случае говорят, что квадратурная формула **имеет алгебраическую степень точности равную m**.
		
		Соответственно для построения квадратурной формулы необходимо построить систему из соотношений (1), по которой мы найдем неизвестные. Затем найти следующее выражение, при котором равенство нарушается.
		
		Из условия следует, что неизвестными значениями являются $A_0, A_1$, $x_0, x_1$. Итого 4 неизвестных. Для их отыскания следует построить систему из 4-ех уравнений по формуле (1). Тогда в соответствии со степенями $x^i$ запишем систему
		$$\begin{cases}
			x^0 : \int\limits_{0}^1 \dfrac{1}{\sqrt{1-x}} dx = A_0 + A_1,\\
			x^1 : \int\limits_{0}^1 \dfrac{x}{\sqrt{1-x}} dx = A_0x_0 + A_1x_1,\\
			x^2 : \int\limits_{0}^1 \dfrac{x^2}{\sqrt{1-x}} dx = A_0x_0^2 + A_1x_1^2,\\
			x^3 : \int\limits_{0}^1 \dfrac{x^3}{\sqrt{1-x}} dx = A_0x_0^3 + A_1x_1^3.\\
		\end{cases}$$ 
		Вычислим значения интегралов из левой части уравнений:
		$$\int\limits_{0}^1 \dfrac{1}{\sqrt{1-x}} dx = -\int\limits_0^1 (1-x)^{-\frac12} d(1-x) = -2(1-x)^{\frac12}\Big|_0^1 = 2.$$
		\begin{multline*}\int\limits_{0}^1 \dfrac{x}{\sqrt{1-x}} dx = \left[\begin{matrix}\sqrt{1-x} = t, & x = 1-t^2, \\ x=1 \to t=0, & x=0 \to t = 1, \\ dx = -2tdt\end{matrix}\right] =\\= \int\limits_1^0\dfrac{1-t^2}{t} \cdot (-2t)dt = -2\int\limits_1^0 1-t^2 dt = -2 \left(t\Big|_ 1^0 - \dfrac{t^3}{3}\Big|_ 1^0\right) = \dfrac43.\end{multline*}
		В последующих интегралах приводится аналогичная замена.
		$$\int\limits_{0}^1 \dfrac{x^2}{\sqrt{1-x}} dx = -2\int\limits_1^0 (1-t^2)^2 dt =-2\int\limits_1^0 1 - 2t^2 + t^4 dt= -2 \left(t\Big|_ 1^0 - \dfrac{2t^3}{3}\Big|_ 1^0 + \dfrac{t^5}{5}\Big|_ 1^0\right) = \dfrac{16}{15}.$$
		\begin{multline*}
			\int\limits_{0}^1 \dfrac{x^3}{\sqrt{1-x}} dx = -2\int\limits_1^0 (1-t^2)^3 dt =-2\int\limits_1^0 1 - 3t^2 + 3t^4 -t^6 dt=\\= -2 \left(t\Big|_ 1^0 - \dfrac{3t^3}{3}\Big|_ 1^0 + \dfrac{3t^5}{5}\Big|_ 1^0 - \dfrac{t^7}{7}\Big|_ 1^0\right) = \dfrac{32}{35}.
		\end{multline*}
		
		В итоге имеем систему нелинейных уравнений $$\begin{cases} A_0 + A_1 = 2, \\ A_0x_0 + A_1x_1 = \dfrac43,\\ A_0x_0^2 + A_1x_1^2 = \dfrac{16}{15}, \\ A_0x_0^3 + A_1x_1^3 = \dfrac{32}{35}. \end{cases}$$
		Первое, второе и третье уравнения домножим на $-x_0$ и прибавим ко второму, третьему и четвертому соответственно. Тогда будем иметь систему с тремя неизвестными 
		$$
		\begin{cases}
			A_1x_1 - A_1x_0 = \dfrac43 - 2x_0,\\
			A_1x_1^2 - A_1x_1x_0 = \dfrac{16}{15} - \dfrac43x_0,\\
			A_1x_1^3 - A_1x_1^2x_0 = \dfrac{32}{35} - \dfrac{16}{15}x_0.
		\end{cases}
		$$
		Поступаем аналогично, но умножим первое и второе уравнения на $-x_1$ и прибавим к второму и третьему соответственно. Тогда имеем более простую нелинейную систему из 2-ух уравнений 
		$$
		\begin{cases}
			x_0 + x_1 - \dfrac32 x_0x_1 = \dfrac45,\\
			x_0 + x_1 - \dfrac54 x_0x_1 = \dfrac67.
		\end{cases}
		$$
		Вычтем из первого уравнения второе и получим $$\dfrac{x_0x_1}{4} = \dfrac{2}{35}\Rightarrow x_0x_1 = \dfrac{8}{35}.$$
		В предположении, что $x_1\ne 0$ (подстановкой в исходную систему можно проверить, что это действительно так), выражаем $x_0$:
		$$x_0 = \dfrac{8}{35x_1}.$$
		Подставляя это выражение в первое уравнение, получим $$\dfrac{8}{35x_1} + x_1 - \dfrac{3}{2}\cdot \dfrac{8}{35x_1} = \dfrac45.$$
		Отсюда получаем квадратное уравнение $$x_1^2 - \dfrac{8}{7} x_1 + \dfrac{8}{35} = 0.$$
		Корнями которого являются значения $$x_1 = \dfrac47 + \dfrac{2\sqrt{30}}{35},\quad x_1 = \dfrac47 - \dfrac{2\sqrt{30}}{35}.$$
		Подстановкой каждого из значений в выражение $x_0 = \dfrac{8}{35x_1}$ можно получить, что $$x_0 = \dfrac47 - \dfrac{2\sqrt{30}}{35},\quad x_0 = \dfrac47 +\dfrac{2\sqrt{30}}{35}.$$
		Таким образом, мы можем выбрать любое из двух значений. Обусловимся тем, что узлы мы строим в порядке возрастания. Поэтому в итоге выберем $$x_0 = \dfrac47 - \dfrac{2\sqrt{30}}{35},\quad x_1 = \dfrac47 +\dfrac{2\sqrt{30}}{35}.$$
		Теперь, используя полученные значения, мы можем из первых двух уравнений исходной системы вычислить $A_0$ и $A_1$. Выпишем эти уравнения, подставив $x_0$, $x_1$:
		$$\begin{cases}
			A_0 + A_1 = 2,\\
			\dfrac{(20-2\sqrt{30})A_0 + (20+2\sqrt{30})A_1}{35}=\dfrac43.
		\end{cases}$$
		Выразим из первого уравнения $A_1 = 2 - A_0$ и подставим во второе:
		$$\dfrac{20A_0-2\sqrt{30}A_0 + 2(20+2\sqrt{30}) - 20A_0 - 4\sqrt{30}A_0}{35}=\dfrac43.$$
		Преобразуем уравнение 
		$$\dfrac{-4\sqrt{30}A_0}{35} = \dfrac43 - \dfrac{40}{35} - \dfrac{2\sqrt{30}}{35}.$$
		$$A_0 = -\dfrac{35}{3\sqrt{30}} + \dfrac{10}{\sqrt{30}} + 1 = -\dfrac{7\sqrt{30}}{18} + \dfrac{\sqrt{30}}{3} + 1=1 - \dfrac{\sqrt{30}}{18}.$$
		Тогда $$A_1 = 1 + \dfrac{\sqrt{30}}{18}.$$
		Итого имеем $$A_0 = 1-\dfrac{\sqrt{30}}{18},\quad A_1 = 1+\dfrac{\sqrt{30}}{18},\quad x_0 = \dfrac47 - \dfrac{2\sqrt{30}}{35},\quad x_0 = \dfrac47 +\dfrac{2\sqrt{30}}{35}.$$
		В итоге получаем квадратурную формулу, имеющую алгебраическую степень точности не ниже 3. Для уточнения построим еще одно уравнение и проверим, удовлетворяют ли полученные нами данные этому уравнению:
		$$x^4 : \int\limits_{0}^1 \dfrac{x^4}{\sqrt{1-x}} dx = A_0x_0^4 + A_1x_1^4.$$
		Как и прежде, вычисляем интеграл с помощью замены:
		\begin{multline*}
			\int\limits_{0}^1 \dfrac{x^4}{\sqrt{1-x}} dx = -2\int\limits_1^0 (1-t^2)^4dx = -2\int\limits_1^0 1 - 4 t^2 + 6 t^4 - 4 t^6 + t^8dt = \\= -2 \left(t\Big|_ 1^0 - \dfrac{4t^3}{3}\Big|_ 1^0 + \dfrac{6t^5}{5}\Big|_ 1^0 - \dfrac{4t^7}{7}\Big|_ 1^0 + \dfrac{t^9}{9}\Big|_ 1^0\right) = \dfrac{256}{315}.\end{multline*}
		Тогда $$A_0x_0^4 + A_1x_1^4 = \dfrac{256}{315}.$$
		
		С помощью компьютерных вычислений посчитаем значение слева.
		\begin{listingcont}
import math
A_0 = (18 - math.sqrt(30)) / 18
A_1 = (18 + math.sqrt(30)) / 18
x_0 = (20 - 2*math.sqrt(30)) / 35
x_1 = (20 + 2*math.sqrt(30)) / 35

A_0 * x_0**4 + A_1 * x_1**4 == 256/315\end{listingcont}
Программа вернула False. \\\\
Таким образом, при степени 4 точность нарушается. Следовательно, мы построили квадратурную формулу с АСТ = 3 вида $$\int\limits_0^1 \dfrac{f(x)}{\sqrt{1-x}}dx = \dfrac{18 - \sqrt{30}}{18}\cdot f\left(\dfrac{20 - 2\sqrt{30}}{35}\right) + \dfrac{18 + \sqrt{30}}{18}\cdot f\left(\dfrac{20 + 2\sqrt{30}}{35}\right).$$
\newpage
\item 
\hypertarget{t4}{}
Для решения задачи понадбятся соотношения (1). Таким образом необходимо строить по одному уравнению из соотношений (1) до тех пор, пока мы не получим неравенство.

Теперь обратим внимание на составляющие элементы формулы (1), а именно $\rho(x), A_k, x_k$. Квадратурная формула для нашей задачи записывается в общем случае в виде
$$I(f) = \int\limits_a^b \rho(x)f(x)dx \approx A_0 f(x_0) + A_1 f(x_1).$$
Из постановки задачи мы выделяем, что $$\rho(x) = e^{-x},\quad A_0 =\dfrac{2+\sqrt2}{4},\quad x_0 =2-\sqrt2,\quad A_1 = \dfrac{2-\sqrt2}{4},\quad x_1 = 2+\sqrt2.$$
Остается лишь записывать соотношения из (1) последовательно, повышая степень $i$ на каждом шаге.

Причем заметим, что по формуле (1) слева мы будем получать интегралы вида $$\int\limits_0^{+\infty} x^i e^{-x} dx = \Gamma(i+1) = i!,$$ что позволяет нам уменьшить количество вычислений.

Берем $i=0$:
$$x^0 : \int\limits_0^{+\infty} e^{-x}dx=0! \overset{?}{=}  A_0 + A_1ю$$
Тогда $$1 \overset{?}{=} A_0 + A_1 = \dfrac{2+\sqrt2}{4} + \dfrac{2-\sqrt2}{4} = 1.$$
Равенство выполняется.

Берем $i=1$:
$$x^1 : \int\limits_0^{+\infty} xe^{-x}dx = 1!\overset{?}{=}  A_0x_0 + A_1x_1ю$$
Тогда $$1 \overset{?}{=} A_0x_0 + A_1x_1 = \dfrac{2+\sqrt2}{4}\cdot (2-\sqrt2) + \dfrac{2-\sqrt2}{4}\cdot (2+\sqrt2) = 1.$$
Равенство выполняется.

Берем $i=2$:
$$x^2 : \int\limits_0^{+\infty} x^2e^{-x}dx=2! \overset{?}{=}  A_0x_0^2 + A_1x_1^2,$$
Тогда $$2 \overset{?}{=} A_0x_0^2 + A_1x_1^2 = \dfrac{2+\sqrt2}{4}\cdot (2-\sqrt2)^2 + \dfrac{2-\sqrt2}{4}\cdot (2+\sqrt2)^2 = 2.$$
Равенство выполняется.

Берем $i=3$:
$$x^3 : \int\limits_0^{+\infty} x^3e^{-x}dx =3!\overset{?}{=}  A_0x_0^3 + A_1x_1^3ю$$
Тогда $$6 \overset{?}{=} A_0x_0^3 + A_1x_1^3 = \dfrac{2+\sqrt2}{4}\cdot (2-\sqrt2)^3 + \dfrac{2-\sqrt2}{4}\cdot (2+\sqrt2)^3 = \dfrac12(4-4\sqrt2 + 2) +\dfrac12(4+4\sqrt2 + 2) = 6.$$
Равенство выполняется.

Берем $i=4$:
$$x^3 : \int\limits_0^{+\infty} x^4e^{-x}dx=4! \overset{?}{=}  A_0x_0^4 + A_1x_1^4.$$
Тогда $$24 \overset{?}{=} A_0x_0^4 + A_1x_1^4 = \dfrac{2+\sqrt2}{4}\cdot (2-\sqrt2)^4 + \dfrac{2-\sqrt2}{4}\cdot (2+\sqrt2)^4 = \dfrac12(4\cdot (6-4\sqrt2 - 2 + 6+4\sqrt2) = 20.$$
Равенство не выполняется.

Таким образом, квадратурная формула имеет АСТ = 3.
	\end{enumerate}
\end{document} 