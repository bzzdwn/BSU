\documentclass[a4paper, 12pt]{article}
\usepackage{cmap}
\usepackage{amssymb}
\usepackage{amsmath}
\usepackage{graphicx}
\usepackage{amsthm}
\usepackage{upgreek}
\usepackage{setspace}
\usepackage{color}
\usepackage{pgfplots}
\pgfplotsset{compat=1.9}
\usepackage[T2A]{fontenc}
\usepackage[utf8]{inputenc}
\usepackage[normalem]{ulem}
\usepackage{mathtext} % русские буквы в формулах
\usepackage[left=2cm,right=2cm, top=2cm,bottom=2cm,bindingoffset=0cm]{geometry}
\usepackage[english,russian]{babel}
\usepackage[unicode]{hyperref}
\newenvironment{Proof} % имя окружения
{\par\noindent{$\blacklozenge$}} % команды для \begin
{\hfill$\scriptstyle\boxtimes$}
\newcommand{\Rm}{\mathbb{R}}
\newcommand{\Cm}{\mathbb{C}}
\newcommand{\Z}{\mathbb{Z}}
\newcommand{\I}{\mathbb{I}}
\newcommand{\N}{\mathbb{N}}
\newcommand{\rank}{\operatorname{rank}}
\newcommand{\Ra}{\Rightarrow}
\newcommand{\ra}{\rightarrow}
\newcommand{\FI}{\Phi}
\newcommand{\Sp}{\text{Sp}}
\renewcommand{\leq}{\leqslant}
\renewcommand{\geq}{\geqslant}
\renewcommand{\alpha}{\upalpha}
\renewcommand{\beta}{\upbeta}
\renewcommand{\gamma}{\upgamma}
\renewcommand{\delta}{\updelta}
\renewcommand{\varphi}{\upvarphi}
\renewcommand{\phi}{\upvarphi}
\renewcommand{\tau}{\uptau}
\renewcommand{\lambda}{\uplambda}
\renewcommand{\psi}{\uppsi}
\renewcommand{\mu}{\upmu}
\renewcommand{\omega}{\upomega}
\renewcommand{\d}{\partial}
\renewcommand{\xi}{\upxi}
\renewcommand{\epsilon}{\upvarepsilon}
\newcommand{\intx}{\int\limits_{x_0}^x}
\newcommand\Norm[1]{\left\| #1 \right\|}
\newcommand{\sumk}{\sum\limits_{k=0}^\infty}
\newcommand{\sumi}{\sum\limits_{i=0}^\infty}
\newtheorem*{theorem}{Теорема}
\newtheorem*{cor}{Следствие}
\newtheorem*{lem}{Лемма}
\begin{document}
	\section*{Отделение корней графическим методом}
	\subsubsection*{Условие}
	Отделить один корень уравнения $$2\sin3x = x^2 - 4x + 3.$$
	\subsubsection*{Алгоритм решения}
	Для решения задачи нам понадобятся утверждения:
	\begin{enumerate}
		\item \textbf{Теорема 1.} Если функция $f(x)\in C[a,b]$ и принимает на его концах значения разных знаков, то на этом отрезке существует по крайней мере один корень уравнения $f(x) = 0$.
			Если при этом функция $f(x)$ будет монотонной на отрезке $[a,b]$, то она может иметь только один корень.
	\end{enumerate}
	Алгоритм решения следующий: с помощью графического метода отделяется корень, проверяется выполнение теоремы 1.\\\\
	Для начала построим графики для данного уравнения, так как в данном случае это легко сделать. Определим две функции $$y_1(x) = 2\sin 3x,\quad y_2(x) = x^2 - 4x+3$$ и построим их графики.
	\begin{center}\begin{tikzpicture}
			\begin{axis}[
				legend pos = north west,
				xlabel = {$x$},
				ylabel = {$y$},
				minor tick num = 2,
				samples=1000,
				xmin = -4,
				xmax = 4,
				ymin = -4,
				ymax = 4,
				grid = major,
				scatter/classes={%
					a={mark=o,draw=black}}
				]
				\legend{$y_1(x)$, $y_2(x)$}
				\addplot[blue] {x^2 - 4*x + 3};
				\addplot[orange] {2*sin(deg(3*x))};
			\end{axis}
	\end{tikzpicture}\end{center}
	Приведем исходное уравнение к виду $f(x) = 0$: $$\underbrace{2\sin 3x - (x^2 - 4x+3)}_{f(x)} = 0.$$
	Область определения функции $f(x)$ совпадает с $\Rm$.
	Таким образом, уравнение $f(x)=0$ имеет 4 корня на отрезке $[-4,4]$ (и на всей числовой прямой). \\\\
	Пусть корнем, для которого мы будем искать приближение, будет корень, лежащий слева (однако по аналогии можно найти приближенное значение любого из остальных корней). Этот корень лежит на отрезке $[0; 1]$. Причем из графика видно, что он располагается до точки, в которой функция $y_1(x) = 2\sin3x$ достигает значения $y_1(x) = 2$, то есть до точки $x = \dfrac\pi6$.
	Таким образом, в качестве отрезка, на котором предположительно располагается исследуемый корень, мы можем взять отрезок $$d = \Big[0; \dfrac\pi6\Big].$$
	Проверим выполнение условий теоремы 1 корня уравнения на отрезке $d$, для этого вычислим значения функции $f(x)$ на концах отрезка $d = \Big[0; \dfrac\pi6\Big]$:
	$$f(0) =2\sin0 - 0^2 + 4\cdot 0 -3 = -3 < 0.$$
	$$f\Big(\dfrac\pi6\Big) = 2\sin \dfrac\pi2 - \dfrac{\pi^2}{36} + 4\dfrac\pi6 - 3 \approx 2 - 0.25 +2-3 = 0.75 > 0.$$
	Функция на концах отрезка меняет знак, значит хотя бы один корень уравнения $f(x) = 0$ лежит в этом отрезке.\\\\
	Исследуем функцию на монотонность. Для этого нам нужно оценить значение производной на отрезке. Определим первую производную исследуемой функции: $$f'(x) = 6\cos3x - 2x + 4.$$ Причем эта функция непрерывна на отрезке $d$, то есть $f\in C\Big[0; \dfrac\pi6\Big]$, так как является результатом сложения непрерывных на этом отрезке функций.\\\\
	Тот факт, что производная не изменяет знак на отрезке докажем аналитически. Разобъем производную на две элементарные функции
	$$f'(x) = \underbrace{6\cos3x}_{z_1(x)} \underbrace{- 2x + 4}_{z_2(x)}.$$
	Берем отрезок $d = \Big[0; \dfrac\pi6\Big]$. 
	\\\\
	Функция $z_1(x) = 6\cos3x$ является на этом отрезке убывающей функцией по свойствам косинуса. Наибольшее значение $$y_1(0)= 6\cos0 = 6,$$ а наименьшее $$y_2\Big(\dfrac\pi6\Big) = 6\cos\dfrac\pi2 = 0.$$
	Таким образом, $z_1(x)$ является строго положительной функцией на отрезке $d$. 
	\\\\
	Рассмотрим функцию $z_2(x) = -2x + 4$. Она также является убывающей на отрезке $d$ функцией по свойствам линейной функции. Ее наибольшее значение $$z_2(0) = 4,$$ а наименьшее $$z_2\Big(\dfrac\pi6\Big) = -\dfrac\pi3 + 4\approx 3.$$
	То есть эта функция также является строго положительной на отрезке $d$. 
	\\\\
	В итоге функция $f'(x)$ состоит из суммы двух строго положительных на отрезке $d$ функций, а следовательно $$3 \leq f'(x) \leq 10 \Rightarrow f'(x) > 0\quad \forall x \in d=\Big[0; \dfrac\pi6\Big].$$ 
	Таким образом, мы доказали, что выбранный нами отрезок числовой прямой содержит ровно один корень исследуемого уравнения. На этом решение задачи отделения корней можно закончить и переходить к отысканию приближенных значений корней, лежащих в этих отрезках с помощью указанных методов.
\end{document}