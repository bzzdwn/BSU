\documentclass[a4paper, 12pt]{article}
\usepackage{cmap}
\usepackage{amssymb}
\usepackage{amsmath}
\usepackage{graphicx}
\usepackage{amsthm}
\usepackage{upgreek}
\usepackage{setspace}
\usepackage{color}
\usepackage{pgfplots}
\pgfplotsset{compat=1.9}
\usepackage[T2A]{fontenc}
\usepackage[utf8]{inputenc}
\usepackage[normalem]{ulem}
\usepackage{mathtext} % русские буквы в формулах
\usepackage[left=2cm,right=2cm, top=2cm,bottom=2cm,bindingoffset=0cm]{geometry}
\usepackage[english,russian]{babel}
\usepackage[unicode]{hyperref}
\newenvironment{Proof} % имя окружения
{\par\noindent{$\blacklozenge$}} % команды для \begin
{\hfill$\scriptstyle\boxtimes$}
\newcommand{\Rm}{\mathbb{R}}
\newcommand{\Cm}{\mathbb{C}}
\newcommand{\Z}{\mathbb{Z}}
\newcommand{\I}{\mathbb{I}}
\newcommand{\N}{\mathbb{N}}
\newcommand{\rank}{\operatorname{rank}}
\newcommand{\Ra}{\Rightarrow}
\newcommand{\ra}{\rightarrow}
\newcommand{\FI}{\Phi}
\newcommand{\Sp}{\text{Sp}}
\renewcommand{\leq}{\leqslant}
\renewcommand{\geq}{\geqslant}
\renewcommand{\alpha}{\upalpha}
\renewcommand{\beta}{\upbeta}
\renewcommand{\gamma}{\upgamma}
\renewcommand{\delta}{\updelta}
\renewcommand{\varphi}{\upvarphi}
\renewcommand{\phi}{\upvarphi}
\renewcommand{\tau}{\uptau}
\renewcommand{\lambda}{\uplambda}
\renewcommand{\psi}{\uppsi}
\renewcommand{\mu}{\upmu}
\renewcommand{\omega}{\upomega}
\renewcommand{\d}{\partial}
\renewcommand{\xi}{\upxi}
\renewcommand{\epsilon}{\upvarepsilon}
\newcommand{\intx}{\int\limits_{x_0}^x}
\newcommand\Norm[1]{\left\| #1 \right\|}
\newcommand{\sumk}{\sum\limits_{k=0}^\infty}
\newcommand{\sumi}{\sum\limits_{i=0}^\infty}
\newtheorem*{theorem}{Теорема}
\newtheorem*{cor}{Следствие}
\newtheorem*{lem}{Лемма}
\begin{document}
	\section*{Погрешность интерполирования при равноотстоящих узлах}
	\subsubsection*{Условие}
	Определить погрешность квадратичной интерполяции функции $f(x) = \ln(x+2)$ на равномерной сетке узлов $x_i \in [-1; 1]$ с шагом $h = 0.1$.
	\subsubsection*{Алгоритм решения}
	Для оценки погрешности интерполирования функции нам понадобятся следующие формулы:
	\begin{enumerate}
		\item остаток интерполирования при равноотстоящих узлах в начале таблицы \begin{eqnarray}
			r_k(x) =h^{k+1} \dfrac{t(t-1)\ldots (t-k)}{(k+1)!}f^{(k+1)}(\xi),\ \xi \in [x_0, x_0 + kh],\ t\in [0,1].
		\end{eqnarray}
		\item остаток интерполирования при равноотстоящих узлах в конце таблицы
		\begin{eqnarray}
			r_k(x) = h^{k+1}\dfrac{t(t+1)\ldots (t+k)}{(k+1)!}f^{(k+1)}(\xi),\ \xi \in [x_n, x_{n}-kh],\ t\in [-1,0].
		\end{eqnarray}
	\end{enumerate}
	Мы построим оценки для остатка интерполирования в начале таблицы и в конце таблицы и сравним полученные результаты.\\\\
	Сперва выпишем все данные, которые нам известны:\begin{itemize}
		\item интерполируемая функция $f(x) = \ln(x+2)$;
		\item сетка узлов на отрезке $[a,b] = [-1, 1]$;
		\item шаг $h = 0.1$;
		\item степень интерполяционного полинома $k=2$ (по условию квадратичная интерполяция).
	\end{itemize}
	Оценим остаток из формулы (1) при $k=2$:
	$$|r_2(x)| =\left|h^{3} \dfrac{t(t-1)(t-2)}{3!}f^{(3)}(\xi)\right|\leq h^{3}\left|\dfrac{t(t-1)(t-2)}{3!}\right|\cdot \underset{x\in[a,b]}{\max}|f^{(3)}(x)|,\ t\in [0,1].$$
	Оценим максимальное значение третьей производной на отрезке. Для этого вычислим третью производную от исходной функции $$f'(x) = \dfrac{1}{x+2},\quad f''(x) = -\dfrac{1}{(x+2)^2},\quad f'''(x) = \dfrac{2}{(x+2)^3}.$$
	Функция $\dfrac{2}{(x+2)^3}$ убывающая, следовательно, ее наибольшее значение в точке $x=-1$: $$\underset{x\in[-1,1]}{\max}|f^{(3)}(x)|= |f'''(-1)|= 2.$$
	Тогда, подставляя полученное значение и известное значение $h$ в формулу для оценки остатка, имеем $$|r_2(x)| \leq \dfrac{1}{3}\cdot10^{-3}\left|t(t-1)(t-2)\right|,\ t\in [0,1].$$
	Оценим значение выражения $\left|t(t-1)(t-2)\right|$. Для этого с помощью производной найдем точку максимума. Но сразу учтем, что $t\in [0,1]$ и при подстановке в это выражение точек $0$ и $1$ значение будет равно нулю, поэтому эти значения нас не интересуют. Рассмотрим функцию $$f(t) = t(t-1)(t-2) = t^3 - 3t^2 + 2t.$$
	Тогда $$f'(t) = 3t^2 - 6t + 2=0.$$
	Отсюда точки подозрительные на экстремумы $$t = 1\pm \dfrac{1}{\sqrt 3}.$$
	Но, так как $0<t<1$, то точка $1 + \dfrac{1}{\sqrt3}$ не подходит.
	Найдем значения в оставшейся точке
	$$f\left(1- \dfrac{1}{\sqrt 3}\right) = -\left(1- \dfrac{1}{\sqrt 3}\right)\cdot \dfrac{1}{\sqrt 3} \cdot \left(-1- \dfrac{1}{\sqrt 3}\right) = -\dfrac{2}{3\sqrt3}.$$
	Таким образом, $$\left|t(t-1)(t-2)\right|\leq \dfrac{2}{3\sqrt3}.$$
	Тогда мы можем вычислить оценку остатка интерполирования:
	$$|r_2(x)| \leq \dfrac{1}{3}\cdot \dfrac{2}{3\sqrt3}\cdot10^{-3}=\dfrac{2}{9\sqrt3}\cdot10^{-3}.$$
	Проделаем все то же самое для остатка интерполирования в конце таблицы из формулы (2): $$|r_2(x)| \leq \dfrac{1}{3}\cdot10^{-3}\left|t(t+1)(t+2)\right|,\ t\in [-1,0].$$
	То есть, нужно лишь оценить значение множителя с $t$. Снова точки $-1$, $0$ не рассматриваем. Рассмотрим функцию $$f(t) = t(t+1)(t+2) = t^3 + 3t^2 + 2t.$$
	Тогда $$f'(t) = 3t^2 + 6t + 2.$$
	Отсюда $$t = -1 \pm \dfrac{1}{\sqrt3}.$$
	Точка $-1 - \dfrac{1}{\sqrt3}$ не подходит. Вычислим значение в оставшейся точке: $$f\left(-1 + \dfrac{1}{\sqrt3}\right) = \left(-1 + \dfrac{1}{\sqrt3}\right)\cdot \dfrac{1}{\sqrt3}\cdot \left(1 + \dfrac{1}{\sqrt3}\right) = -\dfrac{2}{9\sqrt3}\cdot10^{-3}.$$
	То есть значение оценки остатка интерполирования будет таким же, что и в предыдущем случае.
\end{document} 