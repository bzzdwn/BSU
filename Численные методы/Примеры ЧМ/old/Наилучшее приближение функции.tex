\documentclass[a4paper, 12pt]{article}
\usepackage{cmap}
\usepackage{amssymb}
\usepackage{amsmath}
\usepackage{graphicx}
\usepackage{amsthm}
\usepackage{upgreek}
\usepackage{setspace}
\usepackage{color}
\usepackage{pgfplots}
\pgfplotsset{compat=1.9}
\usepackage[T2A]{fontenc}
\usepackage[utf8]{inputenc}
\usepackage[normalem]{ulem}
\usepackage{mathtext} % русские буквы в формулах
\usepackage[left=2cm,right=2cm, top=2cm,bottom=2cm,bindingoffset=0cm]{geometry}
\usepackage[english,russian]{babel}
\usepackage[unicode]{hyperref}
\newenvironment{Proof} % имя окружения
{\par\noindent{$\blacklozenge$}} % команды для \begin
{\hfill$\scriptstyle\boxtimes$}
\newcommand{\Rm}{\mathbb{R}}
\newcommand{\Cm}{\mathbb{C}}
\newcommand{\Z}{\mathbb{Z}}
\newcommand{\I}{\mathbb{I}}
\newcommand{\N}{\mathbb{N}}
\newcommand{\rank}{\operatorname{rank}}
\newcommand{\Ra}{\Rightarrow}
\newcommand{\ra}{\rightarrow}
\newcommand{\FI}{\Phi}
\newcommand{\Sp}{\text{Sp}}
\renewcommand{\leq}{\leqslant}
\renewcommand{\geq}{\geqslant}
\renewcommand{\alpha}{\upalpha}
\renewcommand{\beta}{\upbeta}
\renewcommand{\gamma}{\upgamma}
\renewcommand{\delta}{\updelta}
\renewcommand{\varphi}{\upvarphi}
\renewcommand{\phi}{\upvarphi}
\renewcommand{\tau}{\uptau}
\renewcommand{\lambda}{\uplambda}
\renewcommand{\psi}{\uppsi}
\renewcommand{\mu}{\upmu}
\renewcommand{\omega}{\upomega}
\renewcommand{\d}{\partial}
\renewcommand{\xi}{\upxi}
\renewcommand{\epsilon}{\upvarepsilon}
\newcommand{\intx}{\int\limits_{x_0}^x}
\newcommand\Norm[1]{\left\| #1 \right\|}
\newcommand{\sumk}{\sum\limits_{k=0}^\infty}
\newcommand{\sumi}{\sum\limits_{i=0}^\infty}
\newtheorem*{theorem}{Теорема}
\newtheorem*{cor}{Следствие}
\newtheorem*{lem}{Лемма}
\begin{document}
	\section*{Наилучшее приближение функции}
	\subsubsection*{Условия}
	\begin{enumerate}
		\item Построить наилучшее среднеквадратичное приближение к аналитически заданной функции с помощью алгебраического многочлена первой степени: $$f(x) = x^2,\quad x \in [1,2].$$ Оценить величину наилучшего приближения. (\hyperlink{t1}{Решение})
		\item Построить наилучшее равномерное приближение функции $f(x) = 2^x$, $x\in [-1,1]$ с помощью многочлена первой степени. Найти наилучшее приближение. (\hyperlink{t2}{Решение})
	\end{enumerate}
	
	\newpage
	\subsubsection*{Решения}
	\begin{enumerate}
		\item \hypertarget{t1}{}
		Наилучшее среднеквадратичное приближение алгебраическим многочленом строится в виде \begin{eqnarray}
			\varphi(x) = c_0 + c_1x + \ldots +c_nx^n,
		\end{eqnarray} где коэффициенты являются решениями СЛАУ
		\begin{eqnarray}
			\begin{cases}
				c_0s_0 + c_1s_1 + \ldots + c_ns_n = m_0,\\
				c_0s_1 + c_1s_2 + \ldots + c_ns_{n+1} = m_1,\\
				\dotfill\\
				c_0s_n + c_1s_{n+1} + \ldots + c_ns_{2n} = m_n.
			\end{cases}
		\end{eqnarray}
		\begin{eqnarray}
			s_i = \int\limits_a^b p(x) x^i dx,\quad m_j= \int\limits_a^b p(x) f(x) x^j dx,\quad i=\overline{0,2n}, j=\overline{0,n}.
		\end{eqnarray}
		В нашем случае формулы принимают вид \begin{eqnarray}
			\varphi(x) = c_0 + c_1x,
		\end{eqnarray}
		\begin{eqnarray}
			\begin{cases}
				c_0\int\limits_a^b p(x) dx+ c_1\int\limits_a^b p(x) x dx= \int\limits_a^b p(x)f(x)dx,\\
				c_0\int\limits_a^b p(x)x dx+ c_1\int\limits_a^b p(x) x^2 dx= \int\limits_a^b p(x)f(x)xdx.
			\end{cases}
		\end{eqnarray}
		По условию ничего не сказано про весовую функцию $p(x)$, поэтому принимаем $p(x) = 1$. Тогда, подставляя известные значения в (5), получаем систему вида 
		$$\begin{cases}
			c_0\int\limits_1^2  dx+ c_1\int\limits_1^2 x dx= \int\limits_1^2 x^2dx,\\
			c_0\int\limits_1^2 x dx+ c_1\int\limits_1^2 x^2 dx= \int\limits_1^2 x^3dx.
		\end{cases}$$
		Вычислим все необходимые интегралы $$\int\limits_1^2  dx = 1,\quad \int\limits_1^2 xdx = \dfrac32,\quad \int\limits_1^2 x^2 dx = \dfrac73,\quad \int\limits_1^2 x^3 dx = \dfrac{15}{4}.$$
		Подставим найденные значения в систему:
		$$\begin{cases}
			c_0+ \dfrac32c_1= \dfrac73,\\
			\dfrac32c_0+ c_1\dfrac73= \dfrac{15}{4}.
		\end{cases}$$
		Запишем СЛАУ в виде матрицы и применим метод Гаусса $$\begin{pmatrix}
			1 & \dfrac32 & \vline & \dfrac73\\\\
			\dfrac32 & \dfrac73 & \vline & \dfrac{15}{4}
		\end{pmatrix}
		\sim
		\begin{pmatrix}
			1 & 0 & \vline & -\dfrac{13}{6}\\
			0 & 1 & \vline & 3
		\end{pmatrix}
		$$
		Таким образом, $c_0 = 3$, $c_1 = -\dfrac{13}{6}$. Тогда приближающий многочлен первой степени имеет вид $$\varphi(x) = 3x - \dfrac{13}{6}.$$
		Величину наилучшего приближения оценим по формуле $$\Norm{f(x) - \varphi(x)} = \left(\int\limits_a^b(f(x) - \varphi(x))^2dx\right)^{\frac12}.$$
		Подставим наши функции и получим \begin{multline*}
			\left(\int\limits_1^2\left(x^2 - 3x + \dfrac{13}{6}\right)^2dx\right)^{\frac12} = \left(\int\limits_1^2x^4 + 9x^2+\dfrac{169}{36} - 6x^3 + \dfrac{13}{3}x^2 - 13xdx\right)^{\frac12} =\\= \left(\dfrac{x^5}{5}\Big|_1^2 -6\cdot \dfrac{x^4}{4}\Big|_1^2+ \dfrac{40}{3}\cdot\dfrac{x^3}{3}\Big|_1^2 - 13\cdot \dfrac{x^2}{2}\Big|_1^2 + \dfrac{169}{36}x\Big|_1^2\right)^{\frac12} = \left(\dfrac{1}{180}\right)^{\frac12} \approx 0.0745.
		\end{multline*}
		Графически это будет выглядеть следующим образом:
		\begin{center}\begin{tikzpicture}
				\begin{axis}[
					title = Function Approximation,
					legend pos = north west,
					xlabel = {$x$},
					ylabel = {$y$},
					minor tick num = 2,
					xmin = 1,
					xmax = 2,
					grid = major
					]
					\legend{$x^2$, $\varphi(x)$}
					\addplot[blue] {x^2};
					\addplot[orange] {3*x - 13/6};
				\end{axis}
		\end{tikzpicture}\end{center}
		
		\newpage
		\item 
		\hypertarget{t2}{}
		Все последующие действия справедливы лишь при предположении, что исходная функция выпуклая (по свойствам степенной функции).\\\\
		Для построения наилучшего равномерного приближения многочленом первой степени понадобятся следующие формулы:
		\begin{enumerate}
			\item многочлен наилучшего равномерного приближения в общем виде \begin{eqnarray}
				P_1(x) = c_0 + c_1x
			\end{eqnarray}
			\item необходимое и достаточное условие существования и единственности многочлена \begin{eqnarray}
				f(x_i) - P_n(x_i) = (-1)^i\alpha\Delta,\quad \Delta = \Norm{f(x) - P_n(x)},\quad i = 0,\ldots,n+1,
			\end{eqnarray}
			где $\alpha=1$ или $\alpha = -1$, а $x_i$ --- точки чебышевского альтернанса.
		\end{enumerate}
		Также необходимо определить точки чебышевского альтернанса $x_0,x_1,x_2$ (точки, в которых задана исходная функция, но которые находятся дальше всего от приближающего многочлена). Две из них (первую и последнюю) мы можем задать на концах:
		$$\begin{cases}
			x_0 = -1,\\
			x_2 = 1.
		\end{cases}$$
		Для оставшейся точки мы сформулируем условие следующим образом. Вследствие выпуклости функция $f(x) - P_n(x)$ может иметь только одну внутреннюю точку экстремума. Эту точку и возьмем в качестве оставшейся точки альтернанса. То есть, если функция $f(x)$ дифференцируема, то  \begin{eqnarray}
			f'(x_1) - P_1'(x_1) = 0.
		\end{eqnarray}
		Таким образом, имея 3 условия из (2) и условие (3), составляем систему:
		\begin{eqnarray}
			\begin{cases}
				f(x_0) - P_1(x_0) = \alpha\Delta,\\
				f(x_1) - P_1(x_1) = -\alpha\Delta,\\
				f(x_2) - P_1(x_2) = \alpha\Delta,\\
				f'(x_1) - P_1'(x_1) = 0.
			\end{cases}
		\end{eqnarray}
		Подставим известные нам значения: $$\begin{cases}
			f(-1) - (c_0 + c_1 \cdot (-1)) = \alpha\Delta,\\
			f(x_1) - (c_0 + c_1\cdot x_1) = -\alpha\Delta,\\
			f(1) - (c_0 + c_1 \cdot 1) = \alpha\Delta,\\
			f'(x_1) - c_1 = 0.
		\end{cases}\Rightarrow \begin{cases}
			\dfrac12 - (c_0 - c_1) = \alpha\Delta,\\
			2^{x_1} - (c_0 + c_1\cdot x_1) = -\alpha\Delta,\\
			2 - (c_0 + c_1) = \alpha\Delta,\\
			2^{x_1}\ln2 - c_1 = 0.
		\end{cases}$$
		Вычислим $c_1$, отняв от третьего уравнения первое:
		$$\dfrac{3}{2} - 2c_1 = 0 \Rightarrow c_1 = \dfrac34.$$
		Вычислим $x_1$, подставив в последнее уравнение значение $c_1$:
		$$x_1 = \log_2\dfrac{3}{4\ln 2}\approx 0.11373.$$
		Сложим второе и третье уравнение, чтобы найти $c_0$:
		$$\dfrac{3}{4\ln2} + 2 - 2c_0 - \dfrac34 -\dfrac34 \log_2\dfrac{3}{4\ln 2} = 0 \Rightarrow c_0 = \dfrac{3}{8\ln 2} + \dfrac58 - \dfrac{3}{8}\log_2\dfrac{3}{4\ln 2}\approx 1.12336.$$
		Остается найти $\alpha\Delta$. Мы можем найти это значение как из 1, так и из 3 уравнения. К примеру, возьмем третье уравнение:
		$$\alpha\Delta = 2 - \dfrac{3}{8\ln 2} - \dfrac58 + \dfrac{3}{8}\log_2\dfrac{3}{4\ln 2}\approx 0.87664.$$
		Соответственно $\alpha = 1$, $\Delta \approx 0.87664$ и многочлен наилучшего равномерного приближения имеет вид $$P_1(x) = 0.75x + 1.12336.$$ 
		Графически это будет выглядеть следующим образом:
		\begin{center}\begin{tikzpicture}
				\begin{axis}[
					title = Function Approximation,
					legend pos = north west,
					xlabel = {$x$},
					ylabel = {$y$},
					minor tick num = 2,
					xmin = -1,
					xmax = 1,
					grid = major
					]
					\legend{$2^x$, $P_1(x)$}
					\addplot[blue] {2^x};
					\addplot[orange] {0.75*x + 1.12336};
				\end{axis}
		\end{tikzpicture}\end{center}
	\end{enumerate}
	
\end{document} 