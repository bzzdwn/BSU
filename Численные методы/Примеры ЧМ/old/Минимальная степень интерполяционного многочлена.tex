\documentclass[a4paper, 12pt]{article}
\usepackage{cmap}
\usepackage{amssymb}
\usepackage{amsmath}
\usepackage{graphicx}
\usepackage{amsthm}
\usepackage{upgreek}
\usepackage{setspace}
\usepackage{color}
\usepackage{pgfplots}
\pgfplotsset{compat=1.9}
\usepackage[T2A]{fontenc}
\usepackage[utf8]{inputenc}
\usepackage[normalem]{ulem}
\usepackage{mathtext} % русские буквы в формулах
\usepackage[left=2cm,right=2cm, top=2cm,bottom=2cm,bindingoffset=0cm]{geometry}
\usepackage[english,russian]{babel}
\usepackage[unicode]{hyperref}
\newenvironment{Proof} % имя окружения
{\par\noindent{$\blacklozenge$}} % команды для \begin
{\hfill$\scriptstyle\boxtimes$}
\newcommand{\Rm}{\mathbb{R}}
\newcommand{\Cm}{\mathbb{C}}
\newcommand{\Z}{\mathbb{Z}}
\newcommand{\I}{\mathbb{I}}
\newcommand{\N}{\mathbb{N}}
\newcommand{\rank}{\operatorname{rank}}
\newcommand{\Ra}{\Rightarrow}
\newcommand{\ra}{\rightarrow}
\newcommand{\FI}{\Phi}
\newcommand{\Sp}{\text{Sp}}
\renewcommand{\leq}{\leqslant}
\renewcommand{\geq}{\geqslant}
\renewcommand{\alpha}{\upalpha}
\renewcommand{\beta}{\upbeta}
\renewcommand{\gamma}{\upgamma}
\renewcommand{\delta}{\updelta}
\renewcommand{\varphi}{\upvarphi}
\renewcommand{\phi}{\upvarphi}
\renewcommand{\tau}{\uptau}
\renewcommand{\lambda}{\uplambda}
\renewcommand{\psi}{\uppsi}
\renewcommand{\mu}{\upmu}
\renewcommand{\omega}{\upomega}
\renewcommand{\d}{\partial}
\renewcommand{\xi}{\upxi}
\renewcommand{\epsilon}{\upvarepsilon}
\newcommand{\intx}{\int\limits_{x_0}^x}
\newcommand\Norm[1]{\left\| #1 \right\|}
\newcommand{\sumk}{\sum\limits_{k=0}^\infty}
\newcommand{\sumi}{\sum\limits_{i=0}^\infty}
\newtheorem*{theorem}{Теорема}
\newtheorem*{cor}{Следствие}
\newtheorem*{lem}{Лемма}
\begin{document}
	\section*{Минимальная степень интерполяционного многочлена}
	\subsubsection*{Условие}
	Определить минимальную степень интерполяционного многочлена, гарантирующего при оптимальном распределении узлов на отрезке $[2;5]$ для интерполяционной функции $f(x) = \cos 2x$ величину погрешности $\epsilon \leq 10^{-5}$. Указать соответствующее распределение узлов.
	\subsubsection*{Алгоритм решения}
	Для решения данной задачи потребуются следующие формулы:
	\begin{enumerate}
		\item пусть функция $f(x)\in C^{n+1}[a,b]$ и для нее выполняется неравенство $$|f^{(n+1)}(x)|\leq M, \quad x\in [a,b],$$
		тогда погрешность интерполирования может быть оценена сверху следующим образом
		\begin{eqnarray}
			|r_n(x)| \leq \dfrac{M}{(n+1)!}\cdot \dfrac{(b-a)^{n+1}}{2^{2n+1}}.
		\end{eqnarray}
		\item значения узлов при оптимальном распределении на отрезке $[a,b]$
		\begin{eqnarray}
			x_k = \dfrac{a+b}{2} + \dfrac{b-a}{2}\cos \dfrac{(2k+1)\pi}{2(n+1)},\ k=\overline{0,n}.
		\end{eqnarray}
	\end{enumerate}
	Для отыскания степени $n$ многочлена интерполирования, будем решать неравенство $$|r_n(x)| \leq \dfrac{M}{(n+1)!}\cdot \dfrac{(b-a)^{n+1}}{2^{2n+1}}\leq \epsilon.$$
	Из-за того, что $n$ фигурирует и в качестве факториального значения, и в качестве степени, то решать уравнение придется подбором.\\\\
	Пусть $n=2$, тогда 
	$$|r_2(x)| \leq \dfrac{M}{3!}\cdot \dfrac{(5-2)^{3}}{2^{5}}\leq 10^{-5},\quad |f^{(3)}(x)|\leq M,\quad x\in [2,5]$$
	Оценим значение третьей производной от исходной функции:
	$$f'(x) = -2\sin 2x,\quad f''(x) =-4\cos2x ,\quad f'''(x) = 8\cos2x.$$
	Сделаем грубую оценку производной: $$|f'''(x)| = |8\cos2x| \leq 8 = M,\quad x\in [2;5].$$
	Тогда проверим, верное ли равенство:
	$$\dfrac{8}{6}\cdot \dfrac{27}{32}\leq 10^{-5}.$$
	Очевидно равенство не выполняется.\\\\
	Пусть $n=3$:
	$$|r_3(x)| \leq \dfrac{M}{4!}\cdot \dfrac{(5-2)^{4}}{2^{7}}\leq 10^{-5},\quad |f^{(4)}(x)|\leq M,\quad x\in [2,5]$$
	$$|f^{(4)}(x)| = |-16\sin2x| \leq 16 = M,\quad x\in [2;5].$$
	$$\dfrac{16}{24}\cdot \dfrac{81}{128}\leq 10^{-5}.$$
	Равенство не выполняется.
	\\\\
	Далее избежим оценки производной, считая, что $$|f^{(n+1)}(x)|\leq 2^{n+1}.$$
	Тогда $$|r_n(x)| \leq \dfrac{2^{n+1}}{(n+1)!}\cdot \dfrac{3^{n+1}}{2^{2n+1}} = \dfrac{1}{(n+1)!}\cdot \dfrac{3^{n+1}}{2^{n}}\leq \epsilon.$$
	И так далее подставляем $n=4,5,...,9$. При $n=11$ имеем $$|r_{10}(x)| \leq \dfrac{1}{11!}\cdot \dfrac{3^{11}}{2^{10}}\approx 4.33\cdot 10^{-6}.$$
	Таким образом, минимальная степень интерполяционного многочлена равна $10$.\\\\
	Укажем при этом распределение узлов по формуле (2):
	$$	x_k = \dfrac{7}{2} + \dfrac{3}{2}\cos \dfrac{(2k+1)\pi}{22},\ k=\overline{0,11}.$$
\end{document} 