\documentclass[a4paper, 12pt]{article}
\usepackage{cmap}
\usepackage{amssymb}
\usepackage{amsmath}
\usepackage{graphicx}
\usepackage{amsthm}
\usepackage{upgreek}
\usepackage{setspace}
\usepackage{color}
\usepackage{pgfplots}
\pgfplotsset{compat=1.9}
\usepackage[T2A]{fontenc}
\usepackage[utf8]{inputenc}
\usepackage[normalem]{ulem}
\usepackage{mathtext} % русские буквы в формулах
\usepackage[left=2cm,right=2cm, top=2cm,bottom=2cm,bindingoffset=0cm]{geometry}
\usepackage[english,russian]{babel}
\usepackage[unicode]{hyperref}
\newenvironment{Proof} % имя окружения
{\par\noindent{$\blacklozenge$}} % команды для \begin
{\hfill$\scriptstyle\boxtimes$}
\newcommand{\Rm}{\mathbb{R}}
\newcommand{\Cm}{\mathbb{C}}
\newcommand{\Z}{\mathbb{Z}}
\newcommand{\I}{\mathbb{I}}
\newcommand{\N}{\mathbb{N}}
\newcommand{\rank}{\operatorname{rank}}
\newcommand{\Ra}{\Rightarrow}
\newcommand{\ra}{\rightarrow}
\newcommand{\FI}{\Phi}
\newcommand{\Sp}{\text{Sp}}
\renewcommand{\leq}{\leqslant}
\renewcommand{\geq}{\geqslant}
\renewcommand{\alpha}{\upalpha}
\renewcommand{\beta}{\upbeta}
\renewcommand{\gamma}{\upgamma}
\renewcommand{\delta}{\updelta}
\renewcommand{\varphi}{\upvarphi}
\renewcommand{\phi}{\upvarphi}
\renewcommand{\tau}{\uptau}
\renewcommand{\lambda}{\uplambda}
\renewcommand{\psi}{\uppsi}
\renewcommand{\mu}{\upmu}
\renewcommand{\omega}{\upomega}
\renewcommand{\d}{\partial}
\renewcommand{\xi}{\upxi}
\renewcommand{\epsilon}{\upvarepsilon}
\newcommand{\intx}{\int\limits_{x_0}^x}
\newcommand\Norm[1]{\left\| #1 \right\|}
\newcommand{\sumk}{\sum\limits_{k=0}^\infty}
\newcommand{\sumi}{\sum\limits_{i=0}^\infty}
\newtheorem*{theorem}{Теорема}
\newtheorem*{cor}{Следствие}
\newtheorem*{lem}{Лемма}
\begin{document}
	\section*{Определить АСТ}
	\subsubsection*{Условие}
	Определить алгебраическую степень точности указанной квадратурной формулы
	$$\int\limits_a^b f(x)dx \approx \dfrac{b-a}{4}\left(f(a) + 3f\left(\dfrac{a+2b}{3}\right)\right)$$
	\subsubsection*{Алгоритм решения}
	Для построения для решения понадобятся следующие соотношения. Если \begin{eqnarray}
		\begin{cases}
			\int\limits_a^b \rho(x) x^idx = \sum\limits_{k=0}^{n}A_kx^i_k,\quad i=\overline{0,m},\\
			\int\limits_a^b \rho(x) x^{m+1}dx \ne \sum\limits_{k=0}^{n}A_kx^i_k;
		\end{cases}
	\end{eqnarray}
	то в этом случае говорят, что квадратурная формула \textbf{имеет алгебраическую степень точности равную \textit{m}}.\\\\
	Таким образом, для решения задачи необходимо строить по одному уравнению из соотношений (1) до тех пор, пока мы не получим неравенство.\\\\
	Теперь обратим внимание на составляющие элементы формулы (1), а именно $\rho(x), A_k, x_k$. Квадратурная формула записывается в общем случае в виде $$I(f) = \int\limits_a^b\rho(x)f(x)dx \approx A_0f(x_0) + A_1 f(x_1).$$
	Из условия следует, что $\rho(x) = 1$. Также из условия можно сделать вывод, что $$A_0 = \dfrac{b-a}{4},\quad A_1 = 3\cdot \dfrac{b-a}{4};\quad x_0 = a,\quad x_1 = \dfrac{a+2b}{3}.$$
	Начнем записывать соотношения из (1). Возьмем $i=0$, тогда $$x^0 : \int\limits_a^b dx = A_0 + A_1.$$
	Вычислим интеграл и подставим значения $A_k, x_k$ $$x^0 : b-a \overset{?}{=} \dfrac{b-a}{4} + 3\cdot \dfrac{b-a}{4} =4\cdot \dfrac{ b-a}{4}.$$
	Равенство выполняется. Далее по аналогии берем $i=1$:
	$$x^1 : \int\limits_a^bx dx = A_0x_0 + A_1x_1.$$
	Вычисляем значение интеграла и подставляем неизвестные:
	$$x^1 : \dfrac{b^2-a^2}{2} \overset{?}{=} \dfrac{b-a}{4}\left(a + 3\cdot \dfrac{a+2b}{3}\right) = 2\dfrac{b^2-a^2}{4}.$$
	Равенство выполняется. Берем $i=2$:
	$$x^2 : \int\limits_a^bx^2 dx = A_0x_0^2 + A_1x_1^2.$$
	$$x^2 : \dfrac{b^3-a^3}{3} \overset{?}{=} \dfrac{b-a}{4}\left(a^2 + 3\cdot \dfrac{(a+2b)^2}{9}\right) = \dfrac{b-a}{4}\cdot\left( \dfrac{4a^2 + 4ab + 4b^2}{3}\right) = \dfrac{b^3-a^3}{3}.$$
	Равенство выполняется. Берем $i=3$:
	$$x^3 : \int\limits_a^bx^3 dx = A_0x_0^3 + A_1x_1^3.$$
	$$x^2 : \dfrac{b^4-a^4}{4} \overset{?}{=} \dfrac{b-a}{4}\left(a^3 + 3\cdot \dfrac{(a+2b)^3}{27}\right) = \dfrac{b-a}{4}\cdot\left( \dfrac{10a^3+6a^2b + 12ab^2 + 8b^3}{9}\right).$$
	Отсюда уже видно, что равенство не выполняется, так как справа мы не получим как минимум $-\dfrac14 a^4$.\\\\
	Таким образом, берем последнюю степень, на которой выполнялось равенство. Это $i=2$. Тогда же и АСТ = 2.
\end{document}