\documentclass[a4paper, 12pt]{article}
\usepackage{cmap}
\usepackage{amssymb}
\usepackage{amsmath}
\usepackage{graphicx}
\usepackage{amsthm}
\usepackage{upgreek}
\usepackage{setspace}
\usepackage{color}
\usepackage[T2A]{fontenc}
\usepackage[utf8]{inputenc}
\usepackage[normalem]{ulem}
\usepackage{mathtext} % русские буквы в формулах
\usepackage[left=2cm,right=2cm, top=2cm,bottom=2cm,bindingoffset=0cm]{geometry}
\usepackage[english,russian]{babel}
\usepackage[unicode]{hyperref}
\newenvironment{Proof} % имя окружения
{\par\noindent{$\blacklozenge$}} % команды для \begin
{\hfill$\scriptstyle\boxtimes$}
\newcommand{\Rm}{\mathbb{R}}
\newcommand{\Cm}{\mathbb{C}}
\newcommand{\Z}{\mathbb{Z}}
\newcommand{\I}{\mathbb{I}}
\newcommand{\N}{\mathbb{N}}
\newcommand{\rank}{\operatorname{rank}}
\newcommand{\Ra}{\Rightarrow}
\newcommand{\ra}{\rightarrow}
\newcommand{\FI}{\Phi}
\newcommand{\Sp}{\text{Sp}}
\renewcommand{\leq}{\leqslant}
\renewcommand{\geq}{\geqslant}
\renewcommand{\alpha}{\upalpha}
\renewcommand{\beta}{\upbeta}
\renewcommand{\gamma}{\upgamma}
\renewcommand{\delta}{\updelta}
\renewcommand{\varphi}{\upvarphi}
\renewcommand{\phi}{\upvarphi}
\renewcommand{\tau}{\uptau}
\renewcommand{\lambda}{\uplambda}
\renewcommand{\psi}{\uppsi}
\renewcommand{\mu}{\upmu}
\renewcommand{\omega}{\upomega}
\renewcommand{\d}{\partial}
\renewcommand{\xi}{\upxi}
\renewcommand{\epsilon}{\upvarepsilon}
\newcommand{\intx}{\int\limits_{x_0}^x}
\newcommand\Norm[1]{\left\| #1 \right\|}
\newcommand{\sumk}{\sum\limits_{k=0}^\infty}
\newcommand{\sumi}{\sum\limits_{i=0}^\infty}
\newtheorem*{theorem}{Теорема}
\newtheorem*{cor}{Следствие}
\newtheorem*{lem}{Лемма}
\begin{document}
	\section*{Сплайн-интерполирование естественным кубическим сплайном}
	\subsubsection*{Условие}
	Построить естественный кубический сплайн для функции $y = f(x)$ заданной таблицей значений 
	\begin{center}\begin{tabular}[t]{|c|c|c|c|c|}
			\hline
			$x$ & 0 & 1 & 2 & 4 \\
			\hline
			$f(x)$ & 2 & 3 & 5 & 10 \\
			\hline
	\end{tabular}\end{center}
	Вычислить приближенное значение функции в точке $x=3$.
	\subsubsection*{Алгоритм решения}
	Для решения данной задачи нам понадобятся следующие формулы ($N$ -- количество узлов):\begin{enumerate}
		\item расстояние между $i$-ым и $(i-1)$-ым узлами \begin{eqnarray}
			h_i=x_i - x_{i-1},\qquad i=\overline{1,N}\label{1}
		\end{eqnarray}
		\item формула кубического сплайна\begin{multline}
			S_3(x) = M_{i-1}\dfrac{(x_i - x)^3}{6h_i} + M_{i}\dfrac{(x-x_{i-1})^3}{6h_i} + \left(f_i - M_i\dfrac{h_i^2}{6}\right)\dfrac{x-x_{i-1}}{h_i} +\\+ \left(f_{i-1} - M_{i-1}\dfrac{h_i^2}{6}\right)\dfrac{(x_i - x)}{h_i},\quad x\in [x_{i-1}, x_i],\ i = \overline{1,N}
		\end{multline}
		\item формулы для коэффициентов кубического сплайна
		\begin{multline}
			\dfrac{h_i}{6}M_{i-1} + \dfrac{h_i + h_{i+1}}{3}M_i + \dfrac{h_{i+1}}{6}M_{i+1} = \dfrac{f_{i+1} - f_i}{h_{i+1}} - \dfrac{f_i - f_{i-1}}{h_i},\quad i = \overline {1,N-1}
		\end{multline}
		\item естественные граничные условия для коэффициентов (так как не заданы значения производных) \begin{eqnarray}
			M_0 = 0,\quad M_N = 0.
		\end{eqnarray}
	\end{enumerate}
	Сначала по формуле (1) найдем расстояния между узлами:
	$$\begin{matrix}
		h_1 = x_1 - x_0 = 1,\\
		h_2 = x_2 - x_1 = 1,\\
		h_3 = x_3 - x_2 = 2.
	\end{matrix}$$
	Теперь составим по формулам (3) и (4) СЛАУ для коэффициентов кубического сплайна:
	$$\begin{cases}
		M_0 = 0,\\
		\dfrac{h_1}{6}M_{0} + \dfrac{h_1 + h_{2}}{3}M_1 + \dfrac{h_{2}}{6}M_{2} = \dfrac{f_{2} - f_1}{h_{2}} - \dfrac{f_1 - f_{0}}{h_1},\\
		\dfrac{h_2}{6}M_{1} + \dfrac{h_2 + h_{3}}{3}M_2 + \dfrac{h_{3}}{6}M_{3} = \dfrac{f_{3} - f_2}{h_{3}} - \dfrac{f_2 - f_{1}}{h_2},\\
		M_3 = 0.
	\end{cases}$$
	Подставим в эту систему значения ($h_i$ нам известны, $M_1 = M_3 = 0$):
	$$\begin{cases}
		M_0 = 0,\\
		\dfrac{1 + 1}{3}M_1 + \dfrac{2}{6}M_{2} = \dfrac{5 - 3}{1} - \dfrac{3 - 2}{1},\\\\
		\dfrac{1}{6}M_{1} + \dfrac{1+2}{3}M_2 = \dfrac{10 - 5}{2} - \dfrac{5 - 3}{1},\\
		M_3=0.
	\end{cases}
	\Rightarrow 
	\begin{cases}
		M_0 = 0,\\
		\dfrac{2}{3}M_1 + \dfrac{1}{3}M_{2} = 1,\\\\
		\dfrac{1}{6}M_{1} + M_2 = \dfrac12,\\
		M_3=0.
	\end{cases}$$
	Найдем методом Гаусса коэффициенты $M_1, M_2$:
	$$
	\begin{pmatrix}
		\frac23&\frac16&\vline&1\\
		\frac16&1&\vline&\frac12
	\end{pmatrix}
	\sim
	\begin{pmatrix}
		1&\frac14&\vline&\frac32\\
		0&\frac{23}{4}&\vline&\frac32
	\end{pmatrix}
	\sim 
	\begin{pmatrix}
		1&\frac14&\vline&\frac32\\
		0&1&\vline&\frac{6}{23}
	\end{pmatrix}
	\sim
	\begin{pmatrix}
		1&0&\vline&\frac{33}{23}\\
		0&1&\vline&\frac{6}{23}
	\end{pmatrix}$$
	То есть $$M_0 = 0,\quad M_1 = \dfrac{33}{23},\quad M_2 = \dfrac{6}{23},\quad M_3 = 0.$$
	У нас есть все необходимые значения для того, чтобы построить кубический сплайн, кроме $x_i$, $x_{i-1}$. По условию необходимо вычислить значение в точке $x=3$, она находится между узлами $x_2 = 2$ и $x_3 = 4$. Тогда кубический сплайн по формуле (2) мы будем строить на узлах $x_1, x_2$.\\\\
	В нашем случае формула (2) примет вид 
	\begin{multline*}
		S_3(x) = M_2\dfrac{(x_3 - x)^3}{6h_3} + M_{3}\dfrac{(x-x_2)^3}{6h_3} + \left(f_3 - M_3\dfrac{h_3^2}{6}\right)\dfrac{x-x_2}{h_3} +\\+ \left(f_2 - M_2\dfrac{h_3^2}{6}\right)\dfrac{(x_3 - x)}{h_3},\quad x\in [x_2, x_3].
	\end{multline*}
	Подставляем все известные нам значения:
	$$
	S_3(x) = \dfrac{6}{23}\cdot\dfrac{(4 - x)^3}{12} + 10\cdot\dfrac{x-2}{2} + \left(5 - \dfrac{6}{23}\cdot\dfrac{4}{6}\right)\dfrac{(4 - x)}{2},\quad x\in [2, 4].
	$$
	Сделаем некоторые преобразования для упрощения формулы
	$$
	S_3(x) = \dfrac{(4 - x)^3}{46} + \dfrac{119x}{46}-\dfrac{8}{23},\quad x\in [2, 4].
	$$
	Найдем значение в точке $x=3$:
	$$
	S_3(3) = \dfrac{1}{46} + \dfrac{357}{46} - \dfrac{8}{23} = \dfrac{171}{23}.
	$$
\end{document}