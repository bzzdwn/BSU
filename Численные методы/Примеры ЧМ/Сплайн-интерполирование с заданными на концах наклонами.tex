\documentclass[a4paper, 12pt]{article}
\usepackage{cmap}
\usepackage{amssymb}
\usepackage{amsmath}
\usepackage{graphicx}
\usepackage{amsthm}
\usepackage{upgreek}
\usepackage{setspace}
\usepackage{color}
\usepackage[T2A]{fontenc}
\usepackage[utf8]{inputenc}
\usepackage[normalem]{ulem}
\usepackage{mathtext} % русские буквы в формулах
\usepackage[left=2cm,right=2cm, top=2cm,bottom=2cm,bindingoffset=0cm]{geometry}
\usepackage[english,russian]{babel}
\usepackage[unicode]{hyperref}
\newenvironment{Proof} % имя окружения
{\par\noindent{$\blacklozenge$}} % команды для \begin
{\hfill$\scriptstyle\boxtimes$}
\newcommand{\Rm}{\mathbb{R}}
\newcommand{\Cm}{\mathbb{C}}
\newcommand{\Z}{\mathbb{Z}}
\newcommand{\I}{\mathbb{I}}
\newcommand{\N}{\mathbb{N}}
\newcommand{\rank}{\operatorname{rank}}
\newcommand{\Ra}{\Rightarrow}
\newcommand{\ra}{\rightarrow}
\newcommand{\FI}{\Phi}
\newcommand{\Sp}{\text{Sp}}
\renewcommand{\leq}{\leqslant}
\renewcommand{\geq}{\geqslant}
\renewcommand{\alpha}{\upalpha}
\renewcommand{\beta}{\upbeta}
\renewcommand{\gamma}{\upgamma}
\renewcommand{\delta}{\updelta}
\renewcommand{\varphi}{\upvarphi}
\renewcommand{\phi}{\upvarphi}
\renewcommand{\tau}{\uptau}
\renewcommand{\lambda}{\uplambda}
\renewcommand{\psi}{\uppsi}
\renewcommand{\mu}{\upmu}
\renewcommand{\omega}{\upomega}
\renewcommand{\d}{\partial}
\renewcommand{\xi}{\upxi}
\renewcommand{\epsilon}{\upvarepsilon}
\newcommand{\intx}{\int\limits_{x_0}^x}
\newcommand\Norm[1]{\left\| #1 \right\|}
\newcommand{\sumk}{\sum\limits_{k=0}^\infty}
\newcommand{\sumi}{\sum\limits_{i=0}^\infty}
\newtheorem*{theorem}{Теорема}
\newtheorem*{cor}{Следствие}
\newtheorem*{lem}{Лемма}
\begin{document}
	\section*{Сплайн-интерполирование c заданными на концах наклонами}
	\subsubsection*{Условие}
	Построить кубический сплайн для функции $y = f(x)$ заданной таблицей значений 
	\begin{center}\begin{tabular}[t]{|c|c|c|c|}
		\hline
		$x$ & 0 & 2 & 3 \\
		\hline
		$f(x)$ & 1 & 5 & 7 \\
		\hline
		$f'(x)$ & 1 & - & 0 \\
		\hline
	\end{tabular}\end{center}
	Вычислить приближенное значение функции в точке $x=1$.
	\subsubsection*{Алгоритм решения}
	Для решения данной задачи нам понадобятся следующие формулы ($N$ -- количество узлов):\begin{enumerate}
		\item расстояние между $i$-ым и $(i-1)$-ым узлами \begin{eqnarray}
			h_i=x_i - x_{i-1},\qquad i=\overline{1,N}\label{1}
		\end{eqnarray}
		\item формула кубического сплайна\begin{multline}
			S_3(x) = M_{i-1}\dfrac{(x_i - x)^3}{6h_i} + M_{i}\dfrac{(x-x_{i-1})^3}{6h_i} + \left(f_i - M_i\dfrac{h_i^2}{6}\right)\dfrac{x-x_{i-1}}{h_i} +\\+ \left(f_{i-1} - M_{i-1}\dfrac{h_i^2}{6}\right)\dfrac{(x_i - x)}{h_i},\quad x\in [x_{i-1}, x_i],\ i = \overline{1,N}
		\end{multline}
		\item формулы для коэффициентов кубического сплайна
		\begin{multline}
			\dfrac{h_i}{6}M_{i-1} + \dfrac{h_i + h_{i+1}}{3}M_i + \dfrac{h_{i+1}}{6}M_{i+1} = \dfrac{f_{i+1} - f_i}{h_{i+1}} - \dfrac{f_i - f_{i-1}}{h_i},\quad i = \overline {1,N-1}
		\end{multline}
		\item граничные условия для коэффициентов \begin{eqnarray}
			\begin{cases}
				2M_0 + M_1 = \dfrac{6}{h_1}\left(\dfrac{f_1-f_0}{h_1} - f'(a)\right),\\
				M_{N-1} + 2M_N = \dfrac{6}{h_N}\left(f'(b) - \dfrac{f_N-f_{N-1}}{h_N} \right).
			\end{cases}
		\end{eqnarray}
	\end{enumerate}
	Сначала по формуле (1) найдем расстояния между узлами:
	$$\begin{matrix}
		h_1 = x_1 - x_0 = 2\\
		h_2 = x_2 - x_1 = 1.
	\end{matrix}$$
	Теперь составим по формулам (3) и (4) СЛАУ для коэффициентов кубического сплайна:
	$$\begin{cases}
		2M_0 + M_1 =  \dfrac{6}{h_1}\left(\dfrac{f_1-f_0}{h_1} - f'(a)\right),\\
		\dfrac{h_1}{6}M_{0} + \dfrac{h_1 + h_{2}}{3}M_1 + \dfrac{h_{2}}{6}M_{2} = \dfrac{f_{2} - f_1}{h_{2}} - \dfrac{f_1 - f_{0}}{h_1},\\
		M_{1} + 2M_2 = \dfrac{6}{h_2}\left(f'(b) - \dfrac{f_2-f_{1}}{h_2} \right).
	\end{cases}$$
	Подставим в эту систему значения ($h_i$ нам известны, $f_i$ и $f'(a)$, $f'(b)$ берем из заданной таблицы):
	$$\begin{cases}
		2M_0 + M_1 =  \dfrac{6}{2}\left(\dfrac{5-1}{2} - 1\right),\\
		\dfrac{2}{6}M_{0} + \dfrac{2 +1}{3}M_1 + \dfrac{1}{6}M_{2} = \dfrac{7 -5}{1} - \dfrac{5 - 1}{2},\\
		M_{1} + 2M_2 = \dfrac{6}{1}\left(0 - \dfrac{7-5}{1} \right).
	\end{cases}
	\Rightarrow 
	\begin{cases}
	2M_0 + M_1 =  3,\\
	\dfrac{1}{3}M_{0} + M_1 + \dfrac{1}{6}M_{2} = 0,\\
	M_{1} + 2M_2 = -12.
	\end{cases}$$
	Найдем решение методом Гаусса 
	$$\begin{pmatrix}
		2&1&0&\vline&3\\
		2&6&1&\vline&0\\
		0&1&2&\vline&-12
	\end{pmatrix}
	\sim 
	\begin{pmatrix}
	1&\frac12&0&\vline&\frac32\\
	0&5&1&\vline&-3\\
	0&1&2&\vline&-12
	\end{pmatrix}
	\sim
	\begin{pmatrix}
		1&\frac12&0&\vline&\frac32\\
		0&1&2&\vline&-12\\
		0&0&-9&\vline&-57
	\end{pmatrix}
	\sim
	\begin{pmatrix}
		1&0&0&\vline&\frac76\\
		0&1&0&\vline&\frac{2}{3}\\
		0&0&1&\vline&-\frac{19}{3}
	\end{pmatrix}$$
	То есть $$M_0 = \dfrac76,\quad M_1 = \dfrac23,\quad M_2 = -\dfrac{19}{3}.$$
	У нас есть все необходимые значения для того, чтобы построить кубический сплайн, кроме $x_i$, $x_{i-1}$. По условию необходимо вычислить значение в точке $x=1$, она находится между узлами $x_0 = 0$ и $x_1 = 2$. Тогда кубический сплайн по формуле (2) мы будем строить на узлах $x_0, x_1$.\\\\
	В нашем случае формула (2) примет вид \begin{multline*}
		S_3(x) = M_0\dfrac{(x_1 - x)^3}{6h_1} + M_{1}\dfrac{(x-x_0)^3}{6h_1} + \left(f_1 - M_1\dfrac{h_1^2}{6}\right)\dfrac{x-x_0}{h_1} +\\+ \left(f_0 - M_0\dfrac{h_1^2}{6}\right)\dfrac{(x_1 - x)}{h_1},\quad x\in [x_0, x_1].
	\end{multline*}
	Подставляем все известные нам значения:
	$$
		S_3(x) = \dfrac76\dfrac{(2 - x)^3}{12} + \dfrac23\cdot\dfrac{x^3}{12} + \left(5 - \dfrac23\cdot\dfrac{4}{6}\right)\dfrac{x}{2} + \left(1 - \dfrac76\cdot\dfrac{4}{6}\right)\dfrac{(2 - x)}{2},\quad x\in [0, 2].
	$$
	Сделаем некоторые преобразования для упрощения формулы
	$$
	S_3(x) = \dfrac{7(2 - x)^3}{72} + \dfrac{x^3}{18} + \dfrac{22x}{9} + \dfrac{(2 - x)}{9},\quad x\in [0, 2].
	$$
	Найдем значение в точке $x=1$:
	$$
	S_3(1) = \dfrac{7}{72} + \dfrac{1}{18} + \dfrac{22}{9} + \dfrac{1}{9} = \dfrac{195}{72}.
	$$
\end{document}