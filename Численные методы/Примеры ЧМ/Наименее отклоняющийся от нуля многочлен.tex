\documentclass[a4paper, 12pt]{article}
\usepackage{cmap}
\usepackage{amssymb}
\usepackage{amsmath}
\usepackage{graphicx}
\usepackage{amsthm}
\usepackage{upgreek}
\usepackage{setspace}
\usepackage{color}
\usepackage{pgfplots}
\pgfplotsset{compat=1.9}
\usepackage[T2A]{fontenc}
\usepackage[utf8]{inputenc}
\usepackage[normalem]{ulem}
\usepackage{mathtext} % русские буквы в формулах
\usepackage[left=2cm,right=2cm, top=2cm,bottom=2cm,bindingoffset=0cm]{geometry}
\usepackage[english,russian]{babel}
\usepackage[unicode]{hyperref}
\newenvironment{Proof} % имя окружения
{\par\noindent{$\blacklozenge$}} % команды для \begin
{\hfill$\scriptstyle\boxtimes$}
\newcommand{\Rm}{\mathbb{R}}
\newcommand{\Cm}{\mathbb{C}}
\newcommand{\Z}{\mathbb{Z}}
\newcommand{\I}{\mathbb{I}}
\newcommand{\N}{\mathbb{N}}
\newcommand{\rank}{\operatorname{rank}}
\newcommand{\Ra}{\Rightarrow}
\newcommand{\ra}{\rightarrow}
\newcommand{\FI}{\Phi}
\newcommand{\Sp}{\text{Sp}}
\renewcommand{\leq}{\leqslant}
\renewcommand{\geq}{\geqslant}
\renewcommand{\alpha}{\upalpha}
\renewcommand{\beta}{\upbeta}
\renewcommand{\gamma}{\upgamma}
\renewcommand{\delta}{\updelta}
\renewcommand{\varphi}{\upvarphi}
\renewcommand{\phi}{\upvarphi}
\renewcommand{\tau}{\uptau}
\renewcommand{\lambda}{\uplambda}
\renewcommand{\psi}{\uppsi}
\renewcommand{\mu}{\upmu}
\renewcommand{\omega}{\upomega}
\renewcommand{\d}{\partial}
\renewcommand{\xi}{\upxi}
\renewcommand{\epsilon}{\upvarepsilon}
\newcommand{\intx}{\int\limits_{x_0}^x}
\newcommand\Norm[1]{\left\| #1 \right\|}
\newcommand{\sumk}{\sum\limits_{k=0}^\infty}
\newcommand{\sumi}{\sum\limits_{i=0}^\infty}
\newtheorem*{theorem}{Теорема}
\newtheorem*{cor}{Следствие}
\newtheorem*{lem}{Лемма}
\begin{document}
	\section*{Наименее отклоняющийся от нуля многочлен}
	\subsubsection*{Условие}
	Среди многочленов вида $$P_3(x) = ax^3 + 3x^2 + bx+c$$ найти наименее отклоняющийся от нуля на отрезке $[1,5]$.
	\subsubsection*{Алгоритм решения}
	Для решения данной задачи потребуются следующие формулы:
	\begin{enumerate}
		\item многочлены Чебышева $T_n(x)$, $n\geq 0$ определенные на отрезке $[-1,1]$, задающиеся соотношениями 
		\begin{eqnarray}
			T_0(x)=1,\ T_1(x) = x,\ T_{n+1}(x) = 2xT_n(x) - T_{n-1}(x),\quad n=1,2,\ldots.
		\end{eqnarray}
		\item вид многочленов Чебышева на отрезке $[a,b]$ 
		\begin{eqnarray}
			\hat T_{n+1}(x) = \dfrac{(b-a)^{n+1}}{2^{2n+1}} T_{n+1}\Big(\dfrac{2x - (b+a)}{b-a}\Big),\ x\in [a,b].
		\end{eqnarray}
	\end{enumerate}
	Известно также, что многочлены Чебышева являются наименее отклоняющимися от нуля многочленами степени $n$ на отрезке $[-1,1]$ среди всех многочленов степени $n$ заданных на этом отрезке.\\\\
	Таким образом, нам необходимо, используя формулы (1) и (2), задать многочлен Чебышева на отрезке $[1,5]$, после чего привести его к нужному виду (чтобы коэффициент при $x^2$ был равен 3).\\\\
	Из соотношений (1) выясним, какой вид имеет многочлен Чебышева 3-ей степени:
	$$T_2(x) = 2x^2 - 1,\quad T_3(x) = 4x^2 - 2x - x = 4x^3 - 3x.$$
	Теперь в формулу (2) подставим отрезок $[a,b] = [1,5]$:
	$$T_{n+1}(x) = \dfrac{4^{n+1}}{2^{2n + 1}}\cdot T_{n+1}\left(\dfrac{2x - 6}{4}\right).$$
	Подставим в формулу (2) $n=2$:
	$$\hat T_{3}(x) = \dfrac{4^{3}}{2^{5}}\cdot T_{3}\left(\dfrac{2x - 6}{4}\right)=2T_{3}\left(\dfrac{2x - 6}{4}\right).$$
	Найдем $T_{3}\left(\dfrac{2x - 6}{4}\right)$:
	\begin{multline*}
		T_{3}\left(\dfrac{2x - 6}{4}\right) = 4\left(\dfrac{x}{2} - \dfrac{3}{2}\right)^3 - 3\left(\dfrac{x}{2} - \dfrac{3}{2}\right) = 4\left(\dfrac{x^3}{8} - \dfrac{9x^2}{8} + \dfrac{27x}{8}-\dfrac{27}{8}\right) - \dfrac{3x}{2} + \dfrac{9}{2} =\\ = \dfrac{x^3}{2} - \dfrac{9x^2}{2} + \dfrac{24x}{2} - \dfrac{18}{2}.
	\end{multline*}
	Тогда $$\hat T_3(x) = x^3 - 9x^2 + 24x - 18,\quad x\in [1,5].$$
	Мы получили многочлен наименее отклоняющийся от нуля на отрезке $[1,5]$. Чтобы он удовлетворял указанному виду, домножим его на $-\dfrac13$:
	$$P_3(x) = -\dfrac13 x^3 + 3x^2 - 8x + 6.$$
\end{document} 