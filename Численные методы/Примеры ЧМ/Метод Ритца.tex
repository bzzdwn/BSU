\documentclass[a4paper, 12pt]{article}
\usepackage{cmap}
\usepackage{amssymb}
\usepackage{amsmath}
\usepackage{graphicx}
\usepackage{amsthm}
\usepackage{upgreek}
\usepackage{setspace}
\usepackage{mathtools}
\usepackage{color}
\usepackage{pgfplots}
\pgfplotsset{compat=1.9}
\usepackage[T2A]{fontenc}
\usepackage[utf8]{inputenc}
\usepackage[normalem]{ulem}
\usepackage{mathtext} % русские буквы в формулах
\usepackage[left=2cm,right=2cm, top=2cm,bottom=2cm,bindingoffset=0cm]{geometry}
\usepackage[english,russian]{babel}
\usepackage[unicode]{hyperref}
\newenvironment{Proof} % имя окружения
{\par\noindent{$\blacklozenge$}} % команды для \begin
{\hfill$\scriptstyle\boxtimes$}
\newcommand{\Rm}{\mathbb{R}}
\newcommand{\Cm}{\mathbb{C}}
\newcommand{\Z}{\mathbb{Z}}
\newcommand{\I}{\mathbb{I}}
\newcommand{\N}{\mathbb{N}}
\newcommand{\rank}{\operatorname{rank}}
\newcommand{\Ra}{\Rightarrow}
\newcommand{\ra}{\rightarrow}
\newcommand{\FI}{\Phi}
\newcommand{\Sp}{\text{Sp}}
\renewcommand{\leq}{\leqslant}
\renewcommand{\geq}{\geqslant}
\renewcommand{\alpha}{\upalpha}
\renewcommand{\beta}{\upbeta}
\renewcommand{\gamma}{\upgamma}
\renewcommand{\delta}{\updelta}
\renewcommand{\varphi}{\upvarphi}
\renewcommand{\phi}{\upvarphi}
\renewcommand{\tau}{\uptau}
\renewcommand{\lambda}{\uplambda}
\renewcommand{\psi}{\uppsi}
\renewcommand{\mu}{\upmu}
\renewcommand{\omega}{\upomega}
\renewcommand{\d}{\partial}
\renewcommand{\xi}{\upxi}
\renewcommand{\epsilon}{\upvarepsilon}
\newcommand{\intx}{\int\limits_{x_0}^x}
\newcommand\Norm[1]{\left\| #1 \right\|}
\newcommand{\sumk}{\sum\limits_{k=0}^\infty}
\newcommand{\sumi}{\sum\limits_{i=0}^\infty}
\newtheorem*{theorem}{Теорема}
\newtheorem*{cor}{Следствие}
\newtheorem*{lem}{Лемма}
\begin{document}
	\section*{Метод Ритца}
	\subsubsection*{Условие}
	Методом Ритца при $n=2$ найти решение следующей задачи
	$$\begin{cases}
		u''(x) - xu(x) = x,\ 0\leq x \leq 1,\\
		u(0) = 0,\\
		u(1) = 1.
	\end{cases}$$
	\subsubsection*{Решение}
	Решение задачи ищем в виде 
	$$u_n(x) = \varphi_0(x) + \sum_{i=1}^{n}a_i \varphi_i(x).$$
	Поскольку $n=2$, то имеем $$u_2(x) = \varphi_0(x) + a_1 \varphi_1(x) + a_2 \varphi_2(x).$$
	В качестве системы функций $\{\varphi_i(x)\}$ будем использовать алгебраический базис, но со смещением, чтобы эти функции удовлетворяли граничным условиям. \\\\
	Так как граничные условия первого рода, но неоднородные, то функцию $\varphi_0(x)$ мы выбираем таким образом, чтобы она «вобрала» в себя неоднородность. Строим эту функцию в виде $$\varphi_0(x) = C_0 x + C_1,$$ где $C_0$, $C_1$ --- некоторые константы. Эта функция должна удовлетворять граничным условиям. Тогда подставим ее в граничные условия и получим
	$$\begin{cases}
		\varphi_0(0) = C_0 \cdot 0 + C_1 = 0,\\
		\varphi_0(1) = C_0 + C_1 = 1;
	\end{cases}\Rightarrow \begin{cases}
	C_0 = 1,\\
	C_1 = 0.
	\end{cases}$$
	Таким образом, построили функцию $$\varphi_0(x) = x.$$
	Остальные функции строим в виде $$\varphi_i(x) = (x-a)^i (x-b).$$
	Тогда 
	$$\varphi_1(x) = x(x-1),\ \varphi_2(x) = x^2 (x-1).$$
	Для отыскания коэффициентов $a_{ij}$ строим систему вида 
	$$\sum_{j=1}^{n}c_{ij}a_j = d_i,\ i = \overline {1,n},$$
	$$c_{ij} = \int\limits_a^b (k(x)\varphi_j ' \varphi _i ' + q(x) \varphi_j \varphi_i)dx,\ d_i = -\int_a^b (f\varphi_i + k(x)\varphi_0' \varphi_i' + q(x)\varphi_0 \varphi_i)dx.$$
	В нашем случае эту систему образуют 2 уравнения
	$$\begin{cases}
		c_{11} a_1 + c_{12} a_2 = d_1,\\
		c_{21} a_2 + c_{22} a_2 = d_2.
	\end{cases}$$
	Для вычисления значений $c_{ij}$, $d_i$ нам требуется информация о $k(x), q(x), f(x)$ и $\varphi'_i(x)$. Определим эти значения.\\\\
	Исходная задача ставится для дифференциального оператора 
	$$Lu \equiv -(k(x) u'(x))' + q(x)u)x = -f(x).$$
	Таким образом, из исходного уравнения имеем 
	$$k(x) = 1,\ q(x) = x,\ f(x) = x.$$
	Найдем производные от базисных функций:
	$$\varphi_0'(x) = 0,$$
	$$\varphi_1'(x) = 2x - 1,$$
	$$\varphi_2'(x) = x(3x - 2).$$
	Также вычислим следующие значения:
	$$\varphi_1'(x)\cdot \varphi_2'(x) = (2x-1)x(3x-2) = 6x^3 - 7x^2 + 2x.$$
	Таким образом, коэффициенты для системы получаем следующие:
	$$c_{11} = \int\limits_0^1 \left[(2x-1)^2 + x^3 (x-1)^2\right]dx = \dfrac{7}{20},$$
	$$c_{12} = c_{21} = \int\limits_0^1 \left[(2x-1)(3x-2)x + x^4 (x-1)^2\right]dx =\dfrac{37}{210},$$
	$$c_{22} = \int\limits_0^1 \left[x^2(3x-2)^2 + x^5 (x-1)^2\right]dx = \dfrac{39}{280},$$
	$$d_1 = -\int\limits_0^1 \left[x^2(x-1) + 2x-1+ x^3(x-1)\right]dx = \dfrac{2}{15},$$
	$$d_2 = -\int\limits_0^1 \left[x^3(x-1) + x(3x-2)+ x^4(x-1)\right]dx = \dfrac{1}{12}.$$
	В итоге, подставляя эти значения в систему для отыскания $a_i$, получим
	$$\begin{dcases}
		\dfrac{7}{20} a_1 + \dfrac{37}{210} a_2 = \dfrac{2}{15},\\
		\dfrac{37}{210} a_1 + \dfrac{39}{280} a_2 = \dfrac{1}{12}.
	\end{dcases}$$
	Решая эту систему точным или приближенным методом, получим 
	$$a_1 = \dfrac{1372}{6247}\approx 0.2196,\ a_2 = \dfrac{2002}{6247} \approx 0.3204.$$
	Таким образом, приближенное решение поставленной задачи имеет вид
	$$u_2(x) = x + 0.2196 x(x-1) + 0.3204 x^2(x-1).$$
\end{document} 