\documentclass[a4paper, 12pt]{article}
\usepackage{cmap}
\usepackage{amssymb}
\usepackage{amsmath}
\usepackage{graphicx}
\usepackage{amsthm}
\usepackage{upgreek}
\usepackage{setspace}
\usepackage{color}
\usepackage[T2A]{fontenc}
\usepackage[utf8]{inputenc}
\usepackage[normalem]{ulem}
\usepackage{mathtext} % русские буквы в формулах
\usepackage[left=2cm,right=2cm, top=2cm,bottom=2cm,bindingoffset=0cm]{geometry}
\usepackage[english,russian]{babel}
\usepackage[unicode]{hyperref}
\newenvironment{Proof} % имя окружения
{\par\noindent{$\blacklozenge$}} % команды для \begin
{\hfill$\scriptstyle\boxtimes$}
\newcommand{\Rm}{\mathbb{R}}
\newcommand{\Cm}{\mathbb{C}}
\newcommand{\Z}{\mathbb{Z}}
\newcommand{\I}{\mathbb{I}}
\newcommand{\N}{\mathbb{N}}
\newcommand{\rank}{\operatorname{rank}}
\newcommand{\Ra}{\Rightarrow}
\newcommand{\ra}{\rightarrow}
\newcommand{\FI}{\Phi}
\newcommand{\Sp}{\text{Sp}}
\renewcommand{\leq}{\leqslant}
\renewcommand{\geq}{\geqslant}
\renewcommand{\alpha}{\upalpha}
\renewcommand{\beta}{\upbeta}
\renewcommand{\gamma}{\upgamma}
\renewcommand{\delta}{\updelta}
\renewcommand{\varphi}{\upvarphi}
\renewcommand{\phi}{\upvarphi}
\renewcommand{\tau}{\uptau}
\renewcommand{\lambda}{\uplambda}
\renewcommand{\psi}{\uppsi}
\renewcommand{\mu}{\upmu}
\renewcommand{\omega}{\upomega}
\renewcommand{\d}{\partial}
\renewcommand{\xi}{\upxi}
\renewcommand{\epsilon}{\upvarepsilon}
\newcommand{\intx}{\int\limits_{x_0}^x}
\newcommand\Norm[1]{\left\| #1 \right\|}
\newcommand{\sumk}{\sum\limits_{k=0}^\infty}
\newcommand{\sumi}{\sum\limits_{i=0}^\infty}
\newtheorem*{theorem}{Теорема}
\newtheorem*{cor}{Следствие}
\newtheorem*{lem}{Лемма}
\begin{document}
	\section*{Решение СНУ}
	Рассмотрим СНУ при $n=2$ вида $$f(x) = 0,\quad f=(f_1, f_2)^T,\ x=(x_1,x_2)^T.$$
	$$\begin{cases}
		f_1(x_1,x_2) = 2x_1 - \sin \dfrac{x_1-x_2}{2} = 0,\\
		f_2(x_1,x_2) = 2x_2 - \cos\dfrac{x_1+x_2}{2} = 0.
	\end{cases}$$
	\begin{itemize}
		\item \textbf{Отделение корней.}
		\begin{itemize}
			\item \textbf{Графический метод.}
			$x^* \in [-1, 0]\times[0, 1]$, берем середину отрезка $x^0 = (-0.5, 0.5)$.
			\item \textbf{"Перемена знака".}
			Аналог метода дихотомии. Мы выбираем множетсво точек, в которых одна из функции обращается в ноль. Одну из этих точек подставляем во вторую функцию. Легко видеть, что
			
			$x^* \in [-\pi ,\pi ]\times 0$, $x^0 = (0,0)$.
			
		\end{itemize}
		\item Построение итерационной последовательности. 
		\begin{itemize}
			\item \textbf{Метод простой итерации.}
			Этот метод требует приведения системы к каноническому виду
			$$\begin{cases}
				x_1 = \varphi_1(x_1,x_2),
				x_2 = \varphi_2(x_1,x_2).
			\end{cases}$$
			а затем использование формулы метода простых итераций $x^{k+1} = \varphi(x^k)$. В данном случае мы можем взять $$\begin{cases}
			\varphi_1 = \dfrac{1}{2}\sin \dfrac{x_1-x_2}{2},\\
			\varphi_2 = \dfrac{1}{2}\cos \dfrac{x_1+x_2}{2}.
			\end{cases}$$
			Необходимо проверить достаточное условие сходимости:
			$$\underset{1\leq i \leq 2}{\max}\underset{|x-x^0|\leq \delta }{\max}\{\Big|\dfrac{\d \phi_i}{\d x_1}\Big| + \Big|\dfrac{\d \phi_i}{\d x_2}\Big|\}\leq q = \dfrac{1}{2}< 1$$
			В случае $x_0 = (-0.5, 0.5)$ получим $\delta = \dfrac{1}{2}$.
			\item \textbf{Метод Ньютона.}
			$$\Big(\dfrac{\d f(x^k)}{\d x}\Big)(x^{k+1} - x^k) = -f(x^k).$$
			Необходимым условием реализации данного метода является существование обратной матрицы $$J_k = \begin{vmatrix}
			\frac{\d f_1^k}{\d x_1} & \frac{\d f_1^k}{\d x_2}\\
			\frac{\d f_2^k}{\d x_1} & \frac{\d f_2^k}{\d x_2}			\end{vmatrix}\ne 0.$$
			$$x_1^{k+1} = x_1^k - \dfrac{\begin{vmatrix}
					f_1^k & \frac{\d f_1^k}{\d x_2} \\ f_2^k& \frac{\d f_2^k}{\d x_2} 
			\end{vmatrix}}{J_k}$$
		$$x_2^{k+1} = x_2^k - \dfrac{\begin{vmatrix}
				\frac{\d f_1^k}{\d x_1}&f_1^k  \\ \frac{\d f_2^k}{\d x_1} &f_2^k
		\end{vmatrix}}{J_k}$$
		\end{itemize}
		\item \textbf{Контроль сходимости}
		$$\Delta ^k = \Norm {x^{k+1} - x^k} \leq \epsilon$$
		Оценка невязки:
		$$\Norm{r^k} = \Norm {f^k} \leq \epsilon.$$
	\end{itemize}
	Возьмем случай $\epsilon = 10^{-2}$, $x^0 = \Big(-\dfrac{1}{2}; \dfrac{1}{2}\Big)$.
\end{document}