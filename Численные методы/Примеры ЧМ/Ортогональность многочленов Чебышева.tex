\documentclass[a4paper, 12pt]{article}
\usepackage{cmap}
\usepackage{amssymb}
\usepackage{amsmath}
\usepackage{graphicx}
\usepackage{amsthm}
\usepackage{upgreek}
\usepackage{setspace}
\usepackage{color}
\usepackage[T2A]{fontenc}
\usepackage[utf8]{inputenc}
\usepackage[normalem]{ulem}
\usepackage{mathtext} % русские буквы в формулах
\usepackage[left=2cm,right=2cm, top=2cm,bottom=2cm,bindingoffset=0cm]{geometry}
\usepackage[english,russian]{babel}
\usepackage[unicode]{hyperref}
\newenvironment{Proof} % имя окружения
{\par\noindent{$\blacklozenge$}} % команды для \begin
{\hfill$\scriptstyle\boxtimes$}
\newcommand{\Rm}{\mathbb{R}}
\newcommand{\Cm}{\mathbb{C}}
\newcommand{\Z}{\mathbb{Z}}
\newcommand{\I}{\mathbb{I}}
\newcommand{\N}{\mathbb{N}}
\newcommand{\rank}{\operatorname{rank}}
\newcommand{\Ra}{\Rightarrow}
\newcommand{\ra}{\rightarrow}
\newcommand{\FI}{\Phi}
\newcommand{\Sp}{\text{Sp}}
\renewcommand{\leq}{\leqslant}
\renewcommand{\geq}{\geqslant}
\renewcommand{\alpha}{\upalpha}
\renewcommand{\beta}{\upbeta}
\renewcommand{\gamma}{\upgamma}
\renewcommand{\delta}{\updelta}
\renewcommand{\varphi}{\upvarphi}
\renewcommand{\phi}{\upvarphi}
\renewcommand{\tau}{\uptau}
\renewcommand{\lambda}{\uplambda}
\renewcommand{\psi}{\uppsi}
\renewcommand{\mu}{\upmu}
\renewcommand{\omega}{\upomega}
\renewcommand{\d}{\partial}
\renewcommand{\xi}{\upxi}
\renewcommand{\epsilon}{\upvarepsilon}
\newcommand{\intx}{\int\limits_{x_0}^x}
\newcommand\Norm[1]{\left\| #1 \right\|}
\newcommand{\sumk}{\sum\limits_{k=0}^\infty}
\newcommand{\sumi}{\sum\limits_{i=0}^\infty}
\newtheorem*{theorem}{Теорема}
\newtheorem*{cor}{Следствие}
\newtheorem*{lem}{Лемма}
\begin{document}
	\section*{Ортогональность многочленов Чебышева}
	\subsubsection*{Условие}
	Доказать, что многочлены Чебышева первого рода образуют ортогональную по весу $$p(x)=\dfrac{1}{\sqrt{1-x^2}}$$ на отрезке $[-1, 1]$ систему.
	\subsubsection*{Алгоритм решения}
	В гильбертовом пространстве система функций $\{\varphi_i\}$ ортогональна, если $(\varphi_i, \varphi_j) =0$ $\forall i\ne j$. \\\\
	Возьмем гильбертово пространство $L_2[-1,1]$ с весом $p(x) = \dfrac{1}{\sqrt{1-x^2}}$. В данном случае $$(\varphi_i, \varphi_j) = \int\limits_{-1}^1 p(x)\varphi_i(x)\varphi_j(x) dx.$$
	Также, поскольку система функций является системой многочленов Чебышева, то $$\varphi_k(x) = T_k(x) = \cos (k\arccos x).$$
	Найдем скалярное произведение двух производных функций из системы многочленов Чебышева:
	\begin{multline*}
		(T_i(x), T_j(x)) = \int\limits_{-1}^1 \dfrac{\cos (i\arccos x)\cos(j \arccos x)}{\sqrt{1-x^2}} dx = \left[\begin{matrix}
			\arccos x = t, & x = \cos t\\
			x=-1 \to t=\pi, & x=1 \to t = 0\\
			dx = -\sin t dt
		\end{matrix}\right] =\\ = \int\limits_{0}^\pi \dfrac{\cos (it)\cos(j t)}{\sqrt{1-\cos t^2}}\sin t\ dt = \int\limits_{0}^\pi \cos (it)\cos(j t) dt = \dfrac12\int\limits_{0}^\pi \cos ((i+j)t)+\cos((i-j) t) dt=\\= \dfrac{1}{2(i+j)}\sin ((i+j)t)\Big|_0^\pi + \dfrac{1}{2(i-j)}\sin ((i-j)t)\Big|_0^\pi = 0,\quad i \ne j.
	\end{multline*}
	Таким образом, система многочленов Чебышева при заданных условиях является ортогональной.
	\end{document}