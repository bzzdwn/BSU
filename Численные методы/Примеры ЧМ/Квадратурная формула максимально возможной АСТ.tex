\documentclass[a4paper, 12pt]{article}
\usepackage{cmap}
\usepackage{amssymb}
\usepackage{amsmath}
\usepackage{graphicx}
\usepackage{amsthm}
\usepackage{upgreek}
\usepackage{setspace}
\usepackage{color}
\usepackage{pgfplots}
\pgfplotsset{compat=1.9}
\usepackage[T2A]{fontenc}
\usepackage[utf8]{inputenc}
\usepackage[normalem]{ulem}
\usepackage{mathtext} % русские буквы в формулах
\usepackage[left=2cm,right=2cm, top=2cm,bottom=2cm,bindingoffset=0cm]{geometry}
\usepackage[english,russian]{babel}
\usepackage[unicode]{hyperref}
\newenvironment{Proof} % имя окружения
{\par\noindent{$\blacklozenge$}} % команды для \begin
{\hfill$\scriptstyle\boxtimes$}
\newcommand{\Rm}{\mathbb{R}}
\newcommand{\Cm}{\mathbb{C}}
\newcommand{\Z}{\mathbb{Z}}
\newcommand{\I}{\mathbb{I}}
\newcommand{\N}{\mathbb{N}}
\newcommand{\rank}{\operatorname{rank}}
\newcommand{\Ra}{\Rightarrow}
\newcommand{\ra}{\rightarrow}
\newcommand{\FI}{\Phi}
\newcommand{\Sp}{\text{Sp}}
\renewcommand{\leq}{\leqslant}
\renewcommand{\geq}{\geqslant}
\renewcommand{\alpha}{\upalpha}
\renewcommand{\beta}{\upbeta}
\renewcommand{\gamma}{\upgamma}
\renewcommand{\delta}{\updelta}
\renewcommand{\varphi}{\upvarphi}
\renewcommand{\phi}{\upvarphi}
\renewcommand{\tau}{\uptau}
\renewcommand{\lambda}{\uplambda}
\renewcommand{\psi}{\uppsi}
\renewcommand{\mu}{\upmu}
\renewcommand{\omega}{\upomega}
\renewcommand{\d}{\partial}
\renewcommand{\xi}{\upxi}
\renewcommand{\epsilon}{\upvarepsilon}
\newcommand{\intx}{\int\limits_{x_0}^x}
\newcommand\Norm[1]{\left\| #1 \right\|}
\newcommand{\sumk}{\sum\limits_{k=0}^\infty}
\newcommand{\sumi}{\sum\limits_{i=0}^\infty}
\newtheorem*{theorem}{Теорема}
\newtheorem*{cor}{Следствие}
\newtheorem*{lem}{Лемма}
\begin{document}
	\section*{Квадратурная формула максимально возможной АСТ}
	\subsubsection*{Условие}
	Построить квадратурную формулу максимально возможной алгебраической степени точности вида $$I(f) = \int\limits_{-1}^{1}\rho (x) f(x)dx \approx A_0 f(x_0) + A_1 f(x_1),$$
	приняв $\rho(x) = 1$.
	\subsubsection*{Алгоритм решения}
	Для построения квадратурной формулы понадобятся соотношения \begin{eqnarray}
	\begin{cases}
		\int\limits_a^b \rho(x) x^idx = \sum\limits_{k=0}^{n}A_kx^i_k,\quad i=\overline{0,m},\\
		\int\limits_a^b \rho(x) x^{m+1}dx \ne \sum\limits_{k=0}^{n}A_kx^i_k;
	\end{cases}
	\end{eqnarray}
	причем в этом случае говорят, что квадратурная формула \textbf{имеет алгебраическую степень точности равную \textit{m}}.\\\\
	Соответственно для построения квадратурной формулы необходимо построить систему из соотношений (1), по которой мы найдем неизвестные. Затем найти следующее выражение, при котором равенство нарушается.\\\\
	Из условия следует, что неизвестными значениями являются $A_0, A_1$, $x_0, x_1$. Итого 4 неизвестных. Для их отыскания следует построить систему из 4-ех уравнений по формуле (1). Тогда в соответствии со степенями $x^i$ получаем систему $$\begin{cases}
	x^0 : \int\limits_{-1}^1 dx = A_0 + A_1,\\
	x^1 : \int\limits_{-1}^1 xdx = A_0x_0 + A_1x_1,\\
	x^2 : \int\limits_{-1}^1 x^2dx = A_0x_0^2 + A_1x_1^2,\\
	x^3 : \int\limits_{-1}^1 x^3dx = A_0x_0^3 + A_1x_1^3.\\
	\end{cases}$$ 
	Поменяем местами значения относительно равенства, заодно вычислив значения интегралов:
	$$\begin{cases}
	A_0 + A_1 = 2,\\
	A_0x_0 + A_1x_1 = 0,\\
	A_0x_0^2 + A_1x_1^2 = \frac23,\\
	A_0x_0^3 + A_1x_1^3 = 0.\\
	\end{cases}$$ 
	Итого имеем нелинейную систему из 4-ех уравнений. Для ее решения будем применять эвристики. Заметим, что можно домножить 2-ое уравнение на $x_1^3$ и вычесть из него 4-ое:
	$$A_1x_1(x_1^2 - x_0^2) = 0.$$
	Причем ни $A_1$, ни $x_1$ не обращаются в ноль, иначе тогда либо $A_0=0$, либо $x_0$ = 0. Значит $$(x_1-x_0)(x_1+x_0) = 0.$$
	В предположении, что узлы не совпадают, имеем $$x_1 = -x_0.$$
	Тогда подставляем это во 2-ое уравнение и имеем 
	$$-A_0x_1 + A_1x_1 = x_1(A_1 - A_0) = 0.$$
	Отсюда $$A_0 = A_1.$$
	А из первого уравнения тогда будет следовать, что $$A_0 = A_1 = 1.$$
	Подставим это в третье уравнение и получим $$x_0^2 + x_1^2 = \dfrac23.$$
	Учитывая, что $x_1 = -x_0$, а тогда $x_1^2 = x_0^2$. А значит $$x_0^2 = x_1^2 = \dfrac{1}{3}\Rightarrow x_0 = -\dfrac{1}{\sqrt3},\ x_1 = \dfrac{1}{\sqrt3}.$$
	В итоге система решена и $$\begin{cases}
		A_0 = A_1 = 1,\\
		x_0 = -\dfrac{1}{\sqrt3},\\
		x_1 = \dfrac{1}{\sqrt3}.
	\end{cases}$$
	Таким образом, система имеет единственное решение. Отсюда можно сделать вывод, что \textbf{для всей квадратурной формулы АСТ $\geq 3$.}\\\\
	Теперь нужно показать, что АСТ $= 3$. Для этого необходимо добавить еще одно уравнение и посмотреть, будет ли выполняться равенство:
	$$x^4 : \int\limits_{-1}^1 x^4dx = A_0x_0^4 + A_1x_1^4.$$
	Вычисляя значение интеграла, получим $$A_0x_0^4 + A_1x_1^4 = \dfrac25.$$
	Подставим теперь известные нам значения $A_0,A_1$, $x_0,x_1$:
	$$\dfrac{1}{9} + \dfrac{1}{9} = \dfrac29\ne \dfrac25.$$ 
	Равенство не выполняется, а значит АСТ = 3 является максимально возможностей АСТ.\\\\
	Сама квадратурная формула будет иметь в таком случае вид
	$$I = \int\limits_{-1}^{1} f(x)dx \approx f\left(-\dfrac{1}{\sqrt3}\right) + f\left(\dfrac{1}{\sqrt3}\right).$$
\end{document}