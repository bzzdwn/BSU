\documentclass[a4paper, 12pt]{article}
\usepackage{cmap}
\usepackage{amssymb}
\usepackage{amsmath}
\usepackage{graphicx}
\usepackage{amsthm}
\usepackage{upgreek}
\usepackage{setspace}
\usepackage{color}
\usepackage[T2A]{fontenc}
\usepackage[utf8]{inputenc}
\usepackage[normalem]{ulem}
\usepackage{mathtext} % русские буквы в формулах
\usepackage[left=2cm,right=2cm, top=2cm,bottom=2cm,bindingoffset=0cm]{geometry}
\usepackage[english,russian]{babel}
\usepackage[unicode]{hyperref}
\newenvironment{Proof} % имя окружения
{\par\noindent{$\blacklozenge$}} % команды для \begin
{\hfill$\scriptstyle\boxtimes$}
\newcommand{\Rm}{\mathbb{R}}
\newcommand{\Cm}{\mathbb{C}}
\newcommand{\Z}{\mathbb{Z}}
\newcommand{\I}{\mathbb{I}}
\newcommand{\N}{\mathbb{N}}
\newcommand{\rank}{\operatorname{rank}}
\newcommand{\Ra}{\Rightarrow}
\newcommand{\ra}{\rightarrow}
\newcommand{\FI}{\Phi}
\newcommand{\Sp}{\text{Sp}}
\renewcommand{\leq}{\leqslant}
\renewcommand{\geq}{\geqslant}
\renewcommand{\alpha}{\upalpha}
\renewcommand{\beta}{\upbeta}
\renewcommand{\gamma}{\upgamma}
\renewcommand{\delta}{\updelta}
\renewcommand{\varphi}{\upvarphi}
\renewcommand{\phi}{\upvarphi}
\renewcommand{\tau}{\uptau}
\renewcommand{\lambda}{\uplambda}
\renewcommand{\psi}{\uppsi}
\renewcommand{\mu}{\upmu}
\renewcommand{\omega}{\upomega}
\renewcommand{\d}{\partial}
\renewcommand{\xi}{\upxi}
\renewcommand{\epsilon}{\upvarepsilon}
\newcommand{\intx}{\int\limits_{x_0}^x}
\newcommand\Norm[1]{\left\| #1 \right\|}
\newcommand{\sumk}{\sum\limits_{k=0}^\infty}
\newcommand{\sumi}{\sum\limits_{i=0}^\infty}
\newtheorem*{theorem}{Теорема}
\newtheorem*{cor}{Следствие}
\newtheorem*{lem}{Лемма}
\begin{document}
	\section*{Дифференциальное уравнение для многочленов Чебышева}
	\subsubsection*{Условие}
	Доказать, что многочлены Чебышева удовлетворяют уравнению $$(1-x^2)T''_n(x) - xT'_n(x) + n^2 T_n(x)=0.$$
	\subsubsection*{Алгоритм решения}
	Многочлены Чебышева задаются формулой $$T_n(x) = \cos (n\arccos x).$$
	Чтобы функции являлись решениями дифференциального уравнения, они должны при подстановке в уравнение давать верное равенство. \\\\
	Вычислим первую и вторую производные от многочленов Чебышева:
	$$T'_n(x) = \dfrac{n \sin (n\arccos x)}{\sqrt{1-x^2}};$$
	$$T''_n(x) = \dfrac{n^2\cos (n\arccos x) \cdot (-\frac{1}{\sqrt{1-x^2}})\cdot \sqrt{1-x^2} + \frac{2x}{2\sqrt{1-x^2}} \cdot n \sin (n\arccos x)}{1-x^2}.$$
	Подставим найденные производные в данное по условию дифференциальное уравнение:
	$$(1-x^2)\cdot \dfrac{-n^2\cos (n\arccos x) + \frac{x}{\sqrt{1-x^2}} \cdot n \sin (n\arccos x)}{1-x^2} -x\cdot \dfrac{n \sin (n\arccos x)}{\sqrt{1-x^2}} + n^2 \cos(n\arccos x) = 0.$$
	Равенство выполняется, следовательно, многочлены Чебышева являются решениями данного дифференциального уравнения.
\end{document}