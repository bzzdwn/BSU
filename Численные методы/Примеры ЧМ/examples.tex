\documentclass[a4paper, 12pt]{report}
\usepackage{cmap}
\usepackage{amssymb}
\usepackage{amsmath}
\usepackage{graphicx}
\usepackage{amsthm}
\usepackage{upgreek}
\usepackage{setspace}
\setcounter{secnumdepth}{5}
\setcounter{tocdepth}{5}
\usepackage[T2A]{fontenc}
\usepackage[utf8]{inputenc}
\usepackage[normalem]{ulem}
\usepackage{mathtext} % русские буквы в формулах
\usepackage[left=2cm,right=2cm, top=2cm,bottom=2cm,bindingoffset=0cm]{geometry}
\usepackage[english,russian]{babel}
\usepackage[unicode]{hyperref}
\usepackage{pgfplots}
\pgfplotsset{compat=1.9}
\newenvironment{Proof} % имя окружения
{\par\noindent{$\blacklozenge$}} % команды для \begin
{\hfill$\scriptstyle\boxtimes$}
\newcommand{\Rm}{\mathbb{R}}
\newcommand{\Cm}{\mathbb{C}}
\newcommand{\Z}{\mathbb{Z}}
\newcommand{\I}{\mathbb{I}}
\newcommand{\N}{\mathbb{N}}
\newcommand{\rank}{\operatorname{rank}}
\newcommand{\Ra}{\Rightarrow}
\newcommand{\ra}{\rightarrow}
\newcommand{\FI}{\Phi}
\newcommand{\Sp}{\text{Sp}}
\renewcommand{\leq}{\leqslant}
\renewcommand{\geq}{\geqslant}
\renewcommand{\alpha}{\upalpha}
\renewcommand{\beta}{\upbeta}
\renewcommand{\gamma}{\upgamma}
\renewcommand{\delta}{\updelta}
\renewcommand{\varphi}{\upvarphi}
\renewcommand{\phi}{\upvarphi}
\renewcommand{\tau}{\uptau}
\renewcommand{\theta}{\uptheta}
\renewcommand{\eta}{\upeta}
\renewcommand{\lambda}{\uplambda}
\renewcommand{\psi}{\uppsi}
\renewcommand{\mu}{\upmu}
\renewcommand{\omega}{\upomega}
\renewcommand{\d}{\partial}
\renewcommand{\xi}{\upxi}
\renewcommand{\epsilon}{\upvarepsilon}
\renewcommand{\Re}{\operatorname{Re}}
\newcommand{\intx}{\int\limits_{x_0}^x}
\newcommand\Norm[1]{\left\| #1 \right\|}
\newcommand{\sumk}{\sum\limits_{k=0}^\infty}
\newcommand{\sumi}{\sum\limits_{i=0}^\infty}
\newtheorem*{theorem}{Теорема}
\newtheorem*{cor}{Следствие}
\newtheorem*{lem}{Лемма}
\title{\textbf{\Huge{Численные методы}}\\Примеры решения типовых задач на специальности «прикладная математика»}
\date{}
\begin{document}
	\maketitle
	\tableofcontents{}
	\newpage
	\chapter{Методы решения нелинейных уравнений.}
	\section{Отделение корней. Метод дихотомии.}
	\subsection{Типовые задачи}
	\begin{enumerate}
		\item Отделить один корень уравнения $\tg x = \lg x.$
		\item Отделить один корень уравнения $\sin x = \dfrac1x.$
		\item С точностью $\epsilon=10^{-2}$ найти приближенное решение уравнения $\tg x = \lg x$ по методу дихотомии.
		\item С точностью $\epsilon=10^{-2}$ найти приближенное решение уравнения $\sin x = \dfrac1x$ по методу дихотомии.
		\item Отделить один корень уравнения $x^7 - 3x + 1 = 0$ методом дихотомии.
	\end{enumerate}
	\subsection{Примеры решения}
	\begin{enumerate}
		\item Отделить один корень уравнения $$2\sin3x = x^2 - 4x + 3.$$
		Для решения задачи нам понадобятся утверждения:
		\begin{enumerate}
			\item \textbf{Теорема 1.} \textit{Если функция $f(x)\in C[a,b]$ и принимает на его концах значения разных знаков, то на этом отрезке существует по крайней мере один корень уравнения $f(x) = 0$.
			Если при этом функция $f(x)$ будет монотонной на отрезке $[a,b]$, то она может иметь только один корень.}
		\end{enumerate}
		Алгоритм решения следующий: с помощью графического метода отделяется корень, проверяется выполнение теоремы 1.\\\\
		Для начала построим графики для данного уравнения, так как в данном случае это легко сделать. Определим две функции $$y_1(x) = 2\sin 3x,\quad y_2(x) = x^2 - 4x+3$$ и построим их графики.
		\begin{center}\begin{tikzpicture}
				\begin{axis}[
					legend pos = north west,
					xlabel = {$x$},
					ylabel = {$y$},
					minor tick num = 2,
					samples=1000,
					xmin = -4,
					xmax = 4,
					ymin = -4,
					ymax = 4,
					grid = major,
					scatter/classes={%
						a={mark=o,draw=black}}
					]
					\legend{$y_1(x)$, $y_2(x)$}
					\addplot[blue] {x^2 - 4*x + 3};
					\addplot[orange] {2*sin(deg(3*x))};
				\end{axis}
		\end{tikzpicture}\end{center}
		Приведем исходное уравнение к виду $f(x) = 0$: $$\underbrace{2\sin 3x - (x^2 - 4x+3)}_{f(x)} = 0.$$
		Область определения функции $f(x)$ совпадает с $\Rm$.
		Таким образом, уравнение $f(x)=0$ имеет 4 корня на отрезке $[-4,4]$ (и на всей числовой прямой). \\\\
		Пусть корнем, для которого мы будем искать приближение, будет корень, лежащий слева (однако по аналогии можно найти приближенное значение любого из остальных корней). Этот корень лежит на отрезке $[0; 1]$. Причем из графика видно, что он располагается до точки, в которой функция $y_1(x) = 2\sin3x$ достигает значения $y_1(x) = 2$, то есть до точки $x = \dfrac\pi6$.
		Таким образом, в качестве отрезка, на котором предположительно располагается исследуемый корень, мы можем взять отрезок $$d = \Big[0; \dfrac\pi6\Big].$$
		Проверим выполнение условий теоремы 1 корня уравнения на отрезке $d$, для этого вычислим значения функции $f(x)$ на концах отрезка $d = \Big[0; \dfrac\pi6\Big]$:
		$$f(0) =2\sin0 - 0^2 + 4\cdot 0 -3 = -3 < 0.$$
		$$f\Big(\dfrac\pi6\Big) = 2\sin \dfrac\pi2 - \dfrac{\pi^2}{36} + 4\dfrac\pi6 - 3 \approx 2 - 0.25 +2-3 = 0.75 > 0.$$
		Функция на концах отрезка меняет знак, значит хотя бы один корень уравнения $f(x) = 0$ лежит в этом отрезке.\\\\
		Исследуем функцию на монотонность. Для этого нам нужно оценить значение производной на отрезке. Определим первую производную исследуемой функции: $$f'(x) = 6\cos3x - 2x + 4.$$ Причем эта функция непрерывна на отрезке $d$, то есть $f\in C\Big[0; \dfrac\pi6\Big]$, так как является результатом сложения непрерывных на этом отрезке функций.\\\\
		Тот факт, что производная не изменяет знак на отрезке докажем аналитически. Разобъем производную на две элементарные функции
		$$f'(x) = \underbrace{6\cos3x}_{z_1(x)} \underbrace{- 2x + 4}_{z_2(x)}.$$
		Берем отрезок $d = \Big[0; \dfrac\pi6\Big]$. 
		\\\\
		Функция $z_1(x) = 6\cos3x$ является на этом отрезке убывающей функцией по свойствам косинуса. Наибольшее значение $$y_1(0)= 6\cos0 = 6,$$ а наименьшее $$y_2\Big(\dfrac\pi6\Big) = 6\cos\dfrac\pi2 = 0.$$
		Таким образом, $z_1(x)$ является строго положительной функцией на отрезке $d$. 
		\\\\
		Рассмотрим функцию $z_2(x) = -2x + 4$. Она также является убывающей на отрезке $d$ функцией по свойствам линейной функции. Ее наибольшее значение $$z_2(0) = 4,$$ а наименьшее $$z_2\Big(\dfrac\pi6\Big) = -\dfrac\pi3 + 4\approx 3.$$
		То есть эта функция также является строго положительной на отрезке $d$. 
		\\\\
		В итоге функция $f'(x)$ состоит из суммы двух строго положительных на отрезке $d$ функций, а следовательно $$3 \leq f'(x) \leq 10 \Rightarrow f'(x) > 0\quad \forall x \in d=\Big[0; \dfrac\pi6\Big].$$ 
		Таким образом, мы доказали, что выбранный нами отрезок числовой прямой содержит ровно один корень исследуемого уравнения. На этом решение задачи отделения корней можно закончить и переходить к отысканию приближенных значений корней, лежащих в этих отрезках с помощью указанных методов.
		\item Методом дихотомии найти с точностью $\epsilon=10^{-2}$ приближенное значение корня уравнения $$2\sin3x = x^2 - 4x + 3.$$
		\textbf{Алгоритм решения задачи:}
		\begin{enumerate}
			\item задаем отрезок $[a,b]$, на котором существует единственное решение уравнения $f(x) = 0$;
			\item делим отрезок $[a,b]$ пополам, то есть строим точку $$c = \dfrac{a+b}{2}$$ и смотрим на знак функции $f(x)$ в точке $c$;
			\item из двух новых отрезков $[a,c]$ и $[b,c]$ выбираем тот отрезок, для которого функция $f(x)$ меняет знак на концах отрезка;
			\item повторяем процедуру до тех пор, пока разность концов отрезка не станет ниже заданной точности $\epsilon$.
		\end{enumerate}
		Из предыдущего примера известно, что на отрезке $d = \left[0; \dfrac\pi6\right]$ лежит корень заданного уравнения. Причем, если обозначим снова $$f(x) = 2\sin 3x - x^2 + 4x - 3,$$ то $$f(0) = -3 < 0,\ f\left(\dfrac \pi 6\right)\approx 0.75 > 0.$$
		\begin{enumerate}
			\item Делим отрезок $d$ пополам. Получаем два отрезка $$d_1 = \left[0, \dfrac \pi {12}\right],\ d_2 = \left[\dfrac \pi {12}, \dfrac \pi {6}\right].$$ 
			Мы имеем новую центральную точку $\dfrac \pi {12}$. Вычислим значение функции в ней: $$f\left(\dfrac{\pi}{12}\right) = 2\sin \dfrac{\pi}{4} - \dfrac{\pi^2}{144} + \dfrac{\pi}3 - 3 \approx -0.61 < 0.$$
			Отсюда делаем вывод, что корень лежит на отрезке $d_2$, так как на концах именно этого отрезка функция имеет разные по знаку значения. Условие достижения точности здесь $$\dfrac{\pi}{6} - \dfrac{\pi}{12} = \dfrac \pi 4\approx 0.79 > 10^{-2}.$$
			Мы сделал одну итерацию. Все дальнейшие действия будут выполняться по аналогии.
			\item Делим отрезок $d_2$ пополам. Получаем два отрезка $$d_3 = \left[\dfrac \pi {12}, \dfrac {3\pi} {24}\right],\ d_4 = \left[\dfrac {3\pi} {24}, \dfrac \pi {6}\right].$$ 
			Вычисляем значение функции $f(x)$ в этой точке:
			$$f\left(\dfrac{3\pi}{24}\right) \approx 0.26 > 0.$$
			Значит корень лежит на отрезке $d_3$. Проверяем условие достижения точности $$\dfrac{3\pi}{24} - \dfrac{\pi}{12} = \dfrac \pi{24}\approx 0.13 > 10^{-2}.$$ 
			\item Делим отрезок $d_3$ пополам. Получаем два отрезка $$d_5 = \left[\dfrac \pi {12}, \dfrac {5\pi} {48}\right],\ d_6 = \left[\dfrac {5\pi} {48}, \dfrac {3\pi} {24}\right].$$ 
			Вычисляем значение функции $f(x)$ в этой точке:
			$$f\left(\dfrac{5\pi}{48}\right) \approx -0.13 < 0.$$
			Значит корень лежит на отрезке $d_6$. Проверяем условие достижения точности $$\dfrac{3\pi}{24} - \dfrac{5\pi}{48} = \dfrac \pi{48}\approx 0.06 > 10^{-2}.$$ 
			\item Делим отрезок $d_6$ пополам. Получаем два отрезка $$d_7 = \left[\dfrac {5\pi} {48}, \dfrac {11\pi} {96}\right],\ d_8 = \left[\dfrac {11\pi} {96}, \dfrac {3\pi} {24}\right].$$ 
			Вычисляем значение функции $f(x)$ в этой точке:
			$$f\left(\dfrac{11\pi}{96}\right) \approx 0.07 > 0.$$
			Значит корень лежит на отрезке $d_7$. Проверяем условие достижения точности $$\dfrac{11\pi}{96} - \dfrac{5\pi}{48} = \dfrac \pi{96}\approx 0.03 > 10^{-2}.$$ 
			\item  Делим отрезок $d_7$ пополам. Получаем два отрезка $$d_9 = \left[\dfrac {5\pi} {48}, \dfrac {21\pi} {192}\right],\ d_{10} = \left[\dfrac {21\pi} {192}, \dfrac {11\pi} {96}\right].$$ 
			Вычисляем значение функции $f(x)$ в этой точке:
			$$f\left(\dfrac {21\pi} {192}\right) \approx -0.02 < 0.$$
			Значит корень лежит на отрезке $d_{10}$. Проверяем условие достижения точности $$\dfrac{11\pi}{96}-\dfrac {21\pi} {192}  = \dfrac \pi{192}\approx 0.01 > 10^{-2}.$$ 
			\item  Делим отрезок $d_{10}$ пополам. Получаем два отрезка $$d_{11} = \left[\dfrac {21\pi} {192}, \dfrac {43\pi} {384}\right],\ d_{12} = \left[\dfrac {43\pi} {384}, \dfrac {11\pi} {96}\right].$$ 
			Вычисляем значение функции $f(x)$ в этой точке:
			$$f\left(\dfrac {43\pi} {384}\right) \approx 0.02 > 0.$$
			Значит корень лежит на отрезке $d_{11}$. Проверяем условие достижения точности $$\dfrac {43\pi} {384} - \dfrac{11\pi}{96} = \dfrac \pi{192}\approx 0.008 < 10^{-2}.$$
		\end{enumerate}
		В итоге, разделив еще раз отрезок $d_{11}$ пополам, получим центральную точку $$x = \dfrac{85 \pi}{784},$$ которая и будет являться приближенным решением уравнения с точностью $\epsilon = 10^{-2}$.
		
	\end{enumerate}
	\section{Метод простой итерации решения нелинейного уравнения.}
	\subsection{Типовые задачи}
	\begin{enumerate}
		\item Отделив корень и приведя к виду, удобному для итераций, выбрать начальное приближение, обеспечивающее выполнение условий теоремы о сходимости метода простой итерации уравнения $\tg x = \lg x.$
		\item Отделив корень и приведя к виду, удобному для итераций, выбрать начальное приближение, обеспечивающее выполнение условий теоремы о сходимости метода простой итерации уравнения $\sin x = \dfrac1x.$
		\item С точностью $\epsilon=10^{-2}$ найти приближенное решение уравнения $\tg x = \lg x$ по методу дихотомии.
		\item С точностью $\epsilon=10^{-2}$ найти приближенное решение уравнения $\sin x = \dfrac1x$ по методу дихотомии.
		\item Отделить один корень уравнения $x^7 - 3x + 1 = 0$ методом дихотомии.
	\end{enumerate}
	\subsection{Примеры решения}
	\begin{enumerate}
		\item Отделив корень и приведя к виду, удобному для итераций, выбрать начальное приближение, обеспечивающее выполнение условий теоремы о сходимости метода простой итерации уравнения $$2\sin3x = x^2 - 4x + 3.$$
		Для решения задачи нам понадобятся следующие утверждения:
		\begin{enumerate}
			\item \textbf{Теорема 2.}
			(о сходимости метода простой итерации)
			Пусть выполняются следующие условия:\begin{enumerate}
				\item функция $\varphi(x)$ определена на отрезке \begin{eqnarray}
					|x - x_0| \leq \delta,
				\end{eqnarray} непрерывна на нем и удовлетворяет условию Липшица с постоянным коэффициентом меньше единицы, то есть $\forall x, \widetilde{x}$ $$|\varphi(x) - \varphi(\widetilde{x})| \leq q |x - \widetilde{x}| ,\quad 0 \leq q < 1;$$
				\item для начального приближения $x_0$ верно неравенство \begin{eqnarray}
					|x_0 - \varphi(x_0)| \leq m;
				\end{eqnarray}
				\item числа $\delta, q, m$ удовлетворяют условию 
				\begin{eqnarray}
					\dfrac{m}{1-q}\leq \delta.
				\end{eqnarray}
			\end{enumerate}
			Тогда \begin{itemize}
				\item уравнение $x = \varphi(x)$ в области $(1)$ имеет решение;
				\item последовательность $x_k$ построенная по правилу $x_{k+1} = \varphi(x_k)$ принадлежит отрезку $[x_0 - \delta, x_0 + \delta]$, является сходящейся и ее предел удовлетворяет уравнению $x = \varphi(x)$.
			\end{itemize}
			\item Выполнение условия Липшица можно заменить более строгим условием 
			\begin{eqnarray}
				|\varphi'(x)| \leq q < 1,\quad x \in [x_0-\delta, x_0 + \delta].
			\end{eqnarray}
		\end{enumerate}
		Алгоритм решения следующий: отделяется корень, выбирается канонический вид и начальное приближение, при котором выполняется теорема 2.\\\\
		\textit{Здесь идет этап отделения корней. В данном файле он пропускается, так как этот этап уже разобран в предыдущем пункте}.\\\\
		Из задачи отделения корней уравнения мы имеем, что на отрезке $d = \left[0, \dfrac\pi6\right]$ существует единственный корень уравнения. Приведем исходное уравнение к виду $f(x) = 0$: $$\underbrace{2\sin 3x - (x^2 - 4x+3)}_{f(x)} = 0.$$
		Область определения функции $f(x)$ совпадает с $\Rm$.\\\\
		Нам необходимо задать формулу канонического вида для итерационного процесса $x = \varphi(x)$. Будем строить ее следующим образом. Возьмем наше уравнение $$f(x) = 0,$$ домножим с двух сторон на постоянную $\lambda$ и прибавим с двух сторон $x$, то есть $$x = \underbrace{x + \lambda f(x)}_{\varphi(x)}.$$ Тогда проверим выполнение условия $$|\varphi'(x)| = |1 + \lambda f'(x)| < 1.$$
		Отсюда $$-2< \lambda f'(x)< 0.$$ Нам известно, что на отрезке $d$ производная имеет положительный знак. Тогда $$-\dfrac{2}{M} < \lambda < 0,\quad M = \max_{[0; \frac\pi6]}|f'(x)|.$$
		Оценим значение $M$ аналитически. Вычислим первую производную функции $f(x)$:
		$$f'(x) = 6\cos 3x - 2x + 4.$$
		Разобъем производную на две элементарные функции
		$$f'(x) = \underbrace{6\cos3x}_{z_1(x)} \underbrace{- 2x + 4}_{z_2(x)}.$$
		Берем отрезок $d = \Big[0; \dfrac\pi6\Big]$. 
		\\\\
		Функция $z_1(x) = 6\cos3x$ является на этом отрезке убывающей функцией по свойствам косинуса. Наибольшее значение $$y_1(0)= 6\cos0 = 6,$$ а наименьшее $$y_2\Big(\dfrac\pi6\Big) = 6\cos\dfrac\pi2 = 0.$$
		Таким образом, $z_1(x)$ является строго положительной функцией на отрезке $d$. 
		\\\\
		Рассмотрим функцию $z_2(x) = -2x + 4$. Она также является убывающей на отрезке $d$ функцией по свойствам линейной функции. Ее наибольшее значение $$z_2(0) = 4,$$ а наименьшее $$z_2\Big(\dfrac\pi6\Big) = -\dfrac\pi3 + 4\approx 3.$$
		То есть эта функция также является строго положительной на отрезке $d$. 
		\\\\
		В итоге функция $f'(x)$ состоит из суммы двух строго положительных на отрезке $d$ функций, а следовательно $$3 \leq f'(x) \leq 10 \Rightarrow f'(x) > 0\quad \forall x \in d=\Big[0; \dfrac\pi6\Big].$$
		А тогда $$M = 10.$$
		Тогда мы можем выбрать $$\lambda \in (-0.2; 0).$$ Тогда возьмем, к примеру, $\lambda = -0.05$ (можно выбрать и другое, но лучше выбирать середину полученного интервала).\\\\
		Таким образом, мы можем задать функцию для канонического вида: $$\varphi(x) = x - 0.05 \cdot (2\sin 3x - x^2 + 4x - 3).$$
		Исследуем сходимость построенного итерационного процесса по теореме 2 по соответствующим пунктам:
		\begin{enumerate}
			\item Возьмем $$\Delta = [0; 0.5]\subset d$$ (для удобства вычислений $\Delta$ короче отрезка $d$, но в ином случае лучше брать $\Delta$ не шире $d$).\\\\
			Сперва зададим начальное приближение как середину рассматриваемого отрезка $x_0 = 0.25$. Очевидно на этом отрезке функция определена, непрерывна и дифференцируема (мы показали это ранее). А тогда $\delta = 0.25$.\\\\
			Проверим выполнение условия (4). Для этого продифференцируем функцию $\varphi(x)$:
			$$\varphi'(x) = 1+\lambda f'(x) = 1 - 0.05 \cdot (6\cos 3x - 2x + 4).$$ Ранее мы показали, что $f'(x) > 3$, тогда $$\varphi'(x) < 1 - 0.05 \cdot 3 \leq 0.85 = q.$$
			\item покажем, что выполняется (2):
			Попробуем вычислить значение $m$ аналитически:
			\begin{multline*}
				| 0.25 - (0.25 - 0.05(2\sin0.75 - 0.0625 + 1 -3 ) | = |0.05(2\sin0.75 -1.9375)|\leq\\ \leq |0.05(2\sin\dfrac\pi4 -1.9375)|=|0.05(\sqrt2 - 1.9375)| \approx0.026
			\end{multline*}
			То есть $$m = 0.026.$$
			\item покажем выполнение (3):
			$$\dfrac{0.026}{(1-0.85)}=0.17(3) \leq 0.25$$
		\end{enumerate}
		В итоге мы построили канонический вид для итераций $$x = x - 0.05 \cdot (2\sin 3x - x^2 + 4x - 3)$$
		и выбрали начальное приближение $x_0=0.25$, при котором метод простой итерации будет сходиться.
	\end{enumerate}
\end{document}