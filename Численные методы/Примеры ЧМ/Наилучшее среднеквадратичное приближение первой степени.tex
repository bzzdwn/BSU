\documentclass[a4paper, 12pt]{article}
\usepackage{cmap}
\usepackage{amssymb}
\usepackage{amsmath}
\usepackage{graphicx}
\usepackage{amsthm}
\usepackage{upgreek}
\usepackage{setspace}
\usepackage{color}
\usepackage{pgfplots}
\pgfplotsset{compat=1.9}
\usepackage[T2A]{fontenc}
\usepackage[utf8]{inputenc}
\usepackage[normalem]{ulem}
\usepackage{mathtext} % русские буквы в формулах
\usepackage[left=2cm,right=2cm, top=2cm,bottom=2cm,bindingoffset=0cm]{geometry}
\usepackage[english,russian]{babel}
\usepackage[unicode]{hyperref}
\newenvironment{Proof} % имя окружения
{\par\noindent{$\blacklozenge$}} % команды для \begin
{\hfill$\scriptstyle\boxtimes$}
\newcommand{\Rm}{\mathbb{R}}
\newcommand{\Cm}{\mathbb{C}}
\newcommand{\Z}{\mathbb{Z}}
\newcommand{\I}{\mathbb{I}}
\newcommand{\N}{\mathbb{N}}
\newcommand{\rank}{\operatorname{rank}}
\newcommand{\Ra}{\Rightarrow}
\newcommand{\ra}{\rightarrow}
\newcommand{\FI}{\Phi}
\newcommand{\Sp}{\text{Sp}}
\renewcommand{\leq}{\leqslant}
\renewcommand{\geq}{\geqslant}
\renewcommand{\alpha}{\upalpha}
\renewcommand{\beta}{\upbeta}
\renewcommand{\gamma}{\upgamma}
\renewcommand{\delta}{\updelta}
\renewcommand{\varphi}{\upvarphi}
\renewcommand{\phi}{\upvarphi}
\renewcommand{\tau}{\uptau}
\renewcommand{\lambda}{\uplambda}
\renewcommand{\psi}{\uppsi}
\renewcommand{\mu}{\upmu}
\renewcommand{\omega}{\upomega}
\renewcommand{\d}{\partial}
\renewcommand{\xi}{\upxi}
\renewcommand{\epsilon}{\upvarepsilon}
\newcommand{\intx}{\int\limits_{x_0}^x}
\newcommand\Norm[1]{\left\| #1 \right\|}
\newcommand{\sumk}{\sum\limits_{k=0}^\infty}
\newcommand{\sumi}{\sum\limits_{i=0}^\infty}
\newtheorem*{theorem}{Теорема}
\newtheorem*{cor}{Следствие}
\newtheorem*{lem}{Лемма}
\begin{document}
	\section*{Наилучшее среднеквадратичное приближение многочленом первой степени}
	\subsubsection*{Условие}
	Построить наилучшее среднеквадратичное приближение к аналитически заданной функции с помощью алгебраического многочлена первой степени: $$f(x) = x^2,\quad x \in [1,2].$$ Оценить величину наилучшего приближения.
	\subsubsection*{Алгоритм решения}
	Наилучшее среднеквадратичное приближение алгебраическим многочленом строится в виде \begin{eqnarray}
	\varphi(x) = c_0 + c_1x + \ldots +c_nx^n,
	\end{eqnarray} где коэффициенты являются решениями СЛАУ
	\begin{eqnarray}
		\begin{cases}
			c_0s_0 + c_1s_1 + \ldots + c_ns_n = m_0,\\
			c_0s_1 + c_1s_2 + \ldots + c_ns_{n+1} = m_1,\\
			\dotfill\\
			c_0s_n + c_1s_{n+1} + \ldots + c_ns_{2n} = m_n.
		\end{cases}
	\end{eqnarray}
	\begin{eqnarray}
		s_i = \int\limits_a^b p(x) x^i dx,\quad m_j= \int\limits_a^b p(x) f(x) x^j dx,\quad i=\overline{0,2n}, j=\overline{0,n}.
	\end{eqnarray}
	В нашем случае формулы принимают вид \begin{eqnarray}
		\varphi(x) = c_0 + c_1x,
	\end{eqnarray}
	\begin{eqnarray}
			\begin{cases}
			c_0\int\limits_a^b p(x) dx+ c_1\int\limits_a^b p(x) x dx= \int\limits_a^b p(x)f(x)dx,\\
			c_0\int\limits_a^b p(x)x dx+ c_1\int\limits_a^b p(x) x^2 dx= \int\limits_a^b p(x)f(x)xdx.
		\end{cases}
	\end{eqnarray}
	По условию ничего не сказано про весовую функцию $p(x)$, поэтому принимаем $p(x) = 1$. Тогда, подставляя известные значения в (5), получаем систему вида 
	$$\begin{cases}
		c_0\int\limits_1^2  dx+ c_1\int\limits_1^2 x dx= \int\limits_1^2 x^2dx,\\
		c_0\int\limits_1^2 x dx+ c_1\int\limits_1^2 x^2 dx= \int\limits_1^2 x^3dx.
	\end{cases}$$
	Вычислим все необходимые интегралы $$\int\limits_1^2  dx = 1,\quad \int\limits_1^2 xdx = \dfrac32,\quad \int\limits_1^2 x^2 dx = \dfrac73,\quad \int\limits_1^2 x^3 dx = \dfrac{15}{4}.$$
	Подставим найденные значения в систему:
	$$\begin{cases}
		c_0+ \dfrac32c_1= \dfrac73,\\
		\dfrac32c_0+ c_1\dfrac73= \dfrac{15}{4}.
	\end{cases}$$
	Запишем СЛАУ в виде матрицы и применим метод Гаусса $$\begin{pmatrix}
	1 & \dfrac32 & \vline & \dfrac73\\\\
	\dfrac32 & \dfrac73 & \vline & \dfrac{15}{4}
	\end{pmatrix}
	\sim
	\begin{pmatrix}
		1 & 0 & \vline & -\dfrac{13}{6}\\
		0 & 1 & \vline & 3
	\end{pmatrix}
	$$
	Таким образом, $c_0 = 3$, $c_1 = -\dfrac{13}{6}$. Тогда приближающий многочлен первой степени имеет вид $$\varphi(x) = 3x - \dfrac{13}{6}.$$
	Величину наилучшего приближения оценим по формуле $$\Norm{f(x) - \varphi(x)} = \left(\int\limits_a^b(f(x) - \varphi(x))^2dx\right)^{\frac12}.$$
	Подставим наши функции и получим \begin{multline*}
		\left(\int\limits_1^2\left(x^2 - 3x + \dfrac{13}{6}\right)^2dx\right)^{\frac12} = \left(\int\limits_1^2x^4 + 9x^2+\dfrac{169}{36} - 6x^3 + \dfrac{13}{3}x^2 - 13xdx\right)^{\frac12} =\\= \left(\dfrac{x^5}{5}\Big|_1^2 -6\cdot \dfrac{x^4}{4}\Big|_1^2+ \dfrac{40}{3}\cdot\dfrac{x^3}{3}\Big|_1^2 - 13\cdot \dfrac{x^2}{2}\Big|_1^2 + \dfrac{169}{36}x\Big|_1^2\right)^{\frac12} = \left(\dfrac{1}{180}\right)^{\frac12} \approx 0.0745.
	\end{multline*}
	Графически это будет выглядеть следующим образом:
	\begin{center}\begin{tikzpicture}
			\begin{axis}[
				title = Function Approximation,
				legend pos = north west,
				xlabel = {$x$},
				ylabel = {$y$},
				minor tick num = 2,
				xmin = 1,
				xmax = 2,
				grid = major
				]
				\legend{$x^2$, $\varphi(x)$}
				\addplot[blue] {x^2};
				\addplot[orange] {3*x - 13/6};
			\end{axis}
	\end{tikzpicture}\end{center}
\end{document}