\documentclass[a4paper, 12pt]{report}
\usepackage{cmap}
\usepackage{amssymb}
\usepackage{amsmath}
\usepackage{graphicx}
\usepackage{amsthm}
\usepackage{upgreek}
\usepackage{setspace}
\setcounter{secnumdepth}{5}
\setcounter{tocdepth}{5}
\usepackage[T2A]{fontenc}
\usepackage[utf8]{inputenc}
\usepackage[normalem]{ulem}
\usepackage{mathtext} % русские буквы в формулах
\usepackage[left=2cm,right=2cm, top=2cm,bottom=2cm,bindingoffset=0cm]{geometry}
\usepackage[english,russian]{babel}
\usepackage[unicode]{hyperref}
\newenvironment{Proof} % имя окружения
{\par\noindent{$\blacklozenge$}} % команды для \begin
{\hfill$\scriptstyle\boxtimes$}
\newcommand{\Rm}{\mathbb{R}}
\newcommand{\Cm}{\mathbb{C}}
\newcommand{\Z}{\mathbb{Z}}
\newcommand{\I}{\mathbb{I}}
\newcommand{\N}{\mathbb{N}}
\newcommand{\rank}{\operatorname{rank}}
\newcommand{\Ra}{\Rightarrow}
\newcommand{\ra}{\rightarrow}
\newcommand{\FI}{\Phi}
\newcommand{\Sp}{\text{Sp}}
\renewcommand{\leq}{\leqslant}
\renewcommand{\geq}{\geqslant}
\renewcommand{\alpha}{\upalpha}
\renewcommand{\beta}{\upbeta}
\renewcommand{\gamma}{\upgamma}
\renewcommand{\delta}{\updelta}
\renewcommand{\varphi}{\upvarphi}
\renewcommand{\phi}{\upvarphi}
\renewcommand{\tau}{\uptau}
\renewcommand{\lambda}{\uplambda}
\renewcommand{\psi}{\uppsi}
\renewcommand{\mu}{\upmu}
\renewcommand{\omega}{\upomega}
\renewcommand{\d}{\partial}
\renewcommand{\xi}{\upxi}
\renewcommand{\epsilon}{\upvarepsilon}
\newcommand{\intx}{\int\limits_{x_0}^x}
\newcommand\Norm[1]{\left\| #1 \right\|}
\newcommand{\sumk}{\sum\limits_{k=0}^\infty}
\newcommand{\sumi}{\sum\limits_{i=0}^\infty}
\newtheorem*{theorem}{Теорема}
\newtheorem*{cor}{Следствие}
\newtheorem*{lem}{Лемма}
\date{}
\begin{document}
	\section*{Задачи с коллоквиума.}
	\begin{enumerate}
		\item Среди всех многочленов вида $3x^2+a_1x+a_0$ найти наименее отклоняющийся от нуля на отрезке $[1,2]$.
		\item Применяя метод градиентного спуска к решению системы $$\begin{cases}
		x_1+x_2 = 2,\\
		x_1x_2 = 1,
		\end{cases}$$
		записать условие минимума функционала, т.е. уравнение $\varphi'(t)=0$.
		\item Используя таблицу разделенных разностей, найти сумму конечного ряда нечетных чисел $S(p) = 1+3+5+\ldots + (2p-1)$.
		\item Построить составную кубатурную формулу средних для вычисления интеграла $$\iint\limits_A f(x,y) dxdy$$ используя триагуляцию области $A$. Вычислить значение интеграла при $А=[-1,1]\times[-1,1]$ и $f(x,y) = x^2+y^2$, разбивая $A$ на два треугольника.
		\item Построить интерполяционное приближение таблично заданной функции $f(x)$ системой функций $\phi_i(x)=e^{ix}$, $i=0,1,\ldots$ по таблице \begin{center}\begin{tabular}[t]{|c|c|c|}
				\hline
				$x$ & 0 & 1 \\
				\hline
				$f(x)$ & 2 & 3 \\
				\hline
		\end{tabular}\end{center}
	\item Методом Лобачевского найти корни уравнения $x^2-5x+4=0$ с точностью $\epsilon=10^{-2}$.
	\item Вывести формулу аналогичную формуле простейших трапеций для интеграла $$\int\limits_{-1}^1\dfrac{f(x)}{\sqrt{1-x^2}}dx.$$
	\item Методом простой итерации решить система 
	$$\begin{cases} 
		x+2\sin y=3,\\
		x^2+2y^2=6.
	\end{cases}$$
	\item Определить степень многочлена Лагранжа для сетки равноотстоящих узлов, обеспечивающую точность приближения функции $e^x$ на отрезке $[0; 1]$ не хуже $\epsilon=10^{-3}$.
	\item Построить квадратурную формулу максимально возможной алгебраической степени точности вида $$I(f) = \int\limits_1^2 p(x)f(x)dx\approx A_0f(x_0) + A_1f(x_1).$$
	\item Построить среднеквадратичное приближение функции $f(x) = |x|$ на отрезке $[-1, 1]$ с помощью функции $\varphi(x)=ax^2 + b$.
	\end{enumerate}
\end{document}


